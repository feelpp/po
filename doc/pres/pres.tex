\documentclass{beamer}
\usepackage[T1]{fontenc}
\usepackage[utf8]{inputenc}
\usepackage[frenchb]{babel}
\usepackage{graphicx}
\usepackage{subfig}
\usepackage{colortbl}
\usepackage{tikz}
\usepackage{bm}
\usepackage{listings}

%listins
\lstset{
language=C++,
frame=single,
keywordstyle=\color{magenta}\bfseries,
commentstyle=\color{green}\itshape,
stringstyle=\color{blue},
showstringspaces=true,
showspaces=false,
breaklines=true,
breakatwhitespace=true,
tabsize=2}


\graphicspath{{Images/}}

\usetheme{Hannover}

\beamertemplatenavigationsymbolsempty

\definecolor{bleufonce}{rgb}{0.1,0.1,0.8}
\definecolor{grisbleu}{rgb}{0.8,0.8,0.9}
\definecolor{rougefonce}{rgb}{0.8,0.1,0.1}
\definecolor{grisrouge}{rgb}{0.9,0.8,0.8}
\definecolor{vertfonce}{rgb}{0.1,0.8,0.1}
\definecolor{grisvert}{rgb}{0.8,0.9,0.8}
\definecolor{bleuunistra}{RGB}{15,80,150}
\setbeamercolor{palette quaternary}{fg=white,bg=bleuunistra}
\setbeamercolor{titlelike}{parent=palette quaternary}

\setbeamertemplate{blocks}[rounded][shadow=true] 
\setbeamercolor{block title}{bg=bleufonce,fg=white}
\setbeamercolor{block body}{bg=grisbleu}
\setbeamercolor{block title alerted}{bg=rougefonce,fg=white}
\setbeamercolor{block body alerted}{bg=grisrouge}
\setbeamercolor{block title example}{bg=vertfonce,fg=white}
\setbeamercolor{block body example}{bg=grisvert}

\newcommand{\grad}{{\nabla}}
\newcommand{\laplace}{{\Delta}}
\newcommand{\rot}{{\nabla\times}}
\newcommand{\rott}{{\nabla^2\times}}
\renewcommand{\div}{{\nabla\cdot}}
\newcommand{\restr}{{\big\rvert_{\partial\Omega}}}
\newcommand{\taille}{0.4}
\newcommand{\taillem}{0.5}
\newcommand{\tailleg}{0.7}

\title[Stage]{Parallélisation d'un code de calcul aérodynamique instationnaire et validation par comparaison avec des données existantes.}
\subtitle{Plastic Omnium, M2 CSMI}

\author{Romain HILD}
\institute{Université de Strasbourg}

\begin{document}

\begin{frame}
\includegraphics[scale=0.2]{uds.jpg}\includegraphics[scale=0.15]{po.jpg}
\titlepage
\end{frame}

\section{Problème}
\begin{frame}{Problème}
\begin{block}{On cherche $(\mathbf{v},p)$ solution de :}
\begin{eqnarray}
\label{depart}
\left\{
\begin{aligned}
&\frac{\partial \mathbf{v}}{\partial t} + (\rot  \mathbf{v})\times \mathbf{v} + \grad q + \frac{1}{Re}\rott  \mathbf{v}-\mathbf{f} = 0\\
&\div \mathbf{v} = 0\\
&\mathbf{v}\big\rvert_{t=0} = \mathbf{v}_0\\
&\mathbf{v}\cdot \mathbf{n}\restr = \alpha_0\\
&(\rot  \mathbf{v})\cdot \mathbf{n}\restr = \alpha_1\\
&(\rott  \mathbf{v})\cdot \mathbf{n}\restr = \alpha_2
\end{aligned}
\right.
\end{eqnarray}
où $q = \frac{|\mathbf{v}|^2}{2}+p$.
\end{block}
\end{frame}

\begin{frame}{Décomposition}
\begin{block}{Décomposition des problèmes}
\begin{center}
\begin{tabular}{|c|c|c|}
\hline
Variables & Problèmes & Espaces\\ \hline
$\mathbf{v}$ & $\mathbf{a} + \mathbf{u}$ & $ L^2_\sigma$\\ \hline
$\mathbf{a}$ & $\grad\psi_0 + \rot \mathbf{b}$ & $ L^2_\sigma$\\ \hline
$\psi_0$ & $-\laplace\psi_0 = 0$ & $ H^1/ H(div)$\\ \hline
$(\mathbf{b},\psi_1)$ & $\rott \mathbf{b}= \grad\psi_1$ & $ H(rot)/ D^1$ \\ \hline
$\mathbf{u}$ & $\sum_i c_i\mathbf{g_i}$ & $ D^1$\\ \hline
$(\lambda_i,\mathbf{g_i})$ & $\rott  \mathbf{g_i} = \Lambda_i \mathbf{g_i}$ & $ D^1$ \\ \hline
$\mathbf{\mathbf{g_i}^0}$ & $\grad(\div \mathbf{\mathbf{g_i}^0}) - \laplace \mathbf{\mathbf{g_i}^0} = \Lambda_i \mathbf{g_i}$ & $ H^1_0$\\ \hline
$\psi_i$ & $-\laplace \psi_i = 0$ & $ H^1$\\ \hline
\end{tabular}
\end{center}
La partie difficile est donc repoussé à trouver les coefficients $c_i$. 
\end{block}
\end{frame}

\begin{frame}{Organigramme}
\begin{figure}
\centering
\begin{tikzpicture}[scale=\taille]
\node[draw,scale=\taille,fill=green!50] (a0) at (0,0) {$\alpha_0$} ;
\node[draw,scale=\taille,fill=green!50] (a1) at (4,0) {$\alpha_1$} ;
\node[scale=\taille] at(0,-2) {\ref{psi0}} ;
\node[draw,scale=\taille,fill=gray!50] (pbpsi0) at (0,-2) 
{$
\begin{aligned}
-\laplace\psi_0&=0\\
\grad\psi_0\cdot \mathbf{n}\restr &= \alpha_0
\end{aligned}$} ;
\node[draw,scale=\taille,fill=gray!50] (pbb) at (4,-3) 
{$\begin{aligned}
\rott \mathbf{b} &= \grad\psi_1\\
\div \mathbf{b} &=0\\ 
\mathbf{b}\cdot \mathbf{n}\restr &= 0\\
\rot \mathbf{b}\cdot \mathbf{n}\restr &= 0\\
\rott \mathbf{b}\cdot \mathbf{n}\restr &= \alpha_1
\end{aligned}$} ;
\node[draw,scale=\taille,fill=gray!50] (pbeigen) at (12,-3) 
{$\begin{aligned}
\rott \mathbf{g_i} = \Lambda_i\mathbf{g_i}\\
\mathbf{g_i}\cdot \mathbf{n}\restr = 0\\
\rot \mathbf{g_i}\cdot \mathbf{n}\restr = 0\\
\rott \mathbf{g_i}\cdot \mathbf{n}\restr = 0
\end{aligned}$} ;
\node[draw,scale=\taille,fill=blue!50] (psi0) at (0,-6) {$\grad\psi_0$} ;
\node[draw,scale=\taille,fill=blue!50] (b) at (4,-6) {$\rot \mathbf{b}$} ;
\node[draw,scale=\taille,fill=yellow!50] (lambda) at (11,-6) {$\Lambda_i$} ;
\node[draw,scale=\taille,fill=yellow!50] (gi) at (13,-6) {$\mathbf{g_i}$} ;
\node[draw,scale=\taille,fill=gray!50] (pbgi0) at (16,-6) 
{$\begin{aligned}
\grad(\div \mathbf{\mathbf{g_i}^0})-\laplace \mathbf{\mathbf{g_i}^0} = \Lambda_i \mathbf{g_i}\\
\mathbf{\mathbf{g_i}^0}\restr = 0
\end{aligned}$} ;
\node[draw,scale=\taille,fill=gray!50] (pba) at (2,-8) {$\mathbf{a} = \rot \mathbf{b} + \grad\psi_0$} ;
\node[draw,scale=\taille,fill=green!50] (f) at (7,-8) {$f$} ;
\node[draw,scale=\taille,fill=green!50] (a2) at (8,-8) {$\alpha_2$} ;
\node[draw,scale=\taille,fill=green!50] (ck0) at (9,-8) {$c_k^0$} ;
\node[draw,scale=\taille,fill=yellow!50] (gi0) at (16,-8) {$\mathbf{\mathbf{g_i}^0}$} ;
\node[draw,scale=\taille,fill=gray!50] (pbpsi) at (16,-10) 
{$\begin{aligned}
-\laplace\psi_i = \div \mathbf{\mathbf{g_i}^0}\\
\grad\psi_i\cdot \mathbf{n}\restr = 0
\end{aligned}$} ;
\node[draw,scale=\taille,fill=yellow!50] (psi) at (16,-12) {$\psi_i$} ;
\node[draw,scale=\taillem,fill=blue!50] (a) at (2,-12) {$a$} ;
\node[draw,scale=\taille,fill=gray!50] (pbs) at (9,-12)
{$\begin{aligned}
\frac{\partial c_k}{\partial t} &+ \sum_i\sum_j c_i\lambda_i c_j(\mathbf{g_i}\times \mathbf{g_j}, \mathbf{g_k}) \\
&+ \sum_i c_i\lambda_i(\mathbf{g_i}\times \mathbf{a},\mathbf{g_k}) + \sum_i c_i((\rot \mathbf{a})\times \mathbf{g_i}, \mathbf{g_k}) \\
&+ \frac{1}{Re}c_k\lambda_k^2 = (\mathbf{h_a},\mathbf{g_k}) + \frac{1}{Re}\langle\alpha_2,\psi_k\rangle\\
&c_k(0)=c_k^0
\end{aligned}
$} ;
\node[draw,scale=\taille,fill=blue!50] (ck) at (9,-15) {$c_k$} ;
\node[draw,scale=\taille,fill=gray!50] (pbu) at (9,-16) {$\mathbf{u}=\sum c_k\mathbf{g_k}$} ;
\node[draw,scale=\taillem,fill=blue!50] (u) at (9,-17) {$\mathbf{u}$} ;
\node[draw,scale=\taille,fill=gray!50] (pbv) at (3,-18) {$\mathbf{v}=\mathbf{a}+\mathbf{u}$} ;
\node[draw,scale=\tailleg,fill=red!50] (v) at (3,-19) {$\mathbf{v}$} ;
\node[draw,scale=\taille,fill=gray!50] (pbq) at (9,-19)
{$\begin{aligned}
-\laplace q = \div((\rot \mathbf{v})\times \mathbf{v}) - \div \mathbf{f}\\
\grad q\cdot \mathbf{n}\restr =  \mathbf{f}\cdot \mathbf{n}\restr - \frac{\partial\alpha_0}{\partial t} - ((\rot \mathbf{v})\times \mathbf{v})\cdot \mathbf{n}\restr - \frac{\alpha_2}{Re}
\end{aligned}$} ;
\node[draw,scale=\tailleg,fill=red!50] (q) at(15,-19) {$q$} ;

\draw[->,>=latex] (a0) -- (pbpsi0);
\draw[->,>=latex] (a1) -- (pbb);
\draw[->,>=latex] (pbpsi0) -- (psi0);
\draw[->,>=latex] (pbb) -- (b);
\draw[->,>=latex] (b) -- (pba);
\draw[->,>=latex] (psi0) -- (pba); 
\draw[->,>=latex] (pba) -- (a); 
\draw[->,>=latex] (pbeigen) -- (lambda);
\draw[->,>=latex] (pbeigen) -- (gi);
\draw[->,>=latex] (gi) -- (pbgi0);
\draw[->,>=latex] (pbgi0) -- (gi0);
\draw[->,>=latex] (gi0) -- (pbpsi);
\draw[->,>=latex] (pbpsi) -- (psi);
\draw[->,>=latex] (a) -- (pbs);
\draw[->,>=latex] (f) -- (pbs);
\draw[->,>=latex] (a2) -- (pbs);
\draw[->,>=latex] (ck0) -- (pbs);
\draw[->,>=latex] (lambda) -- (pbs);
\draw[->,>=latex] (gi) -- (pbs);
\draw[->,>=latex] (psi) -- (pbs);
\draw[->,>=latex] (pbs) -- (ck);
\draw[->,>=latex] (ck) -- (pbu);
\draw[->,>=latex] (pbu) -- (u);
\draw[->,>=latex] (u) -- (pbv);
\draw[->,>=latex] (a) -- (pbv);
\draw[->,>=latex] (pbv) -- (v);
\draw[->,>=latex] (v) -- (pbq);
\draw[->,>=latex] (pbq) -- (q);
\end{tikzpicture}
\end{figure}
\end{frame}

\begin{frame}{$\mathbf{v}=\mathbf{a}+\mathbf{u}$}
\begin{block}{Relèvement par $\mathbf{a}$ pour utiliser les travaux de P. Penel}
\begin{center}
\begin{tabular}{c|ccccc}
& $\mathbf{v}$ & = & $\mathbf{a}$ & + & $\mathbf{u}$ \\ \hline
$\div\star$ & 0 & & 0 & & 0\\ \hline
$\star\cdot \mathbf{n}\restr$ & $\alpha_0$ & & $\alpha_0$ & & 0\\ \hline
$\rot\star\cdot \mathbf{n}\restr$ & $\alpha_1$ & & $\alpha_1$ & & 0\\\hline
$\rott\star\cdot \mathbf{n}\restr$ & $\alpha_2$ & & 0 & & $\alpha_2$ 
\end{tabular}
\end{center}
\end{block}
\end{frame}

\section{Galerkin}
\begin{frame}{$\mathbf{u}=\sum c_i\mathbf{g_i}$}
\begin{block}{Décomposition de Galerkin généralisée}
\begin{itemize}
\item les coefficients $c_i$ portent la dimension temporelle.
\item les fonctions de base de $D^1$ portent la dimension spatiale.
\item $D^1$ est engendré par les fonctions propres de l'opérateur rotationnel.
\item les fonctions propres de $\rot\star$ sont aussi celles de $\rott\star$.
\item les valeurs propres de $\rot\star$ sont les racines carrées de celles de $\rott\star$.
\item pour ne pas avoir de multiplicité > 1, on ne garde que les positives : $\lambda_i=\sqrt{\Lambda_i}$
\item le signe est porté par le coefficient $c_k$.
\end{itemize}
\end{block}
\end{frame}

\section{Fonctions propres}
\begin{frame}{$\rott  \mathbf{g_i} = \Lambda_i \mathbf{g_i}$}
\begin{block}{Fonctions propres de $\rott\star$}
\begin{itemize}
\item Problème :
\[
\left\{
\begin{aligned}
&\rott  \mathbf{g_i} = \Lambda_i \mathbf{g_i}\\
&\mathbf{g_i}\cdot \mathbf{n}\restr = 0\\
&\rot \mathbf{g_i}\cdot \mathbf{n}\restr = 0\\
&\rott  \mathbf{g_i}\cdot \mathbf{n}\restr = 0
\end{aligned}
\right.
\]
\item Formulation variationnelle :
\[
\int_\Omega (\rot \mathbf{g})(\rot\bm{\varphi})\ dX = \Lambda\int_\Omega \mathbf{g}\bm{\varphi}\ dX
\]
\end{itemize}
\end{block}
\end{frame}

\begin{frame}{Décomposition des fonctions propres}
\begin{block}{Décomposition des $\mathbf{g_i}$}
\begin{center}
\begin{tabular}{c|ccccc}
& $\mathbf{g_i}$ & = & $\mathbf{\mathbf{g_i}^0}$ & + & $\grad\psi_i$ \\ \hline
$\rott\star$ & $\Lambda_i \mathbf{g_i}$ & & $\grad(\div \mathbf{\mathbf{g_i}^0})-\laplace \mathbf{\mathbf{g_i}^0}$ & & 0\\ \hline
$\div\star$ & 0 & & $\div \mathbf{\mathbf{g_i}^0}$ & & $\laplace\psi_i$\\ \hline
$\star\cdot \mathbf{n}\restr$ & 0 & & 0 & & 0
\end{tabular}
\end{center}
\end{block}
\begin{block}{$\mathbf{g_i}=\mathbf{\mathbf{g_i}^0}+\grad\psi_i$}
\begin{itemize}
\item Problème :
\[
\left\{
\begin{aligned}
\grad(\div \mathbf{\mathbf{g_i}^0})-\laplace \mathbf{\mathbf{g_i}^0} &= \Lambda_i \mathbf{g_i}\\
\mathbf{\mathbf{g_i}^0}\restr &= 0
\end{aligned}
\right.
\]
\item Formulation variationnelle :
\[
-\int_\Omega (\div \mathbf{\mathbf{g_i}^0})(\div\bm{\varphi}) + \int_\Omega \grad \mathbf{\mathbf{g_i}^0}\grad\bm{\varphi} = \int_\Omega \Lambda_i\mathbf{g_i}\bm{\varphi}
\]
\end{itemize}
\end{block}
\end{frame}

\begin{frame}{Décomposition des fonctions propres}
\begin{block}{$\mathbf{g_i}=\mathbf{\mathbf{g_i}^0}+\grad\psi_i$}
\begin{itemize}
\item Problème :
\[
\left\{
\begin{aligned}
-\laplace\psi_i &= \div \mathbf{\mathbf{g_i}^0}\\
\grad\psi_i\cdot \mathbf{n}\restr &= 0
\end{aligned}
\right.
\]
\item Formulation variationnelle :
\[
\int_\Omega \grad\psi_i\grad\varphi = \int_\Omega (\div \mathbf{\mathbf{g_i}^0})\varphi
\]
\end{itemize}
\end{block}
\end{frame}



\section{Relèvement}
\begin{frame}{$\mathbf{a}=\grad\psi_0+\rot \mathbf{b}$}
\label{psi0}
\begin{block}{Décomposition de $\mathbf{a}$}
\begin{center}
\begin{tabular}{c|ccccc}
& $\mathbf{a}$ & = & $\grad\psi_0$ & + & $\rot \mathbf{b}$ \\ \hline
$\div\star$ & 0 & & $\laplace\psi_0$ & & 0\\ \hline
$\star\cdot \mathbf{n}\restr$ & $\alpha_0$ & & $\alpha_0$ & & 0\\ \hline
$\rot\star\cdot \mathbf{n}\restr$ & $\alpha_1$ & & 0 & & $\alpha_1$
\end{tabular}
\end{center}
\end{block}
\begin{columns}[t]
\begin{column}{5cm}
\begin{block}{Problème dans $ H^1$}
\[\left\{
\begin{aligned}
&-\laplace\psi_0 = 0\\
&\grad\psi_0\cdot \mathbf{n}\restr=\alpha_0
\end{aligned}
\right.\]
\end{block}
\end{column}
\begin{column}{5cm}
\begin{block}{Problème dans $ H(div)$}
\[\left\{
\begin{aligned}
\mathbf{w} &= \grad \psi_0\\
\div \mathbf{w} &= 0\\
\mathbf{w}\cdot \mathbf{n}\restr &= \alpha_0
\end{aligned}
\right.\]
\end{block}
\end{column}
\end{columns}
\end{frame}

\begin{frame}{$\grad\psi_0$}
\begin{block}{Problème dans $ H^1$}
\begin{itemize}
\item Problème :
\[\left\{
\begin{aligned}
&-\laplace\psi_0 = 0\\
&\grad\psi_0\cdot \mathbf{n}\restr=\alpha_0
\end{aligned}
\right.\]
\item Formulation variationnelle :
\[
-\int_\Omega \grad\psi_0\cdot\grad\varphi + \int_{\partial\Omega} \alpha_0\varphi = 0
\]
\end{itemize}
\end{block}
\begin{columns}[t]
\begin{column}{5cm}
\begin{exampleblock}{Avantages}
\begin{itemize}
\item[+] Plus simple
\end{itemize}
\end{exampleblock}
\end{column}
\begin{column}{5cm}
\begin{alertblock}{Inconvénients}
\begin{itemize}
\item[$-$] Perte de régularité sur $\grad\psi^0$
\end{itemize}
\end{alertblock}
\end{column}
\end{columns}
\end{frame}

\begin{frame}{$\grad\psi_0$}
\begin{block}{Problème dans $ H(div)$}
\begin{itemize}
\item Problème :
\[\left\{
\begin{aligned}
\mathbf{w} &= \grad \psi^0\\
\div\mathbf{w} &= 0\\
\mathbf{w}\cdot \mathbf{n}\restr &= \alpha_0
\end{aligned}
\right.\]
\item Formulation variationnelle :
\[
-\int_\Omega \mathbf{w}\cdot\bm{\varphi} + \int_\Omega \mathbf{w}\cdot\grad\nu + \int_\Omega \grad\psi^0\cdot\bm{\varphi}  = \int_{\partial\Omega} \alpha_0\nu
\]
\end{itemize}
\end{block}
\begin{columns}[t]
\begin{column}{5cm}
\begin{exampleblock}{Avantages}
\begin{itemize}
\item[+] Gain d'un ordre de régularité sur $\grad\psi^0$
\end{itemize}
\end{exampleblock}
\end{column}
\begin{column}{5cm}
\begin{alertblock}{Inconvénients}
\begin{itemize}
\item[$-$] Utilisation des éléments de Raviert-Thomas en 3D
\item[$-$] Problème bien posé ?
\end{itemize}
\end{alertblock}
\end{column}
\end{columns}
\end{frame}

\begin{frame}{$\rot \mathbf{b}$}
\begin{columns}[t]
\begin{column}{5cm}
\begin{block}{Problème dans $ H(rot)$}
\[\left\{
\begin{aligned}
&\rott \mathbf{b} = \grad\psi_1\\
&\div \mathbf{b} = 0\\
&\mathbf{b}\cdot \mathbf{n}\restr = 0\\
&\rot \mathbf{b}\cdot \mathbf{n}\restr = 0\\
&\grad\psi_1\cdot \mathbf{n}\restr = \alpha_1
\end{aligned}
\right.\]
\end{block}
\end{column}
\begin{column}{5cm}
\begin{block}{$ D^1$}
\[\left\{
\begin{aligned}
&\rott \mathbf{b} = \grad\psi_1\\
&\mathbf{b}\cdot \mathbf{n}\restr = 0\\
&\rot \mathbf{b}\cdot \mathbf{n}\restr = 0\\
&\rott b\cdot \mathbf{n}\restr = \alpha_1
\end{aligned}
\right.\]
où $\psi^1$ solution de
\[\left\{
\begin{aligned}
&-\laplace\psi_1 = 0\\
&\grad\psi_1\cdot \mathbf{n}\restr=\alpha_1
\end{aligned}
\right.\]
\end{block}
\end{column}
\end{columns}
\end{frame}

\begin{frame}{$\rot \mathbf{b}$}
\begin{block}{Formulation variationnelle dans $ H(rot)$}
\begin{align*}
\int_\Omega (\rot \mathbf{b})(\rot\bm{\varphi}) &- \int_{\partial\Omega} (\rot \mathbf{b})(\bm{\varphi}\cdot \mathbf{n}) \\
&+\int_\Omega \psi_1(\div\bm{\varphi}) - \int_{\partial\Omega} \psi_1(\bm{\varphi}\cdot \mathbf{n}) = 0
\end{align*}
\end{block}
\begin{columns}[t]
\begin{column}{5cm}
\begin{exampleblock}{Avantages}
\begin{itemize}
\item[+] Problème plus simple
\end{itemize}
\end{exampleblock}
\end{column}
\begin{column}{5cm}
\begin{alertblock}{Inconvénients}
\begin{itemize}
\item[$-$] Besoin de changer la formulation faible de $u$
\end{itemize}
\end{alertblock}
\end{column}
\end{columns}
\end{frame}

\begin{frame}{$\rot \mathbf{b}$}
\begin{block}{Problème dans $ D^1$}
\begin{itemize}
\item Formulation faible
\[
\int_\Omega (\rot\mathbf{b})\cdot(\rot\bm{\varphi}) + \int_{\partial\Omega} \phi\alpha_1 = \int_\Omega \grad\psi_1\cdot\bm{\varphi}
\]
\item Discrétisation $\mathbf{b}=\sum d_ig_i$
\begin{align*}
d_k\lambda_k^2 = (\grad\psi^1,\mathbf{g_k}) - \langle\alpha_1,\psi_k\rangle
\end{align*}
\end{itemize}
\end{block}
\begin{columns}[t]
\begin{column}{5cm}
\begin{exampleblock}{Avantages}
\begin{itemize}
\item[+] Gain de régularité sur $\rot \mathbf{b}$
\item[+] Réutilisation de la base $\mathbf{g_i}$ 
\end{itemize}
\end{exampleblock}
\end{column}
\begin{column}{5cm}
\begin{alertblock}{Inconvénients}
\begin{itemize}
\item[$-$] Utilisation des éléments de Nedelec en 3D
\end{itemize}
\end{alertblock}
\end{column}
\end{columns}
\end{frame}

\section{Problème spectral}
\begin{frame}{Problème spectral}
\begin{block}{Problème dans $ D^1$}
\[
\left\{
\begin{aligned}
&\frac{\partial \mathbf{u}}{\partial t} + (\rot \mathbf{u})\times \mathbf{u} + (\rot \mathbf{u})\times \mathbf{a} + \left(\rot \mathbf{a}\right)\times \mathbf{u} \\
&+ \grad\pi_\mathbf{a} + \frac{1}{Re}\rott \mathbf{u} - \mathbf{h_a} = 0\\
&\div \mathbf{u} = 0\\
&\mathbf{u}\cdot \mathbf{n}\restr = 0\\
&(\rot \mathbf{u})\cdot \mathbf{n}\restr = 0\\
&(\rott \mathbf{u})\cdot \mathbf{n}\restr = \alpha_2
\end{aligned}
\right.
\]
\end{block}
\end{frame}

\begin{frame}{Problème spectral}
\begin{block}{Forme variationnelle}
\begin{align*}
\int_\Omega \frac{\partial \mathbf{u}}{\partial t}\cdot \bm{\varphi} &+ \int_\Omega ((\rot \mathbf{u})\times \mathbf{u})\cdot \bm{\varphi} + \int_\Omega ((\rot \mathbf{u})\times \mathbf{a})\cdot\bm{\varphi} \\
&+ \int_\Omega ((\rot \mathbf{a})\times \mathbf{u})\cdot\bm{\varphi} + \frac{1}{Re}\int_\Omega (\rot \mathbf{u})\cdot(\rot\bm{\varphi}) \\
&-\frac{1}{Re}\int_{\partial\Omega} \alpha_2\phi = \int_\Omega \mathbf{h_a}\cdot\bm{\varphi}
\end{align*}
\end{block}
On peut maintenant utiliser la relation 
\[
\mathbf{u}\approx\sum_{i=1}^M c_i\mathbf{g_i}
\]
\end{frame}

\begin{frame}{Problème spectral}
\begin{block}{Discrètisation}
\begin{align*}
\sum \frac{\partial c_i}{\partial t}(\mathbf{g_i},\mathbf{g_k}) &+ \sum_i\sum_j c_i\lambda_i c_j(\mathbf{g_i}\times \mathbf{g_j}, \mathbf{g_k}) \\
&+ \sum c_i\lambda_i(\mathbf{g_i}\times \mathbf{a},\mathbf{g_k}) + \sum c_i((\rot \mathbf{a})\times \mathbf{g_i}, \mathbf{g_k}) \\
&+ \frac{1}{Re}\sum c_i\lambda_i\lambda_k(\mathbf{g_i},\mathbf{g_k}) = (\mathbf{h_a},\mathbf{g_k}) + \frac{1}{Re}\langle\alpha_2,\psi_k\rangle
\end{align*}
Orthonormalisation de la base $(\mathbf{g_i})$ : $(\mathbf{g_i},\mathbf{g_k})=\delta_{ik}$
\begin{eqnarray*}
\frac{\partial c_k}{\partial t} + \sum_i\sum_j c_i\lambda_i c_j(\mathbf{g_i}\times \mathbf{g_j}, \mathbf{g_k}) + \sum c_i\lambda_i(\mathbf{g_i}\times \mathbf{a},\mathbf{g_k})\\
+ \sum c_i((\rot \mathbf{a})\times \mathbf{g_i}, \mathbf{g_k}) + \frac{1}{Re}c_k\lambda_k^2 = (\mathbf{h_a},\mathbf{g_k}) + \frac{1}{Re}\langle\alpha_2,\psi_k\rangle
\end{eqnarray*}
\end{block}
\end{frame}

\section{Synthèse}
\begin{frame}{Organigramme}
\begin{figure}
\centering
\begin{tikzpicture}[scale=0.4] 
\node[draw,scale=\taille,fill=gray!50] (pbeigen) at (9,-0.5)
{$\begin{aligned}
\rott \mathbf{g_i} = \Lambda_i\mathbf{g_i}\\
\mathbf{g_i}\cdot \mathbf{n}\restr = 0\\
\rot \mathbf{g_i}\cdot \mathbf{n}\restr = 0\\
\rott \mathbf{g_i}\cdot \mathbf{n}\restr = 0
\end{aligned}$} ;
\node[draw,scale=\taille,fill=green!50] (a0) at (0,-1.5) {$\alpha_0$} ;
\node[draw,scale=\taille,fill=green!50] (a1) at (4,-1.5) {$\alpha_1$} ;
\node[draw,scale=\taille,fill=yellow!50] (lambda) at (8,-3) {$\Lambda_i$} ;
\node[draw,scale=\taille,fill=yellow!50] (gi) at (10,-3) {$\mathbf{g_i}$} ;
\node[draw,scale=\taille,fill=gray!50] (pbgi0) at (14,-3)
{$\begin{aligned}
\grad(\div \mathbf{g_i^0})-\laplace \mathbf{g_i^0} = \Lambda_i \mathbf{g_i}\\
\mathbf{g_i^0}\restr = 0
\end{aligned}$} ;
\node[draw,scale=\taille,fill=gray!50] (pbpsi0) at (0,-3.5)
{$\begin{aligned}
l^0=\grad\psi^0\\
\div\psi^0=0\\
l^0\cdot \mathbf{n}\restr &= \alpha_0
\end{aligned}$} ;
\node[draw,scale=\taille,fill=gray!50] (pbpsi1) at (4,-3.5)
{$\begin{aligned}
l^1=\grad\psi^1\\
\div\psi^1=0\\
l^1\cdot \mathbf{n}\restr &= \alpha_1
\end{aligned}$} ;
\node[draw,scale=\taille,fill=yellow!50] (gi0) at (14,-4.5) {$\mathbf{g_i^0}$} ;
\node[draw,scale=\taille,fill=blue!50] (psi0) at (0,-5.5) {$\grad\psi^0$} ;
\node[draw,scale=\taille,fill=blue!50] (psi1) at (4,-5.5) {$\grad\psi^1$} ;
\node[draw,scale=\taille,fill=gray!50] (pbpsi) at (14,-6)
{$\begin{aligned}
-\laplace\psi_i = \div \mathbf{g_i^0}\\
\grad\psi_i\cdot \mathbf{n}\restr = 0
\end{aligned}$} ;
\node[draw,scale=\taille,fill=yellow!50] (psi) at (14,-7.5) {$\psi_i$} ;
\node[draw,scale=\taille,fill=gray!50] (pbb) at (4,-7.5) 
{$
d_k\lambda_k^2 = (\grad\psi^1,\mathbf{g_k}) - \langle\alpha_1,\psi_k\rangle
$} ;
\node[draw,scale=\taille,fill=blue!50] (b) at (4,-8.5) {$\rot \mathbf{b}$} ;
\node[draw,scale=\taille,fill=gray!50] (pba) at (2,-10.5) {$\mathbf{a} = \rot \mathbf{b} + \grad\psi^0$} ;
\node[draw,scale=\taille,fill=green!50] (f) at (6,-9) {$f$} ;
\node[draw,scale=\taille,fill=green!50] (a2) at (7,-9) {$\alpha_2$} ;
\node[draw,scale=\taille,fill=green!50] (ck0) at (8,-9) {$c_k^0$} ;
\node[draw,scale=\taillem,fill=blue!50] (a) at (2,-12) {$\mathbf{a}$} ;
\node[draw,scale=\taille,fill=gray!50] (pbs) at (9,-12)
{$\begin{aligned}
\frac{\partial c_k}{\partial t} &+ \sum_i\sum_j c_i\lambda_i c_j(\mathbf{g_i}\times \mathbf{g_j}, \mathbf{g_k}) \\
&+ \sum_i c_i\lambda_i(\mathbf{g_i}\times \mathbf{a},\mathbf{g_k}) + \sum_i c_i((\rot \mathbf{a})\times \mathbf{g_i}, \mathbf{g_k}) \\
&+ \frac{1}{Re}c_k\lambda_k^2 = (\mathbf{h_a},\mathbf{g_k}) + \frac{1}{Re}\langle\alpha_2,\psi_k\rangle\\
&c_k(0)=c_k^0
\end{aligned}
$} ;
\node[draw,scale=\taille,fill=blue!50] (ck) at (9,-15) {$c_k$} ;
\node[draw,scale=\taille,fill=gray!50] (pbu) at (9,-16) {$\mathbf{u}=\sum c_kg_k$} ;
\node[draw,scale=\taillem,fill=blue!50] (u) at (9,-17) {$\mathbf{u}$} ;
\node[draw,scale=\taille,fill=gray!50] (pbv) at (3,-17.5) {$\mathbf{v}=\mathbf{a}+\mathbf{u}$} ;
\node[draw,scale=\tailleg,fill=red!50] (v) at (3,-18.5) {$\mathbf{v}$} ;
\node[draw,scale=\taille,fill=gray!50] (pbq) at (9,-18.5)
{$\begin{aligned}
-\laplace q = \div((\rot \mathbf{v})\times \mathbf{v}) - \div \mathbf{f}\\
\grad q\cdot \mathbf{n}\restr =  \mathbf{f}\cdot \mathbf{n}\restr - \frac{\partial\alpha_0}{\partial t} - ((\rot \mathbf{v})\times \mathbf{v})\cdot \mathbf{n}\restr - \frac{\alpha_2}{Re}
\end{aligned}$} ;
\node[draw,scale=\tailleg,fill=red!50] (q) at(15,-18.5) {$p$} ;

\draw[->,>=latex] (a0) -- (pbpsi0);
\draw[->,>=latex] (a1) -- (pbpsi1);
\draw[->,>=latex] (pbpsi0) -- (psi0);
\draw[->,>=latex] (pbpsi1) -- (psi1);
\draw[->,>=latex] (psi1) -- (pbb);
\draw[->,>=latex] (psi) -- (pbb);
\draw[->,>=latex] (lambda) -- (pbb);
\draw[->,>=latex] (gi) -- (pbb);
\draw[->,>=latex] (pbb) -- (b);
\draw[->,>=latex] (b) -- (pba);
\draw[->,>=latex] (psi0) -- (pba);
\draw[->,>=latex] (pba) -- (a);
\draw[->,>=latex] (pbeigen) -- (lambda);
\draw[->,>=latex] (pbeigen) -- (gi);
\draw[->,>=latex] (gi) -- (pbgi0);
\draw[->,>=latex] (pbgi0) -- (gi0);
\draw[->,>=latex] (gi0) -- (pbpsi);
\draw[->,>=latex] (pbpsi) -- (psi);
\draw[->,>=latex] (a) -- (pbs);
\draw[->,>=latex] (f) -- (pbs);
\draw[->,>=latex] (a2) -- (pbs);
\draw[->,>=latex] (ck0) -- (pbs);
\draw[->,>=latex] (lambda) -- (pbs);
\draw[->,>=latex] (gi) -- (pbs);
\draw[->,>=latex] (psi) -- (pbs);
\draw[->,>=latex] (pbs) -- (ck);
\draw[->,>=latex] (ck) -- (pbu);
\draw[->,>=latex] (pbu) -- (u);
\draw[->,>=latex] (u) -- (pbv);
\draw[->,>=latex] (a) -- (pbv);
\draw[->,>=latex] (pbv) -- (v);
\draw[->,>=latex] (v) -- (pbq);
\draw[->,>=latex] (pbq) -- (q);
\end{tikzpicture}
\end{figure}

\end{frame}

\section{Explications}
\begin{frame}{Préliminaires}
\begin{block}{Cylindre}
\begin{center}
\begin{tikzpicture}
\begin{scope}[x={(.7cm,-.3cm)}]
\path (1,0,0);
\pgfgetlastxy{\cylxx}{\cylxy}
\path (0,1,0);
\pgfgetlastxy{\cylyx}{\cylyy}
\path (0,0,1);
\pgfgetlastxy{\cylzx}{\cylzy}
\pgfmathsetmacro{\cylt}{(\cylzy * \cylyx - \cylzx * \cylyy)/ (\cylzy * \cylxx - \cylzx * \cylxy)}
\pgfmathsetmacro{\ang}{atan(\cylt)}
\pgfmathsetmacro{\ct}{1/sqrt(1 + (\cylt)^2)}
\pgfmathsetmacro{\st}{\cylt * \ct}
\begin{scope}[every path/.style={ultra thick}]
\draw (0,0,0) circle[radius=1];
\draw[->] (-1,3,1) -- (0,3,1);
\draw (0,3,1) node [right] {$x$};
\draw[->] (-1,3,1) -- (-1,4,1);
\draw (-1,4,1) node [right] {$y$};
\draw[->] (-1,3,1) -- (-1,3,-1);
\draw (-1,3,-1) node [right] {$z$};
\draw (\ct,\st,0) -- ++(0,0,-8);
\draw (-\ct,-\st,0) -- ++(0,0,-8);
\draw (\ct,\st,-8) arc[start angle=\ang,delta angle=180,radius=1];
\draw[dashed] (\ct,\st,-8) arc[start angle=\ang,delta angle=-180,radius=1];
\draw (0,0,0) node {$\Gamma_1$};
\draw (0,0,-8) node {$\Gamma_2$};
\draw (1,0,-4) node {$\Gamma_3$};
\end{scope}
\end{scope}
\end{tikzpicture}
\end{center}
l=0.5, R=0.05, $\partial\Omega = \Gamma_1+\Gamma_2+\Gamma_3$
\end{block}
\end{frame}

\begin{frame}{Problème}
\begin{block}{On traite le problème aux valeurs propres}
\[
\begin{aligned}
\rott \mathbf{g_i} = -\laplace\bm{g_i} = \Lambda_i\mathbf{g_i}\\
\mathbf{g_i}\cdot \mathbf{n}\restr = 0\\
\rot \mathbf{g_i}\cdot \mathbf{n}\restr = 0\\
\rott \mathbf{g_i}\cdot \mathbf{n}\restr = 0
\end{aligned}
\]
\end{block}

Or, dans le cas du cylindre :
\[
\bm{v}=\begin{pmatrix}
0\\
0\\
v_z
\end{pmatrix}
\text{ et }
\rot \bm{v} = \begin{pmatrix}
\frac{\partial v_z}{\partial y}\\
-\frac{\partial v_z}{\partial x}\\
0
\end{pmatrix}
\]
\end{frame}

\begin{frame}{Problème}
\begin{block}{Décomposition en plusieurs composantes}
On cherche donc à exprimer $\bm{u}=\bm{v}-\bm{a}$ comme $\bm{u}=\sum_i c_i\bm{g_i}$ avec :
\[
\bm{g_i}=\begin{pmatrix}
0\\
0\\
g_i^z
\end{pmatrix}
\text{ et donc }
\rot\bm{g_i}=\begin{pmatrix}
\frac{\partial g_i^z}{\partial y}\\
-\frac{\partial g_i^z}{\partial x}\\
0
\end{pmatrix}
\]
\end{block}
On va donc écrire le problème en décomposant $\bm{g_i}=(g_i^x,g_i^y,g_i^z)$.
\end{frame}

\begin{frame}{Problème}
\begin{align}
-\laplace\bm{g_i}=\begin{pmatrix}-\laplace g_i^x\\-\laplace g_i^y\\-\laplace g_i^z\end{pmatrix}&=\lambda_i^2\begin{pmatrix}g_i^x\\g_i^y\\g_i^z\end{pmatrix} & \Omega\\
\div\bm{g_i}=\partial_x g_i^x+\partial_y g_i^y + \partial_z g_i^z &= 0 &\Omega \label{contdiv}\\
\bm{g_i}\cdot\bm{n} = g_i^xn^x + g_i^yn^y + g_i^zn^z &= 0 &\partial\Omega \label{bc1}\\
\rot\bm{g_i}\cdot \bm{n} = \begin{pmatrix}
\partial_y g_i^z - \partial_z g_i^y\\
\partial_z g_i^x - \partial_x g_i^z\\
\partial_x g_i^y - \partial_y g_i^x
\end{pmatrix}\cdot\begin{pmatrix}n^x\\n^y\\n^z\end{pmatrix} &= 0 &\partial\Omega \label{bc2}
\end{align}
\end{frame}

\begin{frame}{Vecteur normal}
Sur $\Gamma_1$ et $\Gamma_2$, $\bm{n}=(0,0,\pm 1)$ :
\begin{center}
\begin{tikzpicture}[scale=0.5]
\begin{scope}[x={(.7cm,-.3cm)}]
\path (1,0,0);
\pgfgetlastxy{\cylxx}{\cylxy}
\path (0,1,0);
\pgfgetlastxy{\cylyx}{\cylyy}
\path (0,0,1);
\pgfgetlastxy{\cylzx}{\cylzy}
\pgfmathsetmacro{\cylt}{(\cylzy * \cylyx - \cylzx * \cylyy)/ (\cylzy * \cylxx - \cylzx * \cylxy)}
\pgfmathsetmacro{\ang}{atan(\cylt)}
\pgfmathsetmacro{\ct}{1/sqrt(1 + (\cylt)^2)}
\pgfmathsetmacro{\st}{\cylt * \ct}
%\begin{scope}[every path/.style={ultra thick}]
\draw (0,0,0) circle[radius=1];
\draw (\ct,\st,0) -- ++(0,0,-8);
\draw (-\ct,-\st,0) -- ++(0,0,-8);
\draw (\ct,\st,-8) arc[start angle=\ang,delta angle=180,radius=1];
\draw[dashed] (\ct,\st,-8) arc[start angle=\ang,delta angle=-180,radius=1];
\draw[->] (0,0,0) -- (0,0,1);
\draw[dashed,->] (0,0,-8) -- (0,0,-9);
%\end{scope}
\end{scope}
\end{tikzpicture}
\end{center}
Sur $\Gamma_3$ :
\begin{center}
\begin{tikzpicture}[scale=0.5]
\draw (0,0) circle[radius=1];
\draw[->] (0.7071,0.7071) -- (1.4142,1.4142);
\draw (3,0) circle[radius=1];
\draw[->] (3,1) -- (3,2);
\draw (5.5,0) circle[radius=1];
\draw[->] (6.5,0) -- (7.5,0);
\end{tikzpicture}
\end{center}
Par symétrie, on peut se ramener aux deux derniers cas par une rotation. Et donc supposer que $\bm{n}=(1,0,0)$ ou $\bm{n}=(0,1,0)$
\end{frame}

\begin{frame}{Conditions aux bords}
(\ref{bc1}) équivaut donc à :
\begin{align}
g_i^x=g_i^y &= 0 & \Gamma_3\\
g_i^z &= 0 & \Gamma_1,\Gamma_2
\end{align}
et (\ref{bc2}) à :
\begin{align}
\partial_y g_i^z - \partial_z g_i^y = \partial_y g_i^z &= 0 & \Gamma_3\\
\partial_z g_i^x - \partial_x g_i^z = - \partial_x g_i^z &= 0 & \Gamma_3\\
\partial_x g_i^y - \partial_y g_i^x &= 0 & \Gamma_1,\Gamma_2
\end{align}
\end{frame}

\begin{frame}{Formulation variationnelle}
On utilise un multiplicateur de Lagrange pour forcer la condition (\ref{contdiv}).\\
On cherche donc $(\bm{g_i}, \pi_i, \lambda_i)\in [D^1(\Omega)]^3 \times \mathbb{R} \times \mathbb{R}$.
\begin{block}{On obtient la formulation faible suivante :}
\begin{align*}
\int_\Omega \partial_xg_i^x\partial_x\varphi^x + \partial_yg_i^x\partial_y\varphi^x + \partial_zg_i^x\partial_z\varphi^x&\\
+\int_\Omega \partial_xg_i^y\partial_x\varphi^y + \partial_yg_i^y\partial_y\varphi^y + \partial_zg_i^y\partial_z\varphi^y&\\
+\int_\Omega \partial_xg_i^z\partial_x\varphi^z + \partial_yg_i^z\partial_y\varphi^z + \partial_zg_i^z\partial_z\varphi^z&\\
+\int_\Omega (\partial_x g_i^x+ \partial_y g_i^y + \partial_z g_i^z)\xi&\\
+\int_\Omega (\partial_x\varphi^x + \partial_y\varphi^y + \partial_z\varphi^z)\pi_i &= \lambda^2\int_\Omega g_i^z\varphi^z
\end{align*}
\end{block}
\end{frame}

\section{Avec relèvement}
\begin{frame}{Hypothèse}
\begin{alertblock}{Avec Relèvement}
\[
\bm{v}=\bm{a}+\bm{u}=\bm{a}+\sum_i c_i\bm{g_i}
\]
On suppose que
\[
\bm{v} = \begin{pmatrix}0\\0\\v^z\end{pmatrix}\text{ et }\bm{g_i} = \begin{pmatrix}0\\0\\g_i^z\end{pmatrix}
\]

\end{alertblock}
\end{frame}

\begin{frame}{Hypothèse}
Cela simplifie le problème, et on obtient la forme variationnelle suivante :
\begin{align*}
\int_\Omega \partial_xg_i^z\partial_x\varphi^z &+ \partial_yg_i^z\partial_y\varphi^z + \partial_zg_i^z\partial_z\varphi^z\\
&+ \partial_z g_i^z\xi + \partial_z\varphi^z\pi_i = \lambda^2\int_\Omega g_i^z\varphi^z
\end{align*}
Avec $g_i^Z=0$ sur $\Gamma_1$,$\Gamma_2$ et $g_i^z=cst$ sur $\Gamma_3$\\
Alors $\bm{v}=\bm{a}+\sum_i c_i\bm{g_i}$
\end{frame}

\begin{frame}{Valeurs propres}
\begin{block}{Comparaison avec Saks}
\[
\lambda^2_{k,j,m} = \frac{\rho_{kj}^2}{R^2}+\frac{\pi^2m^2}{l^2}
\]
\end{block}
\begin{align*}
\lambda^2_{0,1,1} = \Lambda_{0} &= 2369	& \div\bm{g} &= -1.60651e-09	& ||\div\bm{g}|| &= 6.24909\\
\lambda^2_{0,1,2} = \Lambda_{1} &= 2487.46	& \div\bm{g} &= 7.32721e-07	& ||\div\bm{g}|| &= 12.4931\\
\lambda^2_{0,1,3} = \Lambda_{2} &= 2684.89	& \div\bm{g} &= 4.90275e-09	& ||\div\bm{g}|| &= 18.7299\\
\lambda^2_{0,1,4} = \Lambda_{3} &= 2961.33	& \div\bm{g} &= 1.74436e-06	& ||\div\bm{g}|| &= 24.9565\\
\lambda^2_{0,1,5} = \Lambda_{4} &= 3316.78	& \div\bm{g} &= 1.10662e-08	& ||\div\bm{g}|| &= 31.1678\\
\lambda^2_{0,1,6} = \Lambda_{5} &= 3751.28	& \div\bm{g} &= -3.31484e-06	& ||\div\bm{g}|| &= 37.3597\\
\lambda^2_{0,1,7} = \Lambda_{6} &= 4264.86	& \div\bm{g} &= 1.97088e-08	& ||\div\bm{g}|| &= 43.5285\\
\lambda^2_{0,1,8} = \Lambda_{7} &= 4857.65	& \div\bm{g} &= 5.72063e-06	& ||\div\bm{g}|| &= 49.6785\\
\lambda^2_{0,1,9} = \Lambda_{8} &= 5529.66	& \div\bm{g} &= 3.31517e-08	& ||\div\bm{g}|| &= 55.7955
\end{align*}
\end{frame}

\begin{frame}{Valeurs propres}
\begin{align*}
\lambda^2_{1,1,9} = \Lambda_{30} &= 9124.64	& \div\bm{g} &= -1.66897e-07	& ||\div\bm{g}|| &= 55.404\\
\lambda^2_{1,1,10} = \Lambda_{31} &= 9877.82	& \div\bm{g} &= -7.9746e-08	& ||\div\bm{g}|| &= 61.4882\\
\lambda^2_{1,1,10} = \Lambda_{32} &= 9878.29	& \div\bm{g} &= 5.9413e-08	& ||\div\bm{g}|| &= 61.4728\\
\lambda^2_{0,1,14} = \Lambda_{33} &= 10084.8	& \div\bm{g} &= -2.07731e-05	& ||\div\bm{g}|| &= 85.9147\\
\lambda^2_{2,1,1} = \Lambda_{34} &= 10685.8	& \div\bm{g} &= 9.04227e-08	& ||\div\bm{g}|| &= 6.19571\\
\lambda^2_{2,1,1} = \Lambda_{35} &= 10688.2	& \div\bm{g} &= 1.56108e-08	& ||\div\bm{g}|| &= 6.28396\\
\lambda^2_{1,1,11} = \Lambda_{36} &= 10711	& \div\bm{g} &= -2.3174e-07	& ||\div\bm{g}|| &= 67.4732\\
\lambda^2_{1,1,11} = \Lambda_{37} &= 10711.4	& \div\bm{g} &= -4.91277e-08	& ||\div\bm{g}|| &= 67.535\\
\lambda^2_{2,1,2} = \Lambda_{38} &= 10804.7	& \div\bm{g} &= 1.65283e-07	& ||\div\bm{g}|| &= 12.3054\\
\lambda^2_{2,1,2} = \Lambda_{39} &= 10806.9	& \div\bm{g} &= 3.99449e-08	& ||\div\bm{g}|| &= 12.3059
\end{align*}
\end{frame}

\begin{frame}{Vecteurs propres}
\begin{figure}[H]
	\makebox[\textwidth][c]{
		\subfloat[mode 0]{\includegraphics[scale=0.12]{modeZAR-0}}\ 
		\subfloat[mode 3]{\includegraphics[scale=0.12]{modeZAR-3}}
	}\\
	\makebox[\textwidth][c]{
		\subfloat[mode 9]{\includegraphics[scale=0.12]{modeZAR-9}}\ 
		\subfloat[mode 34]{\includegraphics[scale=0.12]{modeZAR-34}}
	}
	\caption{composante z des fonctions propres}
\end{figure}
\end{frame}

\begin{frame}{Coefficients}
\begin{block}{Problème de Stokes}
\[
c_k = \frac{Re\times (f,g_k)+\langle \alpha_2,\psi_k\rangle}{\lambda_k^2}
\]
avec $\bm{f}=(0,0,1)$
\end{block}
\begin{align*}
c_{0} &= -0.00162348 & c_{1} &= 2.76514e-05\\
c_{2} &= 0.000461607 & c_{3} &= 0.000621154\\
c_{4} &= 0.000261723 & c_{5} &= -0.000181696\\
c_{6} &= -6.11957e-05 & c_{7} &= -3.59883e-05\\
c_{8} &= 8.39453e-05 & c_{9} &= 2.42786e-05\\
c_{10} &= 2.29969e-07 & c_{11} &= -8.20664e-05\\
c_{12} &= 1.98117e-05 & c_{13} &= -0.000165568\\
c_{14} &= -1.08967e-05 & c_{15} &= 4.07849e-05
\end{align*}
\end{frame}

\begin{frame}{Résultats}
\begin{figure}[H]
\makebox[\textwidth][c]{
\includegraphics[scale=0.3]{vZAR}}
\caption{U avec 100 modes propres}
\end{figure}
\end{frame}

\section{Sans relèvement}
\begin{frame}{Autre hypothèse}
\begin{alertblock}{Sans relèvement}
\[\bm{v}=\sum_i d_i\bm{h_i}\]
Avec :
\[
\bm{v}=\begin{pmatrix}
0\\0\\v_z
\end{pmatrix}
\text{ et }\bm{h_i}=\begin{pmatrix}
0\\0\\h_i^z
\end{pmatrix}
\]
\begin{align}
-\laplace\bm{h_i}&=\lambda_i^2\bm{h_i} & \Omega\\
\div\bm{h_i}&=0 & \Omega\label{hdiv}\\
\rot\bm{h_i}\cdot\bm{n}&=0 & \partial\Omega \label{hbord}
\end{align}
\end{alertblock}
La condition (\ref{hbord}) équivaut à $h_i^z=0$ sur $\Gamma_3$.
\end{frame}

\begin{frame}{Autre hypothèse}
On utilise un multiplicateur de Lagrange pour ajouter la contrainte (\ref{hdiv}).\\
On cherche donc $(\bm{h_i},\pi_i,\lambda_i)\in [H^1(\Omega)]^3\cap [L^2_\sigma(\Omega)]^3 \times \mathbb{R} \times \mathbb{R}$.
\begin{block}{On obtient donc la forme variationnelle :}
\begin{align*}
\int_\Omega \partial_xh_i^z\partial_x\varphi^z &+ \partial_yh_i^z\partial_y\varphi^z + \partial_zh_i^z\partial_z\varphi^z\\
&+ \partial_z h_i^z\xi + \partial_z\varphi^z\pi_i = \lambda^2\int_\Omega h_i^z\varphi^z
\end{align*}
Avec $h_i^z=0$ sur $\Gamma_3$.
\end{block}
\begin{alertblock}{Inconvénient}
\begin{itemize}
\item Mathématiquement non prouvé
\item Fonctionne seulement pour le cylindre
\end{itemize}
\end{alertblock}
\end{frame}

\begin{frame}{Valeurs propres}
\begin{block}{Comparaison avec Saks}
\[
\lambda^2_{k,j,m} = \frac{\rho_{kj}^2}{R^2}+\frac{\pi^2m^2}{l^2}
\]
\end{block}
\begin{align*}
\bm{\lambda^2_{0,1,1}=\Lambda_0} &\bf{= 2329.47}	&\div\bm{g} &= 1.94208e-17	&||\div\bm{g}|| &= 0.0983456\\
\lambda^2_{1,1,1}=\Lambda_1 &= 5917.22	&\div\bm{g} &= 3.17129e-18	&||\div\bm{g}|| &= 0.296662\\
\lambda^2_{1,1,1}=\Lambda_2 &= 5917.43	&\div\bm{g} &= -3.4708e-17	&||\div\bm{g}|| &= 0.299039\\
\lambda^2_{2,1,1}=\Lambda_3 &= 10644.7	&\div\bm{g} &= 2.20627e-17	&||\div\bm{g}|| &= 0.683934\\
\lambda^2_{2,1,1}=\Lambda_4 &= 10647.4	&\div\bm{g} &= 2.89804e-17	&||\div\bm{g}|| &= 0.688444\\
\bm{\lambda^2_{0,2,1}=\Lambda_5} &\bf{= 12312.1}	&\div\bm{g} &= -1.04863e-16	&||\div\bm{g}|| &= 0.954444\\
\lambda^2_{3,1,1}=\Lambda_6 &= 16473	&\div\bm{g} &= -1.28681e-17	&||\div\bm{g}|| &= 1.23079\\
\lambda^2_{3,1,1}=\Lambda_7 &= 16474.4	&\div\bm{g} &= -2.9924e-17	&||\div\bm{g}|| &= 1.22237\\
\lambda^2_{1,2,1}=\Lambda_8 &= 19976.6	&\div\bm{g} &= -5.4515e-17	&||\div\bm{g}|| &= 1.8095
\end{align*}
\end{frame}

\begin{frame}{Valeurs propres}
\begin{align*}
\lambda^2_{?,?,?}=\Lambda_{35} &= 51355.4	&\div\bm{g} &= 1.14817e-16	&||\div\bm{g}|| &= 7.14748\\
\lambda^2_{?,?,?}=\Lambda_{36} &= 54891.3	&\div\bm{g} &= 1.21431e-17	&||\div\bm{g}|| &= 50.2826\\
\lambda^2_{?,?,?}=\Lambda_{37} &= 55565.9	&\div\bm{g} &= -2.43729e-16	&||\div\bm{g}|| &= 52.281\\
\lambda^2_{?,?,?}=\Lambda_{38} &= 55976.2	&\div\bm{g} &= -1.73906e-16	&||\div\bm{g}|| &= 58.8682\\
\lambda^2_{?,?,?}=\Lambda_{39} &= 56447.3	&\div\bm{g} &= 1.50813e-16	&||\div\bm{g}|| &= 59.4463\\
\lambda^2_{?,?,?}=\Lambda_{40} &= 57302.4	&\div\bm{g} &= 4.13081e-17	&||\div\bm{g}|| &= 40.4527\\
\lambda^2_{?,?,?}=\Lambda_{41} &= 57360.6	&\div\bm{g} &= -3.61581e-17	&||\div\bm{g}|| &= 36.7049\\
\lambda^2_{?,?,?}=\Lambda_{42} &= 57614.4	&\div\bm{g} &= -8.2833e-17	&||\div\bm{g}|| &= 44.0287\\
\lambda^2_{?,?,?}=\Lambda_{43} &= 57825.8	&\div\bm{g} &= 1.47451e-16	&||\div\bm{g}|| &= 58.9588\\
\lambda^2_{?,?,?}=\Lambda_{44} &= 58151.4	&\div\bm{g} &= -9.36751e-17	&||\div\bm{g}|| &= 62.1296\\
\lambda^2_{?,?,?}=\Lambda_{45} &= 59027	&\div\bm{g} &= 1.70328e-16	&||\div\bm{g}|| &= 54.9454\\
\lambda^2_{?,?,?}=\Lambda_{46} &= 59732.7	&\div\bm{g} &= -1.66371e-16	&||\div\bm{g}|| &= 37.642
\end{align*}
\end{frame}

\begin{frame}{Vecteurs propres}
\begin{figure}[H]
	\makebox[\textwidth][c]{
		\subfloat[mode 0]{\includegraphics[scale=0.12]{modeZ-0}}\ 
		\subfloat[mode 5]{\includegraphics[scale=0.12]{modeZ-5}}
	}\\
	\makebox[\textwidth][c]{
		\subfloat[mode 14]{\includegraphics[scale=0.12]{modeZ-14}}\ 
		\subfloat[mode 16]{\includegraphics[scale=0.12]{modeZ-16}}
	}
	\caption{composante z des fonctions propres}
	\label{resultats}
\end{figure}
\end{frame}

\begin{frame}{Coefficients}
\begin{block}{Problème de Stokes}
\[
c_k = \frac{Re\times (f,g_k)+\langle \alpha_2,\psi_k\rangle}{\lambda_k^2}
\]
avec $\bm{f}=(0,0,1)$
\end{block}
\begin{align*}
\bf{d_{0}} &\bf{= 0.00242664} & d_{1} &= -5.72621e-05\\
d_{2} &= -2.79554e-05 & d_{3} &= 1.17409e-06\\
d_{4} &= 2.4402e-05 & \bf{d_{5}} &\bf{= -0.000211458}\\
d_{6} &= -7.79286e-06 & d_{7} &= -6.61541e-07\\
d_{8} &= -6.61893e-07 & d_{9} &= -4.8829e-06\\
d_{10} &= 1.91622e-05 & d_{11} &= 1.27225e-07\\
d_{12} &= 3.64327e-06 & d_{13} &= -1.34974e-06\\
\bf{d_{14}} &\bf{= 6.02757e-05} & d_{15} &= -3.67763e-07
\end{align*}
\end{frame}

\begin{frame}{Résultats}
\begin{figure}[H]
\makebox[\textwidth][c]{
\includegraphics[scale=0.3]{vZSR}}
\caption{U avec 100 modes propres}
\end{figure}
\end{frame}

\section{Stokes-Curl}
\begin{frame}{Stokes-Curl}
\begin{alertblock}{A partir d'ici,}
\[\bm{v}=\bm{a}+\bm{u}\]
\end{alertblock}

De plus, on ne suppose plus que $\bm{v}=(0,0,v^z)$.
\end{frame}

\begin{frame}{Stokes-Curl}
%\begin{block}{On résout le problème suivant :}
\begin{align}
\rott \bm{g_i} &= \lambda^2\bm{g_i}\label{sc1}\\
\div \bm{g_i} &= 0\label{sc2}\\
\bm{g_i}\cdot\bm{n}&=0\\
\rot\bm{g_i}\cdot\bm{n}&=0
\end{align}
%\end{block}
On a :
\[
\rott\bm{g_i}=-\laplace\bm{g_i}+\grad(\underbrace{\div\bm{g_i}}_{=0})=-\laplace\bm{g_i}
\]
De plus, on utilise un terme de pression pour imposer (\ref{sc2}). On a donc le problème :
\begin{align}
-\laplace \bm{g_i} + \grad p &= \lambda^2_i\bm{g_i}\label{sc11}\\
\div\bm{g_i} &= 0\label{sc22}\\
\bm{g_i}\cdot\bm{n}&=0\\
\rot\bm{g_i}\cdot\bm{n}&=0
\end{align}
\end{frame}

\begin{frame}{Forme variationnelle}
On multiplie (\ref{sc11}) par $\bm{\varphi}\in H(rot)$ et on intègre :
\[
(-\laplace\bm{g_i},\bm{\varphi})+(\grad p,\bm{\varphi}) = \lambda^2_i (\bm{g_i},\bm{\varphi})
\]
On multiplie (\ref{sc22}) par $q\in H^1(\Omega)$ et on intègre :
\[
(\div\bm{g_i},q) = 0
\]
\begin{block}{En intégrant par partie et en additionnant :}
\[
(\grad\bm{g_i},\grad\bm{\varphi}) + (\grad p, \bm{\varphi}) + (\div\bm{g_i},q) = \lambda^2(\bm{g_i},\bm{\varphi})
\]
\end{block}
\end{frame}

\begin{frame}{Conditions aux bords}
En intégrant par partie $(\div\bm{g_i},q)$, on a :
\[
(\div\bm{g_i},q) = (\bm{g_i},\grad q) - \langle \underbrace{\bm{g_i}\cdot \bm{n}}_{=0}, q \rangle
\] 
Pour imposer $\rot\bm{g_i}\cdot\bm{n}=0$ on utilise une méthode de pénalisation, on ajoute donc :
\[
\gamma\int_\Omega (\rot\bm{g_i}\cdot\bm{n})(\rot\bm{\varphi}\cdot\bm{n})
\]
\begin{block}{Ce qui donne la forme variationnelle :}
\[\begin{aligned}
(\grad\bm{g_i},\grad\bm{\varphi}) &+ (\grad p, \bm{\varphi}) + (\bm{g_i}, \grad q)\\
&+ \gamma(\rot\bm{g_i}\cdot\bm{n},\rot\bm{\varphi}\cdot\bm{n}) = \lambda^2(\bm{g_i},\bm{\varphi})
\end{aligned}\]
\end{block}
\end{frame}


\end{document}
