\documentclass[a4paper,11pt]{article} % 11 ou 12pt, article ou report ou book
\usepackage[latin1]{inputenc} % caract�res accentués en UTF8
\usepackage[T1]{fontenc} % idem
\usepackage[francais]{babel} % français (chapter -> chapitre...)
\usepackage{graphicx} % graphiques
\usepackage{amsmath}
\usepackage{amsfonts,amssymb} % symboles AMS
\usepackage{verbatim}
%\usepackage{listings}
\usepackage{color}
\usepackage{geometry}
%\usepackage{dsfont}
\usepackage{fancyhdr}
%\usepackage{lastpage}
\usepackage{here}
%\usepackage{caption}
%\usepackage{subfig}
%\usepackage{animate}
\usepackage[colorlinks=true,urlcolor=blue]{hyperref}
%\usepackage{fancyvrb}
\newcommand{\N}{{\mathbb{N}}}
\newcommand{\Z}{{\mathbb{Z}}}
\newcommand{\R}{{\mathbb{R}}}
\newcommand{\C}{{\mathbb{C}}}
\geometry{hmargin=2.5cm,vmargin=2.5cm}
%\lstset{
%language=C++,
%frame=single,
%keywordstyle=\color{magenta}\bfseries,
%commentstyle=\color{green}\itshape,
%stringstyle=\color{blue},
%showstringspaces=false,
%breaklines=true,
%breakatwhitespace=true,
%tabsize=4}
%\fvset{tabsize=4}
\pagestyle{fancy}
\renewcommand\headrulewidth{1pt}
\renewcommand\footrulewidth{1pt}
\lhead{\leftmark}
\rhead{\rightmark}
\lfoot{Romain HILD\\Jérôme SPECHT}
\cfoot{\thepage/\pageref{LastPage}}
\rfoot{Projet éléments finis\\Plastic Omnium}
\setlength{\headheight}{15pt}
\graphicspath{{Images/}}

\def\restriction#1#2{\mathchoice
              {\setbox1\hbox{${\displaystyle #1}_{\scriptstyle #2}$}
              \restrictionaux{#1}{#2}}
              {\setbox1\hbox{${\textstyle #1}_{\scriptstyle #2}$}
              \restrictionaux{#1}{#2}}
              {\setbox1\hbox{${\scriptstyle #1}_{\scriptscriptstyle #2}$}
              \restrictionaux{#1}{#2}}
              {\setbox1\hbox{${\scriptscriptstyle #1}_{\scriptscriptstyle #2}$}
              \restrictionaux{#1}{#2}}}
\def\restrictionaux#1#2{{#1\,\smash{\vrule height .8\ht1 depth .85\dp1}}_{\,#2}}

\title{\includegraphics[scale=0.5]{uds.jpg}\\Résolution d'un problème aux valeurs propres de l'opérateur rotationnel.}
\author{Romain HILD\\Jérôme SPECHT}
\date{\today\\ \vspace{\fill}\includegraphics[scale=0.5]{po.jpg}\vspace{\fill}}

\begin{document}

\maketitle

\newpage
\tableofcontents

\newpage
\section{Introduction}
Le but de ce projet est la diminution de temps de calcul permettant la simulation des écoulements d'air sur la carrosserie d'une voiture, et plus précisément sur des pièces appelés "spoilers" ou "vortex generators". Ces pièces sont utilisées pour contrôler le flux d'air autour du véhicule, ce qui amène à une diminution de l'émission de $CO_2$ par la voiture. Cette diminution est très importante car dans les années à venir, il y aura beaucoup de normes à ce sujets, il est donc important pour les constructeurs automobiles de s'y préparer dès maintenant.\\
Plastic Omnium possède déjà des codes pour des simulations en trois dimensions, cependant ces codes sont utilisés sur des modèles de plusieurs millions d'éléments et donc les calculs prennent plusieurs semaines sur un nombre important de processeurs.\\
L'idée est de réécrire l'équation de Navier-Stokes en utilisant l'opérateur rotationnel, ce qui permet d'avoir une base de fonctions pour l'espace que l'on peut utiliser à chaque itération en temps et donc diminuer les coûts de calcul.

\hspace{10mm}

\textbf{Commentaire:\\
Concernant la dernière phrase, ce n'est pas le fait de réécrire les équations de NS qui permet d'avoir une base de fonctions propres pour l'espace D1.
Il faut d'abord construire un espace dans lequel on souhaite résoudre les équations de ND : D1.
D1 est intégralement généré par la base de fonctions propres de l'opérateur rotationel (curl).
Ensuite, si on cherche une solution aux équtions de NS dans D1 alors cette solution peut se décomposer comme une combinaison linéaire (somme infinie) de fonctions propres de l'opérateur curl.
L'espace D1 se démarque par un jeu de conditions aux limites non standards.
Tout ceci permet d'arriver à une formulation faible du problème où les termes de gradients de potentiel (pression,...) sont reportés en post-traitement des termes de vitesse.
On obtient également un problème numérique où l'on peut séparer les variables de temps et d'espace.
grossièrement, c'est cela qui nous laisse penser que l'on peut gagner du temps de calcul car on pourra générer une seule fois une base de fonctions et les itérations ne porteront que sur les coefficients temporels.}




\section{Problème}
Nous cherchons $(v,p)$ solutions de l'équation de Navier-Stokes incompressibles adimensionnalisées dans $Q_T=\Omega\times[0,T]$ un ouvert de $\R^3$ et $\partial\Omega$ sa frontière, avec une condition initiale et des conditions aux limites d'imperméabilité généralisée :
\[
(S_1)
\left\{
\begin{aligned}
&\frac{\partial v}{\partial t} + (\nabla\times v)\times v + \nabla q -\frac{1}{Re}\nabla^2\times v-f = 0\\
&div\ v = 0\\
&v\big\rvert_{t=0} = v_0\\
&v\cdot n\big\rvert_{\partial\Omega} = \alpha_0\\
&(\nabla\times v)\cdot n\big\rvert_{\partial\Omega} = \alpha_1\\
&(\nabla^2\times v)\cdot n\big\rvert_{\partial\Omega} = \alpha_2
\end{aligned}
\right.
\]
où $q = \frac{|v|^2}{2}+p$.\\

On cherche les solutions dans l'espace
\[
D^1(\Omega)^3 = \{v\in W^{1,2}(\Omega) | div\ v=0;v\cdot n\big\rvert_{\partial\Omega}=0; (\nabla^2\times v) \cdot n\big\rvert_{\partial\Omega}=0\}
\]

On définit une fonction $a$ dans $W^{1,2}(\Omega)$ tel que :
\[
a=\nabla\times\chi_1+\nabla\chi_0
\]
Cette fonction possède les propriétés suivantes :
\[
div\ a =0\\
a\cdot n\big\rvert_{\partial\Omega} = \alpha_0
\]
avec $\chi_0\in W^{1,2}(\Omega)$ tel que :
\[
(S_2)\left\{
\begin{aligned}
&\Delta\chi_0 = 0\\
&\nabla\chi_0\cdot n\big\rvert_{\partial\Omega}=\alpha_0
\end{aligned}
\right.
\]
et $(\chi_1,\underline{P})\in D^1(\Omega)^3\times W^{1,2}(\Omega)$ tel que :
\[
(S_3)\left\{
\begin{aligned}
&\nabla^2\times\chi_1 + \nabla\underline{P} = 0\\
&div\ \chi_1 = 0\\
&\chi_1\cdot n\big\rvert_{\partial\Omega} = 0\\
&(\nabla\times\chi_1\cdot n_{\partial\Omega} = 0\\
&(\nabla^2\times\chi_1\cdot n_{\partial\Omega} = \alpha_1
\end{aligned}
\right.
\text{ et }
(S_4)\left\{
\begin{aligned}
&\Delta\underline{P} = 0\\
&\nabla\underline{P}\cdot n_{\partial\Omega} = \alpha_1
\end{aligned}
\right.
\]

\textbf{Commentaire:\\
On utilise "a" pour relever le problème (uniquement les 2 premières conditions aux limites) }
\\

Ainsi, on peut décomposer $v=a+u$ et le problème est donc de trouver $(u,q)\in Q_T$ tel que :
\[
(S_5)\left\{
\begin{aligned}
&\frac{\partial u}{\partial t} + (\nabla\times u)\times u + (\nabla\times u)\times a +(\nabla\times a)\times u + \nabla q +\frac{1}{Re}\nabla^2\times u - h = 0\\
&div\ u = 0\\
&u\big\rvert_{t=0} = v_0 - a(0,\cdot)\\
&u\cdot n\big\rvert_{\partial\Omega} = 0\\
&(\nabla\times u)\cdot n\big\rvert_{\partial\Omega} = 0\\
&(\nabla^2\times u)\cdot n\big\rvert_{\partial\Omega} = \alpha_2
\end{aligned}
\right.
\]
où $h=f-\frac{\partial a}{\partial t} - (\nabla\times a)\times a$.\\

On peut décomposer $u$ sous la forme :
\[
u(t,\cdot) = \sum_{i=1}^{\infty} c_i(t)g_i(\cdot)
\]

On va chercher une valeur approché de $u$ qui se présente sous la forme :
\[
u_M(t,\cdot) = \sum_{i=1}^{M} c_i(t)g_i(\cdot)
\]

Les fonctions $g_i$ forment une base de l'espace $D^1(\Omega)^3 = D(curl_{imperm})$. Cette base est constitué des fonctions propres de l'opérateur rotationnel. Elles sont donc solution du problème suivant :
\[
(\lambda_i,g_i)\in\R\times D^1(\Omega)^3\quad (S_6)\left\{
\begin{aligned}
&\nabla^2\times g_i = \lambda_i g_i\\
&div\ g_i = 0\\
&g_i\cdot n\big\rvert_{\partial\Omega} = 0\\
&(\nabla\times g_i)\cdot n\big\rvert_{\partial\Omega} = 0
\end{aligned}
\right.
\]

Toutes fonctions dans $D^1(\Omega)^3$ peut s'écrire $g_i=g_i^0+\nabla\psi_i$ avec $g_i^0\big\rvert_{\partial\Omega} = 0$ et $ \nabla\psi_i\cdot n\big\rvert_{\partial\Omega} = 0$.\\
Ainsi en décomposant dans $(S_6)$, $u$ par sa somme, on obtient le problème suivant où l'on cherche les coefficients $c_i$ :
\[
(S_7)_{1\leq k\leq M}\left\{
\begin{aligned}
&\frac{dc_k}{dt}+\sum_{i=1}^M\sum_{j=1}^M c_i\lambda_i c_j r_{ijk} + \sum_{i=1}^M c_i \lambda_i r_{ik}^a + \sum_{i=1}^M c_j r_{jk}^{\nabla\times a} + \lambda_k^2 c_k = r_k^h - r_k^{\alpha_2}\\
&c_k(0)=c_k^0
\end{aligned}
\right.
\]
Où
\[
\begin{aligned}
&r_{ijk} = \langle g_i\times g_j, g_k\rangle_\Omega,\\
&r_{ik}^a=\langle g_i\times a, g_k\rangle_\Omega,\\
&r_{jk}^{\nabla\times a}=\langle (\nabla\times a)\times g_j, g_k\rangle\rangle_\Omega,\\
&r_k^h=\langle h,g_k\rangle_\Omega,\\
&r_k^{\alpha_2} = \langle \alpha_2,\psi_k\rangle_{\partial\Omega}.
\end{aligned}
\]

Pour résumer, on doit d'abord trouver $a$ pour pouvoir décomposer $v$ en $u+a$, pour cela, on résous d'abord l'équation $(S_4)$ afin de résoudre $(S_3)$ pour trouver $\chi_1$ et $(S_2)$ pour trouver $\chi_0$.\\
Ensuite, on essaye de trouver $u$, on doit donc d'abord calculer les $g_i$ une fois pour toutes au début. On résous donc $(S_6)$ et afin de connaître $g_i^0$ et $\psi_i$ on résout respectivement $\Delta g_i^0=\Delta g_i$ et $ \Delta\psi_i= -div\ g_i^0$.\\
Cela permet de calculer tous les $r_x^y$, et enfin de résoudre l'équation $(S_7)$ pour $1\leq k\leq M$ afin de trouver les $c_i$ et ainsi de connaitre complètement $u$, et donc $v$.\\


\textbf{Commentaire:\\
On génère tout d'abord la base de fonctions propres de l'opérateur curl : les "g"
puis on calcule "a"
avec ça on est capable de résoudre les coefficients "c"
on recompose u comme la somme des c*g
on recompose v = a + u la solution du problème de départ}

\hspace{10mm}

\textbf{Commentaire:\\
Je pourrai expliciter ces ommentaires en séance et/ou par téléphone au besoin}




\end{document}
