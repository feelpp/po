\section{Solution anaylique}
In order to check the validity of our results, we need a function which lies in
$\ZZ$. We turn to J.~Cantarella in \cite{Cantarella2000} to find it, in fact,
the function defined in it is a solution of our eigenproblem in a sphere.\\
The
function $V$ is defined in the spherical coordinates $(r,\theta,\phi)$, where
$r$ is the radial distance, $\theta$ the polar angle and $\phi$ the azimuth 
angle, by :
\[ V(r,\theta,\phi) = u(r,\theta)\hat{r} + v(r,\theta)\hat{\theta} + w(r,\theta)\hat{\phi} \]
with
\begin{gather*}
u(r,\theta) =
\frac{2\lambda}{r^2}\left(\frac{\lambda}{r}\sin(r/\lambda)-\cos(r/\lambda)\right)\cos\theta\\
v(r,\theta) = -\frac{1}{r}\left(\frac{\lambda}{r}\cos(r/\lambda) -
  \frac{\lambda^2}{r^2}\sin(r/\lambda) + \sin(r/\lambda)\right)\sin(\theta)\\
w(r,\theta) = \frac{1}{r}\left(\frac{\lambda}{r}\sin(r/\lambda) - \cos(r/\lambda)\right)\sin\theta
\end{gather*}
where $\lambda$ is the eigenvalue associated, and is equal to the first positive
solution of the equation $x=\tan x$ ($\approx$ 4.4934) for a sphere of radius 1.\\

Unfortunately, at the origin, the function is not define. We have to use a Taylor series around $r=0$. We have then :
\begin{gather*}
u(r,\theta) = -\frac{r^2}{12\lambda}\cos\theta +o(r^4)\\
v(r,\theta) = \left(\frac{2}{3\lambda}-\frac{2r^3}{15\lambda^3}\right)\sin\theta + o(r^4)\\
w(r,\theta) =
\left(\frac{r}{3\lambda^2}-\frac{r^3}{30\lambda^4}\right)\sin\theta + o(r^4)
\end{gather*} 

%%% Local Variables:
%%% TeX-master: "../peps.tex"
%%% eval: (flyspell-mode 1)
%%% ispell-local-dictionary: "english"
%%% End:
