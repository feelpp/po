\section{A mixed formulation}
\subsection{Stokes problem}

In \cite{Dubois2003}, Dubois and all studied the Stokes problem with a
formulation with vorticity-velocity-pressure.\\
They find that under certain hypotheses, the problem is well posed.\\

They supposed that the boundary have been partionated in two ways :
$(\Gamma_m,\Gamma_p)$ and $(\Gamma_\theta, \Gamma_t)$ such that :
\begin{gather*}
\Gamma_\theta = \Gamma_m \mbox{ and } \Gamma_t = \Gamma_p\\
\Gamma = \Gamma_m \cup \Gamma_p \mbox{ and } \Gamma_m \cap \Gamma_t = \emptyset
\end{gather*}
And that there exists no special function between $\Gamma_m$ and $\Gamma_p$.

They also define the following spaces :
\begin{align*}
W &= \{\bm{\varphi}\in H(\mathrm{curl},\Omega), \bm{\varphi}\times\mbf{n} = 0
  \mbox{ on } \Gamma_\theta\}\\
X &= \{\bm{\varphi}\in H(\mathrm{div},\Omega), \bm{\varphi}\cdot\mbf{n} = 0
\mbox{ on } \Gamma_m \}\\
Y &= L^2(\Omega)
\end{align*}

Then the problem : 
\begin{equation}\left\{
\begin{aligned}
\mbf{w}-\curl\mbf{u} &= 0 & \mbox{ in } \Omega,\\
\curl\mbf{w}+\grad p &= f & \mbox{ in } \Omega,\\
\div\mbf{u} &= 0  & \mbox{ in } \Omega,\\
\mbf{u}\cdot\mbf{n} &= 0  & \mbox{ on } \Gamma_m,\\
\mbf{w}\times\mbf{n} &= 0  & \mbox{ on } \Gamma_m,\\
p &= \Pi_0  & \mbox{ on } \Gamma_p,\\
\mbf{n}\times\mbf{u}\times\mbf{n} &= \sigma_0  & \mbox{ on } \Gamma_p
\end{aligned}\right.\end{equation}
admits the following variational formulation :
\begin{equation}\label{pbstokes}\left\{
\begin{aligned}
\mbf{w}\in W,\quad \mbf{u}\in X,\quad p\in Y,&&\\
(\mbf{w},\bm{\varphi}) - (\mbf{u},\curl\bm{\varphi}) &=
\langle\sigma_0,\bm{\varphi}\times\mbf{n}\rangle & \forall\bm{\varphi}\in W,\\
(\curl\mbf{w},\mbf{v})-(p,\div\mbf{v}) &= (\mbf{f},\mbf{v}) -
\langle\Pi_0,\mbf{v}\cdot\mbf{n}\rangle & \forall\bm{v}\in X,\\
(\div\mbf{u},q) &= 0 & \forall q\in Y
\end{aligned}\right.\end{equation}
The problem \ref{pbstokes} admits a unique solution.

\subsection{Eigen Problem}
\subsubsection{A first attempt}

If we start from the Problem \ref{pbcond}, we can use an intermediate variable
$\mbf{w}$ such that the problem becomes :
\begin{pb}\label{pbmixed}
Find $(\mbf{u},\mbf{w})$ and $\lambda$ such that :
\begin{empheq}[left=\empheqlbrace]{align}
\mbf{w} = \curl\mbf{u} & \quad \mbox{in }\Omega\\
\mbf{w} = \lambda\mbf{u} & \quad \mbox{in }\Omega\\
\mbf{w}\cdot\mbf{n} = 0 &  \quad \mbox{on }\Gamma
\end{empheq}
\end{pb}

We know that $\mbf{u}$ belongs to $D^1(\Omega)$ and so in particular to
$H(\mathrm{curl};\Omega)$. Using the De Rham diagram :
\[ H^1(\Omega) \xrightarrow{\grad} H(\mathrm{curl};\Omega) \xrightarrow{\curl}
H(\mathrm{div};\Omega) \xrightarrow{\div} L^2(\Omega) \]
we know that $\mbf{w}$ should belongs to $H(\mathrm{div};\Omega)$.\\

We shall introduce the following notations :
\begin{gather*}
\DDD^k = [\PP_k(T)]^3 \oplus\mbf{x}\PP_k(T) \subset [\PP_{k+1}(T)]^d\\
\DDD_h = \{\mbf{v}_h\in H(\mathrm{div};\Omega) \,|\,
\mbf{v}_h\restr{T}\in\DDD^k(T)\, \forall T\in \TT_h \}
\end{gather*}
Those are the Raviart-Thomas elements, which are $H(\mathrm{div};\Omega)$
conforming.

Then the discrete problem associated to Problem \ref{pbmixed} is 
\begin{pb}\label{pbmixeddiscr}
Find $(\mbf{u},\mbf{w})\in\NN_h\times\DDD_h$ and $\lambda\in\R$ such that
$\forall (\mbf{v},\mbf{\tilde{w}}) \in\NN_h\times\DDD_{h,0}$
\[ \int_\Omega \mbf{w}\cdot\mbf{v}+\mbf{w}\cdot\mbf{\tilde{w}}-\curl\mbf{u}\cdot\mbf{\tilde{w}} = \lambda\int_\Omega \mbf{u}\cdot\mbf{v} \]
\end{pb}

\begin{rk}
An issue with this formulation is that the matrix is not symmetric, and so we can't use the best algorithm to resolve the eigen problem.
\end{rk}

\subsubsection{Using the curl of the curl}

Following the previous remark, we introduce in problem \ref{pbmixed} the curl of $\mbf{w}$ and a gradient of pressure to add the constraint on the divergence :
\begin{pb}\label{pbmixedcurl}
Find $(\mbf{u},\mbf{w})$ and $\lambda$ such that :
\begin{empheq}[left=\empheqlbrace]{align}
\mbf{w} = \curl\mbf{u} & \quad \mbox{in }\Omega\\
\curl\mbf{w} + \grad p = \lambda^2\mbf{u} & \quad \mbox{in }\Omega\\
\div\mbf{u} = 0\\
\mbf{u}\cdot\mbf{n} = 0 &  \quad \mbox{on }\Gamma
\end{empheq}
\end{pb}

This can lead to two differents discrete problems, depending if we take $\mbf{u}\in H(\mathrm{curl},\Omega)$ and $\mbf{w}\in H(\mathrm{div},\Omega)$ or $\mbf{u}\in H(\mathrm{div},\Omega)$ and $\mbf{w}\in H(\mathrm{curl},\Omega)$.

In the first case, we have :
\begin{pb}\label{pbmixedcurldiscrcurl}
Find $(\mbf{u},\mbf{w},p)\in\NN_h\times\DDD_h\times\PP_{c,h}^1$ and $\lambda\in\R$ such that
$\forall (\mbf{v},\mbf{\tilde{w}},q) \in\NN_h\times\DDD_{h,0}\times\PP_{c,h}^1$
\[ \int_\Omega \mbf{w}\cdot\mbf{\tilde{w}} - \curl\mbf{u}\cdot\mbf{\tilde{w}} + \mbf{w}\cdot\curl\mbf{v} + \grad p\cdot\mbf{v} + \mbf{u}\cdot\grad q + \int_\Gamma (\mbf{v}\times\mbf{n})\cdot\mbf{w} - q\underbrace{\mbf{u}\cdot\mbf{n}}_{0}
= \lambda^2\int_\Omega \mbf{u}\cdot\mbf{v} \]
\end{pb}

And in the second case :
\begin{pb}\label{pbmixedcurldiscrdiv}
Find $(\mbf{u},\mbf{w},p)\in\DDD_h\times\NN_h\times\PP_{c,h}^2$ and $\lambda\in\R$ such that
$\forall (\mbf{v},\mbf{\tilde{w}},q) \in\DDD_h\times\NN_{h,0}\times\PP_{c,h}^2$
\[ \int_\Omega \mbf{w}\cdot\mbf{\tilde{w}} - \mbf{u}\cdot\curl\mbf{\tilde{w}} + \curl\mbf{w}\cdot\mbf{v} + p\div\mbf{v} + q\div\mbf{u} - \int_\Gamma (\mbf{\tilde{w}}\times\mbf{n})\cdot\mbf{u} + p\underbrace{\mbf{v}\cdot\mbf{n}}_{0}
= \lambda^2\int_\Omega \mbf{u}\cdot\mbf{v} \]
\end{pb}

Unfortunately, these two formulations leads to a null pivot during the LU
factorization while solving the eigen problem. The two matrices are of the form
:
\begin{align*}
\begin{gathered}
A \\
\begin{pmatrix}
0 & \mbf{u}\cdot\curl\mbf{\tilde{w}} & q\div\mbf{u} \\
\mbf{v}\cdot\curl\mbf{w} & \mbf{w}\cdot\mbf{\tilde{w}} & 0 \\
p\div\mbf{v} & 0 & 0
\end{pmatrix}
\end{gathered}
&
\begin{gathered}
B \\
\begin{pmatrix}
\mbf{u}\cdot\mbf{v} & 0 & 0 \\
0 & 0 & 0 \\
0 & 0 & 0
\end{pmatrix}
\end{gathered}
\end{align*}


%%% Local Variables:
%%% TeX-master: "../peps.tex"
%%% eval: (flyspell-mode 1)
%%% ispell-local-dictionary: "english"
%%% End:
