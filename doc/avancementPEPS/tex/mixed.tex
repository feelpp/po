\section{A mixed formulation}
If we start from the Problem \ref{pbcond}, we can use an intermediate variable
$\mbf{w}$ such that the problem becomes :
\begin{pb}\label{pbmixed}
Find $(\mbf{u},\mbf{w})$ such that :
\begin{empheq}[left=\empheqlbrace]{align}
\mbf{w} = \curl\mbf{u} & \quad \mbox{in }\Omega\\
\mbf{w} = \lambda\mbf{u} & \quad \mbox{in }\Omega\\
\mbf{w}\cdot\mbf{n} = 0 &  \quad \mbox{on }\Gamma
\end{empheq}
\end{pb}

We know that $\mbf{u}$ belongs to $D^1(\Omega)$ and also in particular to
$H(\mathrm{curl};\Omega)$. Using the De Rham diagram :
\[ H^1(\Omega) \xrightarrow{\grad} H(\mathrm{curl};\Omega) \xrightarrow{\curl}
H(\mathrm{div};\Omega) \xrightarrow{\div} L^2(\Omega) \]
we know that $\mbf{w}$ should belongs to $H(\mathrm{div};\Omega)$.\\

We shall introduce the following notations :
\begin{gather*}
\DDD^k = [\PP_k(T)]^3 \oplus\mbf{x}\PP_k(T) \subset [\PP_{k+1}(T)]^d\\
\DDD_h = \{\mbf{v}_h\in H(\mathrm{div};\Omega) \,|\,
\mbf{v}_h\restr{T}\in\DDD^k(T)\, \forall T\in \TT_h \}
\end{gather*}
Those are the Raviart-Thomas elements, which are $H(\mathrm{div};\Omega)$
conforming.

Then the discrete problem associated to Problem \ref{pbmixed} is 
\begin{pb}\label{pbmixeddiscr}
Find $(\mbf{u},\mbf{w})\in\NN_h\times\DDD_h$ and $\lambda\in\R$ such that
$\forall (\mbf{v},\mbf{q}) \in\NN_h\times\DDD_{h,0}$
\[ \int_\Omega \mbf{w}\cdot\mbf{v}+\mbf{w}\cdot{q}-\curl\mbf{u}\cdot\mbf{q} =
\lambda\int_\Omega \mbf{u}\cdot\mbf{v} \]
\end{pb}

%%% Local Variables:
%%% TeX-master: "../peps.tex"
%%% eval: (flyspell-mode 1)
%%% ispell-local-dictionary: "english"
%%% End:
