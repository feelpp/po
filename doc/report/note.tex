\documentclass{article}
\usepackage{mystyle}
\usepackage{xspace}
\usepackage{varioref}
\usepackage{showkeys}
\usepackage{amsthm,amsmath,amsfonts}
\newtheorem{theorem}{Theorem}
\newtheorem{problem}{Problem}
\newtheorem{remark}{Remark}
\newcommand{\ST}{\ensuremath{\; | \;}\xspace}
\newcommand{\Wmp}[2][2]{\ensuremath{W^{{#2},{#1}}}\xspace}
\newcommand{\Hdiv}{\ensuremath{H(\mathrm{div},\Omega)}\xspace}
\newcommand{\Hdivo}{\ensuremath{H_0(\mathrm{div},\Omega)}\xspace}
\newcommand{\Hcurl}{\ensuremath{H(\mathrm{curl},\Omega)}\xspace}
\newcommand{\Hcurlo}{\ensuremath{H_0(\mathrm{curl},\Omega)}\xspace}
\newcommand{\Pkch}{\ensuremath{P^k_{c,h}}\xspace}
\newcommand{\Hp}[1][1]{\ensuremath{H^{#1}(\Omega)}\xspace}
\newcommand{\Hpo}[1][1]{\ensuremath{H_0^{#1}(\Omega)}\xspace}

\begin{document}
\section{Notations}
Notation :
\begin{align*}
\mathrm{div} \mathbf{v} &= \div \mathbf{v} \\
rot(\mathbf{v})&=\nabla \times \mathbf{v}\\
rot(rot(\mathbf{v}))&=\rott \mathbf{v}\\
L^2_\sigma(\Omega) &= \{\mathbf{v} \in L^2(\Omega)\ |\ \div \mathbf{v} = 0\text{ et }\mathbf{v}\cdot \mathbf{n}\restr = 0 \}\\
\Hp &= \{\mathbf{v} \in L^2(\Omega)\ |\ \grad \mathbf{v}\in L^2(\Omega)\}\\
\Hpo &= \{\mathbf{v} \in \Hp\ |\ \mathbf{v}\restr = 0\}\\
D^1(\Omega) &= \{\mathbf{v} \in \Hp\cup L^2_\sigma(\Omega)\ |\ (\rot \mathbf{v}\cdot \mathbf{n})\restr = 0  \}\\
&=\{\mathbf{v} \in \Hp\cup L^2_\sigma(\Omega)\ |\ \mathbf{v}=\mathbf{v_0}+\grad\phi \text{ avec } \mathbf{v_0}\restr = 0,\ \grad\phi\cdot \mathbf{n}\restr = 0 \}
\end{align*}

we shall use the standard Sobolev spaces \Wmp[p]{m} or \Hp[m] if $p=2$
\begin{equation*}
  \Wmp[p]{m} = \{ w \in L^p(\Omega);\ \partial^\alpha w \in L^p(\Omega)\ \forall\;
  |\alpha| \leq m \}
\end{equation*}
$(,)$ denotes the $L^2$ scalar product.
We introduce also
\begin{align*}
  \Hdiv &= \{\mathbf{v} \in L^2(\Omega)\ |\ \div\mathbf{v} \in L^2(\Omega) \}  & \Hdivo & = \{ \mathbf{v} \in \Hdiv \ST \mathbf{v} \cdot \mathbf{n} = 0  \mbox{ on } \partial \Omega \} \\
  \Hcurl &= \{\mathbf{v} \in L^2(\Omega)\ |\ \rot\mathbf{v} \in L^2(\Omega) \}
  & \Hcurlo &= \{\mathbf{v} \in \Hcurl\ |\ \mathbf{v} \times \mathbf{n} =
  \mathbf{0} \mbox{ on } \partial \Omega \}
\end{align*}

\section{Problem statement}
\label{sec:problem-statement}

We consider the following problem
\begin{problem}
  \label{prob:1}
  Find $(\mathbf{u},p)$ such that
  \begin{equation}
    \label{eq:1}
    -\nu \Delta \textbf{u}  + \nabla p = \textbf{f}, \quad    \div \mathbf{u} =
    0 \mbox{ in } \Omega
  \end{equation}
  with the following non-standard boundary conditions
  \begin{equation}
    \label{eq:2}
    \mathbf{u} \cdot \mathbf{n} = 0,\quad \curl \mathbf{u} \cdot \mathbf{n} = 0,
    \quad \curl \curl \mathbf{u} \cdot \mathbf{n} = 0 \mbox{ on } \partial \Omega
  \end{equation}
\end{problem}

\begin{remark}
  \label{rem:6}
  The above boundary conditions corresponds to the classical Dirichlet condition
  : $\mathbf{u} = \mathbf{0}$ . It is possible to prescribe it indirectly by
  imposing : $\mathbf{u}\times \mathbf{n} = \mathbf{0} \mbox{ on } \partial
  \Omega$ and $(\nabla q, \mathbf{u}) = 0 \forall\; q \in \Hp$ , but
  unfortunately, the resulting finite-dimensional problem appears to be
  ill-posed. The second condition imposes weakly that the slip condition
  $\mathbf{u} \cdot \mathbf{n} = 0$, hence we have the no-slip boundary
  condition.
\end{remark}

From \cite{girault90-1}, we introduce
\begin{equation*}
  X = \{ \mathbf{v} \in \Hcurl \ST \curl \mathbf{v} \cdot \mathbf{n} = 0 \mbox{
  on } \partial \Omega \}
\end{equation*}
and we have the theorem
\begin{problem}
  \label{prob:4}
  The problem~\ref{prob:1} has the weak formulation
  Find $(\mathbf{u},p)  \in X \times \Hp$ such that
  for all $(\mathbf{v},q) \in X \times \Hp$
  \begin{equation}
    \label{eq:3}
    \nu (\curl \mathbf{u}, \curl \mathbf{v} ) + (\nabla p, \mathbf{v}) + (\nabla
    q, \mathbf{u} ) = (\mathbf{f}, \mathbf{v})
  \end{equation}
\end{problem}

\begin{remark}
  \label{rem:1}
  Following~\cite{girault90-1}, since $\curl\mathbf{v} \in \Hdiv$ the boundary condition $\curl \mathbf{u}
  \cdot \mathbf{n}=0 \mbox{ on } \partial \Omega $ is properly
  defined. we have that all function $\mathbf{v}$ in $X$ can be
  decomposed in the following form
  \begin{equation}
    \label{eq:4}
    \mathbf{v} = \mathbf{w} + \nabla q
  \end{equation}
  with $q \in \Hp$ and $\mathbf{w} \in [\Hp]^3,\ \div \mathbf{w}
  = 0 $ and $\mathbf{w} \times \mathbf{n} = \mathbf{0}$ on the
  boundary. Conversely, all functions of the form \eqref{eq:4} belong in $X$.
\end{remark}

\begin{remark}
  \label{rem:2}
  We have that
  \begin{equation}
    \label{eq:5}
    (\curl\mathbf{u}, \curl\mathbf{v}) = (\curl \curl \mathbf{u}, \mathbf{v} ),
    \quad \forall\; \mathbf{v} \in X
  \end{equation}
In view of \eqref{eq:4}, we have
\begin{equation}
  \label{eq:6}
  (\curl \curl \mathbf{u}, \nabla q ) =  0\quad \forall\; q \in \Hp
\end{equation}
which is a weak way to prescribe $\curl \curl \mathbf{u} \cdot \mathbf{n} = 0$.
\end{remark}

The variational formulation~\eqref{eq:3} decouples the velocity and pressure: if
we introduce
\begin{equation}
  \label{eq:7}
  U_0 = \{ \mathbf{v} \in X \ST (\nabla q, \mathbf{v}) = 0\quad \forall\; q \in \Hp\}
\end{equation}
we can split~\eqref{eq:3} into
\begin{problem}
  \label{prob:2}
  Find $\mathbf{u} \in U_0 $ such that
  \begin{equation}
    \label{eq:8}
    \nu (\curl \mathbf{u}, \curl \mathbf{v} )  = (\mathbf{f}, \mathbf{v}) \quad
    \forall\; \mathbf{v} \in U_0
  \end{equation}
  Find $p \in \Hp/\mathbb{R}$ such that
  \begin{equation}
    \label{eq:9}
    (\nabla p, \nabla q) = (\mathbf{f}, \nabla q)\quad \forall\; q \in \Hp
  \end{equation}
\end{problem}
\begin{remark}
  \label{rem:3}
  from equation \eqref{eq:9}, $p$ satisfies the Neumann boundary condition
  $\partial_n p = \mathbf{f} \cdot \mathbf{n}$ on the boundary.
\end{remark}
\begin{remark}
  \label{rem:4}
  It is important to stress that, although all the above formulations imply that
  $\mathbf{u}$ belongs to $[\Hp]^3$, this regularity is not explicit~: all they
  require is that $\curl\mathbf{u}$ belong to $L^2$. This leads naturally to
  the idea of using curl-conforming finite elements.
\end{remark}

\section{Discretisation}
\label{sec:discretisation}

We now turn to the discretisation of the problem~\ref{prob:4}. First, following
remark~\ref{rem:4}, we introduce curl-conforming finite element $\mathbb{N}_k$
of degree $k$ to define a discrete subspace of $\Hcurl$.
\begin{equation}
  \label{eq:10}
  N^k_h = \{ \mathbf{v}_h \in \Hcurl \ST \mathbf{v}_h{}_{|_K} \in \mathbb{N}_k \forall\; K
  \in \mathcal{T}_h\}
\end{equation}
then $X^k_h = X \cap N^k_h$ a discrete subspace of $X$
\begin{equation}
  \label{eq:11}
  X^k_h = \{ \mathbf{v}_h \in N^k_h \ST \curl \mathbf{v}_h \cdot \mathbf{n} = 0  \mbox{ on } \Omega \}
\end{equation}
We will require also a discrete space $D^k_h$, a subspace of $\Hdiv$
\begin{equation}
  \label{eq:14}
  D^k_h = \{ \mathbf{v}_h \in \Hcurl \ST \mathbf{v}_h{}_{|_K} \in \mathbb{D}_k \forall\; K
  \in \mathcal{T}_h\}
\end{equation}
where $\mathbb{D}_k$ is a finite element polynomial space which is
div-conforming (e.g. Raviart-Thomas or Brezzi-Douglas-Marini).
And finally $\Pkch$ a discrete subspace of $\Hp$
\begin{equation}
  \label{eq:12}
  \Pkch = \{ q_h \in C^0(\Omega) \ST q_h{}_{|_K} \in \mathbb{P}_k \forall\; K \in \mathcal{T}_h\}
\end{equation}
We have the following relations
\begin{equation}
  \label{eq:13}
  \{\mathbf{v}_h \in N^k_h; \curl \mathbf{v}_h = \mathbf{0} \} = \{\mathbf{v}_h
  \in X^k_h; \curl \mathbf{v}_h = \mathbf{0} \} = \{\nabla q_h; q_h \in \Pkch\}
\end{equation}
and
\begin{equation}
  \label{eq:15}
  \{\mathbf{v}_h \in D^k_h; \div \mathbf{v}_h = 0 \} = \{\curl \mathbf{v}_h;
  \mathbf{v}_h \in N^k_h \}
\end{equation}
\begin{remark}
  \label{rem:5}
  The relation~\eqref{eq:15} implies that all function $\mathbf{v}_h \in X^k_h$
  are of the form
  \begin{equation}
    \label{eq:16}
    \mathbf{v}_h = \mathbf{w}_h + \nabla q_h\quad \mbox{ with } \mathbf{w}_h \in
    N^k_h \mbox{ and } q_h \in \Pkch
  \end{equation}
  which means that in order to construct a basis for $X^k_h$ we take all the
  basis functions of $N^k_h$ and retain only those of $\Pkch$ that do not vanish
  on the boundary of $\Omega$.
\end{remark}

% We can now introduce $U^k_{0h}$ a discrete subspace of $U_0$
% \begin{equation}
%   \label{eq:17}
%   U_{0h} = \{ \mathbf{v}_h \in X^k_h \ST (\nabla q_h, \mathbf{v}_h) = 0\quad \forall\; q_h \in \Pkch\}
% \end{equation}

We now have the following discrete problem to solve
\begin{problem}
  \label{prob:3}
  Find $(\mathbf{u}_h,p_h) \in X^k_{h} \times \Pkch/\mathbb{R}$ such that
  \begin{equation}
    \label{eq:18}
    \nu (\curl \mathbf{u}_h, \curl \mathbf{v}_h ) + (\nabla p_h, \mathbf{v}_h) + (\nabla
    q_h, \mathbf{u}_h ) = (\mathbf{f}, \mathbf{v}_h)\quad \forall\;
    (\mathbf{v_h},q_h) \in X^k_{h} \times \Pkch/\mathbb{R}
  \end{equation}
\end{problem}


\section{Numerical implementation}

One of the questions is how to handle the boundary condition $\curl \mathbf{u}
\cdot \mathbf{n} = 0$ ? as mentioned before this is well defined since $\curl
\mathbf{u} \in \Hp$.
Some ideas:
\begin{itemize}
\item a penalisation method to impose weakly the boundary condition;
\item introduce the vorticity $\mathbf{w} = \curl \mathbf{u}$ as a new unknown
  and impose $\mathbf{w}\cdot \mathbf{n} = 0$. $\mathbf{w}$ would be in $\Hdiv$
  and this would require to solve a really big system involving
  vorticity,velocity and pressure;
\item look into imposing the boundary condition strongly.
\end{itemize}


\bibliographystyle{plain}
\bibliography{ref}
\end{document}
