\chapter*{Conclusion}
Comme on a pu le voir, le sujet est très intéressant mais aussi très complexe, avec plusieurs problèmes à résoudre avant d'arriver à la solution. Le relèvement, la résolution du problème aux valeurs propres et le problème spectral sont pourtant essentiels pour retrouver la bonne solution.\\

La première partie du relèvement, ainsi que la résolution d'un problème similaire avec une seule composante, a permis de retrouver les mêmes résultats que Benjamin Surowiec avec le logiciel FreeFem++. De plus, les temps de calculs pour les problèmes aux valeurs propres ont montré un avantage pour l'utilisation de Feel++ par rapport à FreeFem++.\\

Cependant, il y a encore beaucoup à faire pour que le problème soit résolu pour une géométrie quelconque et de manière rigoureuse. La première chose à faire serait d'utiliser des éléments conformes à l'espace choisi. Ainsi il faudrait utiliser au moins des éléments de Nedelec pour se situer dans $H(rot)$ et être en accord avec la théorie de V. Girault. Une meilleure option encore serait d'utiliser des éléments conformes à $D^1$. Cela demanderait beaucoup de travail mais la résolution du problème serait grandement simplifié.\\

Une autre possibilité serait de relever $\alpha_2$ afin de n'avoir que des conditions homogènes pour $\bm{u}$. Ainsi, $\bm{a}$ porterait toutes les conditions limites, $\alpha_0,\alpha_1$ et $\alpha_2$. On aurait ainsi plus besoin de décomposer les vecteurs propres, évitant de nombreux systèmes à résoudre.\\

Enfin, comme on peut changer de solveur et de pré-conditionneur à l'exécution, il faut tester les différentes possibilités pour voir lequel fonctionne le mieux, notamment préféré l'utilisation de solveurs itératifs. Ainsi, nous ne serons pas bloqué lorsqu'il s'agira d'augmenter le nombre de mailles dans des géométries différentes.

%%% Local Variables:
%%% TeX-master: "../report.tex"
%%% eval: (flyspell-mode 1)
%%% ispell-local-dictionary: "french"
%%% End:
