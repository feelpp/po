\section{Compte-rendu des réunions}
\subsection{Lundi 3 mars}

Lors de la réunion, on a pu relever plusieurs problèmes dans le rapport, quelques fautes mathématiques et des approximations de ma part.\\

La discussion a surtout porté sur le relèvement du problème par la fonction $a=curl\ b+\phi_0$, et notamment sur le problème (\ref{curlb}).\\
La question a été de savoir si on pouvait résoudre directement le problème (\ref{curlb}), étant donné que c'est un problème de Stokes et qu'il existe déjà dans Feel++ une implémentation d'un problème ressemblant (doc/manual/stokes/stokesCurl.cpp). Ou alors de rester sur la formulation en deux partie et d'utiliser les fonctions propres précédemment calculées.\\
Résoudre le problème mixte demanderait d'adapter l'exemple en 3D, et de vérifier si l'on peut l'utiliser dans le cas de $D^1$, et non $H(curl)$.\\

Il a ensuite été question des formats de données lisibles par HyperView. Il faudrait trouver un format permettant la lecture de données paralèlles. Il y a peut-être un moyen de lire plusieurs fichiers .case à la suite, à l'aide de l'option use-sos, à voir si la version de HyperView le permet.\\

Enfin, on a parlé du matèriel disponible à PLASTIC OMNIUM pour lancer les calculs, ma machine virtuelle avec bientôt 64 Go de RAM et le cluster où l'on pourrait virtualiser une debian sur un noeud, soit 12 procs. On a aussi parlé de la possibilité que j'aille à Strasbourg afin de pouvoir utiliser les ressources de calcul de l'université, ou de créer une liasion entre PLASTIC OMNIUM et Strasbourg, afin que je me connecte en ssh.\\

Je dois donc remettre à jour le rapport, en corrigeant les fautes, en ajoutant des détails, et surtout d'écrire l'algorithme du problème, l'enchainement des différents problèmes à résoudre.\\
Il est important que je sois au point sur la formulation du problème avant que Benjamin soit absent.
