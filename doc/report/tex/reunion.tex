\chapter{Compte-rendu des réunions}
\section{Lundi 3 mars}

Lors de la réunion, on a pu relever plusieurs problèmes dans le rapport, quelques fautes mathématiques et des approximations de ma part.\\

La discussion a surtout porté sur le relèvement du problème par la fonction $a=\rot\mathbf{b}+\grad\psi^0$, et notamment sur le problème (\ref{curlb}).\\
La question a été de savoir si on pouvait résoudre directement le problème (\ref{curlb}), étant donné que c'est un problème de Stokes et qu'il existe déjà dans Feel++ une implémentation d'un problème ressemblant (doc/manual/stokes/stokesCurl.cpp). Ou alors de rester sur la formulation en deux partie et d'utiliser les fonctions propres précédemment calculées.\\
Résoudre le problème mixte demanderait d'adapter l'exemple en 3D, et de vérifier si l'on peut l'utiliser dans le cas de $D^1$, et non $H(rot)$.\\

Il a ensuite été question des formats de données lisibles par HyperView. Il faudrait trouver un format permettant la lecture de données paralèlles. Il y a peut-être un moyen de lire plusieurs fichiers .case à la suite, à l'aide de l'option use-sos, à voir si la version de HyperView le permet.\\

Enfin, on a parlé du matèriel disponible à PLASTIC OMNIUM pour lancer les calculs, ma machine virtuelle avec bientôt 64 Go de RAM et le cluster où l'on pourrait virtualiser une debian sur un noeud, soit 12 procs. On a aussi parlé de la possibilité que j'aille à Strasbourg afin de pouvoir utiliser les ressources de calcul de l'université, ou de créer une liasion entre PLASTIC OMNIUM et Strasbourg, afin que je me connecte en ssh.\\

Je dois donc remettre à jour le rapport, en corrigeant les fautes, en ajoutant des détails, et surtout d'écrire l'algorithme du problème, l'enchainement des différents problèmes à résoudre.\\
Il est important que je sois au point sur la formulation du problème avant que Benjamin soit absent.

\section{Lundi 17 mars}

Lors de la réunion, on a revu la décomposition du problème principal en plusieurs problèmes plus simples, notamment les problèmes concernant le relèvement (\ref{psi0}-\ref{curlb}).\\
La question était de savoir dans quel espace on voulait résoudre les problèmes. Par exemple, pour $\psi^0$, il faudrait être dans $H(div)$ pour assurer la régularité. On utiliserait ainsi un problème mixte, avec des multiplicateurs de Lagrange pour la constante, pour trouver $\psi^0$. Il peut être aussi intéressant de comparer les résultats avec ceux obtenues par une projection sur $L^2$.\\

Pour $\mathbf{b}$, il y a plusieurs approches possibles. On peut simplement calculer $\mathbf{b}$ avec un problème mixte dans $H^1$, et ensuite projeter $\rot\mathbf{a}$ sur l’espace choisit pour discrétiser le problème spectral. Ou bien essayer de changer la formulation faible du problème pour que le $\rot\mathbf{a}$ porte plutôt sur les $\mathbf{g}$, ainsi on aurait pas besoin de la  régularité sur $\rot\mathbf{a}$.\\
Une autre possibilité est de ne pas utiliser le problème mixte pour trouver $\mathbf{b}$. On calcul $\psi^1$ de la même manière que $\psi^0$, et ensuite on calcul $\mathbf{b}$ dans $H(rot)$.\\

Ainsi, il est très important que je précise les espaces dans lesquelles se situent les problèmes. De même, le rapport serait plus lisible si je différenciais les champs vectoriels des champs scalaires.\\
Il serait aussi intéressant d’étudier la complexité du problème spectral, en fonction du nombre de fonctions propres calculées, nombre qu’il  faudrait d’ailleurs étudier pour connaitre une limite suffisante et nécessaire.\\

On a aussi mentionné l’intégration de Travis CI au projet, l’efficacité de regrouper les résultats dans un seul fichier ensight, et l’utilité d’utiliser des fonctions propres paramétriques pour pouvoir changer un angle par exemple sur une géométrie et tout de même réutilisé des résultats précédents.\\

Le travail des semaines prochaines portera donc essentiellement sur la définition des problèmes de relèvement, leurs espaces, si on utilise ou non des problèmes mixtes, ainsi que toujours l’amélioration du rapport.

\section{Mardi 01/04}

Lors de la réunion, nous avons pu regarder la formulation du problème de (\ref{pbpsidiv}) dans $H(div)$ et déterminer qu’il faut ajouter une contrainte sur $\psi^0$, par exemple sa valeur moyenne égale à 0, afin de fermer le problème. Comme les éléments de Raviart-Thomas en 3D seront bientôt prêts , je pourrais tester mon implémentation de ce problème.\\
L’autre point sur lequel à porter la discussion est le problème (\ref{gi0}) permettant de trouver $g_i^0$ dans la décomposition des fonctions propres. Ayant des problèmes avec la formulation, nous en sommes venus à nous demander si le problème tel que posé est bien coercif. Cela pourrait expliquer mes difficultés.\\
Toujours dans la décomposition des fonctions propres, le terme $\psi_i$ est à une constante près, ce qui correspond à une translation de la pression, il faudra donc faire attention lors du post-traitement.\\

Mon travail sera donc tout d’abord de vérifier la coercivité du problème (\ref{gi0}), et essayé de trouver une solution à ce problème. Je devrais aussi corriger la formulation du problème de Darcy qui permet de calculer $\grad\psi^0$ dans $H(div)$, c’est-à-dire ajouter une contrainte sur la valeur moyenne.\\
Comme toujours, il faudra que je corrige mon rapport, revois certaines notations, et ajoute des précisions où cela est nécessaire.

\section{Mardi 29/04}

Lors de la réunion, nous avons discuté du problème aux valeurs propres, et des différentes manières de le résoudre. Il a été convenu que je décrive la méthode implémentée par Benjamin en Freefem++, puis l'adapte pour Feel++ et enfin compare les différents résultats afin de valider les autres étapes.\\
Dans le même temps, Christophe Prud'homme va utiliser les travaux de V. Girault \cite{girault90-1} pour mettre au point une autre méthode utilisant des espaces approchant de manière plus précises l'espace $D^1$.\\
Le but sera au final de comparer ces deux méthodes, pour cela il faudra projeter les modes issues d'une des méthodes sur l'espace utilisé par l'autre.\\

Ainsi, mon travail va être de comprendre et de décrire les méthodes utilisées par Benjamin, la résolution du laplacien en 3D avec pénalisation ou tri, ou la résolution, dans chaque dimension, du laplacien puis le tri des solutions.\\
Puis en implémenter au moins une avec Feel++ et donc pouvoir comparer les résultats avec FreeFem++. Pour le moment, je vais donc commencer par la méthode sans pénalisation, et trier les résultats en fonction de la condition aux bords.\\
Il a aussi été question du nombre de modes propres nécessaires et du temps de calculs séquentiels ainsi généré, de la préparation d'un projet PEPS 2 et de ma participation aux web-conférences regroupant les utilisateurs de Feel++.

\section{Mercredi 14/05}

Lors de la réunion de ce matin, nous avons discuté des avancés effectués pour le projet.\\
D’une part,  l’implémentation des espaces $H(div)$ et $H(rot)$ dans Feel++, ainsi que la possibilité d’exporter les résultats dans un seul fichier pour tous les processeurs.\\
D’autre part, j’ai expliqué la méthode consistant à n’utiliser qu’une seule composante pour les modes propres. Cette méthode a encore des problèmes, notamment concernant la divergence, que rien ne force à être nulle.\\

Ainsi, il faudrait que je projette la divergence des modes propres sur $P_0$ discontinue afin de voir sa forme.\\
Afin de la forcer à être nulle, je pourrai essayer d’ajouter cette contrainte de manière faible, en posant $\div\bm{g} = \partial_z \bm{g_z} = 0$.\\
Ou je pourrai utiliser un multiplicateur de Lagrange afin d’ajouter cette contrainte, ce qui reviendrait à un problème de type Stokes avec des champs scalaires.\\
De plus, en utilisant la publication de R. Saks \cite{Saks2005}, je pourrai comparer les vecteurs propres trouvés numériquement, avec leur valeur analytique, en utilisant une bibliothèque de Boost. Je peux aussi comparer avec les résultats obtenus par Benjamin en utilisant FreeFem.\\
Il faut aussi que j’ajoute dans la documentation les explications à ces différentes méthodes.\\
Aussi, il est prévu d’avoir une réunion avec les gens du LNCMI, afin de faire un point sur le projet.

\section{Mercredi 18/06}

Lors de la réunion de cet après-midi, nous avons parlé des différents problèmes survenus lors de la décomposition des modes propres en $\bm{g}=\bm{g_0}+\grad\psi$.\\
Le plus gros problème soulevé est l'adimensionnement des équations mais pas de la géomètrie, ce qui peut conduire à des erreurs lors de la résolution des coefficients $c_i$.\\
Un autre problème est la moyenne nulle des composantes des modes propres, ce qui implique l'impossibilité d'appliquer des forces volumiques sur $\bm{u}$. Cela peut provenir de la symétrie du cylindre, et ainsi des modes propres. Il faudrait donc vérifier si il existe des modes de moyennes non nulles. Une solution est aussi de tester le problème sur une autre géomètrie, telle que la marche.\\
De la même manière, il faut tester $\psi$ avec différentes moyennes, afin de voir le comportements du coefficient $R_{pk}$.\\

Afin de calculer l'erreur sur la condition au bord $\rott\bm{g}\cdot\bm{n}$ et sur les modes propres $\rott\bm{g}-\lambda^2\bm{g}$, il faut pouvoir calculer $\rott\bm{g}$. Si on peut calculer $\rot\bm{g}$ avec Paraview, il faut utiliser une projection sur les fonctions continues et monter en ordre afin de calculer $\rott\bm{g}$.\\
Afin de limiter l'utilisation de la mémoire, il faudrait utiliser une décomposition du domaine afin de limiter la taille des problèmes.\\

Un autre ajout important est le contrôle des différents paramètres du problème, la taille du maillage, le nombre de degré de libertè,... Cela permettra de pouvoir comparer les résultats entre eux précisément.





%%% Local Variables:
%%% TeX-master: "../report.tex"
%%% End:
