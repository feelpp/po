\chapter{Compte-rendu des réunions}
\section{Lundi 3 mars}

Lors de la réunion, on a pu relever plusieurs problèmes dans le rapport, quelques fautes mathématiques et des approximations de ma part.\\

La discussion a surtout porté sur le relèvement du problème par la fonction $a=\rot\mathbf{b}+\grad\psi^0$, et notamment sur le problème (\ref{curlb}).\\
La question a été de savoir si on pouvait résoudre directement le problème (\ref{curlb}), étant donné que c'est un problème de Stokes et qu'il existe déjà dans Feel++ une implémentation d'un problème ressemblant (doc/manual/stokes/stokesCurl.cpp). Ou alors de rester sur la formulation en deux partie et d'utiliser les fonctions propres précédemment calculées.\\
Résoudre le problème mixte demanderait d'adapter l'exemple en 3D, et de vérifier si l'on peut l'utiliser dans le cas de $D^1$, et non $H(rot)$.\\

Il a ensuite été question des formats de données lisibles par HyperView. Il faudrait trouver un format permettant la lecture de données paralèlles. Il y a peut-être un moyen de lire plusieurs fichiers .case à la suite, à l'aide de l'option use-sos, à voir si la version de HyperView le permet.\\

Enfin, on a parlé du matèriel disponible à PLASTIC OMNIUM pour lancer les calculs, ma machine virtuelle avec bientôt 64 Go de RAM et le cluster où l'on pourrait virtualiser une debian sur un noeud, soit 12 procs. On a aussi parlé de la possibilité que j'aille à Strasbourg afin de pouvoir utiliser les ressources de calcul de l'université, ou de créer une liasion entre PLASTIC OMNIUM et Strasbourg, afin que je me connecte en ssh.\\

Je dois donc remettre à jour le rapport, en corrigeant les fautes, en ajoutant des détails, et surtout d'écrire l'algorithme du problème, l'enchainement des différents problèmes à résoudre.\\
Il est important que je sois au point sur la formulation du problème avant que Benjamin soit absent.

\section{Lundi 17 mars}

Lors de la réunion, on a revu la décomposition du problème principal en plusieurs problèmes plus simples, notamment les problèmes concernant le relèvement (\ref{psi0}-\ref{curlb}).\\
La question était de savoir dans quel espace on voulait résoudre les problèmes. Par exemple, pour $\psi^0$, il faudrait être dans $H(div)$ pour assurer la régularité. On utiliserait ainsi un problème mixte, avec des multiplicateurs de Lagrange pour la constante, pour trouver $\psi^0$. Il peut être aussi intéressant de comparer les résultats avec ceux obtenues par une projection sur $L^2$.\\

Pour $\mathbf{b}$, il y a plusieurs approches possibles. On peut simplement calculer $\mathbf{b}$ avec un problème mixte dans $H^1$, et ensuite projeter $\rot\mathbf{a}$ sur l’espace choisit pour discrétiser le problème spectral. Ou bien essayer de changer la formulation faible du problème pour que le $\rot\mathbf{a}$ porte plutôt sur les $\mathbf{g}$, ainsi on aurait pas besoin de la  régularité sur $\rot\mathbf{a}$.\\
Une autre possibilité est de ne pas utiliser le problème mixte pour trouver $\mathbf{b}$. On calcul $\psi^1$ de la même manière que $\psi^0$, et ensuite on calcul $\mathbf{b}$ dans $H(rot)$.\\

Ainsi, il est très important que je précise les espaces dans lesquelles se situent les problèmes. De même, le rapport serait plus lisible si je différenciais les champs vectoriels des champs scalaires.\\
Il serait aussi intéressant d’étudier la complexité du problème spectral, en fonction du nombre de fonctions propres calculées, nombre qu’il  faudrait d’ailleurs étudier pour connaitre une limite suffisante et nécessaire.\\

On a aussi mentionné l’intégration de Travis CI au projet, l’efficacité de regrouper les résultats dans un seul fichier ensight, et l’utilité d’utiliser des fonctions propres paramétriques pour pouvoir changer un angle par exemple sur une géométrie et tout de même réutilisé des résultats précédents.\\

Le travail des semaines prochaines portera donc essentiellement sur la définition des problèmes de relèvement, leurs espaces, si on utilise ou non des problèmes mixtes, ainsi que toujours l’amélioration du rapport.

\section{Mardi 01/04}

Lors de la réunion, nous avons pu regarder la formulation du problème de (\ref{pbpsidiv}) dans $H(div)$ et déterminer qu’il faut ajouter une contrainte sur $\psi^0$, par exemple sa valeur moyenne égale à 0, afin de fermer le problème. Comme les éléments de Raviart-Thomas en 3D seront bientôt prêts , je pourrais tester mon implémentation de ce problème.\\
L’autre point sur lequel à porter la discussion est le problème (\ref{gi0}) permettant de trouver $g_i^0$ dans la décomposition des fonctions propres. Ayant des problèmes avec la formulation, nous en sommes venus à nous demander si le problème tel que posé est bien coercif. Cela pourrait expliquer mes difficultés.\\
Toujours dans la décomposition des fonctions propres, le terme $\psi_i$ est à une constante près, ce qui correspond à une translation de la pression, il faudra donc faire attention lors du post-traitement.\\

Mon travail sera donc tout d’abord de vérifier la coercivité du problème (\ref{gi0}), et essayé de trouver une solution à ce problème. Je devrais aussi corriger la formulation du problème de Darcy qui permet de calculer $\grad\psi^0$ dans $H(div)$, c’est-à-dire ajouter une contrainte sur la valeur moyenne.\\
Comme toujours, il faudra que je corrige mon rapport, revois certaines notations, et ajoute des précisions où cela est nécessaire.
