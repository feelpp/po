\chapter{Pression}
On a à présent à notre disposition $\bm{a}$ et $\bm{u}$. On peut donc retrouver \[ \bm{v}=\bm{a}+\bm{u} \]

En se servant de \ref{fvq}, on peut retrouver la pression en tant que post-traitement.
\begin{align*}
\int_\Omega \grad q\grad\varphi + \int_\Omega \lambda\varphi + \int_\Omega q\nu &= \int_\Omega (\div((\rot \mathbf{v})\times \mathbf{v}) -\div \mathbf{f})\varphi\\
&+ \int_{\partial\Omega} \left(f\cdot \mathbf{n} - \frac{\partial\alpha_0}{\partial t} - ((\rot \mathbf{v})\times \mathbf{v})\cdot \mathbf{n} - \frac{\alpha_2}{Re}\right)\varphi
\end{align*}

\section{Implémentation}
\section{Résultats}
\todo[inline]{comparaison analytique, FreeFem++}

%%% Local Variables:
%%% TeX-master: "../report.tex"
%%% eval: (flyspell-mode 1)
%%% ispell-local-dictionary: "french"
%%% End:
