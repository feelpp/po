\chapter{Relèvement}
On implémente ici le relèvement introduit dans le chapitre \ref{relev}, $\bm{a}=\grad\psi^0+\rot\bm{b}+\bm{e}$. Comme pour le moment le système servant à relever la condition $\rott\bm{a}\cdot\bm{n}=\alpha_2$ n'est pas prêt, on suppose pour le moment que $\alpha_2=0$ et donc $\bm{e}=0$.\\
De plus, dans le cas du cylindre, on a $\alpha_1=0$, ce qui conduit à ce que $\rot \mathbf{b}=0$ et donc comme il ne reste plus que la condition $\bm{a}\cdot\bm{n}=\alpha_0$ à relever, on a \[ \mathbf{a}=\grad\psi^0 \]

\section{Gradient dans $\HH^1$}
\label{impGradh1}
\subsection{Implémentation}
Si l'on veut utiliser le problème (\ref{pbpsi0}) dans $\HH^1$, alors en ajouter une contrainte sur $\psi^0$, par exemple $\int \psi^0 = 0$ on doit utiliser des multiplicateurs de Lagrange. On utilise donc la formulation variationnelle suivante où l'on cherche $(\psi^0,\lambda)\in \HH^1\times\R$ et où $(\varphi,\nu)$ est la fonction de test :
\[ \int_\Omega\grad \psi^0\grad\varphi + \int_\Omega \psi^0\nu + \int_\Omega \lambda\varphi = \int_{\partial\Omega} \alpha_0\varphi \]
On va donc créé un espace de fonction produit correspondant à $\HH^1\times\R$.

\lstinputlisting[linerange={space}]{../../src/psi0.hpp}

On ajoute une fonction permettant de rajouter en option le profil d'entrée en fonction du rayon et de la vitesse. Cela correspond à $\alpha_0$.

\lstinputlisting[linerange={option}]{../../src/psi0.cpp}

Une fois les éléments de l'espace créé, on peut définir la forme bilinéaire de la façon suivante :

\lstinputlisting[linerange={bilinear}]{../../src/psi0.cpp}

Ici, $u$ correspond à $\psi^0$ et $v$ à $\varphi$ et \texttt{inner} est le produit scalaire.\\

On veut que ce qui rentre du cylindre par l'entrée, correspondant à la partie du maillage marquée 1, sorte par l'autre bout du cylindre, marqué 2, et que le tour du cylindre, marqué 3, soit imperméable. Ce qui donne la terme de droite suivant :

\lstinputlisting[linerange={rhs}]{../../src/psi0.cpp}

Une fois le problème résolut, on veut projeter le gradient de $\psi^0$ sur $\LL^2$. Pour cela on résout le problème simple $u=\grad\psi^0$ qui mène à la forme variationnelle suivante :
\[ \int_\Omega \bm{u}\cdot\bm{v} = \int_\Omega \grad\psi^0\cdot\bm{v} \]

\lstinputlisting[linerange={gradpsi0}]{../../src/psi0.cpp}

\subsection{Résultats}
Dans les figure \ref{az},\ref{aIn},\ref{aOut}, on peut observer $\bm{a}$ dans le cylindre. Ici, $\alpha_1=0$ et 
\[ \alpha_0(x,y)= \begin{cases} -2\times v\times\left(1-\frac{x^2+y^2}{r^2}\right) &\mbox{sur } \Gamma_1\\
2\times v\times\left(1-\frac{x^2+y^2}{r^2}\right)&\mbox{sur } \Gamma_2\\
0 &\mbox{sur } \Gamma_3 \end{cases} \]

\begin{figure}[H]
\centering
\includegraphics[scale=0.3]{az}
\caption{composante $z$ de $\bm{a}$}
\label{az}
\end{figure}
\begin{figure}[H]
\centering
\includegraphics[scale=0.5]{aIn}
\caption{entrée du cylindre}
\label{aIn}
\end{figure}
\begin{figure}[H]
\centering
\includegraphics[scale=0.5]{aOut}
\caption{sortie du cylindre}
\label{aOut}
\end{figure}

%\section{Gradient dans $\HH(div)$}

%%% Local Variables:
%%% TeX-master: "../report.tex"
%%% eval: (flyspell-mode 1)
%%% ispell-local-dictionary: "french"
%%% End:
