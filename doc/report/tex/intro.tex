\section{Introduction}

Le but de ce projet est la diminution de temps de calcul permettant la simulation des \'ecoulements d'air sur la carrosserie d'une voiture, et plus pr\'ecis\'ement sur des pi\`eces appel\'es "spoilers" ou "vortex generators". Ces pi\`eces sont utilis\'ees pour contr\^oler le flux d'air autour du v\'ehicule, ce qui am\`ene \`a une diminution de l'\'emission de $CO_2$ par la voiture. Cette diminution est tr\`es importante car dans les ann\'ees \`a venir, il y aura beaucoup de normes \`a ce sujets, il est donc important pour les constructeurs automobiles de s'y pr\'eparer d\`es maintenant.\\
Plastic Omnium poss\`ede d\'ej\`a des codes pour des simulations en trois dimensions, cependant ces codes sont utilis\'es sur des mod\`eles de plusieurs millions d'\'el\'ements et donc les calculs prennent plusieurs semaines sur un nombre important de processeurs.\\
L'id\'ee est de choisir un espace dans lequel résoudre les \'equartions de Navier-Stockes qui est g\'en\'er\'e par  les fonction propres de l'op\'erateur rotationel, ainsi une solution pourra s'\'ecrire comme combinaison lin\'eaire de ces fonctions. Ceci, avec les conditions aux limites de l'espace, permet de séparer les termes de temps et d'espace, et ainsi  réutiliser la base de fonctions et ne faire porter les itérations que sur les coefficients temporels.

