\part{Problème Mathématique}
\label{partProb}
Introduisons d'abord les notations utilisées par la suite :
\begin{align*}
gradient(v)&=(\partial_x v, \partial_y v, \partial_z v)=\grad v\\
gradient(\mathbf{v})&=\begin{pmatrix}
\partial_x v_x & \partial_y v_x & \partial_z v_x\\
\partial_x v_y & \partial_y v_y & \partial_z v_y\\
\partial_x v_z & \partial_y v_z & \partial_z v_z
\end{pmatrix}=\grad\mathbf*{v}\\
divergence(\mathbf{v})&=\frac{\partial v_x}{\partial x}+\frac{\partial v_y}{\partial y}+\frac{\partial v_z}{\partial z}=\div \mathbf{v}\\
rotationnel(\mathbf{v})&=\begin{pmatrix}
\partial_y v_z - \partial_z v_y\\
\partial_z v_x - \partial_x v_z\\
\partial_x v_y - \partial_y v_x
\end{pmatrix}=\rot \mathbf{v}\\
rot(rot(\mathbf{v}))&=\rott \mathbf{v}\\
H^1(\Omega) &= \{v \in L^2(\Omega)\;|\; \grad v\in L^2(\Omega)\}\\
H^1_0(\Omega) &= \{v \in H^1(\Omega)\; |\; v\restr = 0\}\\
H(\mathrm{div}) &= \{\mathbf{v} \in [L^2(\Omega)]^3\; |\; \div\mathbf{v} \in L^2(\Omega) \}\\
H(\mathrm{rot}) &= \{\mathbf{v} \in [L^2(\Omega)]^3\; |\; \rot\mathbf{v} \in L^2(\Omega) \}\\
L^2_\sigma(\Omega) &= \{\mathbf{v} \in [L^2(\Omega)]^3\; |\; \div \mathbf{v} = 0\text{ et }\mathbf{v}\cdot \mathbf{n}\restr = 0 \}\\
D^1(\Omega) &= \{\mathbf{v} \in [H^1(\Omega)]^3\cap L^2_\sigma(\Omega)\; |\; (\rot \mathbf{v}\cdot \mathbf{n})\restr = 0  \}
\end{align*}

Nous cherchons $(\mathbf{v},p)$, correspondant respectivement à la vitesse et à la pression, solutions de l'équation de Navier-Stokes incompressibles adimensionnalisées dans $Q_T=\Omega\times[0,T]$, où $\Omega$ est un ouvert de $\R^3$ et $\partial\Omega$ sa frontière, avec une condition initiale et des conditions aux limites d'imperméabilité généralisée.
\begin{pb}\label{depart}
On cherche $(\mathbf{v},p)$ tel que :
\begin{equation*}
\left\{\begin{aligned}
&\frac{\partial \mathbf{v}}{\partial t} + (\rot  \mathbf{v})\times \mathbf{v} + \grad q + \frac{1}{Re}\rott  \mathbf{v}-\mathbf{f} = 0\\
&\div \mathbf{v} = 0\\
&\mathbf{v}\big\rvert_{t=0} = \mathbf{v}_0\\
&\mathbf{v}\cdot \mathbf{n}\restr = \alpha_0\\
&(\rot  \mathbf{v})\cdot \mathbf{n}\restr = \alpha_1\\
&(\rott  \mathbf{v})\cdot \mathbf{n}\restr = \alpha_2
\end{aligned}\right.
\end{equation*}
où $q = \frac{|\mathbf{v}|^2}{2}+p$.\\
\end{pb}
Ce problème possède des conditions aux bords non standards, plutôt que des conditions de Dirichlet, on utilise les conditions $\nabla^k\times\mathbf{v}\cdot\mathbf{n}\restr$ pour $k=0,1,2$ avec la convention $\nabla^0\times=Id$.\\
Ces conditions ont été assez peu étudiées et très peu utilisées dans l'industrie. On peut citer les travaux de V. Girault \cite{girault90-1}, H. Bellout, J. Neustuppa et P. Penel \cite{Penel2004}. Le principe est une décomposition de type Galerkin que l'on peut rapprocher avec les travaux de E. Deriaz et V. Perrier \cite{Deriaz2009249}.\\

L'intérêt pour Plastic Omnium est de pouvoir séparer les variables en temps et en espace et ainsi pouvoir faire des simulations sur un laps de temps plus long. De plus, les lois de paroi n'étant plus nécessaires, on supprime le besoin de prendre en compte la couche limite lors du maillage.\\

On va maintenant expliquer la stratégie de résolution du problème dans le chapitre \ref{strat}, puis étudié plus en détail chaque étapes de la résolution dans \ref{fv}.

\chapter{Stratégie de résolution}
\label{strat}
D'après \cite{Penel2004}, l'existence et l'unicité de la solution sont assurés si la solution est dans $D^1$, c'est-à-dire $\nabla^k\times\mathbf{v}\cdot\mathbf{n}\restr=0$ pour $k=0,1$. On va donc relever ces conditions aux bords à l'aide d'une fonction $\mathbf{a}$. On cherche à écrire  $\mathbf{v}=\mathbf{u}+\mathbf{a}$, tel que :\\
\begin{equation}\label{v}
\begin{array}{c|ccccc}
& \mathbf{v} & = & \mathbf{a} & + & \mathbf{u}\\ \hline
\div\star & 0 & & 0 & & 0\\ \hline
\star\cdot \mathbf{n}\restr & \alpha_0 & & \alpha_0 & & 0\\ \hline
\rot\star\cdot \mathbf{n}\restr & \alpha_1 & & \alpha_1 & & 0\\ \hline
\rott\star\cdot \mathbf{n}\restr & \alpha_1 & & 0 & & \alpha_2
\end{array}
\end{equation}
Ainsi, on a $\mathbf{u}\in D^1(\Omega)$ qui sert à définir la solution à l'intérieur du domaine et $\mathbf{a}\in [L^2(\Omega)]^3$ qui sert à relever le problème sur les bords du domaine.
\begin{pb}\label{a}
Trouver $\mathbf{a}\in [L^2(\Omega)]^3$ tel que :
\begin{equation*}
\left\{\begin{aligned}
&\mathbf{a}=\rot \mathbf{b}+\grad\psi^0\\
&\div \mathbf{a} =0\\
&\mathbf{a}\cdot \mathbf{n}\restr = \alpha_0\\
&(\rot \mathbf{a})\cdot \mathbf{n}\restr = \alpha_1
\end{aligned}\right.
\end{equation*}
\end{pb}
En appliquant la divergence et les conditions aux bords à la première ligne du problème on obtient le tableau suivant :
\begin{center}
\begin{tabular}{c|ccccc}
& $\mathbf{a}$ & = & $\grad\psi^0$ & + & $\rot \mathbf{b}$ \\ \hline
$\div\star$ & 0 & & $\laplace\psi^0$ & & 0\\ \hline
$\star\cdot \mathbf{n}\restr$ & $\alpha_0$ & & $\alpha_0$ & & 0\\ \hline
$\rot\star\cdot \mathbf{n}\restr$ & $\alpha_1$ & & 0 & & $\alpha_1$
\end{tabular}
\end{center}
En utilisant la première et la deuxième ligne du tableau, on obtient le problème :
\begin{pb}\label{psi0}
Trouver $\psi^0$ tel que :
\begin{equation*}
\left\{\begin{aligned}
&-\laplace\psi^0 = 0\\
&\grad\psi^0\cdot \mathbf{n}\restr=\alpha_0
\end{aligned}\right.
\end{equation*}\end{pb}
Ce problème permet de trouver $\psi^0$ à une constante près, on va donc utiliser un multiplicateur de Lagrange pour ajouter une contrainte sur $\psi^0$, par exemple $\int_\Omega \psi^0 = 0$, cette constante est sans importance car on cherche le gradient de $\psi^0$.\\
Il y a plusieurs manières de résoudre le problème \ref{psi0}, la solution choisie sera discutée plus loin.\\

Soit $\mathbf{b}$ solution du problème mixte suivant
\begin{pb}\label{curlb}
Trouver $(\mathbf{b},\psi^1)$ tel que :
\begin{equation*}
\left\{\begin{aligned}
&\rott \mathbf{b} = \grad\psi^1\\
&\div \mathbf{b} = 0\\
&\mathbf{b}\cdot \mathbf{n}\restr = 0\\
&\rot \mathbf{b}\cdot \mathbf{n}\restr = 0\\
&\grad\psi^1\cdot \mathbf{n}\restr = \alpha_1
\end{aligned}\right.
\end{equation*}\end{pb}

% Il reste encore plusieurs questions concernant la manière de relever $\alpha_1$. En effet, écris de cette manière, $\mathbf{b}\in D^1(\Omega)$ et donc si $\mathbf{b}\cdot\mathbf{n}\restr=0$ alors $\rott\mathbf{b}\cdot\mathbf{n}\restr=0$ aussi. Ce problème est donc encore ouvert.\\

Une fois $\grad\psi^0$, $\rot \mathbf{b}$ connus, on peut retrouver $\mathbf{a}$.\\

On remplace maintenant $\mathbf{v}$ par $\mathbf{u}+\mathbf{a}$ dans le problème \ref{depart} :
\[ \frac{\partial(\mathbf{u}+\mathbf{a})}{\partial t}+(\rot(\mathbf{u}+\mathbf{a}))\times(\mathbf{u}+\mathbf{a}) + \grad (\frac{|\mathbf{u}+\mathbf{a}|^2}{2}+p) + \frac{1}{Re}\rott(\mathbf{u}+\mathbf{a}) - \mathbf{f} = 0 \]
Ce qui donne en notant $\pi_a=\frac{|\mathbf{u}+\mathbf{a}|^2}{2}+p$ :
\[ \frac{\partial \mathbf{u}}{\partial t}+\frac{\partial \mathbf{a}}{\partial t} + (\rot \mathbf{u}+\rot \mathbf{a})\times(\mathbf{u}+\mathbf{a}) + \grad\pi_{\mathbf{a}} + \frac{1}{Re}(\rott \mathbf{u}+\rott \mathbf{a}) - \mathbf{f} = 0 \]
Comme $\rott \mathbf{a} = 0$ et en notant $\mathbf{f_a}=\mathbf{f}-\frac{\partial \mathbf{a}}{\partial t} - (\rot \mathbf{a})\times \mathbf{a}$, on a le problème suivant :
\begin{pb}\label{pbu}
Trouver $\mathbf{u}$ tel que :
\begin{equation*}
\left\{\begin{aligned}
&\frac{\partial \mathbf{u}}{\partial t} + (\rot \mathbf{u})\times \mathbf{u} + (\rot \mathbf{u})\times \mathbf{a} +(\rot \mathbf{a})\times \mathbf{u} + \grad \pi_{\mathbf{a}} +\frac{1}{Re}\rott  \mathbf{u} - \mathbf{f_a} = 0\\
&\div \mathbf{u} = 0\\
&\mathbf{u}\big\rvert_{t=0} = \mathbf{v}_0 - \mathbf{a}(0,\cdot)\\
&\mathbf{u}\cdot \mathbf{n}\restr = 0\\
&(\rot \mathbf{u})\cdot \mathbf{n}\restr = 0\\
&(\rott  \mathbf{u})\cdot \mathbf{n}\restr = 0
\end{aligned}\right.
\end{equation*}\end{pb}

On cherche donc $\mathbf{u}=\mathbf{v}-\mathbf{a}$, et en utilisant une décomposition de Galerkin, on a :
\begin{equation}\label{u}
\mathbf{u}(t,\cdot) = \sum_{i=1}^{\infty} c_i(t)\mathbf{g}_i(\cdot)
\end{equation}
où on sépare les variables en temps et en espace. Les coefficients $c_i$ contiennent toutes les informations dépendantes du temps, tandis que les fonctions $\mathbf{g}_i$ portent la composante spatiale.\\

Comme $\mathbf{u}\in D^1(\Omega)=D(\mathrm{rot}_{imperm})$, d'après \cite{Penel2004}, on peut choisir les fonctions de base
$\mathbf{g}_i$ comme étant les fonctions propres de l'opérateur rotationnel. Ces
fonctions sont les mêmes que celles de l'opérateur $\rott$.

En effet, soit $(\lambda_i,\mathbf{g}_i)$ solutions de $\rot \mathbf{g}_i = \lambda_i\mathbf{g}_i$ et $(\Lambda_i,\mathbf{G}_i)$ solutions de $\rott \mathbf{G}_i = \Lambda_i\mathbf{G_i}$. Alors :
\[ \rott \mathbf{g}_i = \rot(\rot \mathbf{g}_i)=\rot(\lambda_i\mathbf{g}_i)=\lambda^2\mathbf{g}_i \]
Par identification, on voit que $\mathbf{g}_i=\mathbf{i}_i$ et que $\lambda_i=\pm\sqrt\Lambda_i$. Pour n'avoir que des valeurs propres de multiplicité une, on ne garde que les valeurs propres positives, le signe étant porté par le coefficient $c_i$.

On cherche donc à résoudre le problème aux valeurs propres suivant :
\begin{pb}\label{curlcurl}
Trouver$(\lambda_i,\mathbf{g}_i)\in\R\times D^1(\Omega)$ tel que :
\begin{equation*}
\left\{\begin{aligned}
&\rott  \mathbf{g}_i = \lambda_i^2 \mathbf{g}_i\\
&\mathbf{g}_i\cdot \mathbf{n}\restr = 0\\
&\rot \mathbf{g}_i\cdot \mathbf{n}\restr = 0\\
&\rott  \mathbf{g}_i\cdot \mathbf{n}\restr = 0
\end{aligned}\right.
\end{equation*}\end{pb}

En injectant (\ref{u}) dans le problème \ref{pbu}, on obtient :
\begin{align*}
\frac{\partial}{\partial t}\left(\sum_{i=1}^\infty c_i\mathbf{g}_i\right) &+ \left(\rot \left(\sum_{i=1}^\infty c_i\mathbf{g}_i\right)\right)\times \left(\sum_{i=1}^\infty c_i\mathbf{g}_i\right) + \left(\rot \left(\sum_{i=1}^\infty c_i\mathbf{g}_i\right)\right)\times \mathbf{a}&\\
&+ (\rot \mathbf{a})\times \left(\sum_{i=1}^\infty c_i\mathbf{g}_i\right) + \grad \pi_{\mathbf{a}} +\frac{1}{Re}\rott  \left(\sum_{i=1}^\infty c_i\mathbf{g}_i\right) - \mathbf{f_a} =0\\
\end{align*}

En utilisant la linéarité de la dérivée, ainsi que celle de l'opérateur rotationnel et le fait que les fonctions $\bm{g_i}$ soient les modes propres de cet opérateur, on obtient le problème suivant :
\begin{pb}\label{pbc}
Trouver $(c_i)$ tel que :
\begin{align*}
\sum_{i=1}^\infty\frac{\partial c_i}{\partial t}\mathbf{g}_i &+ \sum_{i=1}^\infty\sum_{j=1}^\infty c_i\lambda_ic_j(\mathbf{g}_i\times \mathbf{g}_j) + \sum_{i=1}^\infty c_i\lambda_i(\mathbf{g}_i\times \mathbf{a})\\
& + \sum_{i=1}^\infty c_i((\rot\mathbf{a})\times \mathbf{g}_i) + \grad \pi_{\mathbf{a}} +\frac{1}{Re}\sum_{i=1}^\infty c_i\lambda_i^2\mathbf{g}_i - \mathbf{f_a} = 0\\
\end{align*}
avec \[ \sum_{i=1}^\infty c_i(0)\mathbf{g}_i = \mathbf{v_0}-\mathbf{a}(0,\cdot) \]
\end{pb}

%% Comme les conditions aux bords sont respectées par tous les $\mathbf{g}_i$, $\mathbf{u}$, en tant que combinaison linéaire de ceux-ci, les respectent aussi.\\

Pour résumer, on doit donc :
\begin{enumerate}
\item générer la base $\mathbf{g}_i$ des fonctions propres de l'opérateur curl en résolvant le problème aux valeurs propres \ref{curlcurl} comme expliqué dans le chapitre \ref{eigen}.
\item trouver $\mathbf{a}$ pour pouvoir décomposer $\mathbf{v}$ en $\mathbf{u}+\mathbf{a}$, pour cela, on résout les problèmes \ref{psi0} et \ref{curlb}. Ce qui permet de trouver $\mathbf{a}$ grâce au problème \ref{a}. Cette partie est détaillé dans \ref{relev}.
\item résoudre le problème \ref{pbc} pour trouver les coefficients $c_i$. Cela est expliqué dans \ref{spectre}.
\item recomposer $\mathbf{v}=\mathbf{u}+\mathbf{a}$, et chercher $p$ pour avoir la solution du problème \ref{depart}. Cette dernière partie est montré dans \ref{pression}.
\end{enumerate}
Cette stratégie est présenté dans la figure \ref{org1}.
\begin{figure}[H]
\centering
\begin{tikzpicture}
\node[draw,scale=\taille,fill=green!50] (alpha0) at (0,10) {$\alpha_0$};
\node[draw,scale=\taille,fill=gray!50,label={[xshift=0.9cm]\ref{psi0}}] (pbpsi0) at (0,8) {$\begin{aligned}
-\laplace\psi^0&=0\\
\grad\psi^0\cdot \mathbf{n} &= \alpha_0 \end{aligned}$} ;
\node[draw,scale=\taille,fill=blue!50] (gradpsi0) at (0,5.5) {$\grad\psi^0$} ;
\node[draw,scale=\taille,fill=green!50] (alpha1) at (3,10) {$\alpha_1$};
\node[draw,scale=\taille,fill=gray!50,label={[xshift=1.1cm]\ref{curlb}}] (pbcurlb) at (3,8) {$\begin{aligned}\rott \mathbf{b} = \grad\psi^1\\
\div \mathbf{b} = 0\\
\mathbf{b}\cdot \mathbf{n}\restr = 0\\
\rot \mathbf{b}\cdot \mathbf{n}\restr = 0\\
\grad\psi^1\cdot \mathbf{n}\restr = \alpha_1 \end{aligned}$} ;
\node[draw,scale=\taille,fill=blue!50] (curlb) at (3,5.5) {$(\rot\mathbf{b},\grad\psi^1)$} ;
\node[draw,scale=\taille,fill=gray!50,label={[xshift=1.1cm]\ref{a}}] (pba) at (1.5,4) {$\mathbf{a}=\grad\psi^0+\rot\mathbf{b}$};
\node[draw,scale=\taillem,fill=blue!50] (a) at (1.5,3) {$\mathbf{a}$} ;
\node[draw,scale=\taille,fill=gray!50,label={[xshift=1.3cm]\ref{curlcurl}}] (pbeigen) at (10,8) {$\begin{aligned}
\rott \mathbf{g}_i = \lambda_i^2\mathbf{g}_i\\
\div\mathbf{g}_i = 0\\
\mathbf{g}_i\cdot \mathbf{n}\restr = 0\\
\rot \mathbf{g}_i\cdot \mathbf{n}\restr = 0\\
(\rott \mathbf{g}_i\cdot \mathbf{n}\restr = 0) \end{aligned}$} ;
\node[draw,scale=\taille,fill=yellow!50] (lambdagi) at (10, 5.5) {$(\lambda_i,\mathbf{g}_i)$} ;
\node[draw,scale=\taille,fill=green!50] (alpha2) at (5.75,5.5) {$\alpha_2$};
\node[draw,scale=\taille,fill=green!50] (f) at (6.75,5.5) {$\mathbf{f}$};
\node[draw,scale=\taille,fill=green!50] (c0) at (7.75,5.5) {$c_i(0)$};
\node[draw,scale=\taille,fill=gray!50,label={[xshift=4.2cm]\ref{pbc}}] (pbc) at (8.5,3) {$\begin{aligned}
\sum_{i=1}^\infty\frac{\partial c_i}{\partial t}\bm{g_i} &+ \sum_{i=1}^\infty\sum_{j=1}^\infty c_i\lambda_ic_j(\bm{g_i}\times \bm{g_j}) + \sum_{i=1}^\infty c_i\lambda_i(\bm{g_i}\times \mathbf{a})\\
& + \sum_{i=1}^\infty c_i((\rot\mathbf{a})\times \mathbf{g}_i) + \grad \pi_{\mathbf{a}} +\frac{1}{Re}\sum_{i=1}^\infty c_i\lambda_i^2\mathbf{g}_i - \mathbf{h_a} = 0 \end{aligned}$};
\node[draw,scale=\taillem,fill=blue!50] (u) at (8.5,1) {$\mathbf{u}$} ;
\node[draw,scale=\taille,fill=gray!50] (pbv) at (5,-0.5) {$\mathbf{v}=\mathbf{a}+\mathbf{u}$} ;
\node[draw,scale=\tailleg,fill=red!50] (v) at (5,-2) {$(\mathbf{v},p)$} ;

\draw[->,>=latex] (alpha0) -- (pbpsi0); \draw[->,>=latex] (pbpsi0) -- (gradpsi0); \draw[->,>=latex] (alpha1) -- (pbcurlb); \draw[->,>=latex] (pbcurlb) -- (curlb); \draw[->,>=latex] (gradpsi0) -- (pba); \draw[->,>=latex] (curlb) -- (pba); \draw[->,>=latex] (pba) -- (a); \draw[->,>=latex] (pbeigen) -- (lambdagi); \draw[->,>=latex] (a) -- (pbc); \draw[->,>=latex] (alpha2) -- (pbc); \draw[->,>=latex] (f) -- (pbc); \draw[->,>=latex] (c0) -- (pbc); \draw[->,>=latex] (lambdagi) -- (pbc); \draw[->,>=latex] (pbc) -- (u); \draw[->,>=latex] (u) -- (pbv); \draw[->,>=latex] (a) -- (pbv); \draw[->,>=latex] (pbv) -- (v);
\end{tikzpicture}
\caption{Organigramme de la stratégie de résolution}\label{org1}
\end{figure}

\chapter{Formulations variationnelles}
\label{fv}
Maintenant que l'on sait comment nous allons procéder pour résoudre le problème \ref{depart}, on va examiner plus en détail chacun des problèmes le composant, en particulier, on va chercher les formulations variationnelles de chacun.

\section{Problème aux valeurs propres}
\label{eigen}
On s'intéresse ici plus particulièrement au problème \ref{curlcurl},
\begin{equation*}
(\lambda_i^2,\mathbf{g}_i)\in\R\times D^1(\Omega)\quad \left\{\begin{aligned}
&\rott  \mathbf{g}_i = \lambda_i^2 \mathbf{g}_i\\
&\div \mathbf{g}_i = 0\\
&\mathbf{g}_i\cdot \mathbf{n}\restr = 0\\
&\rot \mathbf{g}_i\cdot \mathbf{n}\restr = 0\\
&(\rott  \mathbf{g}_i\cdot \mathbf{n}\restr = 0)
\end{aligned}\right.
\end{equation*}
qui va nous donner les valeurs et fonctions propres de l'opérateur rotationnel avec les conditions aux limites d'imperméabilité. Cela nous permettra ensuite d'exprimer $\mathbf{u}$ dans la base de $D^1$ formée par ces fonctions.\\

Tout d'abord, notons que d'après P. Penel \cite{Penel2004} lemme 3.3, on a :
\[ D^1 = \{\mathbf{v}=\mathbf{v_0}+\grad\phi\; |\; \mathbf{v_0}\in H^1_0(\Omega),\ -\laplace\phi=\div\bm{v_0},\ \grad\phi\cdot \mathbf{n}\restr = 0 \} \]
Ainsi, sur $\partial\Omega$, $\mathbf{v}=\grad\phi$.\\

On va maintenant chercher la formulation variationnelle, soit $\mathbf{g}\in D^1(\Omega)$ solution du problème \ref{curlcurl}, alors pour tout $\bm{\varphi}\in D^1(\Omega)$ nous avons :
\[ \int_\Omega (\rott \mathbf{g})\cdot\bm{\varphi}\ dX = \int_\Omega\lambda^2 \mathbf{g}\cdot\bm{\varphi}\ dX \]
puis en intégrant par partie :
\[ \int_\Omega (\rot \mathbf{g})\cdot(\rot\bm{\varphi})\ dX + \int_{\partial\Omega} ((\rot \mathbf{g})\times \bm{\varphi})\cdot \mathbf{n}\ d\Gamma = \lambda^2\int_\Omega \mathbf{g}\cdot\bm{\varphi}\ dX \]
On a $\bm{\varphi}=\bm{\varphi}_0+\grad\phi$ et sur $\partial\Omega,\, \bm{\varphi}\restr=\grad\phi$, d'où :
\[ \int_\Omega (\rot \mathbf{g})\cdot(\rot\bm{\varphi})\ dX + \int_{\partial\Omega} ((\rot \mathbf{g})\times \grad\phi)\cdot \mathbf{n}\ d\Gamma = \lambda^2\int_\Omega \mathbf{g}\cdot\bm{\varphi}\ dX \]
En utilisant le théorème de flux-divergence aussi appelé théorème de Green-Ostrogradski :
\[ \int_\Omega (\rot \mathbf{g})\cdot(\rot\bm{\varphi})\ dX + \int_\Omega \div((\rot \mathbf{g})\times \grad\phi)\ dX = \lambda^2\int_\Omega \mathbf{g}\cdot\bm{\varphi}\ dX \]
En utilisant la formule $\div(\mathbf{F}\times \mathbf{G}) = \mathbf{G}\cdot \rot \mathbf{F} - \mathbf{F}\cdot \rot \mathbf{G}$, on a :
\[ \int_\Omega (\rot \mathbf{g})\cdot(\rot\bm{\varphi})\ dX + \int_\Omega \grad\phi\cdot(\rott \mathbf{g})\ dX -\int_\Omega (\rot \mathbf{g})\cdot (\rot\grad\phi)\ dX  = \lambda^2\int_\Omega \mathbf{g}\cdot\bm{\varphi}\ dX \]
Comme le rotationnel d'un gradient est nul, on a :
\[ \int_\Omega (\rot \mathbf{g})\cdot(\rot\bm{\varphi})\ dX + \int_\Omega \grad\phi\cdot(\rott \mathbf{g})\ dX  = \lambda^2\int_\Omega \mathbf{g}\cdot\bm{\varphi}\ dX \]
En intégrant le deuxième terme par partie, on obtient :
\[ \int_\Omega (\rot \mathbf{g})\cdot(\rot\bm{\varphi})\ dX + \int_{\partial\Omega} \phi((\rott \mathbf{g})\cdot \mathbf{n})\ d\Gamma - \int_\Omega \phi(\div(\rott \mathbf{g}))\ dX  = \lambda^2\int_\Omega \mathbf{g}\cdot\bm{\varphi}\ dX \]
Comme $\rott  \mathbf{g}_i\cdot \mathbf{n}\restr = 0$, le deuxième terme s'annule et comme la divergence d'un rotationnel est nulle, le troisième terme s'annule aussi. Le problème \ref{curlcurl} devient donc :
\begin{pb}\label{fveigen}
Trouver $(\lambda_i^2,\mathbf{g}_i)\in \R\times D^1$ tel que $\forall \bm{\varphi}\in D^1(\Omega)$ :
\begin{equation*}
\int_\Omega (\rot \mathbf{g})\cdot(\rot\bm{\varphi})\ dX = \lambda^2\int_\Omega \mathbf{g}\cdot\bm{\varphi}\ dX
\end{equation*}
\end{pb}
On obtient donc $(\lambda_i^2,\mathbf{g}_i)_{i=1,\dots,M}$, où $(\mathbf{g}_i)$ tend vers une base de $D^1(\Omega)$ lorsque $M\rightarrow \infty$.
%\iffalse
\section{Décomposition des fonctions de $D^1$}
\label{decomp}

Comme énoncé plus tôt, tout élément de $D^1$ peut s'écrire $\bm{\varphi} = \bm{\varphi}_0 + \grad\phi$, y compris bien sûr les $\mathbf{g}_i$. Comme on va en avoir besoin pour les problèmes suivants, on va les décomposer en $\mathbf{g}_i=\mathbf{g_i^0}+\grad\psi_i$ avec $\mathbf{g_i^0}\restr = 0$ et $\grad\psi_i\cdot \mathbf{n}\restr = 0$.\\
On applique donc le rotationnel du rotationnel sur cette relation.\\
\[ \rott \mathbf{g_i^0} +\rott\grad\psi_i = \rott \mathbf{g}_i \]
Le dernier terme est nul car c'est le rotationnel d'un gradient. On utilise la formule $\rott \mathbf{v}=\grad(\div \mathbf{v})-\laplace \mathbf{v}$ sur le premier terme :
\[ \grad(\div \mathbf{g_i^0})-\laplace \mathbf{g_i^0} = \lambda_i^2 \mathbf{g}_i \]
On obtient donc le tableau suivant :
\begin{center}
\begin{tabular}{c|ccccc}
& $\mathbf{g}_i$ & = & $\mathbf{g_i^0}$ & + & $\grad\psi_i$ \\ \hline
$\rott\star$ & $\lambda_i^2\mathbf{g}_i$ & & $\grad(\div \mathbf{g_i^0})-\laplace \mathbf{g_i^0}$ & & 0\\ \hline
$\div\star$ & 0 & & $\div \mathbf{g_i^0}$ & & $\laplace\psi_i$\\ \hline
$\star\cdot \mathbf{n}\restr$ & 0 & & 0 & & 0
\end{tabular}
\end{center}

\subsection{$g_i^0$}
En utilisant la première ligne, on parvient au problème :
\begin{equation*}
\left\{\begin{aligned}
\grad(\div \mathbf{g_i^0})-\laplace \mathbf{g_i^0} &= \lambda_i^2\mathbf{g}_i\\
\mathbf{g_i^0}\restr &= 0
\end{aligned}\right.
\end{equation*}
On cherche $\mathbf{g_i^0}$ dans $[H^1_0(\Omega)]^3$. On multiplie donc cette équation par une fonction test de $[H^1_0(\Omega)]^3$ et on intègre :
\[ \int_\Omega \grad(\div \mathbf{g_i^0})\cdot\bm{\varphi} - \int_\Omega \laplace \mathbf{g_i^0}\cdot\bm{\varphi} = \int_\Omega \lambda_i^2\mathbf{g}_i\cdot\bm{\varphi} \]
On utilise ensuite la formule d'intégration par partie $\int_\Omega \grad{\mathbf{u}}\bm{\varphi} = -\int_\Omega \mathbf{u}\div\bm{\varphi} + \int_{\partial\Omega} \mathbf{u}\bm{\varphi}\cdot \mathbf{n}$ sur le premier terme :
\[ -\int_\Omega (\div \mathbf{g_i^0})(\div\bm{\varphi}) + \int_{\partial\Omega} (\div \mathbf{g_i^0})(\bm{\varphi}\cdot \mathbf{n}) - \int_\Omega \laplace \mathbf{g_i^0}\cdot\bm{\varphi} = \int_\Omega \lambda_i^2\mathbf{g}_i\cdot\bm{\varphi} \]
Comme $\bm{\varphi}\in [H^1_0(\Omega)]^3$, la seconde intégrale est nul. On intègre par partie le terme en laplacien :
\[ -\int_\Omega (\div \mathbf{g_i^0})(\div\bm{\varphi}) + \int_\Omega \overline{\grad \mathbf{g_i^0}}:\overline{\grad\bm{\varphi}} - \int_{\partial\Omega} (\overline{\grad \mathbf{g_i^0}}\cdot \mathbf{n})\cdot\bm{\varphi} = \int_\Omega \lambda_i^2\mathbf{g}_i\cdot\bm{\varphi} \]
Encore une fois, comme $\bm{\varphi}\in [H^1_0(\Omega)]^3$, le terme sur les bords s'annule. On obtient donc le problème suivant :
\begin{pb}\label{fvgi0}
Trouver $\mathbf{g_i^0}\in H^1_0$ tel que $\forall \bm{\varphi}\in [H^1_0(\Omega)]^3$ :
\begin{equation*}
-\int_\Omega (\div \mathbf{g_i^0})(\div\bm{\varphi}) + \int_\Omega \overline{\grad \mathbf{g_i^0}}:\overline{\grad\bm{\varphi}} = \int_\Omega \lambda_i^2\mathbf{g}_i\cdot\bm{\varphi}
\end{equation*}\end{pb}

\subsection{Gradient $\psi_i$}
\label{multLagrange}
D'autre part, les deux dernières lignes du tableau nous donnent le problème de Poisson suivant :
\begin{equation*}
\left\{\begin{aligned}
-\laplace\psi_i &= \div \mathbf{g_i^0}\\
\grad\psi_i\cdot \mathbf{n}\restr &= 0
\end{aligned}\right.
\end{equation*}
Cette fois-ci, on cherche $\psi_i$ dans $H^1(\Omega)$. On a donc la forme variationnelle suivante :
\begin{equation}\label{fvpsi}
\int_\Omega \grad\psi_i\cdot\grad\varphi = \int_\Omega (\div \mathbf{g_i^0})\varphi
\end{equation}

Ce problème permet de trouver $\psi_i$ seulement à une constante près, on va donc devoir imposer une constante de notre choix, par exemple pour que $\int_\Omega \psi_i = 0$. Ceci va donc créer une translation dans le résultat, qu'il va falloir corriger en post-traitement.\\
Afin d'appliquer cette contrainte supplémentaire, on va utiliser la méthode des multiplicateurs de Lagrange.\\
Si l'on note $V=H^1(\Omega)$, $a(u,v)=\int \grad u \cdot \grad v$, $l(v)=\int (\div \mathbf{g_i^0})v$ et $J(v)=\frac{1}{2}a(v,v)-l(v)$, alors résoudre l'équation \ref{fvpsi} revient à trouver $u$ tel que :
\[ J(u) = \min_{v\in V} J(v) \]
Si l'on ajoute la contrainte $b(v) = \int v = 0$, alors, avec $\lambda$ un multiplicateur de Lagrange, le problème devient trouver $u$ tel que :
\[ J(u) = \min_{v\in V} J(v) - \lambda b(v) \]
Soit, en ajoutant l'équation de la contrainte multipliée par le multiplicateur de Lagrange correspondant à $\varphi$, et le terme correspondant à la moyenne $m$ de $\psi_i$ :
\[ a(\psi_i,\varphi) + \lambda b(v) + \mu b(u) = l(\varphi) + m b(\mu) \]
Ce qui donne le problème suivant :
\begin{pb}\label{fvpsiml}
Trouver $(\psi_i,\lambda)\in H^1(\Omega)\times L^2(\Omega)$ tel que $\forall (\varphi,\mu)$ :
\begin{equation*}
\int_\Omega \grad\psi_i\cdot\grad\varphi + \int_\Omega \lambda\varphi + \int_\Omega \psi_i\mu = \int_\Omega (\div \mathbf{g_i^0})\varphi + \int_\Omega m\ \mu
\end{equation*}\end{pb}
%\fi
\section{Relèvement}
\label{relev}
On veut maintenant relever les conditions aux limites sur le bord du domaine. On écrit donc $\mathbf{v}=\mathbf{u}+\mathbf{a}$, où $\mathbf{a}$ contient les informations sur le bord du domaine :
\[ \mathbf{a}\cdot\mathbf{n}=\alpha_0\quad \rot\mathbf{a}\cdot\mathbf{n}=\alpha_1 \]

On écrit $\mathbf{a}=\grad\psi^0 + \rot\mathbf{b}$.\\
On va s'intéresser d'abord à $\grad\psi^0$ qui relève la condition $\mathbf{a}\cdot\mathbf{n}=\alpha_0$.\\

\subsection{$\psi^0$ dans $H^1$}
\label{secpsi0hdiv}
Il y a plusieurs alternatives pour résoudre \ref{psi0}, on peut tout d'abord se placer dans $H^1(\Omega)$ et résoudre :
\begin{equation*}
\left\{\begin{aligned}
&-\laplace\psi^0 = 0\\
&\grad\psi^0\cdot \mathbf{n}\restr=\alpha_0
\end{aligned}\right.
\end{equation*}

Il faut faire attention au fait que $\int_{\partial\Omega} \alpha_0$ doit être égale à 0, en effet, on a :
\[ 0=\int_\Omega \laplace \psi^0 = \int_\Omega \div(\grad\psi^0) = \int_{\partial\Omega} \grad\psi^0\cdot \mathbf{n} = \int_{\partial\Omega} \alpha_0 \]

Pour obtenir sa forme variationnelle, on multiplie par une fonction test $\varphi\in H^1(\Omega)$ et on intègre :
\[ \int_\Omega \laplace\psi^0 \varphi = 0 \]
On utilise ensuite la formule de Green pour parvenir à :
\[ -\int_\Omega \grad\psi^0\cdot\grad\varphi + \int_{\partial\Omega} \grad\psi^0\cdot \mathbf{n}\varphi = 0 \]
Or, $\grad\psi^0\cdot \mathbf{n} = \alpha_0$ sur $\partial\Omega$, on obtient donc la forme variationnelle suivante :
\begin{equation*}\label{fvpsi0LM} -\int_\Omega \grad\psi^0\cdot\grad\varphi + \int_{\partial\Omega} \alpha_0\varphi = 0
\end{equation*}
Comme énoncé précédemment, on va devoir utiliser les multiplicateurs de Lagrange pour ajouter la contrainte $\int \psi^0=0$. On utilise donc la même technique que dans \ref{multLagrange} avec $m=0$.\\
Le problème \ref{psi0} devient donc :
\begin{pb}\label{fvpsi0}
Trouver $(\psi^0,\lambda)\in H^1\times L^2(\Omega)$ tel que $\forall (\varphi,\mu)\in H^1\times L^2(\Omega)$ :
\begin{equation*}
\int_\Omega \grad\psi^0\cdot\grad\varphi + \int_\Omega \lambda\varphi + \int_\Omega \psi^0\mu = \int_\Omega \alpha_0\varphi
\end{equation*}\end{pb}
\begin{rk}
Ce problème permet de trouver $\psi^0$, mais il reste encore à calculer son gradient.
\end{rk}
% Si l'on note $V=H^1(\Omega)$, $a(u,v)=\int \grad u \cdot \grad v$, $l(v)=\int \alpha_0v$ et $J(v)=\frac{1}{2}a(v,v)-l(v)$, alors résoudre l'équation \ref{fvpsi0LM} revient à trouver $u$ tel que :
% \[ J(u) = \min_{v\in V} J(v) \]
% Si l'on ajoute la contrainte $b(v) = \int v = 0$, alors, avec $\lambda$ un multiplicateur de Lagrange, le problème devient trouver $u$ tel que :
% \[ J(u) = \min_{v\in V} J(v) - \lambda b(v) \]
% Soit, en ajoutant l'équation de la contrainte multipliée par le multiplicateur de Lagrange correspondant à $\varphi$, et le terme correspondant à la moyenne $m=0$ de $\psi^0$, on doit trouver $(\psi^0,\lambda)\in H^1(\Omega)\times L^2(\Omega)$ tel que $\forall (\varphi,\mu)$ :
% \begin{align}\label{fvpsi0}
% a(\psi_i,\varphi) + \lambda b(v) + \mu b(u) &= l(\varphi) + m b(\mu) \notag \\
% \int_\Omega \grad\psi^0\cdot\grad\varphi + \int_\Omega \lambda\varphi + \int_\Omega \psi^0\mu &= \int_\Omega \alpha_0\varphi
% \end{align}

\subsection{$\psi^0$ dans $H(\mathrm{div})$}

L'autre possibilité pour trouver $\psi^0$ est de chercher $(\mathbf{w},\psi^0)\in H(\mathrm{div})\times L^2(\Omega)$ solution du problème mixte suivant :
\begin{equation*}
\left\{\begin{aligned}
\mathbf{w} &= \grad \psi^0\\
\div \mathbf{w} &= 0\\
\mathbf{w}\cdot \mathbf{n}\restr &= \alpha_0
\end{aligned}\right.
\end{equation*}
Pour obtenir la formulation faible du problème, on multiplie par une fonction test $(\bm{\varphi},\nu)\in H(\mathrm{div})\times L^2(\Omega)$ et on intègre les deux premières équations :
\begin{align*}
\int_\Omega \mathbf{w}\cdot\bm{\varphi} &= \int_\Omega \grad\psi^0\cdot\bm{\varphi}\\
\int_\Omega \div \mathbf{w}\ \nu &= 0
\end{align*}
On intègre par partie la deuxième équation :
\[ \int_\Omega \div \mathbf{w}\ \nu = \int_{\partial\Omega} \mathbf{w}\cdot \mathbf{n}\ q - \int_\Omega \mathbf{w}\cdot\grad\nu = 0  \]
En insérant la condition au bord et la première équation, on obtient la formulation faible :
\[ -\int_\Omega \mathbf{w}\cdot\bm{\varphi} + \int_\Omega \mathbf{w}\cdot\grad\nu + \int_\Omega \grad\psi^0\cdot\bm{\varphi}  = \int_{\partial\Omega} \alpha_0\nu \]

Pour ajouter la contrainte de moyenne nulle, on procède de la même manière que dans \ref{multLagrange}. Le problème \ref{psi0} devient donc :
\begin{pb}\label{fvpsidiv}
Trouver $(\mathbf{w},\psi^0,\lambda)\in H(\mathrm{div})\times L^2(\Omega)\times L^2(\Omega)$ tel que $\forall (\bm{\varphi},\nu,\mu)\in H(\mathrm{div})\times L^2(\Omega)\times L^2(\Omega)$
\begin{equation*}
-\int_\Omega \mathbf{w}\cdot\bm{\varphi} + \int_\Omega \mathbf{w}\cdot\grad\nu + \int_\Omega \grad\psi^0\cdot\bm{\varphi} + \int_\Omega \lambda\varphi + \int_\Omega \psi^0\mu = \int_{\partial\Omega} \alpha_0\nu
\end{equation*}\end{pb}

\begin{rk}
Calculer $\grad\psi^0$ dans $H(\mathrm{div})$ a l'avantage de faire gagner un ordre à la régularité de $\grad\psi^0$ par rapport à résoudre le problème dans $H^1$ pour trouver $\psi^0$ et ensuite calculer son gradient.
\end{rk}

\subsection{$\mathbf{b}$ dans $H(\mathrm{rot})$}
On s'intéresse ensuite à $\rot\mathbf{b}$ qui relève la condition $\rot\mathbf{a}\cdot\mathbf{n}=\alpha_1$.\\
Pour le problème \ref{curlb}, on veut résoudre dans $H(\mathrm{rot})$ le problème mixte suivant : 
\begin{equation*}
\left\{\begin{aligned}
&\rott \mathbf{b} = \grad\psi^1\\
&\div \mathbf{b} = 0\\
&\mathbf{b}\cdot \mathbf{n}\restr = 0\\
&\rot \mathbf{b}\cdot \mathbf{n}\restr = 0\\
&\grad\psi^1\cdot \mathbf{n}\restr = \alpha_1
\end{aligned}\right.
\end{equation*}

Pour avoir la formulation faible, on multiplie par une fonction test de $H(\mathrm{rot})$ et on intègre :
\[ \int_\Omega (\rott \mathbf{b})\cdot\bm{\varphi} = \int_\Omega (\grad\psi^1)\cdot\bm{\varphi} \]
En intégrant par partie le premier terme et en utilisant la formule de Green sur le second, on obtient le problème suivant :
\begin{pb} \label{fvbcurl}
Trouver $(\mathbf{b},\psi^1)\in H(\mathrm{rot})\times L^2$ tel que $\forall \bm{\varphi}\in H(\mathrm{rot})$ :
\begin{equation*}
\int_\Omega (\rot \mathbf{b})\cdot(\rot\bm{\varphi}) - \int_{\partial\Omega} (\rot \mathbf{b})(\bm{\varphi}\cdot \mathbf{n}) + \int_\Omega \psi^1(\div\bm{\varphi}) - \int_{\partial\Omega} \psi^1(\bm{\varphi}\cdot \mathbf{n}) = 0
\end{equation*}\end{pb}

\subsection{$\mathbf{b}$ dans $D^1$}
Une autre solution pour trouver $\mathbf{b}$ est d'utiliser la base de $D^1$ calculée précédemment pour exprimer $\mathbf{b}$. En effet, on a
\[ \div\mathbf{b}=0\quad \mathbf{b}\cdot\mathbf{n}\restr=0\quad \rot\mathbf{b}\cdot\mathbf{n}\restr=0\]
d'où $\mathbf{b}\in D^1$.\\
On peut donc résoudre le problème
\begin{pb}
Trouver $\psi^1$ tel que :
\begin{equation*}
\left\{ \begin{aligned}
-\laplace\psi^1 &= 0\\
\grad\psi^1\cdot\mathbf{n} &= \alpha_1
\end{aligned}\right. \end{equation*}\end{pb}
de la même manière que le problème \ref{psi0}.\\
Ensuite, trouver $\mathbf{b}$ tel que $\forall k=1,\dots,M$ :
\[ \int_\Omega \rott\mathbf{b}\cdot\mathbf{g_k} = \int_\Omega \grad\psi^1\cdot\mathbf{g_k} \]
En réutilisant la même méthode que dans \ref{eigen}, on a :
\[ \int_\Omega \rot\mathbf{b}\cdot\rot\mathbf{g_k} + \int_{\partial\Omega}(\rott\mathbf{b}\cdot\mathbf{n})\psi_k = \int_\Omega \grad\psi^1\cdot\mathbf{g_k} \]
En utilisant le fait que $\rott\mathbf{b}=\grad\psi^1$ et que $\grad\psi^1\cdot\mathbf{n}=\alpha_1$, ainsi qu'en écrivant $\mathbf{b}=\sum d_i\mathbf{g}_i$ et en notant que la base $(\mathbf{g}_i)$ est orthonormale, on obtient le problème suivant :
\begin{pb}\label{pbbd1}
Trouver $(d_i)_{i=1,\dots,M}$ tel que $\forall k=1,\dots,M$ on a :
\[ d_k\lambda_k^2 = \int_\Omega \grad\psi^1\cdot\mathbf{g_k} - \int_{\partial\Omega} \alpha_1\psi_k \]
\end{pb}

%% \subsection{Relèvement de $\alpha_2$}
%% De même, il faut encore trouver un système permettant de relever la condition $\rott\mathbf{a}\cdot\mathbf{n}=\alpha_2$.

\section{Problème spectral}
\label{spectre}
Nous connaissons maintenant $\mathbf{a}$ et les couples $(\lambda_i,\mathbf{g}_i)_{i=1,\dots,M}$, on a donc toutes les briques pour trouver les coefficients $c_i$ de $\mathbf{u}_M$ l'approximation de $\mathbf{u}$ :
\[\mathbf{u}_M= \sum_{i=1}^M c_i\mathbf{g}_i\mbox{ tel que } \mathbf{u}_M\underset{M\rightarrow\infty}{\longrightarrow} \mathbf{u}=\sum_{i=1}^\infty c_i\mathbf{g}_i\]
Pour cela, on utilise l'approximation du problème \ref{pbc}.
\begin{align*}
\sum_{i=1}^M\frac{\partial c_i}{\partial t}\mathbf{g}_i &+ \sum_{i=1}^M\sum_{j=1}^Mc_i\lambda_ic_j(\mathbf{g}_i\times \mathbf{g_j}) + \sum_{i=1}^Mc_i\lambda_i(\mathbf{g}_i\times \mathbf{a})\\
& +  \sum_{i=1}^Mc_i((\rot\mathbf{a})\times \mathbf{g}_i) + \grad \pi_{\mathbf{a}} +\frac{1}{Re}\sum_{i=1}^Mc_i\lambda_i^2\mathbf{g}_i - \mathbf{f_a} = 0\\
\end{align*}

Pour obtenir la forme variationnel du problème, on multiplie par une fonction test $\bm{\varphi}\in D^1(\Omega)$ et on intègre :
\begin{align*}
\sum_{i=1}^M\frac{\partial c_i}{\partial t}\int_\Omega\mathbf{g}_i\cdot\bm{\varphi} &+ \sum_{i=1}^M\sum_{j=1}^Mc_i\lambda_ic_j\int_\Omega(\mathbf{g}_i\times \mathbf{g_j})\cdot\bm{\varphi} + \sum_{i=1}^Mc_i\lambda_i\int_\Omega(\mathbf{g}_i\times \mathbf{a})\cdot\bm{\varphi}\\
& +  \sum_{i=1}^Mc_i\int_\Omega((\rot\mathbf{a})\times \mathbf{g}_i)\cdot\bm{\varphi} + \int_\Omega\grad \pi_{\mathbf{a}}\cdot\bm{\varphi} +\frac{1}{Re}\sum_{i=1}^Mc_i\lambda_i^2\int_\Omega\mathbf{g}_i\cdot\bm{\varphi} - \int_\Omega\mathbf{f_a}\cdot\bm{\varphi} = 0\\
\end{align*}

En utilisant une intégration par partie sur le terme contenant $\grad\pi_{\mathbf{a}}$, on arrive à :
\begin{align*}
\sum_{i=1}^M\frac{\partial c_i}{\partial t}\int_\Omega\mathbf{g}_i\cdot\bm{\varphi} &+ \sum_{i=1}^M\sum_{j=1}^Mc_i\lambda_ic_j\int_\Omega(\mathbf{g}_i\times \mathbf{g_j})\cdot\bm{\varphi} + \sum_{i=1}^Mc_i\lambda_i\int_\Omega(\mathbf{g}_i\times \mathbf{a})\cdot\bm{\varphi}\\
& +  \sum_{i=1}^Mc_i\int_\Omega((\rot\mathbf{a})\times \mathbf{g}_i)\cdot\bm{\varphi} + \int_\Omega\pi_{\mathbf{a}}\div\bm{\varphi} + \int_{\partial\Omega}\pi_{\mathbf{a}}(\bm{\varphi}\cdot\mathbf{n})\\
& \frac{1}{Re}\sum_{i=1}^Mc_i\lambda_i^2\int_\Omega\mathbf{g}_i\cdot\bm{\varphi} = \int_\Omega\mathbf{f_a}\cdot\bm{\varphi}\\
\end{align*}

\begin{rk}
Comme $\bm{\varphi}\in D^1(\Omega)$, $\div\bm{\varphi}=0$ et $\bm{\varphi}\cdot \mathbf{n}=0$ sur $\partial\Omega$, le terme de pression s'annule donc sous cette forme. Il sera recalculé en post-traitement, voir \ref{pression}.\\
\end{rk}
Le terme $\sum_{i=1}^Mc_i\lambda_i^2\int_\Omega\mathbf{g}_i\cdot\bm{\varphi}$ provient de $\int_\Omega \rott\mathbf{u}\cdot\bm{\varphi}$. Or, en utilisant la même méthode que dans le chapitre \ref{eigen}, cela devient $\int_\Omega\rot\mathbf{u}\cdot\rot\bm{\varphi} -\int_{\partial\Omega} \alpha_2\phi$.\\
En repassant à la décomposition de Galerkin, on obtient donc le terme $\sum_{i=1}^Mc_i\lambda_i\int_\Omega\mathbf{g}_i\cdot\rot\bm{\varphi}-\frac{1}{Re}\int_{\partial\Omega} \alpha_2\phi$.\\
\begin{rk}
Cependant, on a que $\rott\mathbf{u}\cdot\mathbf{n}\restr=\sum c_i\lambda_i^2 \mathbf{g}_i\cdot\mathbf{n}\restr = 0 \neq\alpha_2$.\\
On voit donc que $\alpha_2$ devrait être relever de la même manière que $\alpha_0$ et $\alpha_1$, mais comme cela a été remarqué tardivement, on a fait le choix de continuer à résoudre le système de la manière décrite dans ce rapport.\\
Pareillement, $\mathbf{b}$ appartient aussi à $D^1$, on ne devrait donc pas avoir $\rott\mathbf{b}\cdot\mathbf{n}\restr=\alpha_1$.
\end{rk}
Lorsqu'on prend une fonction de test dans $D^1$, on cherche à ce que l'équation soit vrai pour toutes les fonctions de $D^1$. Cela revient au même que de résoudre cette équation pour toutes les fonctions de la base de $D^1$. On utilise donc encore une fois les fonctions propres de l'opérateur rotationnel.\\
On cherche donc $\forall k=1,\dots,M$ :
\begin{align*}
\sum_{i=1}^M\frac{\partial c_i}{\partial t}\int_\Omega\mathbf{g}_i\cdot\mathbf{g_k} &+ \sum_{i=1}^M\sum_{j=1}^Mc_i\lambda_ic_j\int_\Omega(\mathbf{g}_i\times \mathbf{g_j})\cdot\mathbf{g_k} + \sum_{i=1}^Mc_i\lambda_i\int_\Omega(\mathbf{g}_i\times \mathbf{a})\cdot\mathbf{g_k}\\
& +  \sum_{i=1}^Mc_i\int_\Omega((\rot\mathbf{a})\times \mathbf{g}_i)\cdot\mathbf{g_k} +\frac{1}{Re}\sum_{i=1}^Mc_i\lambda_i\lambda_k\int_\Omega\mathbf{g}_i\cdot\mathbf{g_k}-\frac{1}{Re}\int_{\partial\Omega} \alpha_2\phi_k = \int_\Omega\mathbf{f_a}\cdot\mathbf{g_k}\\
\end{align*}
Comme la base $(\mathbf{g}_i)$ est orthonormale, on a $\forall k=1,\dots,M$ :
\begin{align*}
\frac{\partial c_k}{\partial t} &+ \frac{1}{Re}c_k\lambda_k^2 + \sum_{i=1}^M\sum_{j=1}^Mc_i\lambda_ic_j\int_\Omega(\mathbf{g}_i\times \mathbf{g_j})\cdot\mathbf{g_k}\\
& + \sum_{i=1}^Mc_i\lambda_i\int_\Omega(\mathbf{g}_i\times \mathbf{a})\cdot\mathbf{g_k} +  \sum_{i=1}^Mc_i\int_\Omega((\rot\mathbf{a})\times \mathbf{g}_i)\cdot\mathbf{g_k} = \int_\Omega\mathbf{f_a}\cdot\mathbf{g_k}+\frac{1}{Re}\int_{\partial\Omega} \alpha_2\phi_k\\
\end{align*}
En utilisant les notations suivantes :
\begin{align*}
R_{ijk} &= \lambda_i\int_\Omega(\mathbf{g}_i\times \mathbf{g_j})\cdot\mathbf{g_k} & R_{iak} &= \lambda_i\int_\Omega(\mathbf{g}_i\times \mathbf{a})\cdot\mathbf{g_k}\\
R_{raij} &= \int_\Omega((\rot\mathbf{a})\times \mathbf{g}_i)\cdot\mathbf{g_k} & R_{hk} &= \int_\Omega\mathbf{f_a}\cdot\mathbf{g_k}\\
R_{pk} &= \int_{\partial\Omega} \alpha_2\phi_k
\end{align*}
Le problème \ref{pbc} devient :
\begin{pb}\label{fvc}
Trouver $(c_i)_{i=1,\dots,M}$ tel que $\forall k=1,\dots,M$ :
\begin{equation*}
\frac{\partial c_k}{\partial t} + \frac{1}{Re}c_k\lambda_k^2 + \sum_{i=1}^M\sum_{j=1}^Mc_ic_jR_{ijk} + \sum_{i=1}^Mc_iR_{iak} + \sum_{i=1}^Mc_iR_{raij} = R_{hk} + \frac{1}{Re}R_{pk}
\end{equation*}\end{pb}

\begin{rk}
$R_{ijk}$ et $R_{pk}$ ne sont à calculer qu'une seule fois. Si $\mathbf{a}$ dépend du temps, alors, les autres termes doivent être recalculés à chaque itération en temps.
\end{rk}
\begin{rk}
On a autant d'équations à résoudre et d'inconnus que de fonctions propres, c'est-à-dire $M$. Afin d'avoir une bonne approximation, on a besoin du plus grand nombre de fonctions de bases possible. Il faut donc trouver $M$ pour avoir un bon ratio entre précision et temps de calcul.
\end{rk}

\section{Pression : Post-traitement de la vitesse}
\label{pression}
Pour retrouver la vitesse $\mathbf{v}$, il suffit maintenant d'additionner $\mathbf{a}$ et $\mathbf{u}$.\\
Le terme correspondant à la pression ayant été relayé en post-traitement de la vitesse, il faut le recalculer à partir de l'équation du problème \ref{depart}.

On applique la divergence sur cette équation et on utilise le fait que $\mathbf{v}$ soit à divergence nulle, et que la divergence d'un rotationnel soit toujours nulle. On a alors :
\begin{equation*}
-\laplace q = \div((\rot \mathbf{v})\times \mathbf{v}) - \div \mathbf{f}
\end{equation*}

Pour obtenir une condition au bord, on utilise la composante normale de l'équation (\ref{depart}) et les conditions aux bords de $\mathbf{v}$ :
\[ \grad q\cdot \mathbf{n}\restr =  \mathbf{f}\cdot \mathbf{n}\restr - \frac{\partial\alpha_0}{\partial t} - ((\rot \mathbf{v})\times \mathbf{v})\cdot \mathbf{n}\restr - \frac{\alpha_2}{Re} \]
On cherche maintenant la forme variationnelle du problème :
\[ \int_\Omega -\laplace q\varphi = \int_\Omega (\div((\rot \mathbf{v})\times \mathbf{v}) -\div \mathbf{f})\varphi \]
En intégrant par partie le terme de gauche, on a :
\[ \int_\Omega \grad q\grad\varphi - \int_{\partial\Omega} (\grad q\cdot \mathbf{n})\varphi = \int_\Omega (\div((\rot \mathbf{v})\times \mathbf{v}) -\div \mathbf{f})\varphi \]

Toujours de même manière, on va trouver la pression à une constante près, on utilise donc encore une fois les multiplicateur de Lagrange afin de fixer cette constante. Comme dans \ref{multLagrange}, on obtient donc au final :
\begin{pb}\label{fvq}
Trouver $p=q-\frac{\mathbf{v}\cdot\mathbf{v}}{2} \in L^2(\Omega)$ tel que $\forall \varphi\in L^2(\Omega)$, on a :
\begin{align*}
\int_\Omega \grad q\grad\varphi + \int_\Omega \lambda\varphi + \int_\Omega q\nu &= \int_\Omega (\div((\rot \mathbf{v})\times \mathbf{v}) -\div \mathbf{f})\varphi\\
&+ \int_{\partial\Omega} \left(f\cdot \mathbf{n} - \frac{\partial\alpha_0}{\partial t} - ((\rot \mathbf{v})\times \mathbf{v})\cdot \mathbf{n} - \frac{\alpha_2}{Re}\right)\varphi
\end{align*}\end{pb}

\chapter{Récapitulatif}

Les différentes étapes pour résoudre le problème sont donc :
\begin{enumerate}
\item calculer les valeurs et fonctions propres de l'opérateur rotationnel avec le problème \ref{fveigen},
\item décomposer ces fonctions propres en résolvant \ref{fvgi0} et \ref{fvpsiml},
\item trouver $\psi^0$, pour cela, on peut :
\begin{itemize}
\item résoudre \ref{fvpsi0} pour avoir $\psi^0\in H^1$,
\item résoudre \ref{fvpsidiv} pour avoir $\psi^0\in H(\mathrm{div})$,
\end{itemize}
\item résoudre \ref{fvbcurl} ou \ref{pbbd1} pour trouver $\rot\mathbf{b}$,
%\item résoudre un système à déterminer pour trouver $\mathbf{e}$,
\item recomposer $\mathbf{a}$ grâce à $\psi^0$ et $\rot\mathbf{b}$,
\item résoudre le problème spectral \ref{fvc} afin de trouver les coefficients $c_i$ pour $i=0\dots M$,
\item reconstruire $\mathbf{u}=\sum c_i \mathbf{g}_i$ et $\mathbf{v}=\mathbf{a}+\mathbf{u}$,
\item calculer la pression en post-traitement avec le problème \ref{fvq}.
\end{enumerate}

On a ainsi trouver notre solution $(\mathbf{v},p)$.\\
L'étape une est très coûteuse en temps de calcul, mais elle ne dépend que de la géométrie, ainsi, on peut réutiliser les fonctions propres même si l'on change certains paramètres. On a donc à réaliser cette étape seulement une fois par géométrie.\\
Les autres étapes dépendants de différents paramètres, on devra de nouveau les exécuter si l'on change les paramètres, notamment l'étape 6, qui prend en compte tous les paramètres et qui est elle aussi coûteuse en temps de calcul.\\

La figure \ref{org3} présente graphiquement les problèmes à résoudre.\\

\begin{figure}
\centering
\begin{tikzpicture}[scale=\taille]
\node[draw,scale=\taille,fill=green!50] (di) at (3,8) {Données initiales} ;
\node[draw,scale=\taille,fill=gray!50] (pb) at (3,7) {Problèmes à résoudre} ;
\node[draw,scale=\taille,fill=blue!50] (si) at (3,6) {Solutions intermédiaires} ;
\node[draw,scale=\taille,fill=yellow!50] (sim) at (3,5) {Sol. inter. dépendantes de la géométrie} ;
\node[draw,scale=\taille,fill=red!50] (sf) at (3,4) {Solutions finales} ;
\node[scale=\taille,text width=10cm] (coblig) at (15,8) {Chemins obligatoires} ;
\node[scale=\taille,text width=10cm] (cpurple) at (15,7.5) {{\color{purple} Chemins pour $\psi^0\in H^1$}} ;
\node[scale=\taille,text width=10cm] (cgreen) at (15,7) {{\color{green} Chemins pour $\psi^0\in H(\mathrm{div})$}} ;
\node[scale=\taille,text width=10cm] (cmagenta) at (15,6.5) {{\color{magenta} Chemins pour $\rot \mathbf{b} \in H(\mathrm{rot})$}} ;
\node[scale=\taille,text width=10cm] (ccyan) at (15,6) {{\color{cyan} Chemins pour $\rot \mathbf{b} \in D^1$}} ;
\node[scale=\taille,text width=10cm] (cred) at (15,5.5) {{\color{red} Chemins pour $\psi^1 \in H^1$}} ;
\node[scale=\taille,text width=10cm] (corange) at (15,5) {{\color{orange} Chemins pour $\psi^1 \in H(\mathrm{div})$}} ;

\node[draw,scale=\taille,fill=green!50] (a0) at (0.75,2) {$\alpha_0$} ;
\node[draw,scale=\taille,fill=gray!50,label={[xshift=-0.8cm]\ref{fvpsi0}}] (pbpsi0lp) at (-0.5,-2)
{$\begin{aligned}
-\laplace\psi^0&=0\\
\grad\psi^0\cdot \mathbf{n} &= \alpha_0
\end{aligned}$} ;
\node[draw,scale=\taille,fill=blue!50] (psi0) at (-0.5,-3.5) {$\psi^0$} ;
\node[draw,scale=\taille,fill=gray!50] (pbgradpsi0) at (-0.5,-4.75) {$w=\grad\psi^0$} ;
\node[draw,scale=\taille,fill=gray!50,label={[xshift=0.8cm]\ref{fvpsidiv}}] (pbpsi0div) at (2,-2)
{$\begin{aligned}
\mathbf{w}&=\grad\psi^0\\
\div\mathbf{w}&=0\\
\mathbf{w}\cdot \mathbf{n} &= \alpha_0
\end{aligned}$} ;
\node[draw,scale=\taille,fill=blue!50] (gradpsi0) at (0.75,-8.75) {$\grad\psi^0$} ;

\node[draw,scale=\taille,fill=green!50] (a1) at (5.5,2) {$\alpha_1$} ;
\node[draw,scale=\taille,fill=gray!50,label={[xshift=1.0cm]\ref{fvbcurl}}] (pbbcurl) at (4,-5)
{$\begin{aligned}
\rott \mathbf{b} &= \grad\psi^1\\
\div \mathbf{b} &=0\\
\mathbf{b}\cdot \mathbf{n} &= 0\\
\rot \mathbf{b}\cdot \mathbf{n} &= 0\\
\grad \psi^1\cdot \mathbf{n} &= \alpha_1
\end{aligned}$} ;
\node[draw,scale=\taille,fill=gray!50,label={[xshift=-0.8cm]\ref{fvpsi0}}] (pbpsi1lp) at (7,-2)
{$\begin{aligned}
-\laplace\psi^1&=0\\
\grad\psi^1\cdot \mathbf{n} &= \alpha_1
\end{aligned}$} ;
\node[draw,scale=\taille,fill=blue!50] (psi1) at (7,-3.5) {$\psi^1$} ;
\node[draw,scale=\taille,fill=gray!50] (pbgradpsi1) at (7,-4.75) {$w=\grad\psi^1$} ;
\node[draw,scale=\taille,fill=gray!50,label={[xshift=0.8cm]\ref{fvpsidiv}}] (pbpsi1div) at (9.5,-2)
{$\begin{aligned}
\mathbf{w}&=\grad\psi^1\\
\div \mathbf{w}&=0\\
\mathbf{w}\cdot \mathbf{n} &= \alpha_1
\end{aligned}$} ;
\node[draw,scale=\taille,fill=blue!50] (gradpsi1) at (8.25,-6) {$\grad\psi^1$} ;
\node[draw,scale=\taille,fill=gray!50,label={[xshift=1.8cm]\ref{pbbd1}}
] (pbb) at (8,-7.5) 
{$
d_k\lambda_k^2 = (\grad\psi^1,\mathbf{g_k}) - \langle\alpha_1,\psi_k\rangle
$} ;
\node[draw,scale=\taille,fill=blue!50] (b) at (5.5,-8.75) {$\rot \mathbf{b}$} ;

\node[draw,scale=\taille,fill=gray!50,label={[xshift=1.3cm]\ref{fveigen}}] (pbeigen) at (13.5,2)
{$\begin{aligned}
\rott \mathbf{g}_i = \lambda_i^2\mathbf{g}_i\\
\div\mathbf{g}_i = 0\\
\mathbf{g}_i\cdot \mathbf{n}\restr = 0\\
\rot \mathbf{g}_i\cdot \mathbf{n}\restr = 0\\
(\rott \mathbf{g}_i\cdot \mathbf{n}\restr = 0)
\end{aligned}$} ;
\node[draw,scale=\taille,fill=yellow!50] (lambdagi) at (13.5,-0.5) {$(\lambda_i,\mathbf{g}_i)$} ;
\node[draw,scale=\taille,fill=gray!50,label={[xshift=1.6cm]\ref{fvgi0}}] (pbgi0) at (13.5,-3)
{$\begin{aligned}
\grad(\div \mathbf{g_i^0})-\laplace \mathbf{g_i^0} &= \lambda_i^2\mathbf{g}_i\\
\mathbf{g_i^0} &= 0
\end{aligned}$} ;
\node[draw,scale=\taille,fill=yellow!50] (gi0) at (13.5,-4.5) {$\mathbf{g_i^0}$} ;
\node[draw,scale=\taille,fill=gray!50,label={[xshift=-0.9cm]\ref{fvpsiml}}] (pbpsi) at (13,-6)
{$\begin{aligned}
-\laplace\psi_i = \div \mathbf{g_i^0}\\
\grad\psi_i\cdot \mathbf{n}\restr = 0
\end{aligned}$} ;
\node[draw,scale=\taille,fill=yellow!50] (psi) at (12.5,-7.5) {$\psi_i$} ;

\node[draw,scale=\taille,fill=gray!50,label={[xshift=-1.1cm]\ref{a}}] (pba) at (2,-10.5) {$\mathbf{a} = \rot \mathbf{b} + \grad\psi^0$} ;
\node[draw,scale=\taillem,fill=blue!50] (a) at (2,-12) {$\mathbf{a}$} ;

\node[draw,scale=\taille,fill=green!50] (f) at (7,-9) {$f$} ;
\node[draw,scale=\taille,fill=green!50] (a2) at (8,-9) {$\alpha_2$} ;
\node[draw,scale=\taille,fill=green!50] (ck0) at (9,-9) {$c_k^0$} ;
\node[draw,scale=\taille,fill=gray!50,label={[xshift=3.3cm]\ref{fvc}}] (pbs) at (9,-12)
{$\begin{aligned}
\frac{\partial c_k}{\partial t} &+ \frac{1}{Re}c_k\lambda_k^2 + \sum_i\sum_j c_i c_j R_{ijk}\\
&+ \sum_i c_i R_{iak} + \sum_i c_i R_{raik} = R_{fk} + \frac{1}{Re}R_{pk}\\
c_k(0)&=c_k^0
\end{aligned}
$} ;
\node[draw,scale=\taille,fill=blue!50] (ck) at (9,-15) {$c_k$} ;
\node[draw,scale=\taille,fill=gray!50,label={[xshift=0.6cm]\ref{u}}] (pbu) at (9,-16) {$\mathbf{u}=\sum c_kg_k$} ;
\node[draw,scale=\taillem,fill=blue!50] (u) at (9,-17) {$\mathbf{u}$} ;
\node[draw,scale=\taille,fill=gray!50,label={[xshift=0.5cm]\ref{v}}] (pbv) at (3,-18) {$\mathbf{v}=\mathbf{a}+\mathbf{u}$} ;
\node[draw,scale=\tailleg,fill=red!50] (v) at (3,-19) {$\mathbf{v}$} ;
\node[draw,scale=\taille,fill=gray!50,label={[xshift=3.6cm]\ref{fvq}}] (pbq) at (9,-19)
{$\begin{aligned}
-\laplace q = \div((\rot \mathbf{v})\times \mathbf{v}) - \div \mathbf{f}\\
\grad q\cdot \mathbf{n}\restr =  \mathbf{f}\cdot \mathbf{n}\restr - \frac{\partial\alpha_0}{\partial t} - ((\rot \mathbf{v})\times \mathbf{v})\cdot \mathbf{n}\restr - \frac{\alpha_2}{Re}
\end{aligned}$} ;
\node[draw,scale=\tailleg,fill=red!50] (q) at(15,-19) {$p$} ;

\draw[->,>=latex] (9,8) -- (coblig) ;
\draw[->,>=latex,purple] (9,7.5) -- (cpurple) ;
\draw[->,>=latex,green] (9,7) -- (cgreen) ;
\draw[->,>=latex,magenta] (9,6.5) -- (cmagenta) ;
\draw[->,>=latex,cyan] (9,6) -- (ccyan) ;
\draw[->,>=latex,red] (9,5.5) -- (cred) ;
\draw[->,>=latex,orange] (9,5) -- (corange) ;

\draw (a0) -- (0.75,0);
\draw[->,>=latex] (gradpsi0) -- (pba) ;
\draw[->,>=latex,purple] (0.75,0) -| (pbpsi0lp) ;
\draw[->,>=latex,purple] (pbpsi0lp) -- (psi0) ;
\draw[->,>=latex,purple] (psi0) -- (pbgradpsi0) ;
\draw[->,>=latex,purple] (pbgradpsi0) -- (gradpsi0) ;
\draw[->,>=latex,green] (0.75,0) -| (pbpsi0div) ;
\draw[->,>=latex,green] (pbpsi0div) -- (gradpsi0) ;

\draw (a1) -- (5.5,1);
\draw[->,>=latex,magenta] (5.5,1) -| (pbbcurl) ;
\draw[->,>=latex,magenta] (pbbcurl) -- (b) ;
\draw[cyan] (5.5,1) -| (8.25,0);
\draw[->,>=latex,orange] (8.25,0) -| (pbpsi1div);
\draw[->,>=latex,red] (8.25,0) -| (pbpsi1lp);
\draw[->,>=latex,red] (pbpsi1lp) -- (psi1);
\draw[->,>=latex,red] (psi1) -- (pbgradpsi1);
\draw[->,>=latex,red] (pbgradpsi1) -- (gradpsi1);
\draw[->,>=latex,orange] (pbpsi1div) -- (gradpsi1);
\draw[->,>=latex,cyan] (gradpsi1) -- (pbb);
\draw[->,>=latex,cyan] (psi) -- (pbb);
\draw[->,>=latex,cyan] (lambdagi) to[out=180,in=40] (pbb.15);
\draw[->,>=latex,cyan] (pbb) -- (b);
\draw[->,>=latex] (b) -- (pba);

\draw[->,>=latex] (pba) -- (a);
\draw[->,>=latex] (pbeigen) -- (lambdagi);
\draw[->,>=latex] (lambdagi) -- (pbgi0);
\draw[->,>=latex] (pbgi0) -- (gi0);
\draw[->,>=latex] (gi0) -- (pbpsi);
\draw[->,>=latex] (pbpsi) -- (psi);
\draw[->,>=latex] (a) -- (pbs);
\draw[->,>=latex] (f) -- (pbs);
\draw[->,>=latex] (a2) -- (pbs);
\draw[->,>=latex] (ck0) -- (pbs);
\draw[->,>=latex] (lambdagi) to[out=-10,in=20] (pbs.east);
\draw[->,>=latex] (psi) -- (pbs);
\draw[->,>=latex] (pbs) -- (ck);
\draw[->,>=latex] (ck) -- (pbu);
\draw[->,>=latex] (pbu) -- (u);
\draw[->,>=latex] (u) -- (pbv);
\draw[->,>=latex] (a) -- (pbv);
\draw[->,>=latex] (pbv) -- (v);
\draw[->,>=latex] (v) -- (pbq);
\draw[->,>=latex] (pbq) -- (q);
\end{tikzpicture}
\caption{Organigramme présentant les différents chemins possibles}
\label{org3}
\end{figure}

%% \begin{figure}
%% \centering
%% \begin{tikzpicture}[scale=\taille]
%% \node[draw,scale=\taille,fill=green!50] (di) at (12,5) {Données initiales} ;
%% \node[draw,scale=\taille,fill=gray!50] (pb) at (12,4) {Problèmes à résoudre} ;
%% \node[draw,scale=\taille,fill=blue!50] (si) at (12,3) {Solutions intermédiaires} ;
%% \node[draw,scale=\taille,fill=yellow!50] (sim) at (12,2) {Sol. inter. dép. de la géométrie} ;
%% \node[draw,scale=\taille,fill=red!50] (sf) at (12,1) {Solutions finales} ;
%% \node[scale=\taille,text width=10cm] (ccyan) at (5,4) {{\color{cyan} Chemins pour $\psi^0\in H^1$}} ;
%% \node[scale=\taille,text width=10cm] (cmagenta) at (5,3.5) {{\color{magenta} Chemins pour $\psi^0\in H(\mathrm{div})$}} ;

%% \node[draw,scale=\taille,fill=green!50] (a0) at (0.75,2) {$\alpha_0$} ;
%% \node[draw,scale=\taille,fill=gray!50,label={[xshift=-0.7cm](\ref{fvpsi0})}] (pbpsi0lp) at (-0.5,-2)
%% {$\begin{aligned}
%% -\laplace\psi^0&=0\\
%% \grad\psi^0\cdot \mathbf{n} &= \alpha_0
%% \end{aligned}$} ;
%% \node[draw,scale=\taille,fill=blue!50] (psi0) at (-0.5,-3.5) {$\psi^1$} ;
%% \node[draw,scale=\taille,fill=gray!50] (pbgradpsi0) at (-0.5,-4.75) {$w=\grad\psi^0$} ;
%% \node[draw,scale=\taille,fill=gray!50,label={[xshift=0.8cm](\ref{fvpsidiv})}] (pbpsi0div) at (2,-2)
%% {$\begin{aligned}
%% \mathbf{w}&=\grad\psi^0\\
%% \div\mathbf{w}&=0\\
%% \mathbf{w}\cdot \mathbf{n} &= \alpha_0
%% \end{aligned}$} ;
%% \node[draw,scale=\taille,fill=blue!50] (gradpsi0) at (0.75,-6.5) {$\grad\psi^0$} ;

%% \node[draw,scale=\taille,fill=green!50] (a1) at (5,2) {$\alpha_1$} ;
%% \node[draw,scale=\taille,fill=gray!50,label={[xshift=1.0cm,yshift=-0.1cm](\ref{fvbcurl})}] (pbbcurl) at (5,-2)
%% {$\begin{aligned}
%% \rott \mathbf{b} &= \grad\psi^1\\
%% \div \mathbf{b} &=0\\
%% \mathbf{b}\cdot \mathbf{n} &= 0\\
%% \rot \mathbf{b}\cdot \mathbf{n} &= 0\\
%% \grad \psi^1\cdot \mathbf{n} &= \alpha_1
%% \end{aligned}$} ;
%% \node[draw,scale=\taille,fill=blue!50] (b) at (5,-6.5) {$\rot \mathbf{b}$} ;

%% \node[draw,scale=\taille,fill=green!50] (a2) at (8,2) {$\alpha_2$} ;
%% \node[draw,scale=\taille,fill=gray!50] (pbe) at (8,-2)
%% {{\Huge ?}} ;
%% \node[draw,scale=\taille,fill=blue!50] (e) at (8,-6.5) {$\mathbf{e}$} ;

%% \node[draw,scale=\taille,fill=gray!50,label={[xshift=-1.3cm](\ref{a})}] (pba) at (3,-9) {$\mathbf{a} = \grad\psi^0 + \rot \mathbf{b} + \mathbf{e}$} ;
%% \node[draw,scale=\taillem,fill=blue!50] (a) at (3,-12) {$\mathbf{a}$} ;

%% \node[draw,scale=\taille,fill=gray!50,label={[xshift=1.0cm](\ref{fveigen})}] (pbeigen) at (11.5,-2)
%% {$\begin{aligned}
%% \rott \mathbf{g}_i = \lambda_i^2\mathbf{g}_i\\
%% \div\mathbf{g}_i = 0\\
%% \mathbf{g}_i\cdot \mathbf{n}\restr = 0\\
%% \rot \mathbf{g}_i\cdot \mathbf{n}\restr = 0\\
%% (\rott \mathbf{g}_i\cdot \mathbf{n}\restr = 0)
%% \end{aligned}$} ;
%% \node[draw,scale=\taille,fill=yellow!50] (lambdagi) at (11.5,-6.5) {$(\lambda_i^2,\mathbf{g}_i)$} ;

%% \node[draw,scale=\taille,fill=green!50] (f) at (8,-9) {$f$} ;
%% \node[draw,scale=\taille,fill=green!50] (ck0) at (9,-9) {$c_k^0$} ;
%% \node[draw,scale=\taille,fill=gray!50,label={[xshift=3.2cm](\ref{fvc})}] (pbs) at (9,-12)
%% {$\begin{aligned}
%% \frac{\partial c_k}{\partial t} &+ \sum_i\sum_j c_i\lambda_i c_j(\mathbf{g}_i\times \mathbf{g_j}, \mathbf{g_k}) \\
%% &+ \sum_i c_i\lambda_i(\mathbf{g}_i\times \mathbf{a},\mathbf{g_k}) + \sum_i c_i((\rot \mathbf{a})\times \mathbf{g}_i, \mathbf{g_k}) \\
%% &+ \frac{1}{Re}c_k\lambda_k^2 = (\mathbf{f_a},\mathbf{g_k})
%% \end{aligned}
%% $} ;
%% \node[draw,scale=\taille,fill=blue!50] (ck) at (9,-15) {$c_k$} ;
%% \node[draw,scale=\taille,fill=gray!50,label={[xshift=0.7cm](\ref{u})}] (pbu) at (9,-16) {$\mathbf{u}=\sum c_kg_k$} ;
%% \node[draw,scale=\taillem,fill=blue!50] (u) at (9,-17) {$\mathbf{u}$} ;
%% \node[draw,scale=\taille,fill=gray!50,label={[xshift=0.6cm](\ref{v})}] (pbv) at (3,-18) {$\mathbf{v}=\mathbf{a}+\mathbf{u}$} ;
%% \node[draw,scale=\tailleg,fill=red!50] (v) at (3,-19) {$\mathbf{v}$} ;
%% \node[draw,scale=\taille,fill=gray!50,label={[xshift=3.1cm](\ref{fvq})}] (pbq) at (9,-19)
%% {$\begin{aligned}
%% -\laplace q = \div((\rot \mathbf{v})\times \mathbf{v}) - \div \mathbf{f}\\
%% \grad q\cdot \mathbf{n}\restr =  \mathbf{f}\cdot \mathbf{n}\restr - \frac{\partial\alpha_0}{\partial t} - ((\rot \mathbf{v})\times \mathbf{v})\cdot \mathbf{n}\restr - \frac{\alpha_2}{Re}
%% \end{aligned}$} ;
%% \node[draw,scale=\tailleg,fill=red!50] (q) at(15,-19) {$p$} ;

%% \draw[->,>=latex,cyan] (-1,4) -- (ccyan) ; \draw[->,>=latex,magenta] (-1,3.5) -- (cmagenta) ;
%% \draw (a0) -- (0.75,0); \draw[->,>=latex] (gradpsi0) -- (pba) ; \draw[->,>=latex,cyan] (0.75,0) -| (pbpsi0lp) ; \draw[->,>=latex,cyan] (pbpsi0lp) -- (psi0) ; \draw[->,>=latex,cyan] (psi0) -- (pbgradpsi0) ; \draw[->,>=latex,cyan] (pbgradpsi0) -- (gradpsi0) ; \draw[->,>=latex,magenta] (0.75,0) -| (pbpsi0div) ; \draw[->,>=latex,magenta] (pbpsi0div) -- (gradpsi0) ; \draw[->,>=latex] (gradpsi0) -- (pba) ;
%% \draw[->,>=latex] (a1) -- (pbbcurl) ; \draw[->,>=latex] (pbbcurl) -- (b) ; \draw[->,>=latex] (b) -- (pba) ;
%% \draw[->,>=latex] (a2) -- (pbe) ; \draw[->,>=latex] (pbe) -- (e) ; \draw[->,>=latex] (e) -- (pba) ;
%% \draw[->,>=latex] (pba) -- (a);
%% \draw[->,>=latex] (pbeigen) -- (lambdagi); \draw[->,>=latex] (lambdagi) -- (pbs) ;
%% \draw[->,>=latex] (a) -- (pbs); \draw[->,>=latex] (f) -- (pbs); \draw[->,>=latex] (ck0) -- (pbs);
%% \draw[->,>=latex] (pbs) -- (ck); \draw[->,>=latex] (ck) -- (pbu); \draw[->,>=latex] (pbu) -- (u); \draw[->,>=latex] (u) -- (pbv); \draw[->,>=latex] (a) -- (pbv); \draw[->,>=latex] (pbv) -- (v); \draw[->,>=latex] (v) -- (pbq); \draw[->,>=latex] (pbq) -- (q);
%% \end{tikzpicture}
%% \caption{Organigramme présentant les différents problèmes à résoudre}
%% \label{org3}
%% \end{figure}

%%% Local Variables:
%%% TeX-master: "../report.tex"
%%% eval: (flyspell-mode 1)
%%% ispell-local-dictionary: "french"
%%% End:
