\part{Problème}
\label{partProb}
Introduisons d'abord les notations utilisées par la suite :
\begin{align*}
gradient(v)&=(\partial_x v, \partial_y v, \partial_z v)=\grad v\\
gradient(\bm{v})&=\begin{pmatrix}
\partial_x \bm{v_x} & \partial_y \bm{v_x} & \partial_z \bm{v_x}\\
\partial_x \bm{v_y} & \partial_y \bm{v_y} & \partial_z \bm{v_y}\\
\partial_x \bm{v_z} & \partial_y \bm{v_z} & \partial_z \bm{v_z}
\end{pmatrix}=\grad\bm{v}\\
divergence(\mathbf{v})&=\frac{\partial v_x}{\partial x}+\frac{\partial v_y}{\partial y}+\frac{\partial v_z}{\partial z}=\div \mathbf{v}\\
rot(\mathbf{v})&=\begin{pmatrix}
\partial_y v_z - \partial_z v_y\\
\partial_z v_x - \partial_x v_z\\
\partial_x v_y - \partial_y v_x
\end{pmatrix}=\rot \mathbf{v}\\
rot(rot(\mathbf{v}))&=\rott \mathbf{v}\\
H^1(\Omega) &= \{\mathbf{v} \in L^2(\Omega)\ |\ \grad \mathbf{v}\in L^2(\Omega)\}\\
H^1_0(\Omega) &= \{\mathbf{v} \in H^1(\Omega)\ |\ \mathbf{v}\restr = 0\}\\
H(div) &= \{\mathbf{v} \in L^2(\Omega)\ |\ \div\mathbf{v} \in L^2(\Omega) \}\\
H(rot) &= \{\mathbf{v} \in L^2(\Omega)\ |\ \rot\mathbf{v} \in L^2(\Omega) \}\\
L^2_\sigma(\Omega) &= \{\mathbf{v} \in L^2(\Omega)\ |\ \div \mathbf{v} = 0\text{ et }\mathbf{v}\cdot \mathbf{n}\restr = 0 \}\\
D^1(\Omega) &=\{\bm{v}\in L^2(\Omega)\ |\ \grad\bm{v}\in L^2(\Omega),\ \div\bm{v}=0,\ \bm{v}\cdot\bm{n}\restr = 0,\ \rot\bm{v}\cdot\bm{n}\restr = 0\}\\
&= \{\mathbf{v} \in H^1(\Omega)\cap L^2_\sigma(\Omega)\ |\ (\rot \mathbf{v}\cdot \mathbf{n})\restr = 0  \}\\
&=\{\mathbf{v}=\mathbf{v_0}+\grad\phi\ |\ \mathbf{v_0}\in H^1_0(\Omega),\ -\laplace\phi=\div\bm{v_0},\ \grad\phi\cdot \mathbf{n}\restr = 0 \}
\end{align*}

Nous cherchons $(\mathbf{v},p)$, correspondant respectivement à la vitesse et à la pression, solutions de l'équation de Navier-Stokes incompressibles adimensionnalisées dans $Q_T=\Omega\times[0,T]$, où $\Omega$ est un ouvert de $\R^3$ et $\partial\Omega$ sa frontière, avec une condition initiale et des conditions aux limites d'imperméabilité généralisée :
\begin{equation}\label{depart}
\left\{\begin{aligned}
&\frac{\partial \mathbf{v}}{\partial t} + (\rot  \mathbf{v})\times \mathbf{v} + \grad q + \frac{1}{Re}\rott  \mathbf{v}-\mathbf{f} = 0\\
&\div \mathbf{v} = 0\\
&\mathbf{v}\big\rvert_{t=0} = \mathbf{v}_0\\
&\mathbf{v}\cdot \mathbf{n}\restr = \alpha_0\\
&(\rot  \mathbf{v})\cdot \mathbf{n}\restr = \alpha_1\\
&(\rott  \mathbf{v})\cdot \mathbf{n}\restr = \alpha_2
\end{aligned}\right.
\end{equation}
où $q = \frac{|\mathbf{v}|^2}{2}+p$.\\

On va maintenant expliquer la décomposition du problème en une suite de problème plus simple dans le chapitre \ref{local}, puis étudié plus en détail chacun de ces problèmes dans \ref{fv}.

\chapter{Problème local}
\label{local}

Afin d'utiliser les travaux de H. Bellout, J. Neustupa et P. Penel \cite{Penel2004} qui ont étudié ce type d'équations dans $D^1$, on cherche à écrire $\mathbf{v}=\mathbf{u}+\mathbf{a}$, tel que :\\
\begin{equation}\label{v}
\begin{array}{c|ccccc}
& \mathbf{v} & = & \mathbf{a} & + & \mathbf{u}\\ \hline
\div\star & 0 & & 0 & & 0\\ \hline
\star\cdot \mathbf{n}\restr & \alpha_0 & & \alpha_0 & & 0\\ \hline
\rot\star\cdot \mathbf{n}\restr & \alpha_1 & & \alpha_1 & & 0\\ \hline
\rott\star\cdot \mathbf{n}\restr & \alpha_1 & & \alpha_2 & & 0
\end{array}
\end{equation}
Ainsi, on a $\mathbf{u}\in [D^1(\Omega)]^3$ et $\mathbf{a}\in [L_\sigma^2(\Omega)]^3$ sert à relever le problème.

On veut $\mathbf{a}$ tel que :
\begin{equation}\label{a}
\left\{\begin{aligned}
&\mathbf{a}=\rot \mathbf{b}+\grad\psi^0+\bm{e}\\
&\div \mathbf{a} =0\\
&\mathbf{a}\cdot \mathbf{n}\restr = \alpha_0\\
&(\rot \mathbf{a})\cdot \mathbf{n}\restr = \alpha_1\\
&(\rott \bm{a})\cdot \bm{n}\restr = \alpha_2
\end{aligned}\right.
\end{equation}
En appliquant la divergence et les conditions aux bords à la première ligne du problème on obtient le tableau suivant :
\begin{center}
\begin{tabular}{c|ccccccc}
& $\mathbf{a}$ & = & $\grad\psi^0$ & + & $\rot \mathbf{b}$ & + & $\bm{e}$ \\ \hline
$\div\star$ & 0 & & $\laplace\psi^0$ & & 0 & & 0\\ \hline
$\star\cdot \mathbf{n}\restr$ & $\alpha_0$ & & $\alpha_0$ & & 0 & & 0\\ \hline
$\rot\star\cdot \mathbf{n}\restr$ & $\alpha_1$ & & 0 & & $\alpha_1$ & & 0\\ \hline
$\rott\star\cdot \mathbf{n}\restr$ & $\alpha_2$ & & 0 & & 0 & & $\alpha_2$
\end{tabular}
\end{center}
En utilisant la première et la deuxième ligne du tableau, on voit que $\psi^0$ est solution de l'équation :
\begin{equation}\label{psi0}
\left\{\begin{aligned}
&-\laplace\psi^0 = 0\\
&\grad\psi^0\cdot \mathbf{n}\restr=\alpha_0
\end{aligned}\right.
\end{equation}
Ce problème permet de trouver $\psi^0$ à une constante près, on va donc utiliser un multiplicateur de Lagrange pour ajouter une contrainte sur $\psi^0$, par exemple $\int_\Omega \psi^0 = 0$, cette constante est sans importance car on cherche le gradient de $\psi^0$.\\
Il y a plusieurs manières de résoudre le problème (\ref{psi0}), la solution choisie sera discutée plus loin.\\

On cherche $b$ tel que :
\begin{equation}\label{curlb}
\left\{\begin{aligned}
&\rott \mathbf{b} = \grad\psi^1\\
&\div \mathbf{b} = 0\\
&\mathbf{b}\cdot \mathbf{n}\restr = 0\\
&\rot \mathbf{b}\cdot \mathbf{n}\restr = 0\\
&\grad\psi^1\cdot \mathbf{n}\restr = \alpha_1
\end{aligned}\right.
\end{equation}

Il reste encore plusieurs questions concernant la manière de relever $\alpha_1$ et $\alpha_2$. En effet, écris de cette manière, $\bm{b},\bm{e}\in [D^1(\Omega)]^3$ et donc si $\bm{b}\cdot\bm{n}=0$ alors $\rott\bm{b}\cdot\bm{n}=0$ aussi, et de même pour $\bm{e}$. Ces problèmes sont donc encore ouverts.\\

Une fois $\grad\psi^0$, $\rot \mathbf{b}$ et $\bm{e}$ connus, on peut retrouver $\mathbf{a}$.\\

On cherche donc $\mathbf{u}=\mathbf{v}-\mathbf{a}$, et en utilisant la décomposition de Galerkin généralisée, on a :
\begin{equation}\label{u}
\mathbf{u}(t,\cdot) = \sum_{i=1}^{\infty} c_i(t)\mathbf{g_i}(\cdot)
\end{equation}

Comme $\mathbf{u}\in [D^1(\Omega)]^3=D(curl_{imperm})$, on peut choisir les fonctions de base
$\mathbf{g_i}$ comme étant les fonctions propres de l'opérateur rotationnel. Ces
fonctions sont les mêmes que celles de l'opérateur $\rott$.

En effet, soit $(\lambda_i,\mathbf{g_i})$ solutions de $\rot \mathbf{g_i} = \lambda_i\mathbf{g_i}$ et $(\Lambda_i,\mathbf{G_i})$ solutions de $\rott \mathbf{G_i} = \Lambda_i\mathbf{G_i}$. Alors :
\[ \rott \mathbf{g_i} = \rot(\rot \mathbf{g_i})=\rot(\lambda_i\mathbf{g_i})=\lambda^2\mathbf{g_i} \]
Par identification, on voit que $\mathbf{g_i}=\mathbf{G_i}$ et que $\lambda_i=\pm\sqrt\Lambda_i$. Pour n'avoir que des valeurs propres de multiplicité une, on ne garde que les valeurs propres positives, le signe étant porté par le coefficient $c_i$.

On cherche donc $(\lambda_i,\mathbf{g_i})\in\R\times [D^1(\Omega)]^3$ solutions du problème suivant :
\begin{equation}\label{curlcurl}
\left\{\begin{aligned}
&\rott  \mathbf{g_i} = \lambda_i^2 \mathbf{g_i}\\
&\mathbf{g_i}\cdot \mathbf{n}\restr = 0\\
&\rot \mathbf{g_i}\cdot \mathbf{n}\restr = 0\\
&\rott  \mathbf{g_i}\cdot \mathbf{n}\restr = 0
\end{aligned}\right.
\end{equation}

On remplace maintenant $\mathbf{v}$ par $\mathbf{u}+\mathbf{a}$ dans (\ref{depart}) :
\[ \frac{\partial(\mathbf{u}+\mathbf{a})}{\partial t}+(\rot(\mathbf{u}+\mathbf{a}))\times(\mathbf{u}+\mathbf{a}) + \grad (\frac{|\mathbf{u}+\mathbf{a}|^2}{2}+p) + \frac{1}{Re}\rott(\mathbf{u}+\mathbf{a}) - \mathbf{f} = 0 \]
Ce qui donne en notant $\pi_a=\frac{|\mathbf{u}+\mathbf{a}|^2}{2}+p$ :
\[ \frac{\partial \mathbf{u}}{\partial t}+\frac{\partial \mathbf{a}}{\partial t} + (\rot \mathbf{u}+\rot \mathbf{a})\times(\mathbf{u}+\mathbf{a}) + \grad\pi_{\mathbf{a}} + \frac{1}{Re}(\rott \mathbf{u}+\rott \mathbf{a}) - \mathbf{f} = 0 \]
Comme $\rott \mathbf{a} = 0$ et en notant $\mathbf{h_a}=\mathbf{f}-\frac{\partial \mathbf{a}}{\partial t} - (\rot \mathbf{a})\times \mathbf{a}$, on a le problème suivant :
\begin{equation}\label{pbu}
\left\{\begin{aligned}
&\frac{\partial \mathbf{u}}{\partial t} + (\rot \mathbf{u})\times \mathbf{u} + (\rot \mathbf{u})\times \mathbf{a} +(\rot \mathbf{a})\times \mathbf{u} + \grad \pi_{\mathbf{a}} +\frac{1}{Re}\rott  \mathbf{u} - \mathbf{h_a} = 0\\
&\div \mathbf{u} = 0\\
&\mathbf{u}\big\rvert_{t=0} = \mathbf{v}_0 - \mathbf{a}(0,\cdot)\\
&\mathbf{u}\cdot \mathbf{n}\restr = 0\\
&(\rot \mathbf{u})\cdot \mathbf{n}\restr = 0\\
&(\rott  \mathbf{u})\cdot \mathbf{n}\restr = \alpha_2
\end{aligned}\right.
\end{equation}

Pour résumer, on doit donc :
\begin{enumerate}
\item générer la base $\mathbf{g_i}$ des fonctions propres de l'opérateur curl en résolvant l'équation (\ref{curlcurl}) comme expliqué dans le chapitre \ref{eigen}.
\item trouver $\mathbf{a}$ pour pouvoir décomposer $\mathbf{v}$ en $\mathbf{u}+\mathbf{a}$, pour cela, on résout les équations (\ref{psi0}) et (\ref{curlb}) et un système encore à définir pour $\bm{e}$. Ce qui permet de trouver $\mathbf{a}$ grâce à (\ref{a}). Cette partie est détaillé dans \ref{relev}.
\item résoudre l'équation (\ref{pbu}) pour trouver les coefficients $c_i$. Cela est expliqué dans \ref{spectre}.
\item recomposer $\mathbf{v}=\mathbf{u}+\mathbf{a}$, et chercher $p$ pour avoir la solution du problème (\ref{depart}). Cette dernière partie est montré dans \ref{pression}.
\end{enumerate}

\chapter{Forme variationnelle}
\label{fv}
\section{Problème aux valeurs propres}
\label{eigen}

On s'intéresse ici plus particulièrement au problème (\ref{curlcurl}).
\begin{equation}\label{pbeigen}
(\Lambda_i,\mathbf{g_i})\in\R\times [D^1(\Omega)]^3\quad \left\{\begin{aligned}
&\rott  \mathbf{g_i} = \Lambda_i \mathbf{g_i}\\
&\mathbf{g_i}\cdot \mathbf{n}\restr = 0\\
&\rot \mathbf{g_i}\cdot \mathbf{n}\restr = 0\\
&\rott  \mathbf{g_i}\cdot \mathbf{n}\restr = 0
\end{aligned}\right.
\end{equation}
On note d'abord que tout élément $\bm{\varphi}\in [D^1(\Omega)]^3$ peut s'écrire de la manière suivante :
\[ \bm{\varphi} = \bm{\varphi}_0 + \grad\phi\text{ et } \bm{\varphi}\restr = \grad\phi \]

On va maintenant chercher la formulation variationnelle de ce problème.\\
Soit $\mathbf{g}\in [D^1(\Omega)]^3$ solution de (6), alors pour tout $\bm{\varphi}\in [D^1(\Omega)]^3$ nous avons :
\[ \int_\Omega (\rott \mathbf{g})\cdot\bm{\varphi}\ dX = \int_\Omega\Lambda \mathbf{g}\cdot\bm{\varphi}\ dX \]
puis en intégrant par partie :
\[ \int_\Omega (\rot \mathbf{g})\cdot(\rot\bm{\varphi})\ dX + \int_{\partial\Omega} ((\rot \mathbf{g})\times \bm{\varphi})\cdot \mathbf{n}\ d\Gamma = \Lambda\int_\Omega \mathbf{g}\cdot\bm{\varphi}\ dX \]
or sur $\partial\Omega,\quad \bm{\varphi}\restr=\grad\phi$, d'où :
\[ \int_\Omega (\rot \mathbf{g})\cdot(\rot\bm{\varphi})\ dX + \int_{\partial\Omega} ((\rot \mathbf{g})\times \grad\phi)\cdot \mathbf{n}\ d\Gamma = \Lambda\int_\Omega \mathbf{g}\cdot\bm{\varphi}\ dX \]
En utilisant le théorème de flux-divergence aussi appelé théorème de Green-Ostrogradski :
\[ \int_\Omega (\rot \mathbf{g})\cdot(\rot\bm{\varphi})\ dX + \int_\Omega \div((\rot \mathbf{g})\times \grad\phi)\ dX = \Lambda\int_\Omega \mathbf{g}\cdot\bm{\varphi}\ dX \]
En utilisant la formule $\div(\mathbf{F}\times \mathbf{G}) = \mathbf{G}\cdot \rot \mathbf{F} - \mathbf{F}\cdot \rot \mathbf{G}$, on a :
\[ \int_\Omega (\rot \mathbf{g})\cdot(\rot\bm{\varphi})\ dX + \int_\Omega \grad\phi\cdot(\rott \mathbf{g})\ dX -\int_\Omega (\rot \mathbf{g})\cdot (\rot\grad\phi)\ dX  = \Lambda\int_\Omega \mathbf{g}\cdot\bm{\varphi}\ dX \]
Comme le rotationnel d'un gradient est nul, on a :
\[ \int_\Omega (\rot \mathbf{g})\cdot(\rot\bm{\varphi})\ dX + \int_\Omega \grad\phi\cdot(\rott \mathbf{g})\ dX  = \Lambda\int_\Omega \mathbf{g}\cdot\bm{\varphi}\ dX \]
En intégrant le deuxième terme par partie, on obtient :
\[ \int_\Omega (\rot \mathbf{g})\cdot(\rot\bm{\varphi})\ dX + \int_{\partial\Omega} \phi((\rott \mathbf{g})\cdot \mathbf{n})\ d\Gamma - \int_\Omega \phi(\div(\rott \mathbf{g}))\ dX  = \Lambda\int_\Omega \mathbf{g}\cdot\bm{\varphi}\ dX \]
Comme $\rott  \mathbf{g_i}\cdot \mathbf{n}\restr = 0$, le deuxième terme s'annule et comme la divergence d'un rotationnel est nulle, le troisième terme s'annule aussi, ce qui laisse pour tous $\bm{\varphi}\in [D^1(\Omega)]^3$ :
\begin{equation}\label{fveigen}
\int_\Omega (\rot \mathbf{g})\cdot(\rot\bm{\varphi})\ dX = \Lambda\int_\Omega \mathbf{g}\cdot\bm{\varphi}\ dX
\end{equation}

On obtient donc $(\Lambda_i,\mathbf{g_i})$, où $(\mathbf{g_i})$ forme une base de $[D^1(\Omega)]^3$.
\iffalse
\section{Décomposition des $\mathbf{g_i}$}
\label{decomp}

Comme énoncé plus tôt, tout élément de $D^1$ peut s'écrire $\bm{\varphi} = \bm{\varphi}_0 + \grad\phi$, y compris bien sûr les $\mathbf{g_i}$. Comme on va en avoir besoin pour les problèmes suivants, on va les décomposer en $\mathbf{g_i}=\mathbf{g_i^0}+\grad\psi_i$ avec $\mathbf{g_i^0}\restr = 0$ et $\grad\psi_i\cdot \mathbf{n}\restr = 0$.\\
On applique donc le rotationnel du rotationnel sur cette relation.\\
\[ \rott \mathbf{g_i^0} +\rott\grad\psi_i = \rott \mathbf{g_i} \]
Le dernier terme est nul car c'est le rotationnel d'un gradient. On utilise la formule $\rott \mathbf{v}=\grad(\div \mathbf{v})-\laplace \mathbf{v}$ sur le premier terme :
\[ \grad(\div \mathbf{g_i^0})-\laplace \mathbf{g_i^0} = \Lambda_i \mathbf{g_i} \]
On obtient donc le tableau suivant :
\begin{center}
\begin{tabular}{c|ccccc}
& $\mathbf{g_i}$ & = & $\mathbf{g_i^0}$ & + & $\grad\psi_i$ \\ \hline
$\rott\star$ & $\Lambda_i\mathbf{g_i}$ & & $\grad(\div \mathbf{g_i^0})-\laplace \mathbf{g_i^0}$ & & 0\\ \hline
$\div\star$ & 0 & & $\div \mathbf{g_i^0}$ & & $\laplace\psi_i$\\ \hline
$\star\cdot \mathbf{n}\restr$ & 0 & & 0 & & 0
\end{tabular}
\end{center}

\subsection{$g_i^0$}
En utilisant la première ligne, on parvient au problème :
\begin{equation}\label{gi0}
\left\{\begin{aligned}
\grad(\div \mathbf{g_i^0})-\laplace \mathbf{g_i^0} &= \Lambda_i\mathbf{g_i}\\
\mathbf{g_i^0}\restr &= 0
\end{aligned}\right.
\end{equation}
On cherche $\mathbf{g_i^0}$ dans $[H^1_0(\Omega)]^3$. On multiplie donc cette équation par une fonction test de $[H^1_0(\Omega)]^3$ et on intègre :
\[ \int_\Omega \grad(\div \mathbf{g_i^0})\cdot\bm{\varphi} - \int_\Omega \laplace \mathbf{g_i^0}\cdot\bm{\varphi} = \int_\Omega \Lambda_i\mathbf{g_i}\cdot\bm{\varphi} \]
On utilise ensuite la formule d'intégration par partie $\int_\Omega \grad{\mathbf{u}}\bm{\varphi} = -\int_\Omega \mathbf{u}\div\bm{\varphi} + \int_{\partial\Omega} \mathbf{u}\bm{\varphi}\cdot \mathbf{n}$ sur le premier terme :
\[ -\int_\Omega (\div \mathbf{g_i^0})(\div\bm{\varphi}) + \int_{\partial\Omega} (\div \mathbf{g_i^0})(\bm{\varphi}\cdot \mathbf{n}) - \int_\Omega \laplace \mathbf{g_i^0}\cdot\bm{\varphi} = \int_\Omega \Lambda_i\mathbf{g_i}\cdot\bm{\varphi} \]
Comme $\bm{\varphi}\in [H^1_0(\Omega)]^3$, la seconde intégrale est nul. On intègre par partie le terme en laplacien :
\[ -\int_\Omega (\div \mathbf{g_i^0})(\div\bm{\varphi}) + \int_\Omega \overline{\grad \mathbf{g_i^0}}:\overline{\grad\bm{\varphi}} - \int_{\partial\Omega} (\overline{\grad \mathbf{g_i^0}}\cdot \mathbf{n})\cdot\bm{\varphi} = \int_\Omega \Lambda_i\mathbf{g_i}\cdot\bm{\varphi} \]
Encore une fois, comme $\bm{\varphi}\in [H^1_0(\Omega)]^3$, le terme sur les bords s'annule. On obtient donc la forme variationnelle suivante :
\begin{equation}\label{fvgi0}
-\int_\Omega (\div \mathbf{g_i^0})(\div\bm{\varphi}) + \int_\Omega \overline{\grad \mathbf{g_i^0}}:\overline{\grad\bm{\varphi}} = \int_\Omega \Lambda_i\mathbf{g_i}\cdot\bm{\varphi}
\end{equation}

\subsection{Gradient $\psi_i$}
\label{multLagrange}

D'autre part, les deux dernières lignes du tableau nous donnent le problème de Poisson suivant :
\begin{equation}\label{psi}
\left\{\begin{aligned}
-\laplace\psi_i &= \div \mathbf{g_i^0}\\
\grad\psi_i\cdot \mathbf{n}\restr &= 0
\end{aligned}\right.
\end{equation}
Cette fois-ci, on cherche $\psi_i$ dans $[H^1(\Omega)]^1$. On a donc la forme variationnelle suivante :
\begin{equation}\label{fvpsi}
\int_\Omega \grad\psi_i\cdot\grad\varphi = \int_\Omega (\div \mathbf{g_i^0})\varphi
\end{equation}

Ce problème permet de trouver $\psi_i$ seulement à une constante près, on va donc devoir imposer une constante de notre choix, par exemple pour que $\int_\Omega \psi_i = 0$. Ceci va donc créer une translation dans le résultat, qu'il va falloir corriger en post-traitement.\\
Afin d'appliquer cette contrainte supplémentaire, on va utiliser la méthode des multiplicateurs de Lagrange.\\
Si l'on note $V=[H^1(\Omega)]^1$, $a(u,v)=\int \grad u \cdot \grad v$, $l(v)=\int (\div \mathbf{g_i^0})v$ et $J(v)=\frac{1}{2}a(v,v)-l(v)$, alors résoudre l'équation \ref{fvpsi} revient à trouver $u$ tel que :
\[ J(u) = \min_{v\in V} J(v) \]
Si l'on ajoute la contrainte $b(v) = \int v = 0$, alors, avec $\lambda$ un multiplicateur de Lagrange, le problème devient trouver $u$ tel que :
\[ J(u) = \min_{v\in V} J(v) - \lambda b(v) \]
Soit, en ajoutant l'équation de la contrainte multipliée par le multiplicateur de Lagrange correspondant à $\varphi$, on doit trouver $(\psi_i,\lambda)\in H^1(\Omega)\times L^2(\Omega)$ tel que $\forall (\varphi,\mu)$ :
\begin{align}\label{fvpsiml}
a(\psi_i,\varphi) + \lambda b(v) + \mu b(u) &= l(\varphi) \notag \\
\int_\Omega \grad\psi_i\cdot\grad\varphi + \int_\Omega \lambda\varphi + \int_\Omega \psi_i\mu &= \int_\Omega (\div \mathbf{g_i^0})\varphi
\end{align}
\fi
\section{Relèvement}
\label{relev}

\subsection{Gradient dans $H^1$}
\label{secpsi0hdiv} \label{multLagrange}

On cherche maintenant $\mathbf{a}$, on va donc s'intéresser d'abord à $\grad\psi^0$.\\
Il y a plusieurs alternatives pour résoudre (\ref{psi0}), on peut tout d'abord se placer dans $[H^1(\Omega)]^1$ et de résoudre :
\begin{equation}\label{pbpsi0}
\left\{\begin{aligned}
&-\laplace\psi^0 = 0\\
&\grad\psi^0\cdot \mathbf{n}\restr=\alpha_0
\end{aligned}\right.
\end{equation}

Il faut faire attention au fait que $\int \alpha_0$ doit être égale à 0, en effet, on a :
\[ 0=\int_\Omega \laplace \psi^0 = \int_\Omega \div(\grad\psi^0) = \int_{\partial\Omega} \grad\psi^0\cdot \mathbf{n} = \int_{\partial\Omega} \alpha_0 \]

Pour obtenir sa forme variationnelle, on multiplie par une fonction test $\varphi\in [H^1(\Omega)]^1$ et on intègre :
\[ \int_\Omega \laplace\psi^0 \varphi = 0 \]
On utilise ensuite la formule de Green pour parvenir à :
\[ -\int_\Omega \grad\psi^0\cdot\grad\varphi + \int_{\partial\Omega} \grad\psi^0\cdot \mathbf{n}\varphi = 0 \]
Or, $\grad\psi^0\cdot \mathbf{n} = \alpha_0$ sur $\partial\Omega$, on obtient donc la forme variationnelle suivante :
\begin{equation}\label{fvpsi0LM} -\int_\Omega \grad\psi^0\cdot\grad\varphi + \int_{\partial\Omega} \alpha_0\varphi = 0
\end{equation}

Comme énoncé précédemment, on va devoir utiliser les multiplicateurs de Lagrange pour ajouter la contrainte $\int \psi^0=0$.
Si l'on note $V=[H^1(\Omega)]^1$, $a(u,v)=\int \grad u \cdot \grad v$, $l(v)=\int \alpha_0v$ et $J(v)=\frac{1}{2}a(v,v)-l(v)$, alors résoudre l'équation \ref{fvpsi0LM} revient à trouver $u$ tel que :
\[ J(u) = \min_{v\in V} J(v) \]
Si l'on ajoute la contrainte $b(v) = \int v = 0$, alors, avec $\lambda$ un multiplicateur de Lagrange, le problème devient trouver $u$ tel que :
\[ J(u) = \min_{v\in V} J(v) - \lambda b(v) \]
Soit, en ajoutant l'équation de la contrainte multipliée par le multiplicateur de Lagrange correspondant à $\varphi$, et le terme correspondant à la moyenne $m=0$ de $\psi^0$, on doit trouver $(\psi^0,\lambda)\in H^1(\Omega)\times L^2(\Omega)$ tel que $\forall (\varphi,\mu)$ :
\begin{align}\label{fvpsi0}
a(\psi_i,\varphi) + \lambda b(v) + \mu b(u) &= l(\varphi) + m b(\mu) \notag \\
\int_\Omega \grad\psi^0\cdot\grad\varphi + \int_\Omega \lambda\varphi + \int_\Omega \psi^0\mu &= \int_\Omega \alpha_0\varphi
\end{align}

\subsection{Gradient dans $H(div)$}

L'autre possibilité pour trouver $\psi^0$ est de chercher $(\mathbf{w},\psi^0)\in [H(div)]^3\times [L^2(\Omega)]^1$ solution du problème de Darcy suivant :
\begin{equation}\label{pbpsidiv}
\left\{\begin{aligned}
\mathbf{w} &= \grad \psi^0\\
\div \mathbf{w} &= 0\\
\mathbf{w}\cdot \mathbf{n}\restr &= \alpha_0
\end{aligned}\right.
\end{equation}
Pour obtenir la formulation faible du problème, on multiplie par une fonction test $(\bm{\varphi},\nu)\in [H(div)]^3\times [L^2(\Omega)]^1$ et on intègre les deux premières équations :
\begin{align*}
\int_\Omega \mathbf{w}\cdot\bm{\varphi} &= \int_\Omega \grad\psi^0\cdot\bm{\varphi}\\
\int_\Omega \div \mathbf{w}\ \nu &= 0
\end{align*}
On intègre par partie la deuxième équation :
\[ \int_\Omega \div \mathbf{w}\ \nu = \int_{\partial\Omega} \mathbf{w}\cdot \mathbf{n}\ q - \int_\Omega \mathbf{w}\cdot\grad\nu = 0  \]
En insérant la condition au bord et la première équation, on obtient la formulation faible :
\[ -\int_\Omega \mathbf{w}\cdot\bm{\varphi} + \int_\Omega \mathbf{w}\cdot\grad\nu + \int_\Omega \grad\psi^0\cdot\bm{\varphi}  = \int_{\partial\Omega} \alpha_0\nu \]

 Ainsi, de la même manière que dans \ref{multLagrange}, on parvient à la formulation suivante :
\begin{equation}\label{fvpsidiv}
-\int_\Omega \mathbf{w}\cdot\bm{\varphi} + \int_\Omega \mathbf{w}\cdot\grad\nu + \int_\Omega \grad\psi^0\cdot\bm{\varphi} + \int_\Omega \lambda\varphi + \int_\Omega \psi^0\mu = \int_{\partial\Omega} \alpha_0\nu
\end{equation}

Calculer $\grad\psi^0$ dans $H(div)$ a l'avantage de faire gagner un ordre à la régularité de $\grad\psi^0$ par rapport à résoudre le problème dans $H^1$ pour trouver $\psi^0$ et ensuite calculer son gradient.\\

\subsection{Rotationnel dans $H(rot)$}

Pour le problème (\ref{curlb}), on veut résoudre dans $[H(rot)]^3$ le problème mixte suivant : 
\begin{equation}\label{pbbcurl}
\left\{\begin{aligned}
&\rott \mathbf{b} = \grad\psi^1\\
&\div \mathbf{b} = 0\\
&\mathbf{b}\cdot \mathbf{n}\restr = 0\\
&\rot \mathbf{b}\cdot \mathbf{n}\restr = 0\\
&\grad\psi^1\cdot \mathbf{n}\restr = \alpha_1
\end{aligned}\right.
\end{equation}

Pour avoir la formulation faible, on multiplie par une fonction test de $[H(rot)]^3$ et on intègre :
\[ \int_\Omega (\rott \mathbf{b})\cdot\bm{\varphi} = \int_\Omega (\grad\psi^1)\cdot\bm{\varphi} \]
En intégrant par partie le premier terme et en utilisant la formule de Green sur le second, on obtient :
\begin{equation} \label{fvbcurl}
\int_\Omega (\rot \mathbf{b})\cdot(\rot\bm{\varphi}) - \int_{\partial\Omega} (\rot \mathbf{b})(\bm{\varphi}\cdot \mathbf{n}) + \int_\Omega \psi^1(\div\bm{\varphi}) - \int_{\partial\Omega} \psi^1(\bm{\varphi}\cdot \mathbf{n}) = 0
\end{equation}

Comme on peut le voir dans la formulation, $\alpha_1$ n'apparaît pas, il faut donc chercher un autre problème. De plus, comme expliqué précédemment, $\bm{b}$ appartient à $D^1$, et ne peut donc pas satisfaire la condition $\rott\bm{b}\cdot \bm{n}$ :
\[ \rott\bm{b}\cdot \bm{n} = \rott\left(\sum h_i \bm{g_i}\right)\cdot\bm{n} = \sum h_i \rott \bm{g_i}\cdot\bm{n} = \sum h_i\lambda^2 \underbrace{\bm{g_i}\cdot\bm{n}}_{=0} = 0 \]

De même, il faut encore trouver un système permettant de relever la condition $\alpha_2$.

\section{Problème spectral}
\label{spectre}

On cherche maintenant $\mathbf{u}\in [D^1(\Omega)]^3$ solution de (\ref{pbu}), on veut donc obtenir la forme variationnelle du problème, pour cela, on multiplie par une fonction test $\bm{\varphi}\in [D^1(\Omega)]^3$ et on intègre :
\begin{align*}
\int_\Omega \frac{\partial \mathbf{u}}{\partial t}\cdot \bm{\varphi} &+ \int_\Omega ((\rot \mathbf{u})\times \mathbf{u})\cdot \bm{\varphi} + \int_\Omega ((\rot \mathbf{u})\times \mathbf{a})\cdot\bm{\varphi} + \int_\Omega ((\rot \mathbf{a})\times \mathbf{u})\cdot\bm{\varphi} \\
&+ \int_\Omega \grad\pi_{\mathbf{a}}\cdot\bm{\varphi} + \frac{1}{Re}\int_\Omega (\rott \mathbf{u})\cdot\bm{\varphi} = \int_\Omega \mathbf{h_a}\cdot\bm{\varphi}
\end{align*}
En utilisant une intégration par partie sur l'avant dernier terme et la même méthode que dans le chapitre \ref{eigen} pour le dernier terme du membre de gauche, on arrive à :
\begin{align*}
\int_\Omega \frac{\partial \mathbf{u}}{\partial t}\cdot \bm{\varphi} &+ \int_\Omega ((\rot \mathbf{u})\times \mathbf{u})\cdot \bm{\varphi} + \int_\Omega ((\rot \mathbf{u})\times \mathbf{a})\cdot\bm{\varphi} + \int_\Omega ((\rot \mathbf{a})\times \mathbf{u})\cdot\bm{\varphi}\\
&+ \int_\Omega \pi_{\mathbf{a}}(\div\bm{\varphi}) + \int_{\partial \Omega} \pi_{\mathbf{a}}(\bm{\varphi}\cdot \mathbf{n}) + \frac{1}{Re}\int_\Omega (\rot \mathbf{u})\cdot(\rot\bm{\varphi}) -\frac{1}{Re}\int_{\partial\Omega} ((\rott \mathbf{u})\cdot \mathbf{n})\phi = \int_\Omega \mathbf{h_a}\cdot\bm{\varphi}
\end{align*}
Où $\phi$ provient de la décomposition de la fonction de test en $\bm{\varphi}=\bm{\varphi}_0+\grad\phi$.\\
Or, $\bm{\varphi}\in [D^1(\Omega)]^3$ donc $\div\bm{\varphi}=0$ et $\bm{\varphi}\cdot \mathbf{n}=0$ sur $\partial\Omega$, le terme de pression s'annule donc sous cette forme. De plus, $\rott \mathbf{u}\cdot \mathbf{n}=0$ sur $\partial\Omega$. On a donc :
\begin{equation}\label{fvu}
\begin{aligned}
\int_\Omega \frac{\partial \mathbf{u}}{\partial t}\cdot \bm{\varphi} &+ \int_\Omega ((\rot \mathbf{u})\times \mathbf{u})\cdot \bm{\varphi} + \int_\Omega ((\rot \mathbf{u})\times \mathbf{a})\cdot\bm{\varphi} \\
&+ \int_\Omega ((\rot \mathbf{a})\times \mathbf{u})\cdot\bm{\varphi} + \frac{1}{Re}\int_\Omega (\rot \mathbf{u})\cdot(\rot\bm{\varphi}) = \int_\Omega \mathbf{h_a}\cdot\bm{\varphi}
\end{aligned}
\end{equation}

\label{discr}

On cherche maintenant à exprimer $\mathbf{u}$ dans la base $(\mathbf{g_i})$, c'est-à-dire $\mathbf{u}=\sum_i c_ig_i$.\\
De plus, prendre une fonction de test $\bm{\varphi}$ dans $[D^1(\Omega)]^3$ revient à prendre chaque fonction de la base que l'on a calculé dans le chapitre \ref{eigen}.\\
On rappelle que $\rot \mathbf{g_i}=\lambda_i\mathbf{g_i}$. En notant $\int_\Omega \mathbf{f}\cdot\mathbf{g}=(f,g)$, on obtient pour le problème (\ref{fvu}), trouver $c_i$, $i=0\dots M$ tel que $\forall k=0\dots M$ :
\begin{align*}
\frac{\partial}{\partial t}\sum_i (c_i\mathbf{g_i}, \mathbf{g_k}) &+ \sum_i\sum_j (c_i\lambda_ic_j\mathbf{g_i}\times \mathbf{g_j},\mathbf{g_k}) + \sum_i(c_i\lambda_i\mathbf{g_i}\times \mathbf{a},\mathbf{g_k})\\
&+ \sum_i ((\rot \mathbf{a})\times c_i\mathbf{g_i},\mathbf{g_k}) + \frac{1}{Re}\sum_i(c_i\lambda_i\mathbf{g_i}, \lambda_k\mathbf{g_k}) = (\mathbf{h_a},\mathbf{g_k})
\end{align*}
Comme les termes $c_i$ ne porte que sur la dimension temporelle, on peut les
sortir des intégrales, tout comme les $\lambda$.\\

Par ailleurs, on rappel que la base $(\mathbf{g_i})$ est orthonormale. On obtient donc :
\begin{equation}\label{fvspec}
\begin{aligned}
\frac{\partial c_k}{\partial t} &+ \sum_i\sum_j c_i\lambda_i c_j(\mathbf{g_i}\times \mathbf{g_j}, \mathbf{g_k}) + \sum_i c_i\lambda_i(\mathbf{g_i}\times \mathbf{a},\mathbf{g_k})\\
&+ \sum_i c_i((\rot \mathbf{a})\times \mathbf{g_i}, \mathbf{g_k}) + \frac{1}{Re}c_k\lambda_k^2 = (\mathbf{h_a},\mathbf{g_k})
\end{aligned}
\end{equation}

\section{Pression}
\label{pression}

Pour retrouver la vitesse $\mathbf{v}$, il suffit maintenant d'additionner $\mathbf{a}$ et $\mathbf{u}$.\\
Le terme correspondant à la pression ayant été relayé en post-traitement de la vitesse, il faut le recalculer à partir de l'équation (\ref{depart}).

On applique la divergence sur cette équation et cela nous donne :
\[ \frac{\partial}{\partial t}\div \mathbf{v} + \div((\rot \mathbf{v})\times \mathbf{v}) + \div\grad q + \frac{1}{Re}\div(\rott \mathbf{v}) - \div \mathbf{f} = 0 \]
Dans cette équation, la divergence de $\mathbf{v}$ et la divergence du rotationnel de $\mathbf{v}$ s'annulent. Il nous reste donc :
\begin{eqnarray}\label{q}
-\laplace q = \div((\rot \mathbf{v})\times \mathbf{v}) - \div \mathbf{f}
\end{eqnarray}

Pour obtenir une condition au bord, on utilise la composante normale de l'équation (\ref{depart}) :
\[ \frac{\partial}{\partial t}(\mathbf{v}\cdot \mathbf{n})\restr + ((\rot \mathbf{v})\times \mathbf{v})\cdot \mathbf{n}\restr + \grad q\cdot \mathbf{n}\restr +\frac{1}{Re}(\rott \mathbf{v})\cdot \mathbf{n}\restr - \mathbf{f}\cdot \mathbf{n}\restr = 0 \]
En utilisant les conditions aux bords de $\mathbf{v}$, on obtient :
\[ \grad q\cdot \mathbf{n}\restr =  \mathbf{f}\cdot \mathbf{n}\restr - \frac{\partial\alpha_0}{\partial t} - ((\rot \mathbf{v})\times \mathbf{v})\cdot \mathbf{n}\restr - \frac{\alpha_2}{Re} \]
On cherche maintenant la forme variationnelle du problème :
\[ \int_\Omega -\laplace q\varphi = \int_\Omega (\div((\rot \mathbf{v})\times \mathbf{v}) -\div \mathbf{f})\varphi \]
En intégrant par partie le terme de gauche, on a :
\[ \int_\Omega \grad q\grad\varphi - \int_{\partial\Omega} (\grad q\cdot \mathbf{n})\varphi = \int_\Omega (\div((\rot \mathbf{v})\times \mathbf{v}) -\div \mathbf{f})\varphi \]

Toujours de même manière, on va trouver la pression à une constante près, on utilise donc encore une fois les multiplicateur de Lagrange afin de fixer cette constante. Comme dans \ref{multLagrange}, on obtient donc au final :
\begin{equation}\label{fvq}
\begin{aligned}
\int_\Omega \grad q\grad\varphi + \int_\Omega \lambda\varphi + \int_\Omega q\nu &= \int_\Omega (\div((\rot \mathbf{v})\times \mathbf{v}) -\div \mathbf{f})\varphi\\
&+ \int_{\partial\Omega} \left(f\cdot \mathbf{n} - \frac{\partial\alpha_0}{\partial t} - ((\rot \mathbf{v})\times \mathbf{v})\cdot \mathbf{n} - \frac{\alpha_2}{Re}\right)\varphi
\end{aligned}
\end{equation}

\chapter{Récapitulatif}

Les différentes étapes pour résoudre le problème sont donc :
\begin{enumerate}
\item calculer les valeurs et fonctions propres de l'opérateur rotationnel avec (\ref{fveigen}),
\item trouver $\psi^0$, pour cela, on peut :
\begin{itemize}
\item résoudre (\ref{fvpsi0}) pour avoir $\psi^0\in H^1$,
\item résoudre (\ref{fvpsidiv}) pour avoir $\psi^0\in H(div)$,
\end{itemize}
\item résoudre (\ref{fvbcurl}) pour trouver $\rot\bm{b}$,
\item résoudre un système à déterminer pour trouver $\bm{e}$,
\item recomposer $\bm{a}$ grâce à $\psi^0$, $\rot\bm{b}$ et $\bm{e}$,
\item résoudre le problème spectral (\ref{fvspec}) afin de trouver les coefficients $c_i$ pour $i=0\dots M$,
\item reconstruire $\bm{u}=\sum c_i \bm{g_i}$ et $\bm{v}=\bm{a}+\bm{u}$,
\item calculer la pression en post-traitement avec (\ref{fvq}).
\end{enumerate}

On a ainsi trouver notre solution $(\mathbf{v},p)$.\\
L'étape une est très coûteuse en temps de calcul, mais elle ne dépend que de la géométrie, ainsi, on peut réutiliser les fonctions propres même si l'on change certains paramètres. On a donc à réaliser cette étape seulement une fois par géométrie.\\
Les autres étapes dépendants de différents paramètres, on devra de nouveau les exécuter si l'on change les paramètres, notamment l'étape 6, qui prend en compte tous les paramètres et qui est elle aussi coûteuse en temps de calcul.\\

La figure \ref{org3} présente graphiquement les problèmes à résoudre.\\

\begin{figure}
\centering
\begin{tikzpicture}[scale=\taille]
\node[draw,scale=\taille,fill=green!50] (di) at (12,5) {Données initiales} ;
\node[draw,scale=\taille,fill=gray!50] (pb) at (12,4) {Problèmes à résoudre} ;
\node[draw,scale=\taille,fill=blue!50] (si) at (12,3) {Solutions intermédiaires} ;
\node[draw,scale=\taille,fill=yellow!50] (sim) at (12,2) {Sol. inter. dép. de la géométrie} ;
\node[draw,scale=\taille,fill=red!50] (sf) at (12,1) {Solutions finales} ;
\node[scale=\taille,text width=10cm] (ccyan) at (5,4) {{\color{cyan} Chemins pour $\psi^0\in H^1$}} ;
\node[scale=\taille,text width=10cm] (cmagenta) at (5,3.5) {{\color{magenta} Chemins pour $\psi^0\in H(div)$}} ;

\node[draw,scale=\taille,fill=green!50] (a0) at (0.75,2) {$\alpha_0$} ;
\node[draw,scale=\taille,fill=gray!50,label={[xshift=-0.7cm](\ref{pbpsi0})}] (pbpsi0lp) at (-0.5,-2)
{$\begin{aligned}
-\laplace\psi^0&=0\\
\grad\psi^0\cdot \mathbf{n} &= \alpha_0
\end{aligned}$} ;
\node[draw,scale=\taille,fill=blue!50] (psi0) at (-0.5,-3.5) {$\psi^1$} ;
\node[draw,scale=\taille,fill=gray!50] (pbgradpsi0) at (-0.5,-4.75) {$w=\grad\psi^0$} ;
\node[draw,scale=\taille,fill=gray!50,label={[xshift=0.8cm](\ref{pbpsidiv})}] (pbpsi0div) at (2,-2)
{$\begin{aligned}
\mathbf{w}&=\grad\psi^0\\
\div\mathbf{w}&=0\\
\mathbf{w}\cdot \mathbf{n} &= \alpha_0
\end{aligned}$} ;
\node[draw,scale=\taille,fill=blue!50] (gradpsi0) at (0.75,-6.5) {$\grad\psi^0$} ;

\node[draw,scale=\taille,fill=green!50] (a1) at (5,2) {$\alpha_1$} ;
\node[draw,scale=\taille,fill=gray!50,label={[xshift=1.0cm,yshift=-0.1cm](\ref{pbbcurl})}] (pbbcurl) at (5,-2)
{$\begin{aligned}
\rott \mathbf{b} &= \grad\psi^1\\
\div \mathbf{b} &=0\\
\mathbf{b}\cdot \mathbf{n} &= 0\\
\rot \mathbf{b}\cdot \mathbf{n} &= 0\\
\grad \psi^1\cdot \mathbf{n} &= \alpha_1
\end{aligned}$} ;
\node[draw,scale=\taille,fill=blue!50] (b) at (5,-6.5) {$\rot \mathbf{b}$} ;

\node[draw,scale=\taille,fill=green!50] (a2) at (8,2) {$\alpha_2$} ;
\node[draw,scale=\taille,fill=gray!50] (pbe) at (8,-2)
{{\Huge ?}} ;
\node[draw,scale=\taille,fill=blue!50] (e) at (8,-6.5) {$\bm{e}$} ;

\node[draw,scale=\taille,fill=gray!50,label={[xshift=-1.3cm](\ref{a})}] (pba) at (3,-9) {$\mathbf{a} = \grad\psi^0 + \rot \mathbf{b} + \bm{e}$} ;
\node[draw,scale=\taillem,fill=blue!50] (a) at (3,-12) {$\mathbf{a}$} ;

\node[draw,scale=\taille,fill=gray!50,label={[xshift=1.0cm](\ref{pbeigen})}] (pbeigen) at (11.5,-2)
{$\begin{aligned}
\rott \mathbf{g_i} = \Lambda_i\mathbf{g_i}\\
\div\mathbf{g_i} = 0\\
\mathbf{g_i}\cdot \mathbf{n}\restr = 0\\
\rot \mathbf{g_i}\cdot \mathbf{n}\restr = 0\\
(\rott \mathbf{g_i}\cdot \mathbf{n}\restr = 0)
\end{aligned}$} ;
\node[draw,scale=\taille,fill=yellow!50] (lambdagi) at (11.5,-6.5) {$(\Lambda_i,\mathbf{g_i})$} ;

\node[draw,scale=\taille,fill=green!50] (f) at (8,-9) {$f$} ;
\node[draw,scale=\taille,fill=green!50] (ck0) at (9,-9) {$c_k^0$} ;
\node[draw,scale=\taille,fill=gray!50,label={[xshift=3.2cm](\ref{fvspec})}] (pbs) at (9,-12)
{$\begin{aligned}
\frac{\partial c_k}{\partial t} &+ \sum_i\sum_j c_i\lambda_i c_j(\mathbf{g_i}\times \mathbf{g_j}, \mathbf{g_k}) \\
&+ \sum_i c_i\lambda_i(\mathbf{g_i}\times \mathbf{a},\mathbf{g_k}) + \sum_i c_i((\rot \mathbf{a})\times \mathbf{g_i}, \mathbf{g_k}) \\
&+ \frac{1}{Re}c_k\lambda_k^2 = (\mathbf{h_a},\mathbf{g_k})
\end{aligned}
$} ;
\node[draw,scale=\taille,fill=blue!50] (ck) at (9,-15) {$c_k$} ;
\node[draw,scale=\taille,fill=gray!50,label={[xshift=0.7cm](\ref{u})}] (pbu) at (9,-16) {$\mathbf{u}=\sum c_kg_k$} ;
\node[draw,scale=\taillem,fill=blue!50] (u) at (9,-17) {$\mathbf{u}$} ;
\node[draw,scale=\taille,fill=gray!50,label={[xshift=0.6cm](\ref{v})}] (pbv) at (3,-18) {$\mathbf{v}=\mathbf{a}+\mathbf{u}$} ;
\node[draw,scale=\tailleg,fill=red!50] (v) at (3,-19) {$\mathbf{v}$} ;
\node[draw,scale=\taille,fill=gray!50,label={[xshift=3.1cm](\ref{q})}] (pbq) at (9,-19)
{$\begin{aligned}
-\laplace q = \div((\rot \mathbf{v})\times \mathbf{v}) - \div \mathbf{f}\\
\grad q\cdot \mathbf{n}\restr =  \mathbf{f}\cdot \mathbf{n}\restr - \frac{\partial\alpha_0}{\partial t} - ((\rot \mathbf{v})\times \mathbf{v})\cdot \mathbf{n}\restr - \frac{\alpha_2}{Re}
\end{aligned}$} ;
\node[draw,scale=\tailleg,fill=red!50] (q) at(15,-19) {$p$} ;

\draw[->,>=latex,cyan] (-1,4) -- (ccyan) ; \draw[->,>=latex,magenta] (-1,3.5) -- (cmagenta) ;
\draw (a0) -- (0.75,0); \draw[->,>=latex] (gradpsi0) -- (pba) ; \draw[->,>=latex,cyan] (0.75,0) -| (pbpsi0lp) ; \draw[->,>=latex,cyan] (pbpsi0lp) -- (psi0) ; \draw[->,>=latex,cyan] (psi0) -- (pbgradpsi0) ; \draw[->,>=latex,cyan] (pbgradpsi0) -- (gradpsi0) ; \draw[->,>=latex,magenta] (0.75,0) -| (pbpsi0div) ; \draw[->,>=latex,magenta] (pbpsi0div) -- (gradpsi0) ; \draw[->,>=latex] (gradpsi0) -- (pba) ;
\draw[->,>=latex] (a1) -- (pbbcurl) ; \draw[->,>=latex] (pbbcurl) -- (b) ; \draw[->,>=latex] (b) -- (pba) ;
\draw[->,>=latex] (a2) -- (pbe) ; \draw[->,>=latex] (pbe) -- (e) ; \draw[->,>=latex] (e) -- (pba) ;
\draw[->,>=latex] (pba) -- (a);
\draw[->,>=latex] (pbeigen) -- (lambdagi); \draw[->,>=latex] (lambdagi) -- (pbs) ;
\draw[->,>=latex] (a) -- (pbs); \draw[->,>=latex] (f) -- (pbs); \draw[->,>=latex] (ck0) -- (pbs);
\draw[->,>=latex] (pbs) -- (ck); \draw[->,>=latex] (ck) -- (pbu); \draw[->,>=latex] (pbu) -- (u); \draw[->,>=latex] (u) -- (pbv); \draw[->,>=latex] (a) -- (pbv); \draw[->,>=latex] (pbv) -- (v); \draw[->,>=latex] (v) -- (pbq); \draw[->,>=latex] (pbq) -- (q);
\end{tikzpicture}
\caption{Organigramme présentant les différents problèmes à résoudre}
\label{org3}
\end{figure}

%%% Local Variables:
%%% TeX-master: "../report.tex"
%%% eval: (flyspell-mode 1)
%%% ispell-local-dictionary: "french"
%%% End:
