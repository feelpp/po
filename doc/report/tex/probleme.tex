\section{Problème}
Notation : 
\begin{eqnarray*}
divergence(v)=\diverg v\\
curl(v)=\rot v\\
curl(curl(v))=\rott v
\end{eqnarray*}

Nous cherchons $(v,p)$, correspondant respectivement à la vitesse et à la pression, solutions de l'équation de Navier-Stokes incompressibles adimensionnalisées dans $Q_T=\Omega\times[0,T]$ un ouvert de $\R^3$ et $\partial\Omega$ sa frontière, avec une condition initiale et des conditions aux limites d'imperméabilité généralisée :
\begin{eqnarray}
\label{depart}
\left\{
\begin{aligned}
&\frac{\partial v}{\partial t} + (\rot  v)\times v + \grad q + \frac{1}{Re}\rott  v-f = 0\\
&\diverg v = 0\\
&v\big\rvert_{t=0} = v_0\\
&v\cdot n\restr = \alpha_0\\
&(\rot  v)\cdot n\restr = \alpha_1\\
&(\rott  v)\cdot n\restr = \alpha_2
\end{aligned}
\right.
\end{eqnarray}
où $q = \frac{|v|^2}{2}+p$.\\

Soit l'espace
\[
\LL^2_\sigma(\Omega) = \{v \in \LL^2(\Omega)_| \diverg v = 0\text{ et }v\cdot n\restr = 0 \}
\]
Alors, on cherche la solution dans l'espace
\[
D^1(\Omega) = \{v\in W^{1,2}(\Omega) \cup \LL^2(\Omega) | (\rot v\cdot n)\restr = 0  \}
\]

\subsection{Problème local}

Afin d'utiliser les travaux de O. Mezureau, J. Neustupa et P. Penel\footnote{\href{http://www.researchgate.net/publication/254219219\_A\_weak\_solvability\_to\_the\_steady\_NavierStokes\_equations\_for\_compressible\_barotropic\_fluid\_with\_generalized\_impermeability\_boundary\_conditions}{A weak solvability to the steady Navier-Stokes equations for compressible barotropic fluid with generalized impermeability boundary conditions}}, on souhaite se ramener à un problème où $\alpha_0=\alpha_1=0$.\\
On veut donc écrire $v=u+a$ avec $a$ pour relever le problème.
\[
\begin{array}{cccccc}
& v & = & a & + & u\\
\diverg\star & 0 & & 0 & & 0\\
\star\cdot n\restr & \alpha_0 & & \alpha_0 & & 0\\
\rot\star\cdot n\restr & \alpha_1 & & \alpha_1 & & 0\\
\rott\star\cdot n\restr & \alpha_1 & & 0 & & \alpha_2 
\end{array}
\]
On a donc $a$ tel que :
\begin{eqnarray}
\label{aegal}
\left\{
\begin{aligned}
&a=\rot b+\grad\psi_0\\
&\diverg a =0\\
&a\cdot n\restr = \alpha_0\\
&(\rot a)\cdot n\restr = \alpha_1
\end{aligned}
\right.
\end{eqnarray}
Alors $\psi_0$ est solution de l'équation :
\begin{eqnarray}
\label{psi0}
\left\{
\begin{aligned}
&-\laplace\psi_0 = 0\\
&\grad\psi_0\cdot n\restr=\alpha_0
\end{aligned}
\right.
\end{eqnarray}
On obtient alors $\psi_0$ à une constante près, cela nous permet néanmoins de calculer $\grad\psi_0$.\\
Et $b$ est solution du problème mixte :
\begin{eqnarray}
\label{curlb}
\left\{
\begin{aligned}
&\rott b = \grad\psi_1\\
&\diverg b = 0\\
&b\cdot n\restr = 0\\
&\rot b\cdot n\restr = 0\\
&\grad\psi_1\cdot n\restr = \alpha_1
\end{aligned}
\right.
\end{eqnarray}

Une fois $\grad\psi_0$ et $b$ connus, on peut calculer $\rot b$ et donc retrouver $a$.\\

On cherche donc $u=v-a$, et en utiliant la décomposition de Galerkin généralisée, on a :
\[
u(t,\cdot) = \sum_{i=1}^{\infty} c_i(t)g_i(\cdot)
\]
Comme $u\in D^1(\Omega)=D(curl_{imperm})$, on peut choisir les fonctions de base $g_i$ comme étant les fonctions propres de l'opérateur rotationnel. Ces fonctions sont les mêmes que celles de l'opérateur rotationnel au carré.\\
En effet, soit $(\lambda_i,g_i)$ solutions de $\rot g_i = \lambda_ig_i$ et $(\Lambda_i,G_i)$ solutions de $\rott G_i = \Lambda_iG_i$. Alors :
\[
\rott g_i = \rot(\rot g_i)=\rot(\lambda_ig_i)=\lambda^2g_i = \Lambda_iG_i
\]
On voit donc que $g_i=G_i$ et que $\lambda_i=\pm\sqrt\Lambda_i$. Pour n'avoir que des valeurs propres de multiplicité une, on ne garde que les valeurs propres positives, le signe étant porté par le coefficient $c_i$.

On cherche donc $(\lambda_i,g_i)\in\R\times D^1(\Omega)$ solutions du problème suivant :
\begin{eqnarray}
\label{curlcurl}
\left\{
\begin{aligned}
&\rott  g_i = \lambda_i^2 g_i\\
&g_i\cdot n\restr = 0\\
&\rot g_i\cdot n\restr = 0\\
&\rott  g_i\cdot n\restr = 0
\end{aligned}
\right.
\end{eqnarray}

On remplace maintenant $v$ par $u+a$ dans (\ref{depart}) :
\[
\frac{\partial(u+a)}{\partial t}+(\rot(u+a))\times(u+a) + \grad (\frac{|u+a|^2}{2}+p) + \frac{1}{Re}\rott(u+a) - f = 0
\]
Ce qui donne en notant $\pi_a=\frac{|u+a|^2}{2}+p$ :
\[
\frac{\partial u}{\partial t}+\frac{\partial a}{\partial t} + (\rot u+\rot a)\times(u+a) + \grad\pi_a + \frac{1}{Re}(\rott u+\rott a) - f = 0
\]
Comme $\rott a = 0$ et en notant $h=f-\frac{\partial a}{\partial t} - (\rot a)\times a$, on a le problème suivant :
\begin{eqnarray}
\label{uci}
\left\{
\begin{aligned}
&\frac{\partial u}{\partial t} + (\rot u)\times u + (\rot u)\times a +(\rot a)\times u + \grad \pi_a +\frac{1}{Re}\rott  u - h = 0\\
&\diverg u = 0\\
&u\big\rvert_{t=0} = v_0 - a(0,\cdot)\\
&u\cdot n\restr = 0\\
&(\rot u)\cdot n\restr = 0\\
&(\rott  u)\cdot n\restr = \alpha_2
\end{aligned}
\right.
\end{eqnarray}

Pour trouver les coefficients $c_i$, on remplace $u$ par sa décomposition en $\sum_i c_ig_i$. On cherche alors à résoudre le problème suivant :
\begin{eqnarray}
\label{cigi}
\left\{
\begin{aligned}
&\sum_i g_i\frac{\partial c_i}{\partial t} + \left(\rot\left(\sum_i c_ig_i\right)\right)\times\left(\sum_jc_jg_j\right)+\left(\rot\left(\sum_ic_ig_i\right)\right)\times a&\\
&+ \left(\rot a\right)\times\left(\sum_ic_ig_i\right) + \grad\pi_a + \frac{1}{Re}\rott\left(\sum_ic_ig_i\right) - h = 0\\
&\diverg\left(\sum_ic_ig_i\right) = 0\\
&\left(\sum_ic_ig_i\right)\cdot n\restr = 0\\
&\rot\left(\sum_ic_ig_i\right)\cdot n\restr = 0\\
&\rott\left(\sum_ic_ig_i\right)\cdot n\restr = \alpha_2
\end{aligned}
\right.
\end{eqnarray}

Pour résumer, on doit donc :
\begin{enumerate}
\item trouver $a$ pour pouvoir décomposer $v$ en $u+a$, pour cela, on résout les équations (\ref{psi0}) et (\ref{curlb}). Ce qui permet de trouver $a$ grâce à (\ref{aegal}). Cette partie est détaillé dans \ref{relev}.
\item générer la base $g_i$ des fonctions propres de l'opérateur curl en résolvant l'équation (\ref{curlcurl}) comme expliqué dans le chapitre \ref{eigen}.
\item résoudre l'équation (\ref{cigi}) pour trouver les coefficients $c_i$. Cela est expliqué dans \ref{spectre}.
\item recomposer $v=u+a$, et chercher $p$ pour avoir la solution du problème (\ref{depart}). Cette dernière partie est montré dans \ref{pression}.
\end{enumerate}

\subsection{Forme variationnelle}
\subsubsection{Relèvement}
\label{relev}

Le problème (\ref{aegal}) sert à retrouver des conditions aux limites pour lesquels on sait résoudre le problème. Pour trouver $a$, il nous faut d'abord résoudre les problèmes (\ref{psi0}) et (\ref{curlb}).\\

(\ref{psi0}) est un problème de Laplace, pour obtenir sa forme variationnelle, on multiplie par une fonction test et on intègre :
\[
\int_\Omega \laplace\psi_0 \varphi = 0
\]
On utilise ensuite la formule de Green pour parvenir à :
\[
-\int_\Omega \grad\psi_0\cdot\grad\varphi + \int_{\partial\Omega} \grad\psi_0\cdot n\varphi = 0
\]
Or, $\grad\psi_0\cdot n = \alpha_0$ sur $\partial\Omega$, on obtient donc la forme variationnelle suivante :
\begin{eqnarray}
\label{fvpsi}
-\int_\Omega \grad\psi_0\cdot\grad\varphi + \int_{\partial\Omega} \alpha_0\varphi = 0
\end{eqnarray}
Cela permet de connaître $\grad\psi_0$.\\

On résout maintenant le problème (\ref{curlb}) :
\begin{eqnarray*}
\left\{
\begin{aligned}
&\rott b = \grad\psi_1\\
&\diverg b = 0\\
&b\cdot n\restr = 0\\
&\rot b\cdot n\restr = 0\\
&\grad\psi_1\cdot n\restr = \alpha_1
\end{aligned}
\right.
\end{eqnarray*}
Toujours de la même manière, on multiplie par une fonction test et on intègre :
\[
\int_\Omega (\rott b) \varphi = \int_\Omega (\grad\psi_1)\varphi
\]
En intégrant par partie le premier terme et en utilisant la formule de Green sur le second, on obtient :
\begin{eqnarray}
\label{fvrelev}
\int_\Omega (\rot b)(\rot\varphi) - \int_{\partial\Omega} (\rot b)(\varphi\cdot n) +\int_\Omega \psi_1(\diverg\varphi) - \int_{\partial\Omega} \psi_1(\varphi\cdot n) = 0
\end{eqnarray}

\subsubsection{Problème aux valeurs propres}
\label{eigen}

On s'intéresse ici plus particulièrement au problème (\ref{curlcurl}).
\begin{eqnarray*}
(\lambda_i,g_i)\in\R\times D^1(\Omega)\quad \left\{
\begin{aligned}
&\rott  g_i = \lambda_i g_i\\
&g_i\cdot n\restr = 0\\
&\rot g_i\cdot n\restr = 0\\
&\rott  g_i\cdot n\restr = 0
\end{aligned}
\right.
\end{eqnarray*}
On note d'abord que tout élément $\varphi\in D^1(\Omega)$ peut s'écrire de la manière suivante :
\[
\varphi = \varphi_0 + \grad\phi\text{ et } \varphi\restr = \grad\phi
\]

On va maintenant chercher la formulation variationnelle de ce problème.\\
Soit $g\in D^1(\Omega)$ solution de (6), alors pour tout $\varphi\in D^1(\Omega)$ nous avons :
\[
\int_\Omega (\rott g)\varphi\ dX = \int_\Omega\lambda g\varphi\ dX
\]
puis en intégrant par partie :
\[
\int_\Omega (\rot g)(\rot\varphi)\ dX + \int_{\partial\Omega} ((\rot g)\times \varphi)\cdot n\ d\Gamma = \lambda\int_\Omega g\varphi\ dX
\]
or sur $\partial\Omega,\quad \varphi\restr=\grad\phi$, d'où : 
\[
\int_\Omega (\rot g)(\rot\varphi)\ dX + \int_{\partial\Omega} ((\rot g)\times \grad\phi)\cdot n\ d\Gamma = \lambda\int_\Omega g\varphi\ dX
\]
En utilisant le théorème de flux-divergence aussi appelé théorème de Green-Ostrogradski :
\[
\int_\Omega (\rot g)(\rot\varphi)\ dX + \int_\Omega \diverg((\rot g)\times \grad\phi)\ dX = \lambda\int_\Omega g\varphi\ dX
\]
En utilisant la formule $div(F\times G) = G\cdot \rot F - F\cdot \rot G$, on a :
\[
\int_\Omega (\rot g)(\rot\varphi)\ dX + \int_\Omega \grad\phi\cdot(\rott g)\ dX -\int_\Omega (\rot g)\cdot (\rot\grad\phi)\ dX  = \lambda\int_\Omega g\varphi\ dX
\]
Comme le rotationnel d'un gradient est nul, on a :
\[
\int_\Omega (\rot g)(\rot\varphi)\ dX + \int_\Omega \grad\phi\cdot(\rott g)\ dX  = \lambda\int_\Omega g\varphi\ dX
\]
En intégrant le deuxième terme par partie, on obtient : 
\[
\int_\Omega (\rot g)(\rot\varphi)\ dX + \int_{\partial\Omega} \phi((\rott g)\cdot n)\ d\Gamma - \int_\Omega \phi(\diverg(\rott g))\ dX  = \lambda\int_\Omega g\varphi\ dX
\]
Comme $\rott  g_i\cdot n\restr = 0$, le deuxième terme s'annule et comme la divergence d'un rotationnel est nulle, le troisième terme s'annule aussi, ce qui laisse pour tous $\phi\in D^1(\Omega)$ :
\begin{eqnarray}
\label{fveigen}
\int_\Omega (\rot g)(\rot\varphi)\ dX = \lambda\int_\Omega g\varphi\ dX
\end{eqnarray}

\subsubsection{Problème spectral}
\label{spectre}

Pour obtenir la forme variationnelle de (\ref{cigi}), on revient au problème équivalent (\ref{uci}) pour plus de clarté. On multiplie donc par une fonction test et on intègre :
%En partant de l'équation (\ref{uci}), on peut obtenir la forme variationnelle en multipliant par une fonction test et en intégrant :
\begin{align*}
\int_\Omega \frac{\partial u}{\partial t}\cdot \varphi &+ \int_\Omega ((\rot u)\times u)\cdot \varphi + \int_\Omega ((\rot u)\times a)\cdot\varphi + \int_\Omega ((\rot a)\times u)\cdot\varphi \\
&+ \int_\Omega \grad\pi_a\cdot\varphi + \frac{1}{Re}\int_\Omega (\rott u)\cdot\varphi = \int_\Omega h\cdot\varphi
\end{align*}
En utilisant une intégration par partie sur l'avant dernier terme et la même méthode que dans le chapitre \ref{eigen} pour le dernier terme du membre de gauche, on arrive à :
\begin{align*}
\int_\Omega \frac{\partial u}{\partial t}\cdot \varphi &+ \int_\Omega ((\rot u)\times u)\cdot \varphi + \int_\Omega ((\rot u)\times a)\cdot\varphi + \int_\Omega ((\rot a)\times u)\cdot\varphi\\ 
&+ \int_\Omega \pi_a(\diverg\varphi) + \int_{\partial \Omega} \pi_a(\varphi\cdot n) + \frac{1}{Re}\int_\Omega (\rot u)\cdot(\rot\varphi) -\frac{1}{Re}\int_{\partial\Omega} ((\rott u)\cdot n)\phi = \int_\Omega h\cdot\varphi
\end{align*}
Or, $\varphi\in D^1$ donc $\diverg\varphi=0$ et $\varphi\cdot n=0$ sur $\partial\Omega$, le terme de pression s'annule donc sous cette forme. De plus, $\rott u\cdot n=\alpha_2$ sur $\partial\Omega$. On a donc :
\begin{eqnarray}
\label{fvu}
\begin{aligned}
\int_\Omega \frac{\partial u}{\partial t}\cdot \varphi &+ \int_\Omega ((\rot u)\times u)\cdot \varphi + \int_\Omega ((\rot u)\times a)\cdot\varphi \\
&+ \int_\Omega ((\rot a)\times u)\cdot\varphi + \frac{1}{Re}\int_\Omega (\rot u)\cdot(\rot\varphi) -\frac{1}{Re}\int_{\partial\Omega} \alpha_2\phi = \int_\Omega h\cdot\varphi
\end{aligned}
\end{eqnarray}

Où $phi$ provient de la décomposition de la fonction de test en $\varphi=\varphi_0+\grad\psi$. En pratique, on utilise les fonctions propres de l'opérateur rotationnel en tant que fonctions de test.
On décrit donc la décomposition des fonctions propres $g_i$ dans le prochain chapitre.

\subsubsection{Décomposition des $g_i$}
\label{decomp}

Une fois les fonctions propres connues, on a besoin de les décomposer en $g_i=g_i^0+\grad\psi_i$ avec $g_i^0\restr = 0$ et $\grad\psi_i\cdot n\restr = 0$. Les problèmes situés dans cette section prennent place dans $H^1_0$, contrairement aux autres problèmes qui sont dans $D^1$.\\
On applique donc le rotationnel du rotationnel sur cette relation.\\
\[
\rott g_i^0 +\rott\grad\psi_i = \rott g_i
\]
Le dernier terme est nul car c'est le rotationnel d'un gradient. On utilise la formule $\rott v=\grad(\diverg v)-\laplace v$ sur les deux termes restants :
\[
\grad(\diverg g_i^0)-\laplace g_i^0 = \grad(\diverg g_i)-\laplace g_i
\]
Comme la divergence des $g_i$ est nul, on obtient le problème :
\[
\left\{
\begin{aligned}
\grad(\diverg g_i^0)-\laplace g_i^0 &= -\laplace g_i\\
g_i^0\restr &= 0
\end{aligned}
\right.
\]
On multiplie donc cette équation par une fonction test de $H^1_0$ et on intègre :
\[
\int_\Omega \grad(\diverg g_i^0)\varphi - \int_\Omega \laplace g_i^0\varphi = -\int_\Omega \laplace g_i\varphi
\]
On utilise ensuite la formule d'intégration par partie $\int_\Omega \grad u\varphi = -\int_\Omega u\diverg\varphi + \int_{\partial\Omega} u\varphi\cdot n$ sur le premier terme : 
\[
-\int_\Omega (\diverg g_i^0)(\diverg\varphi) + \int_{\partial\Omega} (\diverg g_i^0)(\varphi\cdot n) - \int_\Omega \laplace g_i^0\varphi = -\int_\Omega \laplace g_i\varphi
\]
Comme $\varphi\in H^1_0$, la seconde intégrale est nul. On intègre par partie les termes en laplacien :
\[
-\int_\Omega (\diverg g_i^0)(\diverg\varphi) + \int_\Omega \grad g_i^0\grad\varphi - \int_{\partial\Omega} (\grad g_i^0\cdot n)\varphi = \int_\Omega \grad g_i\grad\varphi - \int_{\partial\Omega} (\grad g_i\cdot n)\varphi
\]
Encore une fois, comme $\varphi\in H^1_0$, les termes sur les bords s'annulent. On obtient donc la forme variationnelle suivante :
\begin{eqnarray}
\label{fvgi0}
-\int_\Omega (\diverg g_i^0)(\diverg\varphi) + \int_\Omega \grad g_i^0\grad\varphi = \int_\Omega \grad g_i\grad\varphi
\end{eqnarray}

D'autre part, on sait que la divergence des $g_i$ est nulle, on a donc :
\[
\diverg (g_i^0+\grad\psi_i)=0
\]
Ce qui nous donne le problème de Poisson suivant :
\[
\left\{
\begin{aligned}
-\laplace\psi_i &= \diverg g_i^0\\
\grad\psi_i\cdot n\restr &= 0
\end{aligned}
\right.
\]
On a donc la forme variationnelle suivante :
\begin{eqnarray}
\label{fvpsii}
\int_\Omega \grad\psi_i\grad\varphi = \int_\Omega (\diverg g_i^0)\varphi
\end{eqnarray}

\subsubsection{Pression}
\label{pression}

Pour retrouver la vitesse $v$, il suffit maintenant d'additionner $a$ et $u$.\\
Le terme correspondant à la pression ayant été perdu lors du passage à la forme variationnelle, il faut le recalculer à partir de l'équation (\ref{depart}).\\

On applique la divergence sur cette équation et cela nous donne :
\[
\frac{\partial}{\partial t}\diverg v + \diverg((\rot v)\times v) + \diverg\grad q + \frac{1}{Re}\diverg(\rott v) - \diverg f = 0
\]
Dans cette équation, la divergence de $v$ et la divergence du rotationnel de $v$ s'annulent. Il nous reste donc :
\[
-\laplace q = \diverg((\rot v)\times v) - \diverg f
\]

Pour obtenir une condition au bord, on utilise la composante normale de l'équation (\ref{depart}) :
\[
\frac{\partial}{\partial t}(v\cdot n)\restr + ((\rot v)\times v)\cdot n\restr + \grad q\cdot n\restr +\frac{1}{Re}(\rott v)\cdot n\restr - f\cdot n\restr = 0
\]
En utilisant les conditions aux bords de $v$, on obtient :
\[
\grad q\cdot n\restr =  f\cdot n\restr - \frac{\partial\alpha_0}{\partial t} - ((\rot v)\times v)\cdot n\restr - \frac{\alpha_2}{Re}
\]
On cherche maintenant la forme variationnelle du problème :
\[
\int_\Omega -\laplace q\varphi = \int_\Omega (\diverg((\rot v)\times v) -\diverg f)\varphi
\] 
En intégrant par partie le terme de gauche, on a :
\[
\int_\Omega \grad q\grad\varphi - \int_{\partial\Omega} (\grad q\cdot n)\varphi = \int_\Omega (\diverg((\rot v)\times v) -\diverg f)\varphi
\]
On obtient donc au final :
\begin{eqnarray}
\label{fvsol}
\int_\Omega \grad q\grad\varphi = \int_\Omega (\diverg((\rot v)\times v) -\diverg f)\varphi + \int_{\partial\Omega} \left(f\cdot n - \frac{\partial\alpha_0}{\partial t} - ((\rot v)\times v)\cdot n - \frac{\alpha_2}{Re}\right)\varphi
\end{eqnarray}

\subsection{Discrétisation}

On peut maintenant remplacer $u$ par la somme $\sum_ic_ig_i$ dans (\ref{fvu}). De plus, prendre une fonction de test $\varphi$ dans $D^1$ revient à prendre chaque fonction de la base que l'on a calculé dans le chapitre \ref{eigen}.\\
On rappel que $\rot g_i=\lambda_ig_i$. En notant $\int_\Omega f\cdot g=(f,g)$, on a donc :
\begin{align*}
\frac{\partial}{\partial t}\sum_i (c_ig_i, g_k) &+ \sum_i\sum_j (c_i\lambda_ic_jg_i\times g_j,g_k) + \sum_i(c_i\lambda_ig_i\times a,g_k)\\
&+ \sum_i ((\rot a)\times c_ig_i,g_k) + \frac{1}{Re}\sum_i(c_i\lambda_ig_i, \lambda_kg_k) = (h,g_k) + \frac{1}{Re}\langle\alpha_2,\psi_k\rangle
\end{align*}
Comme les termes $c_i$ ne porte que sur la dimension temporelle, on peut les sortir des intégrales, tout comme les $\lambda$.\\
Par ailleurs, on suppose la base $(g_i)$ orthonormale. On obtient donc :
\begin{eqnarray}
\label{fvspec}
\begin{aligned}
\frac{\partial c_k}{\partial t} &+ \sum_i\sum_j c_i\lambda_i c_j(g_i\times g_j, g_k) + \sum_i c_i\lambda_i(g_i\times a,g_k)\\
&+ \sum_i c_i((\rot a)\times g_i, g_k) + \frac{1}{Re}c_k\lambda_k^2 = (h,g_k) + \frac{1}{Re}\langle\alpha_2,\psi_k\rangle
\end{aligned}
\end{eqnarray}
 
\subsection{Conclusion}

On utilise d'abord (\ref{fvpsi}) et $\alpha_0$ pour trouver $\psi_0$ et (\ref{fvrelev}) et $\alpha_1$ pour $\rot b$, en ajoutant ces deux résultats, on trouve $a$.\\
On peut maintenant calculer les valeurs et fonctions propres de l'opérateur rotationnel avec (\ref{fveigen}).\\
(\ref{fvgi0}) et (\ref{fvpsii}) permettent de décomposer les fonctions propres.\\
Une fois cela fait, on peut calculer $u$ en utilisant les $g_i$ et leurs valeurs propres $\lambda_i$ ainsi que $a$ et $\alpha_2$ dans (\ref{fvspec}).\\
Connaissant $a$ et $u$, on peut les additionner pour avoir $v$.\\
Une fois la vitesse connue, on utilise (\ref{fvsol}) pour trouver la pression. 

\begin{figure}
\centering
\begin{tikzpicture}[scale=0.7]
\node[draw,scale=0.7,fill=green!50] (di) at (15,4) {Données initiales} ;
\node[draw,scale=0.7,fill=gray!50] (pb) at (15,3) {Problèmes à résoudre} ;
\node[draw,scale=0.7,fill=blue!50] (si) at (15,2) {Solutions intermédiaires} ;
\node[draw,scale=0.7,fill=yellow!50] (sim) at (15,1) {Sol. inter. dépendantes du maillage} ;
\node[draw,scale=0.7,fill=red!50] (sf) at (15,0) {Solutions finales} ;

\node[draw,scale=0.7,fill=green!50] (a0) at (0,0) {$\alpha_0$} ;
\node[draw,scale=0.7,fill=green!50] (a1) at (4,0) {$\alpha_1$} ;
\node[draw,scale=0.7,fill=gray!50] (pbpsi0) at (0,-2) 
{$\begin{aligned}
-\laplace\psi_0&=0\\
\grad\psi_0\cdot n\restr &= \alpha_0
\end{aligned}$} ;
\node[draw,scale=0.7,fill=gray!50] (pbb) at (4,-3) 
{$\begin{aligned}
\rott b &= \grad\psi_0\\
\diverg b &=0\\ 
b\cdot n\restr &= 0\\
\rot b\cdot n\restr &= 0\\
\rott b\cdot n\restr &= \alpha_1
\end{aligned}$} ;
\node[draw,scale=0.7,fill=gray!50] (pbeigen) at (12,-3) 
{$\begin{aligned}
\rott g_i = \lambda_i^2g_i\\
g_i\cdot n\restr = 0\\
\rot g_i\cdot n\restr = 0\\
\rott g_i\cdot n\restr = 0
\end{aligned}$} ;
\node[draw,scale=0.7,fill=blue!50] (psi0) at (0,-6) {$\grad\psi_0$} ;
\node[draw,scale=0.7,fill=blue!50] (b) at (4,-6) {$\rot b$} ;
\node[draw,scale=0.7,fill=yellow!50] (lambda) at (11,-6) {$\lambda_i$} ;
\node[draw,scale=0.7,fill=yellow!50] (gi) at (13,-6) {$g_i$} ;
\node[draw,scale=0.7,fill=gray!50] (pbgi0) at (16,-6) 
{$\begin{aligned}
\grad(\diverg_i^0)-\laplace g_i^0 = -\laplace g_i\\
g_i^0\restr = 0
\end{aligned}$} ;
\node[draw,scale=0.7,fill=gray!50] (pba) at (2,-8) {$a = \rot b + \grad\psi_0$} ;
\node[draw,scale=0.7,fill=green!50] (f) at (7,-8) {$f$} ;
\node[draw,scale=0.7,fill=green!50] (a2) at (8,-8) {$\alpha_2$} ;
\node[draw,scale=0.7,fill=green!50] (ck0) at (9,-8) {$c_k^0$} ;
\node[draw,scale=0.7,fill=yellow!50] (gi0) at (16,-8) {$g_i^0$} ;
\node[draw,scale=0.7,fill=gray!50] (pbpsi) at (16,-10) 
{$\begin{aligned}
-\laplace\psi_i = \diverg g_i^0\\
\grad\psi_i\cdot n\restr = 0
\end{aligned}$} ;
\node[draw,scale=0.7,fill=yellow!50] (psi) at (16,-12) {$\psi_i$} ;
\node[draw,scale=0.85,fill=blue!50] (a) at (2,-12) {$a$} ;
\node[draw,scale=0.7,fill=gray!50] (pbs) at (9,-12)
{$\begin{aligned}
\frac{\partial c_k}{\partial t} &+ \sum_i\sum_j c_i\lambda_i c_j(g_i\times g_j, g_k) \\
&+ \sum_i c_i\lambda_i(g_i\times a,g_k) + \sum_i c_i((\rot a)\times g_i, g_k) \\
&+ \frac{1}{Re}c_k\lambda_k^2 = (h,g_k) + \frac{1}{Re}\langle\alpha_2,\psi_k\rangle\\
&c_k(0)=c_k^0
\end{aligned}
$} ;
\node[draw,scale=0.7,fill=blue!50] (ck) at (9,-15) {$c_k$} ;
\node[draw,scale=0.7,fill=gray!50] (pbu) at (9,-16) {$u=\sum c_kg_k$} ;
\node[draw,scale=0.85,fill=blue!50] (u) at (9,-17) {$u$} ;
\node[draw,scale=0.7,fill=gray!50] (pbv) at (4,-18) {$v=a+u$} ;
\node[draw,fill=red!50] (v) at (4,-19) {$v$} ;
\node[draw,scale=0.7,fill=gray!50] (pbq) at (10,-19)
{$\begin{aligned}
-\laplace q = \diverg((\rot v)\times v) - \diverg f\\
\grad q\cdot n\restr =  f\cdot n\restr - \frac{\partial\alpha_0}{\partial t} - ((\rot v)\times v)\cdot n\restr - \frac{\alpha_2}{Re}
\end{aligned}$} ;
\node[draw,fill=red!50] (q) at(16,-19) {$q$} ;

\draw[->] (a0) -- (pbpsi0);
\draw[->] (a1) -- (pbb);
\draw[->] (pbpsi0) -- (psi0);
\draw[->] (pbb) -- (b);
\draw[->] (b) -- (pba);
\draw[->] (psi0) -- (pba); 
\draw[->] (pba) -- (a); 
\draw[->] (pbeigen) -- (lambda);
\draw[->] (pbeigen) -- (gi);
\draw[->] (gi) -- (pbgi0);
\draw[->] (pbgi0) -- (gi0);
\draw[->] (gi0) -- (pbpsi);
\draw[->] (pbpsi) -- (psi);
\draw[->] (a) -- (pbs);
\draw[->] (f) -- (pbs);
\draw[->] (a2) -- (pbs);
\draw[->] (ck0) -- (pbs);
\draw[->] (lambda) -- (pbs);
\draw[->] (gi) -- (pbs);
\draw[->] (psi) -- (pbs);
\draw[->] (pbs) -- (ck);
\draw[->] (ck) -- (pbu);
\draw[->] (pbu) -- (u);
\draw[->] (u) -- (pbv);
\draw[->] (a) -- (pbv);
\draw[->] (pbv) -- (v);
\draw[->] (v) -- (pbq);
\draw[->] (pbq) -- (q);
\end{tikzpicture}
\caption{Organigramme de synthèse}
\label{org}
\end{figure}
