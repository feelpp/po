\section{Problème}
Notation :
\begin{align*}
divergence(\mathbf{v})&=\diverg \mathbf{v}\\
rot(\mathbf{v})&=\rot \mathbf{v}\\
rot(rot(\mathbf{v}))&=\rott \mathbf{v}\\
\LL^2_\sigma(\Omega) &= \{\mathbf{v} \in \LL^2(\Omega) | \diverg \mathbf{v} = 0\text{ et }\mathbf{v}\cdot n\restr = 0 \}\\
\HH^1(\Omega) &= \{\mathbf{v} \in \LL^2(\Omega) | \grad \mathbf{v}\in\LL^2(\Omega)\}\\
\HH^1_0(\Omega) &= \{\mathbf{v} \in \HH^1(\Omega) | \mathbf{v}\restr = 0\}\\
\HH(div) &= \{\mathbf{v} \in \LL^2(\Omega) | \diverg\mathbf{v} \in \LL^2(\Omega) \}\\
\HH(curl) &= \{\mathbf{v} \in \LL^2(\Omega) | \rot\mathbf{v} \in \LL^2(\Omega) \}\\
\DD^1 &= \{\mathbf{v} \in \HH^1(\Omega)\cup \LL^2(\Omega) | (\rot \mathbf{v}\cdot n)\restr = 0  \}
\end{align*}

Nous cherchons $(\mathbf{v},p)$, correspondant respectivement à la vitesse et à la pression, solutions de l'équation de Navier-Stokes incompressibles adimensionnalisées dans $Q_T=\Omega\times[0,T]$ un ouvert de $\R^3$ et $\partial\Omega$ sa frontière, avec une condition initiale et des conditions aux limites d'imperméabilité généralisée :
\begin{equation}
\label{depart}
\left\{
\begin{aligned}
&\frac{\partial \mathbf{v}}{\partial t} + (\rot  \mathbf{v})\times \mathbf{v} + \grad q + \frac{1}{Re}\rott  \mathbf{v}-\mathbf{f} = 0\\
&\diverg \mathbf{v} = 0\\
&\mathbf{v}\big\rvert_{t=0} = \mathbf{v}_0\\
&\mathbf{v}\cdot n\restr = \alpha_0\\
&(\rot  \mathbf{v})\cdot n\restr = \alpha_1\\
&(\rott  \mathbf{v})\cdot n\restr = \alpha_2
\end{aligned}
\right.
\end{equation}
où $q = \frac{|\mathbf{v}|^2}{2}+p$.\\
Alors, on cherche la solution dans l'espace $\LL^2_\sigma$.\\

On va maintenant expliquer la décomposition du problème en une suite de problème plus simple dans le chapitre \ref{local}, puis étudié plus en détail chacun de ces problèmes dans \ref{fv}. Enfin on verra la discrétisation du problème dans la section \ref{discr}.

\subsection{Problème local}
\label{local}

Afin d'utiliser les travaux de O. Mezureau, J. Neustupa et P. Penel\footnote{\href{http://www.researchgate.net/publication/254219219\_A\_weak\_solvability\_to\_the\_steady\_NavierStokes\_equations\_for\_compressible\_barotropic\_fluid\_with\_generalized\_impermeability\_boundary\_conditions}{A weak solvability to the steady Navier-Stokes equations for compressible barotropic fluid with generalized impermeability boundary conditions}} qui ont étudié ce type d'équations dans $\DD^1$, on cherche à écrire $\mathbf{v}=\mathbf{u}+\mathbf{a}$, tel que :\\
\begin{equation}
\label{v}
\begin{array}{c|ccccc}
& \mathbf{v} & = & \mathbf{a} & + & \mathbf{u}\\ \hline
\diverg\star & 0 & & 0 & & 0\\ \hline
\star\cdot n\restr & \alpha_0 & & \alpha_0 & & 0\\ \hline
\rot\star\cdot n\restr & \alpha_1 & & \alpha_1 & & 0\\ \hline
\rott\star\cdot n\restr & \alpha_1 & & 0 & & \alpha_2
\end{array}
\end{equation}
Ainsi, on a $\mathbf{u}\in\DD^1$ et $\mathbf{a}$ sert à relever le problème.

On veut $\mathbf{a}$ tel que :
\begin{equation}
\label{a}
\left\{
\begin{aligned}
&\mathbf{a}=\rot \mathbf{b}+\grad\psi^0\\
&\diverg \mathbf{a} =0\\
&\mathbf{a}\cdot n\restr = \alpha_0\\
&(\rot \mathbf{a})\cdot n\restr = \alpha_1
\end{aligned}
\right.
\end{equation}
En appliquant la divergence et les conditions aux bords à la première ligne du problème on obtient le tableau suivant :
\begin{center}
\begin{tabular}{c|ccccc}
& $\mathbf{a}$ & = & $\grad\psi^0$ & + & $\rot \mathbf{b}$ \\ \hline
$\diverg\star$ & 0 & & $\laplace\psi^0$ & & 0\\ \hline
$\star\cdot n\restr$ & $\alpha_0$ & & $\alpha_0$ & & 0\\ \hline
$\rot\star\cdot n\restr$ & $\alpha_1$ & & 0 & & $\alpha_1$
\end{tabular}
\end{center}
En utilisant la première et la deuxième ligne du tableau, on voit que $\psi^0$ est solution de l'équation :
\begin{equation}
\label{psi0}
\left\{
\begin{aligned}
&-\laplace\psi^0 = 0\\
&\grad\psi^0\cdot n\restr=\alpha_0
\end{aligned}
\right.
\end{equation}
Ce problème permet de trouver $\psi^0$ à une constante près, on va donc utiliser un multiplicateur de Lagrange pour ajouter une contrainte sur $\psi^0$, par exemple $\int_\Omega u = 0$, cette constante est sans importance car on cherche le gradient de $\psi^0$.\\

Et $b$ est solution du problème :
\begin{equation}
\label{curlb}
\left\{
\begin{aligned}
&\rott \mathbf{b} = \grad\psi^1\\
&\diverg \mathbf{b} = 0\\
&\mathbf{b}\cdot n\restr = 0\\
&\rot \mathbf{b}\cdot n\restr = 0\\
&\rott \mathbf{b}\cdot n\restr = \alpha_1
\end{aligned}
\right.
\end{equation}
où $\psi^1$ est solution de :
\begin{equation}
\label{psi1}
\left\{
\begin{aligned}
&-\laplace\psi^1 = 0\\
&\grad\psi^1\cdot n\restr=\alpha_1
\end{aligned}
\right.
\end{equation}
Il y a plusieurs manières de résoudre les problèmes (\ref{psi0}-\ref{curlb}-\ref{psi1}), la solution choisie sera discutée plus loin.\\

Une fois $\grad\psi^0$ et $\rot \mathbf{b}$ connus, on peut retrouver $\mathbf{a}$.\\

On cherche donc $\mathbf{u}=\mathbf{v}-\mathbf{a}$, et en utilisant la décomposition de Galerkin généralisée, on a :
\begin{equation}
\label{u}
\mathbf{u}(t,\cdot) = \sum_{i=1}^{\infty} c_i(t)\mathbf{g_i}(\cdot)
\end{equation}
Comme $\mathbf{u}\in D^1(\Omega)=D(curl_{imperm})$, on peut choisir les fonctions de base
$\mathbf{g_i}$ comme étant les fonctions propres de l'opérateur rotationnel. Ces
fonctions sont les mêmes que celles de l'opérateur $\rott$.

En effet, soit $(\lambda_i,\mathbf{g_i})$ solutions de $\rot \mathbf{g_i} = \lambda_i\mathbf{g_i}$ et $(\Lambda_i,\mathbf{G_i})$ solutions de $\rott \mathbf{G_i} = \Lambda_i\mathbf{G_i}$. Alors :
\[
\rott \mathbf{g_i} = \rot(\rot \mathbf{g_i})=\rot(\lambda_i\mathbf{g_i})=\lambda^2\mathbf{g_i} = \Lambda_i\mathbf{G_i}
\]
On voit donc que $\mathbf{g_i}=\mathbf{G_i}$ et que $\lambda_i=\pm\sqrt\Lambda_i$. Pour n'avoir que des valeurs propres de multiplicité une, on ne garde que les valeurs propres positives, le signe étant porté par le coefficient $c_i$.

On cherche donc $(\lambda_i,\mathbf{g_i})\in\R\times D^1(\Omega)$ solutions du problème suivant :
\begin{equation}
\label{curlcurl}
\left\{
\begin{aligned}
&\rott  \mathbf{g_i} = \lambda_i^2 \mathbf{g_i}\\
&\mathbf{g_i}\cdot n\restr = 0\\
&\rot \mathbf{g_i}\cdot n\restr = 0\\
&\rott  \mathbf{g_i}\cdot n\restr = 0
\end{aligned}
\right.
\end{equation}

On remplace maintenant $\mathbf{v}$ par $\mathbf{u}+\mathbf{a}$ dans (\ref{depart}) :
\[
\frac{\partial(\mathbf{u}+\mathbf{a})}{\partial t}+(\rot(\mathbf{u}+\mathbf{a}))\times(\mathbf{u}+\mathbf{a}) + \grad (\frac{|\mathbf{u}+\mathbf{a}|^2}{2}+p) + \frac{1}{Re}\rott(\mathbf{u}+\mathbf{a}) - \mathbf{f} = 0
\]
Ce qui donne en notant $\pi_a=\frac{|\mathbf{u}+\mathbf{a}|^2}{2}+p$ :
\[
\frac{\partial \mathbf{u}}{\partial t}+\frac{\partial \mathbf{a}}{\partial t} + (\rot \mathbf{u}+\rot \mathbf{a})\times(\mathbf{u}+\mathbf{a}) + \grad\pi_{\mathbf{a}} + \frac{1}{Re}(\rott \mathbf{u}+\rott \mathbf{a}) - \mathbf{f} = 0
\]
Comme $\rott \mathbf{a} = 0$ et en notant $\mathbf{h_a}=\mathbf{f}-\frac{\partial \mathbf{a}}{\partial t} - (\rot \mathbf{a})\times \mathbf{a}$, on a le problème suivant :
\begin{equation}
\label{pbu}
\left\{
\begin{aligned}
&\frac{\partial \mathbf{u}}{\partial t} + (\rot \mathbf{u})\times \mathbf{u} + (\rot \mathbf{u})\times \mathbf{a} +(\rot \mathbf{a})\times \mathbf{u} + \grad \pi_{\mathbf{a}} +\frac{1}{Re}\rott  \mathbf{u} - \mathbf{h_a} = 0\\
&\diverg \mathbf{u} = 0\\
&\mathbf{u}\big\rvert_{t=0} = \mathbf{v}_0 - \mathbf{a}(0,\cdot)\\
&\mathbf{u}\cdot n\restr = 0\\
&(\rot \mathbf{u})\cdot n\restr = 0\\
&(\rott  \mathbf{u})\cdot n\restr = \alpha_2
\end{aligned}
\right.
\end{equation}

Pour résumer, on doit donc :
\begin{enumerate}
\item générer la base $\mathbf{g_i}$ des fonctions propres de l'opérateur curl en résolvant l'équation (\ref{curlcurl}) comme expliqué dans le chapitre \ref{eigen}.
\item trouver $\mathbf{a}$ pour pouvoir décomposer $\mathbf{v}$ en $\mathbf{u}+\mathbf{a}$, pour cela, on résout les équations (\ref{psi0}) et (\ref{curlb}). Ce qui permet de trouver $\mathbf{a}$ grâce à (\ref{a}). Cette partie est détaillé dans \ref{relev}.
\item résoudre l'équation (\ref{pbu}) pour trouver les coefficients $c_i$. Cela est expliqué dans \ref{spectre}.
\item recomposer $\mathbf{v}=\mathbf{u}+\mathbf{a}$, et chercher $p$ pour avoir la solution du problème (\ref{depart}). Cette dernière partie est montré dans \ref{pression}.
\end{enumerate}

\subsection{Forme variationnelle}
\label{fv}
\subsubsection{Problème aux valeurs propres}
\label{eigen}

On s'intéresse ici plus particulièrement au problème (\ref{curlcurl}).
\begin{equation}
\label{pbeigen}
(\lambda_i,\mathbf{g_i})\in\R\times \DD^1(\Omega)\quad \left\{
\begin{aligned}
&\rott  \mathbf{g_i} = \lambda_i \mathbf{g_i}\\
&\mathbf{g_i}\cdot n\restr = 0\\
&\rot \mathbf{g_i}\cdot n\restr = 0\\
&\rott  \mathbf{g_i}\cdot n\restr = 0
\end{aligned}
\right.
\end{equation}
On note d'abord que tout élément $\varphi\in \DD^1(\Omega)$ peut s'écrire de la manière suivante :
\[
\varphi = \varphi_0 + \grad\phi\text{ et } \varphi\restr = \grad\phi
\]

On va maintenant chercher la formulation variationnelle de ce problème.\\
Soit $\mathbf{g}\in \DD^1(\Omega)$ solution de (6), alors pour tout $\varphi\in \DD^1(\Omega)$ nous avons :
\[
\int_\Omega (\rott \mathbf{g})\cdot\varphi\ dX = \int_\Omega\lambda \mathbf{g}\cdot\varphi\ dX
\]
puis en intégrant par partie :
\[
\int_\Omega (\rot \mathbf{g})\cdot(\rot\varphi)\ dX + \int_{\partial\Omega} ((\rot \mathbf{g})\times \varphi)\cdot n\ d\Gamma = \lambda\int_\Omega \mathbf{g}\cdot\varphi\ dX
\]
or sur $\partial\Omega,\quad \varphi\restr=\grad\phi$, d'où :
\[
\int_\Omega (\rot \mathbf{g})\cdot(\rot\varphi)\ dX + \int_{\partial\Omega} ((\rot \mathbf{g})\times \grad\phi)\cdot n\ d\Gamma = \lambda\int_\Omega \mathbf{g}\cdot\varphi\ dX
\]
En utilisant le théorème de flux-divergence aussi appelé théorème de Green-Ostrogradski :
\[
\int_\Omega (\rot \mathbf{g})\cdot(\rot\varphi)\ dX + \int_\Omega \diverg((\rot \mathbf{g})\times \grad\phi)\ dX = \lambda\int_\Omega \mathbf{g}\cdot\varphi\ dX
\]
En utilisant la formule $div(F\times G) = G\cdot \rot F - F\cdot \rot G$, on a :
\[
\int_\Omega (\rot \mathbf{g})\cdot(\rot\varphi)\ dX + \int_\Omega \grad\phi\cdot(\rott \mathbf{g})\ dX -\int_\Omega (\rot \mathbf{g})\cdot (\rot\grad\phi)\ dX  = \lambda\int_\Omega \mathbf{g}\cdot\varphi\ dX
\]
Comme le rotationnel d'un gradient est nul, on a :
\[
\int_\Omega (\rot \mathbf{g})\cdot(\rot\varphi)\ dX + \int_\Omega \grad\phi\cdot(\rott \mathbf{g})\ dX  = \lambda\int_\Omega \mathbf{g}\cdot\varphi\ dX
\]
En intégrant le deuxième terme par partie, on obtient :
\[
\int_\Omega (\rot \mathbf{g})\cdot(\rot\varphi)\ dX + \int_{\partial\Omega} \phi((\rott \mathbf{g})\cdot n)\ d\Gamma - \int_\Omega \phi(\diverg(\rott \mathbf{g}))\ dX  = \lambda\int_\Omega \mathbf{g}\cdot\varphi\ dX
\]
Comme $\rott  \mathbf{g_i}\cdot n\restr = 0$, le deuxième terme s'annule et comme la divergence d'un rotationnel est nulle, le troisième terme s'annule aussi, ce qui laisse pour tous $\varphi\in D^1(\Omega)$ :
\begin{equation}
\label{fveigen}
\int_\Omega (\rot \mathbf{g})\cdot(\rot\varphi)\ dX = \lambda\int_\Omega \mathbf{g}\cdot\varphi\ dX
\end{equation}

On obtient donc $(\lambda_i,g_i)$, où $(g_i)$ forme une base de $\DD^1$. Comme énoncé plus tôt, tout élément de $\DD^1$ peut s'écrire $\varphi = \varphi_0 + \grad\phi$, y compris bien sûr les $g_i$. Comme on va en avoir besoin pour les problèmes suivants, le prochain chapitre explique comment les décomposer.

\subsubsection{Décomposition des $\mathbf{g_i}$}
\label{decomp}

Une fois les fonctions propres connues, on a besoin de les décomposer en $\mathbf{g_i}=\mathbf{g_i^0}+\grad\psi_i$ avec $\mathbf{g_i^0}\restr = 0$ et $\grad\psi_i\cdot n\restr = 0$.\\
On applique donc le rotationnel du rotationnel sur cette relation.\\
\[
\rott \mathbf{g_i^0} +\rott\grad\psi_i = \rott \mathbf{g_i}
\]
Le dernier terme est nul car c'est le rotationnel d'un gradient. On utilise la formule $\rott v=\grad(\diverg v)-\laplace v$ sur les deux termes restants :
\[
\grad(\diverg \mathbf{g_i^0})-\laplace \mathbf{g_i^0} = \grad(\diverg g_i)-\laplace \mathbf{g_i}
\]
Comme la divergence des $\mathbf{g_i}$ est nul, on obtient le tableau suivant :
\begin{center}
\begin{tabular}{c|ccccc}
& $\mathbf{g_i}$ & = & $\mathbf{g_i^0}$ & + & $\grad\psi_i$ \\ \hline
$\rott\star$ & $-\laplace g_i$ & & $\grad(\diverg \mathbf{g_i^0})-\laplace \mathbf{g_i^0}$ & & 0\\ \hline
$\diverg\star$ & 0 & & $\diverg \mathbf{g_i^0}$ & & $\laplace\psi_i$\\ \hline
$\star\cdot n\restr$ & 0 & & 0 & & 0
\end{tabular}
\end{center}
En utilisant la première et la dernière ligne on parvient au problème :
\begin{equation}
\label{gi0}
\left\{
\begin{aligned}
\grad(\diverg \mathbf{g_i^0})-\laplace \mathbf{g_i^0} &= -\laplace \mathbf{g_i}\\
\mathbf{g_i^0}\restr &= 0
\end{aligned}
\right.
\end{equation}
Comme $\rott\mathbf{g_i^0}\not=0$ et $\mathbf{g_i^0}\restr=0$, on cherche $g_i^0$ dans $\HH^1_0$. On multiplie donc cette équation par une fonction test de $\HH^1_0$ et on intègre :
\[
\int_\Omega \grad(\diverg \mathbf{g_i^0})\cdot\varphi - \int_\Omega \laplace \mathbf{g_i^0}\cdot\varphi = -\int_\Omega \laplace \mathbf{g_i}\cdot\varphi
\]
On utilise ensuite la formule d'intégration par partie $\int_\Omega \grad{\mathbf{u}}\varphi = -\int_\Omega \mathbf{u}\diverg\varphi + \int_{\partial\Omega} \mathbf{u}\varphi\cdot n$ sur le premier terme :
\[
-\int_\Omega (\diverg \mathbf{g_i^0})(\diverg\varphi) + \int_{\partial\Omega} (\diverg \mathbf{g_i^0})(\varphi\cdot n) - \int_\Omega \laplace \mathbf{g_i^0}\cdot\varphi = -\int_\Omega \laplace \mathbf{g_i}\cdot\varphi
\]
Comme $\varphi\in H^1_0$, la seconde intégrale est nul. On intègre par partie les termes en laplacien :
\[
-\int_\Omega (\diverg \mathbf{g_i^0})(\diverg\varphi) + \int_\Omega \overline{\grad \mathbf{g_i^0}}\cdot\overline{\grad\varphi} - \int_{\partial\Omega} (\overline{\grad \mathbf{g_i^0}}\cdot n)\cdot\varphi = \int_\Omega \overline{\grad \mathbf{g_i}}\cdot\overline{\grad\varphi} - \int_{\partial\Omega} (\overline{\grad \mathbf{g_i}}\cdot n)\cdot\varphi
\]
Encore une fois, comme $\varphi\in H^1_0$, les termes sur les bords s'annulent. On obtient donc la forme variationnelle suivante :
\begin{equation}
\label{fvgi0}
-\int_\Omega (\diverg \mathbf{g_i^0})(\diverg\varphi) + \int_\Omega \overline{\grad \mathbf{g_i^0}}\cdot\overline{\grad\varphi} = \int_\Omega \overline{\grad \mathbf{g_i}}\cdot\overline{\grad\varphi}
\end{equation}

D'autre part, les deux dernières lignes du tableau nous donnent le problème de Poisson suivant :
\begin{equation}
\label{psi}
\left\{
\begin{aligned}
-\laplace\psi_i &= \diverg \mathbf{g_i^0}\\
\grad\psi_i\cdot n\restr &= 0
\end{aligned}
\right.
\end{equation}
Cette fois-ci, on cherche $\grad\psi_i$ dans $\HH^1$. On a donc la forme variationnelle suivante :
\begin{equation}
\label{fvpsi}
\int_\Omega \grad\psi_i\cdot\grad\varphi = \int_\Omega (\diverg \mathbf{g_i^0})\varphi
\end{equation}

\subsubsection{Relèvement}
\label{relev}

On cherche maintenant $\mathbf{a}$, on va donc s'interesser d'abord à $\grad\psi^0$.\\
Il y a plusieurs alternatives pour résoudre (\ref{psi0}), on peut tout d'abord chercher $(\mathbf{l},\psi^0)\in\HH(div)\times\LL^2$ solution du problème de Darcy suivant :
\begin{equation}
\label{pbpsidiv}
\left\{
\begin{aligned}
\mathbf{l} &= \grad \psi^0\\
\diverg \mathbf{l} &= 0\\
\mathbf{l}\cdot n\restr &= \alpha_0
\end{aligned}
\right.
\end{equation}
Pour obtenir la formulation faible du problème, on multiplie par une fonction test $(\varphi,\nu)\in\HH(div)\times\LL^2$ et on intègre les deux premières équations :
\begin{align*}
\int_\Omega \mathbf{l}\cdot\varphi &= \int_\Omega \grad\psi^0\cdot\varphi\\
\int_\Omega \diverg \mathbf{l}\ \nu &= 0
\end{align*}
On intègre par partie la deuxième équation :
\[
\int_\Omega \diverg \mathbf{l}\ \nu = \int_{\partial\Omega} \mathbf{l}\cdot n\ q - \int_\Omega \mathbf{l}\cdot\grad\nu = 0 
\]
En insérant la condition au bord et la première équation, on obtient la formulation faible :
\begin{equation}
\label{fvpsidiv}
-\int_\Omega \mathbf{l}\cdot\varphi + \int_\Omega \mathbf{l}\cdot\grad\nu + \int_\Omega \grad\psi^0\cdot\varphi  = \int_{\partial\Omega} \alpha_0\nu
\end{equation}

L'autre possibilité pour résoudre (\ref{psi0}) est de se placer dans $\HH^1$. 
\begin{equation}
\label{pbpsi0}
\left\{
\begin{aligned}
&-\laplace\psi^0 = 0\\
&\grad\psi^0\cdot n\restr=\alpha_0
\end{aligned}
\right.
\end{equation}

Ainsi pour obtenir sa forme variationnelle, on multiplie par une fonction test $\varphi\in\HH^1$ et on intègre :
\[
\int_\Omega \laplace\psi^0 \varphi = 0
\]
On utilise ensuite la formule de Green pour parvenir à :
\[
-\int_\Omega \grad\psi^0\cdot\grad\varphi + \int_{\partial\Omega} \grad\psi^0\cdot n\varphi = 0
\]
Or, $\grad\psi^0\cdot n = \alpha_0$ sur $\partial\Omega$, on obtient donc la forme variationnelle suivante :
\begin{equation}
\label{fvpsi0}
-\int_\Omega \grad\psi^0\cdot\grad\varphi + \int_{\partial\Omega} \alpha_0\varphi = 0
\end{equation}
La première technique a l'avantage de faire gagner un ordre à la régularité de $\grad\psi^0$.\\

Pour le problème (\ref{curlb}), on peut le résoudre de deux manières différentes.\\
Soit de résoudre dans $\HH(curl)$ le problème mixte suivant : 
\begin{equation}
\label{pbbcurl}
\left\{
\begin{aligned}
&\rott \mathbf{b} = \grad\psi^1\\
&\diverg \mathbf{b} = 0\\
&\mathbf{b}\cdot n\restr = 0\\
&\rot \mathbf{b}\cdot n\restr = 0\\
&\grad\psi^1\cdot n\restr = \alpha_1
\end{aligned}
\right.
\end{equation}
et en changeant la formulation variationnelle du problème (\ref{pbu}) pour que $\rot a$ n'y apparaisse plus.\\
Pour avoir la formulation faible, on multiplie par une fonction test de $\HH(curl)$ et on intègre :
\[
\int_\Omega (\rott \mathbf{b})\cdot\varphi = \int_\Omega (\grad\psi^1)\cdot\varphi
\]
En intégrant par partie le premier terme et en utilisant la formule de Green sur le second, on obtient :
\begin{equation}
\label{fvbcurl}
\int_\Omega (\rot \mathbf{b})\cdot(\rot\varphi) - \int_{\partial\Omega} (\rot \mathbf{b})(\varphi\cdot n) + \int_\Omega \psi^1(\diverg\varphi) - \int_{\partial\Omega} \psi^1(\varphi\cdot n) = 0
\end{equation}

Soit résoudre les deux problèmes (\ref{curlb}) et (\ref{psi1}) séparément, en réutilisant la technique de (\ref{psi0}) pour (\ref{psi1}), et en cherchant $\rot b$ dans $\DD^1$, cela permet de réutiliser les fonctions propres de l'opérateur $\rott$, et d'avoir une plus grande régularité de $\rot \mathbf{b}$.\\
On cherche donc à résoudre :
\[
\left\{
\begin{aligned}
&\rott \mathbf{b} = \grad\psi^1\\
&\diverg \mathbf{b} = 0\\
&\mathbf{b}\cdot n\restr = 0\\
&\rot \mathbf{b}\cdot n\restr = 0\\
&\rott \mathbf{b}\cdot n\restr = \alpha_1
\end{aligned}
\right.
\]
De la même manière que dans le chapitre \ref{eigen}, on obtient la forme variationnelle :
\[
\int_\Omega (\rot\mathbf{b})\cdot(\rot\varphi) + \int_{\partial\Omega} \phi((\rott\mathbf{b})\cdot n) - \int_\Omega \phi(\diverg(\rot \mathbf{b}))  = \int_\Omega \grad\psi\cdot\varphi
\]
Le troisième terme est toujours nul, tandis que le deuxième peut s'écrire avec la condition aux bords :
\begin{equation}
\label{bd1}
\int_\Omega (\rot\mathbf{b})\cdot(\rot\varphi) + \int_{\partial\Omega} \phi\alpha_1 = \int_\Omega \grad\psi_1\cdot\varphi
\end{equation}

\subsubsection{Problème spectral}
\label{spectre}

On cherche maintenant $\mathbf{u}\in\DD^1$ solution de (\ref{pbu}), on veut donc obtenir la forme variationnelle du problème, pour cela, on multiplie par une fonction test $\varphi\in\DD^1$ et on intègre :
\begin{align*}
\int_\Omega \frac{\partial \mathbf{u}}{\partial t}\cdot \varphi &+ \int_\Omega ((\rot \mathbf{u})\times \mathbf{u})\cdot \varphi + \int_\Omega ((\rot \mathbf{u})\times \mathbf{a})\cdot\varphi + \int_\Omega ((\rot \mathbf{a})\times \mathbf{u})\cdot\varphi \\
&+ \int_\Omega \grad\pi_{\mathbf{a}}\cdot\varphi + \frac{1}{Re}\int_\Omega (\rott \mathbf{u})\cdot\varphi = \int_\Omega \mathbf{h_a}\cdot\varphi
\end{align*}
En utilisant une intégration par partie sur l'avant dernier terme et la même méthode que dans le chapitre \ref{eigen} pour le dernier terme du membre de gauche, on arrive à :
\begin{align*}
\int_\Omega \frac{\partial \mathbf{u}}{\partial t}\cdot \varphi &+ \int_\Omega ((\rot \mathbf{u})\times \mathbf{u})\cdot \varphi + \int_\Omega ((\rot \mathbf{u})\times \mathbf{a})\cdot\varphi + \int_\Omega ((\rot \mathbf{a})\times \mathbf{u})\cdot\varphi\\
&+ \int_\Omega \pi_{\mathbf{a}}(\diverg\varphi) + \int_{\partial \Omega} \pi_{\mathbf{a}}(\varphi\cdot n) + \frac{1}{Re}\int_\Omega (\rot \mathbf{u})\cdot(\rot\varphi) -\frac{1}{Re}\int_{\partial\Omega} ((\rott \mathbf{u})\cdot n)\phi = \int_\Omega \mathbf{h_a}\cdot\varphi
\end{align*}
Or, $\varphi\in D^1$ donc $\diverg\varphi=0$ et $\varphi\cdot n=0$ sur $\partial\Omega$, le terme de pression s'annule donc sous cette forme. De plus, $\rott \mathbf{u}\cdot n=\alpha_2$ sur $\partial\Omega$. On a donc :
\begin{equation}
\label{fvu}
\begin{aligned}
\int_\Omega \frac{\partial \mathbf{u}}{\partial t}\cdot \varphi &+ \int_\Omega ((\rot \mathbf{u})\times \mathbf{u})\cdot \varphi + \int_\Omega ((\rot \mathbf{u})\times \mathbf{a})\cdot\varphi \\
&+ \int_\Omega ((\rot \mathbf{a})\times \mathbf{u})\cdot\varphi + \frac{1}{Re}\int_\Omega (\rot \mathbf{u})\cdot(\rot\varphi) -\frac{1}{Re}\int_{\partial\Omega} \alpha_2\phi = \int_\Omega \mathbf{h_a}\cdot\varphi
\end{aligned}
\end{equation}

Où $\phi$ provient de la décomposition de la fonction de test en $\varphi=\varphi_0+\grad\psi$.

\subsubsection{Pression}
\label{pression}

Pour retrouver la vitesse $\mathbf{v}$, il suffit maintenant d'additionner $\mathbf{a}$ et $\mathbf{u}$.\\
Le terme correspondant à la pression ayant été perdu lors du passage à la forme variationnelle, il faut le recalculer à partir de l'équation (\ref{depart}).

On applique la divergence sur cette équation et cela nous donne :
\[
\frac{\partial}{\partial t}\diverg \mathbf{v} + \diverg((\rot \mathbf{v})\times \mathbf{v}) + \diverg\grad q + \frac{1}{Re}\diverg(\rott \mathbf{v}) - \diverg \mathbf{f} = 0
\]
Dans cette équation, la divergence de $\mathbf{v}$ et la divergence du rotationnel de $\mathbf{v}$ s'annulent. Il nous reste donc :
\begin{eqnarray}
\label{q}
-\laplace q = \diverg((\rot \mathbf{v})\times \mathbf{v}) - \diverg \mathbf{f}
\end{eqnarray}

Pour obtenir une condition au bord, on utilise la composante normale de l'équation (\ref{depart}) :
\[
\frac{\partial}{\partial t}(\mathbf{v}\cdot n)\restr + ((\rot \mathbf{v})\times \mathbf{v})\cdot n\restr + \grad q\cdot n\restr +\frac{1}{Re}(\rott \mathbf{v})\cdot n\restr - \mathbf{f}\cdot n\restr = 0
\]
En utilisant les conditions aux bords de $\mathbf{v}$, on obtient :
\[
\grad q\cdot n\restr =  \mathbf{f}\cdot n\restr - \frac{\partial\alpha_0}{\partial t} - ((\rot \mathbf{v})\times \mathbf{v})\cdot n\restr - \frac{\alpha_2}{Re}
\]
On cherche maintenant la forme variationnelle du problème :
\[
\int_\Omega -\laplace q\varphi = \int_\Omega (\diverg((\rot \mathbf{v})\times \mathbf{v}) -\diverg \mathbf{f})\varphi
\]
En intégrant par partie le terme de gauche, on a :
\[
\int_\Omega \grad q\grad\varphi - \int_{\partial\Omega} (\grad q\cdot n)\varphi = \int_\Omega (\diverg((\rot \mathbf{v})\times \mathbf{v}) -\diverg \mathbf{f})\varphi
\]

\todo[inline]{Encore une fois la pression est à une constante près, il faudra
faire attention!}

On obtient donc au final :
\begin{eqnarray}
\label{fvq}
\int_\Omega \grad q\grad\varphi = \int_\Omega (\diverg((\rot \mathbf{v})\times \mathbf{v}) -\diverg \mathbf{f})\varphi + \int_{\partial\Omega} \left(f\cdot n - \frac{\partial\alpha_0}{\partial t} - ((\rot \mathbf{v})\times \mathbf{v})\cdot n - \frac{\alpha_2}{Re}\right)\varphi
\end{eqnarray}

\subsection{Discrétisation}
\label{discr}

Dans les problèmes (\ref{bd1}) et (\ref{fvu}), on va chercher à expimer $\mathbf{b}$ et $\mathbf{u}$ dans la base $(\mathbf{g_i})$, c'est-à-dire $\mathbf{b}=\sum_i d_ig_i$ et $\mathbf{u}=\sum_i c_ig_i$. De plus, prendre une fonction de test $\varphi$ dans $D^1$ revient à prendre chaque fonction de la base que l'on a calculé dans le chapitre \ref{eigen}.\\
On rappel que $\rot \mathbf{g_i}=\lambda_i\mathbf{g_i}$. En notant $\int_\Omega \mathbf{f}\mathbf{g}=(f,g)$ et $\int_{\partial\Omega}a\ b=\langle a,b\rangle$, on obtient pour le problème (\ref{fvu}) :
\begin{align*}
\frac{\partial}{\partial t}\sum_i (c_i\mathbf{g_i}, \mathbf{g_k}) &+ \sum_i\sum_j (c_i\lambda_ic_j\mathbf{g_i}\times \mathbf{g_j},\mathbf{g_k}) + \sum_i(c_i\lambda_i\mathbf{g_i}\times \mathbf{a},\mathbf{g_k})\\
&+ \sum_i ((\rot \mathbf{a})\times c_i\mathbf{g_i},\mathbf{g_k}) + \frac{1}{Re}\sum_i(c_i\lambda_i\mathbf{g_i}, \lambda_k\mathbf{g_k}) = (\mathbf{h_a},\mathbf{g_k}) + \frac{1}{Re}\langle\alpha_2,\psi_k\rangle
\end{align*}
Comme les termes $c_i$ ne porte que sur la dimension temporelle, on peut les
sortir des intégrales, tout comme les $\lambda$.\\

Par ailleurs, on rappel que la base $(\mathbf{g_i})$ est orthonormale. On obtient donc :
\begin{equation}
\label{fvspec}
\begin{aligned}
\frac{\partial c_k}{\partial t} &+ \sum_i\sum_j c_i\lambda_i c_j(\mathbf{g_i}\times \mathbf{g_j}, \mathbf{g_k}) + \sum_i c_i\lambda_i(\mathbf{g_i}\times \mathbf{a},\mathbf{g_k})\\
&+ \sum_i c_i((\rot \mathbf{a})\times \mathbf{g_i}, \mathbf{g_k}) + \frac{1}{Re}c_k\lambda_k^2 = (\mathbf{h_a},\mathbf{g_k}) + \frac{1}{Re}\langle\alpha_2,\psi_k\rangle
\end{aligned}
\end{equation}

Pour le problème (\ref{bd1}), on obtient de la même manière :
\begin{equation}
\label{discbd1}
d_k\lambda_k^2 = (\grad\psi^1,\mathbf{g_k}) - \langle\alpha_1,\psi_k\rangle
\end{equation}

\subsection{Conclusion}

On calcul d'abord les valeurs et fonctions propres de l'oprérateur rotationnel avec (\ref{fveigen}), puis on décompose les fonctions à l'aide de (\ref{fvgi0}) et (\ref{fvpsi}).\\
On cherche ensuite à trouver $\mathbf{a}$, pour cela, on a plusieurs solutions :
\begin{itemize}
\item utiliser (\ref{fvpsi0}) pour trouver $\grad\psi^0$  et (\ref{fvbcurl}) pour $\rot \mathbf{b}$.
\item utiliser (\ref{fvpsidiv}) pour trouver $\grad\psi^0$ et $\grad\psi^1$, puis utiliser ce dernier dans (\ref{discbd1}) pour trouver $\rot \mathbf{b}$.
\end{itemize}
Une fois $\rot \mathbf{b}$ et $\grad\psi^0$ connus, on les additionne pour avoir $\mathbf{a}$.\\
On peut maintenant calculer les coefficients $c_i$ avec (\ref{fvspec}), ce qui permet de reconstruire $\mathbf{u}$ en utilisant (\ref{u}).\\
Connaissant $\mathbf{a}$ et $\mathbf{u}$, on peut les additionner pour avoir $\mathbf{v}$.\\
Une fois la vitesse connue, on utilise (\ref{fvq}) pour trouver la pression.\\

On a ainsi trouver notre solution $(\mathbf{v},p)$.\\
Si l'on veut résoudre le problème mixte (\ref{pbbcurl}) pour trouver $\mathbf{b}$, la figure \ref{org} montre les imbrications des différents problèmes, tandis que si l'on veut trouver $\mathbf{b}$ dans $\DD^1$, on utilise le schèma illustré par la figure \ref{org2}.\\

\begin{figure}
\centering
\begin{tikzpicture}[scale=\taille]
\node[draw,scale=\taille,fill=green!50] (di) at (15,4) {Données initiales} ;
\node[draw,scale=\taille,fill=gray!50] (pb) at (15,3) {Problèmes à résoudre} ;
\node[draw,scale=\taille,fill=blue!50] (si) at (15,2) {Solutions intermédiaires} ;
\node[draw,scale=\taille,fill=yellow!50] (sim) at (15,1) {Sol. inter. dépendantes de la géomètrie} ;
\node[draw,scale=\taille,fill=red!50] (sf) at (15,0) {Solutions finales} ;

\node[draw,scale=\taille,fill=green!50] (a0) at (0,0) {$\alpha_0$} ;
\node[draw,scale=\taille,fill=green!50] (a1) at (4,0) {$\alpha_1$} ;
\node[draw,scale=\taille,fill=gray!50,label={[xshift=1.0cm](\ref{pbpsi0})}] (pbpsi0) at (0,-2)
{$\begin{aligned}
-\laplace\psi^0&=0\\
\grad\psi^0\cdot n\restr &= \alpha_0
\end{aligned}$} ;
\node[draw,scale=\taille,fill=gray!50,label={[xshift=1.1cm](\ref{pbbcurl})}] (pbb) at (4,-3)
{$\begin{aligned}
\rott \mathbf{b} &= \grad\psi^1\\
\diverg \mathbf{b} &=0\\
\mathbf{b}\cdot n\restr &= 0\\
\rot \mathbf{b}\cdot n\restr &= 0\\
\rott \mathbf{b}\cdot n\restr &= \alpha_1
\end{aligned}$} ;
\node[draw,scale=\taille,fill=gray!50,label={[xshift=1.0cm](\ref{pbeigen})}] (pbeigen) at (12,-3)
{$\begin{aligned}
\rott \mathbf{g_i} = \lambda_i^2\mathbf{g_i}\\
\mathbf{g_i}\cdot n\restr = 0\\
\rot \mathbf{g_i}\cdot n\restr = 0\\
\rott \mathbf{g_i}\cdot n\restr = 0
\end{aligned}$} ;
\node[draw,scale=\taille,fill=blue!50] (psi0) at (0,-6) {$\grad\psi^0$} ;
\node[draw,scale=\taille,fill=blue!50] (b) at (4,-6) {$\rot \mathbf{b}$} ;
\node[draw,scale=\taille,fill=yellow!50] (lambda) at (11,-6) {$\lambda_i$} ;
\node[draw,scale=\taille,fill=yellow!50] (gi) at (13,-6) {$\mathbf{g_i}$} ;
\node[draw,scale=\taille,fill=gray!50,label={[xshift=1.3cm](\ref{gi0})}] (pbgi0) at (16,-6)
{$\begin{aligned}
\grad(\diverg \mathbf{g_i^0})-\laplace \mathbf{g_i^0} = -\laplace \mathbf{g_i}\\
\mathbf{g_i^0}\restr = 0
\end{aligned}$} ;
\node[draw,scale=\taille,fill=gray!50,label={[xshift=1.0cm](\ref{a})}] (pba) at (2,-8) {$\mathbf{a} = \rot \mathbf{b} + \grad\psi^0$} ;
\node[draw,scale=\taille,fill=green!50] (f) at (7,-8) {$f$} ;
\node[draw,scale=\taille,fill=green!50] (a2) at (8,-8) {$\alpha_2$} ;
\node[draw,scale=\taille,fill=green!50] (ck0) at (9,-8) {$c_k^0$} ;
\node[draw,scale=\taille,fill=yellow!50] (gi0) at (16,-8) {$\mathbf{g_i^0}$} ;
\node[draw,scale=\taille,fill=gray!50,label={[xshift=0.8cm](\ref{psi})}] (pbpsi) at (16,-10)
{$\begin{aligned}
-\laplace\psi_i = \diverg \mathbf{g_i^0}\\
\grad\psi_i\cdot n\restr = 0
\end{aligned}$} ;
\node[draw,scale=\taille,fill=yellow!50] (psi) at (16,-12) {$\psi_i$} ;
\node[draw,scale=\taillem,fill=blue!50] (a) at (2,-12) {$\mathbf{a}$} ;
\node[draw,scale=\taille,fill=gray!50,label={[xshift=2.9cm](\ref{fvspec})}] (pbs) at (9,-12)
{$\begin{aligned}
\frac{\partial c_k}{\partial t} &+ \sum_i\sum_j c_i\lambda_i c_j(\mathbf{g_i}\times \mathbf{g_j}, \mathbf{g_k}) \\
&+ \sum_i c_i\lambda_i(\mathbf{g_i}\times \mathbf{a},\mathbf{g_k}) + \sum_i c_i((\rot \mathbf{a})\times \mathbf{g_i}, \mathbf{g_k}) \\
&+ \frac{1}{Re}c_k\lambda_k^2 = (\mathbf{h_a},\mathbf{g_k}) + \frac{1}{Re}\langle\alpha_2,\psi_k\rangle\\
&c_k(0)=c_k^0
\end{aligned}
$} ;
\node[draw,scale=\taille,fill=blue!50] (ck) at (9,-15) {$c_k$} ;
\node[draw,scale=\taille,fill=gray!50,label={[xshift=0.7cm](\ref{u})}] (pbu) at (9,-16) {$\mathbf{u}=\sum c_kg_k$} ;
\node[draw,scale=\taillem,fill=blue!50] (u) at (9,-17) {$\mathbf{u}$} ;
\node[draw,scale=\taille,fill=gray!50,label={[xshift=0.6cm](\ref{v})}] (pbv) at (3,-18) {$\mathbf{v}=\mathbf{a}+\mathbf{u}$} ;
\node[draw,scale=\tailleg,fill=red!50] (v) at (3,-19) {$\mathbf{v}$} ;
\node[draw,scale=\taille,fill=gray!50,label={[xshift=3.1cm](\ref{q})}] (pbq) at (9,-19)
{$\begin{aligned}
-\laplace q = \diverg((\rot \mathbf{v})\times \mathbf{v}) - \diverg \mathbf{f}\\
\grad q\cdot n\restr =  \mathbf{f}\cdot n\restr - \frac{\partial\alpha_0}{\partial t} - ((\rot \mathbf{v})\times \mathbf{v})\cdot n\restr - \frac{\alpha_2}{Re}
\end{aligned}$} ;
\node[draw,scale=\tailleg,fill=red!50] (q) at(15,-19) {$p$} ;

\draw[->,>=latex] (a0) -- (pbpsi0);
\draw[->,>=latex] (a1) -- (pbb);
\draw[->,>=latex] (pbpsi0) -- (psi0);
\draw[->,>=latex] (pbb) -- (b);
\draw[->,>=latex] (b) -- (pba);
\draw[->,>=latex] (psi0) -- (pba);
\draw[->,>=latex] (pba) -- (a);
\draw[->,>=latex] (pbeigen) -- (lambda);
\draw[->,>=latex] (pbeigen) -- (gi);
\draw[->,>=latex] (gi) -- (pbgi0);
\draw[->,>=latex] (pbgi0) -- (gi0);
\draw[->,>=latex] (gi0) -- (pbpsi);
\draw[->,>=latex] (pbpsi) -- (psi);
\draw[->,>=latex] (a) -- (pbs);
\draw[->,>=latex] (f) -- (pbs);
\draw[->,>=latex] (a2) -- (pbs);
\draw[->,>=latex] (ck0) -- (pbs);
\draw[->,>=latex] (lambda) -- (pbs);
\draw[->,>=latex] (gi) -- (pbs);
\draw[->,>=latex] (psi) -- (pbs);
\draw[->,>=latex] (pbs) -- (ck);
\draw[->,>=latex] (ck) -- (pbu);
\draw[->,>=latex] (pbu) -- (u);
\draw[->,>=latex] (u) -- (pbv);
\draw[->,>=latex] (a) -- (pbv);
\draw[->,>=latex] (pbv) -- (v);
\draw[->,>=latex] (v) -- (pbq);
\draw[->,>=latex] (pbq) -- (q);
\end{tikzpicture}
\caption{Organigramme de synthèse}
\label{org}
\end{figure}

\begin{figure}
\centering
\begin{tikzpicture}[scale=\taille]
\node[draw,scale=\taille,fill=green!50] (di) at (15,4) {Données initiales} ;
\node[draw,scale=\taille,fill=gray!50] (pb) at (15,3) {Problèmes à résoudre} ;
\node[draw,scale=\taille,fill=blue!50] (si) at (15,2) {Solutions intermédiaires} ;
\node[draw,scale=\taille,fill=yellow!50] (sim) at (15,1) {Sol. inter. dépendantes de la géomètrie} ;
\node[draw,scale=\taille,fill=red!50] (sf) at (15,0) {Solutions finales} ;

\node[draw,scale=\taille,fill=gray!50,label={[xshift=1.0cm](\ref{pbeigen})}] (pbeigen) at (9,0)
{$\begin{aligned}
\rott \mathbf{g_i} = \lambda_i^2\mathbf{g_i}\\
\mathbf{g_i}\cdot n\restr = 0\\
\rot \mathbf{g_i}\cdot n\restr = 0\\
\rott \mathbf{g_i}\cdot n\restr = 0
\end{aligned}$} ;
\node[draw,scale=\taille,fill=green!50] (a0) at (0,-1.5) {$\alpha_0$} ;
\node[draw,scale=\taille,fill=green!50] (a1) at (4,-1.5) {$\alpha_1$} ;
\node[draw,scale=\taille,fill=yellow!50] (lambda) at (8,-3) {$\lambda_i$} ;
\node[draw,scale=\taille,fill=yellow!50] (gi) at (10,-3) {$\mathbf{g_i}$} ;
\node[draw,scale=\taille,fill=gray!50,label={[xshift=1.3cm](\ref{gi0})}] (pbgi0) at (14,-3)
{$\begin{aligned}
\grad(\diverg \mathbf{g_i^0})-\laplace \mathbf{g_i^0} = -\laplace \mathbf{g_i}\\
\mathbf{g_i^0}\restr = 0
\end{aligned}$} ;
\node[draw,scale=\taille,fill=gray!50,label={[xshift=1.0cm](\ref{pbpsidiv})}] (pbpsi0) at (0,-3.5)
{$\begin{aligned}
l^0=\grad\psi^0\\
\diverg\psi^0=0\\
l^0\cdot n\restr &= \alpha_0
\end{aligned}$} ;
\node[draw,scale=\taille,fill=gray!50,label={[xshift=1.1cm](\ref{pbpsidiv})}] (pbpsi1) at (4,-3.5)
{$\begin{aligned}
l^1=\grad\psi^1\\
\diverg\psi^1=0\\
l^1\cdot n\restr &= \alpha_1
\end{aligned}$} ;
\node[draw,scale=\taille,fill=yellow!50] (gi0) at (14,-4.5) {$\mathbf{g_i^0}$} ;
\node[draw,scale=\taille,fill=blue!50] (psi0) at (0,-5.5) {$\grad\psi^0$} ;
\node[draw,scale=\taille,fill=blue!50] (psi1) at (4,-5.5) {$\grad\psi^1$} ;
\node[draw,scale=\taille,fill=gray!50,label={[xshift=0.8cm](\ref{psi})}] (pbpsi) at (14,-6)
{$\begin{aligned}
-\laplace\psi_i = \diverg \mathbf{g_i^0}\\
\grad\psi_i\cdot n\restr = 0
\end{aligned}$} ;
\node[draw,scale=\taille,fill=yellow!50] (psi) at (14,-7.5) {$\psi_i$} ;
\node[draw,scale=\taille,fill=gray!50,label={[xshift=-1.6cm](\ref{discbd1})}] (pbb) at (4,-7.5) 
{$
d_k\lambda_k^2 = (\grad\psi^1,\mathbf{g_k}) - \langle\alpha_1,\psi_k\rangle
$} ;
\node[draw,scale=\taille,fill=blue!50] (b) at (4,-8.5) {$\rot \mathbf{b}$} ;
\node[draw,scale=\taille,fill=gray!50,label={[xshift=1.0cm](\ref{a})}] (pba) at (2,-10.5) {$\mathbf{a} = \rot \mathbf{b} + \grad\psi^0$} ;
\node[draw,scale=\taille,fill=green!50] (f) at (6,-9) {$f$} ;
\node[draw,scale=\taille,fill=green!50] (a2) at (7,-9) {$\alpha_2$} ;
\node[draw,scale=\taille,fill=green!50] (ck0) at (8,-9) {$c_k^0$} ;
\node[draw,scale=\taillem,fill=blue!50] (a) at (2,-12) {$\mathbf{a}$} ;
\node[draw,scale=\taille,fill=gray!50,label={[xshift=2.9cm](\ref{fvspec})}] (pbs) at (9,-12)
{$\begin{aligned}
\frac{\partial c_k}{\partial t} &+ \sum_i\sum_j c_i\lambda_i c_j(\mathbf{g_i}\times \mathbf{g_j}, \mathbf{g_k}) \\
&+ \sum_i c_i\lambda_i(\mathbf{g_i}\times \mathbf{a},\mathbf{g_k}) + \sum_i c_i((\rot \mathbf{a})\times \mathbf{g_i}, \mathbf{g_k}) \\
&+ \frac{1}{Re}c_k\lambda_k^2 = (\mathbf{h_a},\mathbf{g_k}) + \frac{1}{Re}\langle\alpha_2,\psi_k\rangle\\
&c_k(0)=c_k^0
\end{aligned}
$} ;
\node[draw,scale=\taille,fill=blue!50] (ck) at (9,-15) {$c_k$} ;
\node[draw,scale=\taille,fill=gray!50,label={[xshift=0.7cm](\ref{u})}] (pbu) at (9,-16) {$\mathbf{u}=\sum c_kg_k$} ;
\node[draw,scale=\taillem,fill=blue!50] (u) at (9,-17) {$\mathbf{u}$} ;
\node[draw,scale=\taille,fill=gray!50,label={[xshift=0.6cm](\ref{v})}] (pbv) at (3,-18) {$\mathbf{v}=\mathbf{a}+\mathbf{u}$} ;
\node[draw,scale=\tailleg,fill=red!50] (v) at (3,-19) {$\mathbf{v}$} ;
\node[draw,scale=\taille,fill=gray!50,label={[xshift=3.1cm](\ref{q})}] (pbq) at (9,-19)
{$\begin{aligned}
-\laplace q = \diverg((\rot \mathbf{v})\times \mathbf{v}) - \diverg \mathbf{f}\\
\grad q\cdot n\restr =  \mathbf{f}\cdot n\restr - \frac{\partial\alpha_0}{\partial t} - ((\rot \mathbf{v})\times \mathbf{v})\cdot n\restr - \frac{\alpha_2}{Re}
\end{aligned}$} ;
\node[draw,scale=\tailleg,fill=red!50] (q) at(15,-19) {$p$} ;

\draw[->,>=latex] (a0) -- (pbpsi0);
\draw[->,>=latex] (a1) -- (pbpsi1);
\draw[->,>=latex] (pbpsi0) -- (psi0);
\draw[->,>=latex] (pbpsi1) -- (psi1);
\draw[->,>=latex] (psi1) -- (pbb);
\draw[->,>=latex] (psi) -- (pbb);
\draw[->,>=latex] (lambda) -- (pbb);
\draw[->,>=latex] (gi) -- (pbb);
\draw[->,>=latex] (pbb) -- (b);
\draw[->,>=latex] (b) -- (pba);
\draw[->,>=latex] (psi0) -- (pba);
\draw[->,>=latex] (pba) -- (a);
\draw[->,>=latex] (pbeigen) -- (lambda);
\draw[->,>=latex] (pbeigen) -- (gi);
\draw[->,>=latex] (gi) -- (pbgi0);
\draw[->,>=latex] (pbgi0) -- (gi0);
\draw[->,>=latex] (gi0) -- (pbpsi);
\draw[->,>=latex] (pbpsi) -- (psi);
\draw[->,>=latex] (a) -- (pbs);
\draw[->,>=latex] (f) -- (pbs);
\draw[->,>=latex] (a2) -- (pbs);
\draw[->,>=latex] (ck0) -- (pbs);
\draw[->,>=latex] (lambda) -- (pbs);
\draw[->,>=latex] (gi) -- (pbs);
\draw[->,>=latex] (psi) -- (pbs);
\draw[->,>=latex] (pbs) -- (ck);
\draw[->,>=latex] (ck) -- (pbu);
\draw[->,>=latex] (pbu) -- (u);
\draw[->,>=latex] (u) -- (pbv);
\draw[->,>=latex] (a) -- (pbv);
\draw[->,>=latex] (pbv) -- (v);
\draw[->,>=latex] (v) -- (pbq);
\draw[->,>=latex] (pbq) -- (q);
\end{tikzpicture}
\caption{Organigramme de synthèse}
\label{org2}
\end{figure}


