\section{Problème}
Notation : 
\begin{eqnarray*}
divergence(v)=\nabla\cdot v\\
curl(v)=\nabla\times v\\
curl(curl(v))=\nabla^2\times v
\end{eqnarray*}

Nous cherchons $(v,p)$, correspondant respectivement à la vitesse et à la pression, solutions de l'équation de Navier-Stokes incompressibles adimensionnalisées dans $Q_T=\Omega\times[0,T]$ un ouvert de $\R^3$ et $\partial\Omega$ sa frontière, avec une condition initiale et des conditions aux limites d'imperméabilité généralisée :
\begin{eqnarray}
\label{depart}
\left\{
\begin{aligned}
&\frac{\partial v}{\partial t} + (\nabla\times  v)\times v + \nabla q + \frac{1}{Re}\nabla^2\times  v-f = 0\\
&\nabla\cdot v = 0\\
&v\big\rvert_{t=0} = v_0\\
&v\cdot n\big\rvert_{\partial\Omega} = \alpha_0\\
&(\nabla\times  v)\cdot n\big\rvert_{\partial\Omega} = \alpha_1\\
&(\nabla^2\times  v)\cdot n\big\rvert_{\partial\Omega} = \alpha_2
\end{aligned}
\right.
\end{eqnarray}
où $q = \frac{|v|^2}{2}+p$.\\

On définit l'espace
\[
\LL^2_\sigma(\Omega) = \{v \in \LL^2(\Omega)_| \nabla\cdot v = 0\text{ et }v\cdot n\big\rvert_{\partial\Omega} = 0 \}
\]
Alors, on cherche les solutions dans l'espace
\[
D^1(\Omega) = \{v\in W^{1,2}(\Omega) \cup \LL^2(\Omega) | (\nabla\times v\cdot n)\big\rvert_{\partial\Omega} = 0  \}
\]

Afin de pouvoir résoudre ce problème, on veut se ramener à un problème où $\alpha_0=\alpha_1=0$. On introduit donc une fonction $a$ dans $W^{1,2}(\Omega)$ pour relever le problème tel que :
\begin{eqnarray}
\label{aegal}
\left\{
\begin{aligned}
&a=\nabla\times b+\nabla\psi_0\\
&\nabla\cdot a =0\\
&a\cdot n\big\rvert_{\partial\Omega} = \alpha_0\\
&(\nabla\times a)\cdot n\big\rvert_{\partial\Omega} = \alpha_1
\end{aligned}
\right.
\end{eqnarray}
Comme $\nabla\cdot a=0$, on obtient l'équation suivante pour $\psi_0\in W^{1,2}(\Omega)$ :
\begin{eqnarray}
\label{psi0}
\left\{
\begin{aligned}
&-\Delta\psi_0 = 0\\
&\nabla\psi_0\cdot n\big\rvert_{\partial\Omega}=\alpha_0
\end{aligned}
\right.
\end{eqnarray}

Pour connaitre $\nabla\times b$, il faut résoudre le problème :
\begin{eqnarray}
\label{curlb}
\left\{
\begin{aligned}
&\nabla^2\times b = \nabla\psi_1\\
&\nabla\cdot b = 0\\
&b\cdot n\big\rvert_{\partial\Omega} = 0\\
&\nabla\times b\cdot n\big\rvert_{\partial\Omega} = 0\\
&\nabla^2\times b\cdot n\big\rvert_{\partial\Omega} = \alpha_1
\end{aligned}
\right.
\end{eqnarray}

Une fois $\psi_0$ et $\nabla\times b$ connus, cela permet de retrouver $a$.\\

Ainsi, on peut décomposer $v=a+u$ et l'équatioin (\ref{depart}) devient :
\[
\frac{\partial(u+a)}{\partial t}+(\nabla\times(u+a))\times(u+a) + \nabla q + \frac{1}{Re}\nabla^2\times(u+a) - f = 0
\]
Ce qui donne :
\[
\frac{\partial u}{\partial t}+\frac{\partial a}{\partial t} + (\nabla\times u+\nabla\times a)\times(u+a) +\frac{1}{Re}(\nabla^2\times u+\nabla^2\times a) - f = 0
\]
Comme $\nabla^2\times a = 0$ et en notant $h=f-\frac{\partial a}{\partial t} - (\nabla\times a)\times a$, on obtient le problème suivant où l'objectif est donc de trouver $(u,q)\in Q_T$ tel que :
\begin{eqnarray}
\label{uci}
\left\{
\begin{aligned}
&\frac{\partial u}{\partial t} + (\nabla\times u)\times u + (\nabla\times u)\times a +(\nabla\times a)\times u + \nabla q  +\frac{1}{Re}\nabla^2\times  u - h = 0\\
&\nabla\cdot u = 0\\
&u\big\rvert_{t=0} = v_0 - a(0,\cdot)\\
&u\cdot n\big\rvert_{\partial\Omega} = 0\\
&(\nabla\times u)\cdot n\big\rvert_{\partial\Omega} = 0\\
&(\nabla^2\times  u)\cdot n\big\rvert_{\partial\Omega} = \alpha_2
\end{aligned}
\right.
\end{eqnarray}

Comme $u$ appartient à $D^1(\Omega)$, on peut décomposer $u$ sous la forme :
\[
u(t,\cdot) = \sum_{i=1}^{\infty} c_i(t)g_i(\cdot)
\]
On va chercher une valeur approché de $u$ qui se présente sous la forme :
\[
u_M(t,\cdot) = \sum_{i=1}^{M} c_i(t)g_i(\cdot)
\]

Les fonctions $g_i$ forment une base de l'espace $D^1(\Omega) = D(curl_{imperm})$. Cette base est constitué des fonctions propres de l'opérateur rotationnel. Elles sont donc solutions du problème suivant :
\begin{eqnarray}
\label{curlcurl}
(\lambda_i,g_i)\in\R\times D^1(\Omega)\quad \left\{
\begin{aligned}
&\nabla^2\times  g_i = \lambda_i g_i\\
&g_i\cdot n\big\rvert_{\partial\Omega} = 0\\
&\nabla\times g_i\cdot n\big\rvert_{\partial\Omega} = 0\\
&\nabla^2\times  g_i\cdot n\big\rvert_{\partial\Omega} = 0
\end{aligned}
\right.
\end{eqnarray}

Toutes fonctions dans $D^1(\Omega)$ peut s'écrire $g_i=g_i^0+\nabla\psi_i$ avec $g_i^0\big\rvert_{\partial\Omega} = 0$ et $ \nabla\psi_i\cdot n\big\rvert_{\partial\Omega} = 0$.\\
Ainsi en décomposant dans (\ref{uci}), $u$ par sa somme, on obtient le problème suivant où l'on cherche les coefficients $c_i$ :\\
\begin{eqnarray}
\label{cigi}
\left\{
\begin{aligned}
&\frac{dc_k}{dt}+\sum_{i=1}^M\sum_{j=1}^M c_i\lambda_i c_j r_{ijk} + \sum_{i=1}^M c_i\lambda_i r_{ik}^a + \sum_{i=1}^M c_j r_{jk}^{\nabla\times a} + \lambda_k^2 c_k = r_k^h - r_k^{\alpha_2}\\
&c_k(0)=c_k^0
\end{aligned}
\right.\quad
1\leq k\leq M
\end{eqnarray}
Où\\
\[
\begin{aligned}
&r_{ijk} = \langle g_i\times g_j, g_k\rangle_\Omega,\\
&r_{ik}^a=\langle g_i\times a, g_k\rangle_\Omega,\\
&r_{jk}^{\nabla\times a}=\langle (\nabla\times a)\times g_j, g_k\rangle_\Omega,\\
&r_k^h=\langle h,g_k\rangle_\Omega,\\
&r_k^{\alpha_2} = \langle \alpha_2,\psi_k\rangle_{\partial\Omega}.
\end{aligned}
\]

Pour résumer, on doit donc :
\begin{itemize}
\item générer la base $g_i$ des fonctions propres de l'opérateur curl en résolvant l'équation (\ref{curlcurl}) comme expliqué dans le chapitre \ref{eigen}.
\item les décomposer en $g_i=g_i^0+\nabla\psi_i$ comme montré dans \ref{decomp}.
\item trouver $u$, pour cela on calcul tous les $r_x^y$ et on résous l'équation (\ref{cigi}) pour $1\leq k\leq M$ afin de connaitre les $c_k$, cela est expliqué dans \ref{spectre}.
\item trouver $a$ pour pouvoir décomposer $v$ en $u+a$, pour cela, on résout les équations (\ref{psi0}) et (\ref{curlb}). Ce qui permet de trouver $a$ grâce à (\ref{aegal}). Cette partie est détaillé dans \ref{relev}.
\item recomposer $v=u+a$, la solution du problème (\ref{depart}).
\end{itemize}

\subsection{Problème aux valeurs propres}
\label{eigen}

On s'intéresse ici plus particulièrement au problème (\ref{curlcurl}).
\begin{eqnarray*}
(\lambda_i,g_i)\in\R\times D^1(\Omega)\quad \left\{
\begin{aligned}
&\nabla^2\times  g_i = \lambda_i g_i\\
&g_i\cdot n\big\rvert_{\partial\Omega} = 0\\
&\nabla\times g_i\cdot n\big\rvert_{\partial\Omega} = 0\\
&\nabla^2\times  g_i\cdot n\big\rvert_{\partial\Omega} = 0
\end{aligned}
\right.
\end{eqnarray*}
On note d'abord que tout élément $\varphi\in D^1(\Omega)$ peut s'écrire de la manière suivante :
\[
\varphi = \varphi_0 + \nabla\phi\text{ et } \varphi\big\rvert_{\partial\Omega} = \nabla\phi
\]

On va maintenant chercher la formulation variationnelle de ce problème.\\
Soit $g\in D^1(\Omega)$ solution de (6), alors pour tout $\varphi\in D^1(\Omega)$ nous avons :
\[
\int_\Omega \nabla^2\times g\cdot\varphi\ dX = \int_\Omega\lambda g\cdot \varphi\ dX
\]
puis en intégrant par partie :
\[
\int_\Omega \nabla\times g\cdot \nabla\times\varphi\ dX + \int_{\partial\Omega} (\nabla\times g\times \varphi)\cdot n\ d\Gamma = \lambda\int_\Omega g\cdot \varphi\ dX
\]
or sur $\partial\Omega,\quad \varphi\big\rvert_{\partial\Omega}=\nabla\phi$, d'où : 
\[
\int_\Omega \nabla\times g\cdot \nabla\times\varphi\ dX + \int_{\partial\Omega} (\nabla\times g\times \nabla\phi)\cdot n\ d\Gamma = \lambda\int_\Omega g\cdot \varphi\ dX
\]
En utilisant le théorème de flux-divergence aussi appelé théorème de Green-Ostrogradski :
\[
\int_\Omega \nabla\times g\cdot \nabla\times\varphi\ dX + \int_\Omega \nabla\cdot(\nabla\times g\times \nabla\phi)\ dX = \lambda\int_\Omega g\cdot \varphi\ dX
\]
En utilisant la formule $div(F\times G) = G\cdot \nabla\times F - F\cdot \nabla\times G$, on a :
\[
\int_\Omega \nabla\times g\cdot \nabla\times\varphi\ dX + \int_\Omega \nabla\phi\cdot \nabla^2\times g\ dX -\int_\Omega \nabla\times g\cdot \nabla\times\nabla\phi\ dX  = \lambda\int_\Omega g\cdot \varphi\ dX
\]
Comme le rotationel d'un gradient est nul, on a :
\[
\int_\Omega \nabla\times g\cdot \nabla\times\varphi\ dX + \int_\Omega \nabla\phi\cdot \nabla^2\times g\ dX  = \lambda\int_\Omega g\cdot \varphi\ dX
\]
En intégrant le deuxième terme par partie, on obtient : 
\[
\int_\Omega \nabla\times g\cdot \nabla\times\varphi\ dX + \int_{\partial\Omega} \phi(\nabla^2\times g\cdot n)\ d\Gamma - \int_\Omega \phi(\nabla\cdot(\nabla^2\times g))\ dX  = \lambda\int_\Omega g\cdot \varphi\ dX
\]
Comme $\nabla^2\times  g_i\cdot n\big\rvert_{\partial\Omega} = 0$, le deuxième terme s'annule et comme la divergence d'un rotationel est nulle, le troisième terme s'annule aussi, ce qui laisse pour tous $\phi\in D^1(\Omega)$ :
\begin{eqnarray}
\label{fveigen}
\int_\Omega \nabla\times g\cdot \nabla\times\varphi\ dX = \lambda\int_\Omega g\cdot \varphi\ dX
\end{eqnarray}

\subsection{Décomposition des $g_i$}
\label{decomp}

Une fois les fonctions propres connues, on a besoin de les décomposer en $g_i=g_i^0+\nabla\psi_i$. On applique donc le rotationnel du rotationnel sur cette relation.\\
\[
\nabla^2\times g_i = \nabla^2\times g_i^0 +\nabla^2\times\nabla\psi_i
\]
Le dernier terme est nul car c'est le rotationnel d'un gradient. On utilise la formule $\nabla^2\times v=\nabla(\nabla\cdot v)-\Delta v$ sur les deux termes restants :
\[
\nabla(\nabla\cdot g_i)-\Delta g_i = \nabla(\nabla\cdot g_i^0)-\Delta g_i^0
\]
Comme la divergence des $g_i$ est nul, on obtient :
\[
-\Delta g_i = \nabla(\nabla\cdot g_i^0)-\Delta g_i^0
\]
D'autre part, on sait que la divergence des $g_i$ est nulle, on a donc :
\[
\nabla\cdot (g_i^0+\nabla\psi_i)=0
\]
Ce qui nous donne :
\[
\nabla\cdot g_i^0=-\Delta\psi_i
\]
En réinjectant cette égalité dans l'équation obtenue précédemment, on arrive à :
\begin{eqnarray*}
\left\{
\begin{aligned}
-\nabla(\Delta\psi_i)-\Delta g_i^0 &= -\Delta g_i\\
g_i^0\big\rvert_{\partial\Omega} &= 0\\
\nabla\psi_i\cdot n\big\rvert_{\partial\Omega} &= 0
\end{aligned}
\right.
\end{eqnarray*}

\subsection{Problème spectral}
\label{spectre}

En partant de l'équation (\ref{uci}), on remplace $u$ par la somme $\sum_{i=1}^Mc_ig_i$ qui tend vers $u$ lorsque $M$ tend vers l'infini.
\begin{align*}
\frac{\partial (\sum_{i=1}^Mg_ic_i)}{\partial t} &+ \left(\nabla\times\left(\sum_{i=1}^Mc_ig_i\right)\right)\times\left(\sum_{i=1}^Mc_ig_i\right)  + \left(\nabla\times\left(\sum_{i=1}^Mc_ig_i\right)\right)\times a &\\
&+ \left(\nabla\times a\right)\times\left(\sum_{i=1}^Mc_ig_i\right) + \nabla q  + \frac{1}{Re}\nabla^2\times\left(\sum_{i=1}^Mc_ig_i\right) - h = 0
\end{align*}
En utilisant le fait que $\nabla\times g_i=\lambda_i g_i$, on arrive à :
\begin{align*}
\sum_{i=1}^Mg_i\frac{\partial c_i}{\partial t} &+ \left(\sum_{i=1}^M c_i\lambda_i g_i\right)\times\left(\sum_{i=1}^Mc_ig_i\right) + \left(\sum_{i=1}^Mc_i\lambda_ig_i\right)\times a &\\
&+ \sum_{i=1}^Mc_i\left(\nabla\times a\right)g_i +\nabla q + \frac{1}{Re}\sum_{i=1}^Mc_i\lambda_i^2g_i - h = 0
\end{align*}
en intégrant par chaque $g_k$ de la base de $D^1(\Omega)$, et en intégrant, on obtient :
\begin{align*}
\sum_{i=1}^M\frac{\partial c_i}{\partial t}r_{ik} +\sum_{i=1}^M\sum_{j=1}^Mc_ic_j\lambda_ir_{ijk} +\sum_{i=1}^Mc_i\lambda_ir_{ik}^a +\sum_{i=1}^Mc_ir_{ik}^{\nabla\times a} + \frac{1}{Re}\sum_{i=1}^Mc_i\lambda_i^2r_{ik} = r_k^h
\end{align*}

\subsection{Relèvement}
\label{relev}

Le problème (\ref{aegal}) sert à retrouver des conditions aux limites pour lesquels on sait résoudre le problème. Pour trouver $a$, il nous faut d'abord résoudre les problèmes (\ref{psi0}) et (\ref{curlb}).\\

(\ref{psi0}) est un problème de Laplace, pour obtenir sa forme variationnelle, on multiplie par une fonction test et on intègre :
\[
\int_\Omega \Delta\psi_0 \varphi = 0
\]
On utilise ensuite la formule de Green pour parvenir à :
\[
-\int_\Omega \nabla\psi\nabla\varphi + \int_{\partial\Omega} \nabla\psi_0\cdot n\ \varphi = 0
\]
Or, $\nabla\psi_0\cdot n = \alpha_0$ sur $\partial\Omega$, on obtient donc la forme variationnelle suivante :
\begin{eqnarray}
\label{fvpsi}
-\int_\Omega \nabla\psi_0\ \nabla\varphi + \int_{\partial\Omega} \alpha_0\ \varphi = 0
\end{eqnarray}
Cela permet de connaitre $\nabla\psi_0$.\\

On résout maintenant le problème (\ref{curlb}) :
\begin{eqnarray*}
\left\{
\begin{aligned}
&\nabla^2\times b = \nabla\psi_1\\
&\nabla\cdot b = 0\\
&b\cdot n\big\rvert_{\partial\Omega} = 0\\
&\nabla\times b\cdot n\big\rvert_{\partial\Omega} = 0\\
&\nabla^2\times b\cdot n\big\rvert_{\partial\Omega} = \alpha_1
\end{aligned}
\right.
\end{eqnarray*}
Toujours de la même manière, on multiplie par une fonction test et on intégre :
\[
\int_\Omega (\nabla^2\times b) \varphi = \int_\Omega (\nabla\psi_1)\varphi
\]
En intégrant par partie le premier terme et en utilisant la formule de Green sur le second, on obtient :
\begin{eqnarray}
\label{fvrelev}
\int_\Omega (\nabla\times b)(\nabla\times\varphi) - \int_{\partial\Omega} (\nabla\times b)(\varphi\cdot n) +\int_\Omega \psi_1(\nabla\cdot\varphi) - \int_{\partial\Omega} \psi_1(\varphi\cdot n) = 0
\end{eqnarray}

