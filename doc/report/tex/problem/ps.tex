\section{Problème spectral}
\label{spectre}
Nous connaissons maintenant $\mathbf{a}$ et les couples $(\lambda_i,\mathbf{g}_i)_{i=1,\dots,M}$, on a donc toutes les briques pour trouver les coefficients $c_i$ de $\mathbf{u}_M$ l'approximation de $\mathbf{u}$ :
\[\mathbf{u}_M= \sum_{i=1}^M c_i\mathbf{g}_i\mbox{ tel que } \mathbf{u}_M\underset{M\rightarrow\infty}{\longrightarrow} \mathbf{u}=\sum_{i=1}^\infty c_i\mathbf{g}_i\]
Pour cela, on utilise l'approximation du problème \ref{pbc}.
\begin{align*}
\sum_{i=1}^M\frac{\partial c_i}{\partial t}\mathbf{g}_i &+ \sum_{i=1}^M\sum_{j=1}^Mc_ic_j(\curl\mathbf{g}_i\times \mathbf{g_j}) + \sum_{i=1}^Mc_i(\curl\mathbf{g}_i\times \mathbf{a})\\
& +  \sum_{i=1}^Mc_i((\curl\mathbf{a})\times \mathbf{g}_i) + \grad \pi_{\mathbf{a}} +\frac{1}{Re}\sum_{i=1}^Mc_i\curll\mathbf{g}_i - \mathbf{f_a} = 0\\
\end{align*}

Pour obtenir la forme variationnel du problème, on multiplie par une fonction test $\bm{\varphi}\in D^1(\Omega)$ et on intègre :
\begin{align*}
\sum_{i=1}^M\frac{\partial c_i}{\partial t}\int_\Omega\mathbf{g}_i\cdot\bm{\varphi} &+ \sum_{i=1}^M\sum_{j=1}^Mc_ic_j\int_\Omega(\curl\mathbf{g}_i\times \mathbf{g_j})\cdot\bm{\varphi} + \sum_{i=1}^Mc_i\int_\Omega(\curl\mathbf{g}_i\times \mathbf{a})\cdot\bm{\varphi}\\
& +  \sum_{i=1}^Mc_i\int_\Omega((\curl\mathbf{a})\times \mathbf{g}_i)\cdot\bm{\varphi} + \int_\Omega\grad \pi_{\mathbf{a}}\cdot\bm{\varphi} +\frac{1}{Re}\sum_{i=1}^Mc_i\int_\Omega\curll\mathbf{g}_i\cdot\bm{\varphi} - \int_\Omega\mathbf{f_a}\cdot\bm{\varphi} = 0\\
\end{align*}

En utilisant une intégration par partie sur le terme contenant $\grad\pi_{\mathbf{a}}$, on arrive à :
\begin{align*}
\sum_{i=1}^M\frac{\partial c_i}{\partial t}\int_\Omega\mathbf{g}_i\cdot\bm{\varphi} &+ \sum_{i=1}^M\sum_{j=1}^Mc_ic_j\int_\Omega(\curl\mathbf{g}_i\times \mathbf{g_j})\cdot\bm{\varphi} + \sum_{i=1}^Mc_i\int_\Omega(\curl\mathbf{g}_i\times \mathbf{a})\cdot\bm{\varphi}\\
& +  \sum_{i=1}^Mc_i\int_\Omega((\curl\mathbf{a})\times \mathbf{g}_i)\cdot\bm{\varphi} + \int_\Omega\pi_{\mathbf{a}}\div\bm{\varphi} + \int_{\partial\Omega}\pi_{\mathbf{a}}(\bm{\varphi}\cdot\mathbf{n})\\
& \frac{1}{Re}\sum_{i=1}^Mc_i\int_\Omega\curll\mathbf{g}_i\cdot\bm{\varphi} = \int_\Omega\mathbf{f_a}\cdot\bm{\varphi}\\
\end{align*}

\begin{rk}
Comme $\bm{\varphi}\in D^1(\Omega)$, $\div\bm{\varphi}=0$ et $\bm{\varphi}\cdot \mathbf{n}=0$ sur $\partial\Omega$, le terme de pression s'annule donc sous cette forme. Il sera recalculé en post-traitement, voir \ref{pression}.\\
\end{rk}
Ce qui donne :
\begin{align*}
\sum_{i=1}^M\frac{\partial c_i}{\partial t}\int_\Omega\mathbf{g}_i\cdot\bm{\varphi} &+ \sum_{i=1}^M\sum_{j=1}^Mc_ic_j\int_\Omega(\curl\mathbf{g}_i\times \mathbf{g_j})\cdot\bm{\varphi} + \sum_{i=1}^Mc_i\int_\Omega(\curl\mathbf{g}_i\times \mathbf{a})\cdot\bm{\varphi}\\
& +  \sum_{i=1}^Mc_i\int_\Omega((\curl\mathbf{a})\times \mathbf{g}_i)\cdot\bm{\varphi} + \int_\Omega\pi_{\mathbf{a}}\div\bm{\varphi} + \textcolor{red}{\frac{1}{Re}\sum_{i=1}^Mc_i\int_\Omega\curll\mathbf{g}_i\cdot\bm{\varphi}}\\
& = \int_\Omega\mathbf{f_a}\cdot\bm{\varphi}\\
\end{align*}

Le terme $\sum_{i=1}^Mc_i\int_\Omega\curll\mathbf{g}_i\cdot\bm{\varphi}$ provient de $\int_\Omega \curll\mathbf{u}\cdot\bm{\varphi}$.\\
En utilisant plusieurs intégrations par parties et la décomposition de $\bm{\varphi}$ dans $D^1(\Omega)$, cela devient $\int_\Omega\curl\mathbf{u}\cdot\curl\bm{\varphi} -\int_{\partial\Omega} \alpha_2\phi$.\\
En repassant à la décomposition de Galerkin, on obtient donc le terme $\sum_{i=1}^Mc_i\lambda_i\int_\Omega\mathbf{g}_i\cdot\curl\bm{\varphi}-\int_{\partial\Omega} \alpha_2\phi$.\\
\begin{rk}
  Cependant, on a que $\alpha_2=\curll\mathbf{u}\cdot\mathbf{n}\restr{\Gamma}=\sum c_i\lambda_i^2 \mathbf{g}_i\cdot\mathbf{n}\restr{\Gamma} = 0$.\\
Si l'on veut que cette condition soit différente de 0, $\alpha_2$ devrait être relever de la même manière que $\alpha_0$ et $\alpha_1$, mais comme cela a été remarqué tardivement, on a fait le choix de continuer à résoudre le système de la manière décrite dans ce rapport.\\
Pareillement, $\mathbf{b}$ appartient aussi à $D^1$, on ne devrait donc pas avoir $\curll\mathbf{b}\cdot\mathbf{n}\restr{\Gamma}=\alpha_1$.
\end{rk}
Lorsqu'on prend une fonction de test dans $D^1$, on cherche à ce que l'équation soit vrai pour toutes les fonctions de $D^1$. Cela revient au même que de résoudre cette équation pour toutes les fonctions de la base de $D^1$. On utilise donc encore une fois les fonctions propres de l'opérateur rotationnel.\\
On cherche donc $\forall k=1,\dots,M$ :
\begin{align*}
\sum_{i=1}^M\frac{\partial c_i}{\partial t}\int_\Omega\mathbf{g}_i\cdot\mathbf{g_k} &+ \sum_{i=1}^M\sum_{j=1}^Mc_ic_j\int_\Omega(\curl\mathbf{g}_i\times \mathbf{g_j})\cdot\mathbf{g_k} + \sum_{i=1}^Mc_i\int_\Omega(\curl\mathbf{g}_i\times \mathbf{a})\cdot\mathbf{g_k}\\
& +  \sum_{i=1}^Mc_i\int_\Omega((\curl\mathbf{a})\times \mathbf{g}_i)\cdot\mathbf{g_k} +\frac{1}{Re}\sum_{i=1}^Mc_i\lambda_i^2\int_\Omega\mathbf{g}_i\cdot\mathbf{g_k}-\frac{1}{Re}\int_{\partial\Omega} \alpha_2\phi_k = \int_\Omega\mathbf{f_a}\cdot\mathbf{g_k}\\
\end{align*}
Comme la base $(\mathbf{g}_i)$ est orthonormale, on a $\forall k=1,\dots,M$ :
\begin{align*}
\frac{\partial c_k}{\partial t} &+ \frac{1}{Re}c_k\lambda_k^2 + \sum_{i=1}^M\sum_{j=1}^Mc_ic_j\int_\Omega(\curl\mathbf{g}_i\times \mathbf{g_j})\cdot\mathbf{g_k}\\
& + \sum_{i=1}^Mc_i\int_\Omega(\curl\mathbf{g}_i\times \mathbf{a})\cdot\mathbf{g_k} +  \sum_{i=1}^Mc_i\int_\Omega((\curl\mathbf{a})\times \mathbf{g}_i)\cdot\mathbf{g_k} = \int_\Omega\mathbf{f_a}\cdot\mathbf{g_k}+\frac{1}{Re}\int_{\partial\Omega} \alpha_2\phi_k\\
\end{align*}
En utilisant les notations suivantes :
\begin{align*}
  R_{ijk} &= \int_\Omega(\curl\mathbf{g}_i\times \mathbf{g_j})\cdot\mathbf{g_k} & R_{iak} &= \int_\Omega(\curl\mathbf{g}_i\times \mathbf{a})\cdot\mathbf{g_k}\\
  &= \lambda_i\int_\Omega(\curl\mathbf{g}_i\times \mathbf{g_j})\cdot\mathbf{g_k} & &= \lambda_i\int_\Omega(\curl\mathbf{g}_i\times \mathbf{a})\cdot\mathbf{g_k}\\
R_{raij} &= \int_\Omega((\curl\mathbf{a})\times \mathbf{g}_i)\cdot\mathbf{g_k} & R_{hk} &= \int_\Omega\mathbf{f_a}\cdot\mathbf{g_k}\\
R_{pk} &= \int_{\partial\Omega} \alpha_2\phi_k
\end{align*}
Le problème \ref{pbc} devient :
\begin{pb}\label{fvc}
Trouver $(c_i)_{i=1,\dots,M}$ tel que $\forall k=1,\dots,M$ :
\begin{equation*}
\frac{\partial c_k}{\partial t} + \frac{1}{Re}c_k\lambda_k^2 + \sum_{i=1}^M\sum_{j=1}^Mc_ic_jR_{ijk} + \sum_{i=1}^Mc_iR_{iak} + \sum_{i=1}^Mc_iR_{raij} = R_{hk} + \frac{1}{Re}R_{pk}
\end{equation*}\end{pb}

\begin{rk}
$R_{ijk}$ et $R_{pk}$ ne sont à calculer qu'une seule fois. Si $\mathbf{a}$ dépend du temps, alors, les autres termes doivent être recalculés à chaque itération en temps.
\end{rk}
\begin{rk}
On a autant d'équations à résoudre et d'inconnus que de fonctions propres, c'est-à-dire $M$. Afin d'avoir une bonne approximation, on a besoin du plus grand nombre de fonctions de bases possible. Il faut donc trouver $M$ pour avoir un bon ratio entre précision et temps de calcul.
\end{rk}

%%% Local Variables:
%%% TeX-master: "../report.tex"
%%% eval: (flyspell-mode 1)
%%% ispell-local-dictionary: "french"
%%% End:
