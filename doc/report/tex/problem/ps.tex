\section{Spectral Problem}
\label{spectre}
Now that we have $\mbf{a}$ and the eigenpairs $(\lambda_i,\mbf{g}_i)_{i=1,\dots,M}$, we have all the pieces to find the coefficients $c_i$ of $\mbf{u}_M$, the approximation of $\mbf{u}$ :
\[\mathbf{u}_M= \sum_{i=1}^M c_i\mathbf{g}_i\mbox{ tel que } \mathbf{u}_M\underset{M\rightarrow\infty}{\longrightarrow} \mathbf{u}=\sum_{i=1}^\infty c_i\mathbf{g}_i\]
For this, we use the problem \ref{pbc}.
\begin{align*}
\sum_{i=1}^M\frac{\partial c_i}{\partial t}\mathbf{g}_i &+ \sum_{i=1}^M\sum_{j=1}^Mc_ic_j((\curl\mathbf{g}_i)\times \mathbf{g_j}) + \sum_{i=1}^Mc_i((\curl\mathbf{g}_i)\times \mathbf{a})\\
& +  \sum_{i=1}^Mc_i((\curl\mathbf{a})\times \mathbf{g}_i) + \grad \pi_{\mathbf{a}} +\frac{1}{Re}\sum_{i=1}^Mc_i\curll\mathbf{g}_i - \mathbf{f_a} = 0\\
\end{align*}

To get the weak formulation of the problem, we multiply by a test function $\bm{\varphi}\in D^1(\Omega)$ and we integrate :
\begin{align*}
\sum_{i=1}^M\frac{\partial c_i}{\partial t}\int_\Omega\mathbf{g}_i\cdot\bm{\varphi} &+ \sum_{i=1}^M\sum_{j=1}^Mc_ic_j\int_\Omega((\curl\mathbf{g}_i)\times \mathbf{g_j})\cdot\bm{\varphi} + \sum_{i=1}^Mc_i\int_\Omega((\curl\mathbf{g}_i)\times \mathbf{a})\cdot\bm{\varphi}\\
& +  \sum_{i=1}^Mc_i\int_\Omega((\curl\mathbf{a})\times \mathbf{g}_i)\cdot\bm{\varphi} + \int_\Omega\grad \pi_{\mathbf{a}}\cdot\bm{\varphi} +\frac{1}{Re}\sum_{i=1}^Mc_i\int_\Omega(\curll\mathbf{g}_i)\cdot\bm{\varphi} - \int_\Omega\mathbf{f_a}\cdot\bm{\varphi} = 0\\
\end{align*}

By using an integration by parts on $\int_\Omega\grad \pi_{\mathbf{a}}\cdot\bm{\varphi}$,we get :
\begin{align*}
\sum_{i=1}^M\frac{\partial c_i}{\partial t}\int_\Omega\mathbf{g}_i\cdot\bm{\varphi} &+ \sum_{i=1}^M\sum_{j=1}^Mc_ic_j\int_\Omega((\curl\mathbf{g}_i)\times \mathbf{g_j})\cdot\bm{\varphi} + \sum_{i=1}^Mc_i\int_\Omega((\curl\mathbf{g}_i)\times \mathbf{a})\cdot\bm{\varphi}\\
& +  \sum_{i=1}^Mc_i\int_\Omega((\curl\mathbf{a})\times \mathbf{g}_i)\cdot\bm{\varphi} + \int_\Omega\pi_{\mathbf{a}}\div\bm{\varphi} + \int_{\partial\Omega}\pi_{\mathbf{a}}(\bm{\varphi}\cdot\mathbf{n})\\
& \frac{1}{Re}\sum_{i=1}^Mc_i\int_\Omega(\curll\mathbf{g}_i)\cdot\bm{\varphi} = \int_\Omega\mathbf{f_a}\cdot\bm{\varphi}\\
\end{align*}

\begin{rk}
As $\bm{\varphi}\in D^1(\Omega)$, $\div\bm{\varphi}=0$ and $\bm{\varphi}\cdot \mathbf{n}=0$ on $\Gamma$, the pressure term doesn't appear anymore in this formulation. We'll need to compute it in post-traitment, see \ref{pression}.
\end{rk}
Which leads to :
\begin{align*}
\sum_{i=1}^M\frac{\partial c_i}{\partial t}\int_\Omega\mathbf{g}_i\cdot\bm{\varphi} &+ \sum_{i=1}^M\sum_{j=1}^Mc_ic_j\int_\Omega((\curl\mathbf{g}_i)\times \mathbf{g_j})\cdot\bm{\varphi} + \sum_{i=1}^Mc_i\int_\Omega((\curl\mathbf{g}_i)\times \mathbf{a})\cdot\bm{\varphi}\\
& +  \sum_{i=1}^Mc_i\int_\Omega((\curl\mathbf{a})\times \mathbf{g}_i)\cdot\bm{\varphi} + \frac{1}{Re}\sum_{i=1}^Mc_i\int_\Omega(\curll\mathbf{g}_i)\cdot\bm{\varphi}\\
& = \int_\Omega\mathbf{f_a}\cdot\bm{\varphi}\\
\end{align*}

Multiplying by a test function in $D^1$, means that we want the equation to be true for all functions in $D^1$. So this is the same that solving this equation for all functions of a basis of $D^1$. This is where we use the basis $\{\mbf{g}_i\}_{i=1,\dots,M}$.\\
We want $\forall k=1,\dots,M$ :
\begin{align*}
\sum_{i=1}^M\frac{\partial c_i}{\partial t}\int_\Omega\mathbf{g}_i\cdot\mathbf{g_k} &+ \sum_{i=1}^M\sum_{j=1}^Mc_ic_j\int_\Omega((\curl\mathbf{g}_i)\times \mathbf{g_j})\cdot\mathbf{g_k} + \sum_{i=1}^Mc_i\int_\Omega((\curl\mathbf{g}_i)\times \mathbf{a})\cdot\mathbf{g_k}\\
& +  \sum_{i=1}^Mc_i\int_\Omega((\curl\mathbf{a})\times \mathbf{g}_i)\cdot\mathbf{g_k} +\frac{1}{Re}\sum_{i=1}^Mc_i\int_\Omega(\curll\mathbf{g}_i)\cdot\mathbf{g_k} = \int_\Omega\mathbf{f_a}\cdot\mathbf{g_k}\\
\end{align*}
We know that in $\ZZ_h$, $\int(\curll\mbf{g}_i)\cdot\mbf{g_k}=\int(\curl\mbf{g}_i)\cdot(\curl\mbf{g}_k)$ and that the basis $\{\mbf{g}_i\}_{i=1,\dots,M}$ is orthogonal, so we have for $\forall k=1,\dots,M$ :
\begin{align*}
  \frac{\partial c_k}{\partial t} &+ \sum_{i=1}^M\sum_{j=1}^Mc_ic_j\int_\Omega((\curl\mathbf{g}_i)\times \mathbf{g_j})\cdot\mathbf{g_k} + \sum_{i=1}^Mc_i\int_\Omega((\curl\mathbf{g}_i)\times \mathbf{a})\cdot\mathbf{g_k}\\
  &+  \sum_{i=1}^Mc_i\int_\Omega((\curl\mathbf{a})\times \mathbf{g}_i)\cdot\mathbf{g_k} + \frac{1}{Re}\sum_{i=1}^M c_i\int_\Omega(\curl\mbf{g}_i)\cdot(\curl\mbf{g}_k) = \int_\Omega\mathbf{f_a}\cdot\mathbf{g_k}
\end{align*}
We will now use the following notations :
\begin{align*}
  R_{ijk} &= \int_\Omega((\curl\mathbf{g}_i)\times \mathbf{g}_j)\cdot\mathbf{g}_k & R_{iak} &= \int_\Omega((\curl\mathbf{g}_i)\times \mathbf{a})\cdot\mathbf{g}_k\\
  R_{raij} &= \int_\Omega((\curl\mathbf{a})\times \mathbf{g}_i)\cdot\mathbf{g}_k & R_{fk} &= \int_\Omega\mathbf{f_a}\cdot\mathbf{g}_k\\
  R_{ik} &= \int_{\partial\Omega} (\curl\mbf{g}_i)\cdot(\curl\mbf{g}_k)
\end{align*}
\begin{rk}
  Since we couldn't know if $\curll\mbf{g}_i=\lambda^2_i \mbf{g}_i$ (see remark \ref{curll}), we have to add the term $R_{ik}$, but in practice, this term is $\delta_{ik}\lambda^2_k$, so we don't have to compute it.
\end{rk}
The problem \ref{pbc} becomes :
\begin{pb}\label{fvc}
Find $(c_i)_{i=1,\dots,M}$ such that $\forall k=1,\dots,M$ :
\begin{equation*}
\frac{\partial c_k}{\partial t} + \frac{1}{Re}c_k\lambda_k^2 + \sum_{i=1}^M\sum_{j=1}^Mc_ic_jR_{ijk} + \sum_{i=1}^Mc_iR_{iak} + \sum_{i=1}^Mc_iR_{raij} = R_{fk}
\end{equation*}\end{pb}

\begin{rk}
  We need to compute $R_{ijk}$ only once. If $\mbf{a}$ depends on time, then we'll need to recompute the others terms at each iteration.
\end{rk}
\begin{rk}
  We have that :
  \[ R_{ijk}=\int_\Omega((\curl(\mbf{g}_i)\times\mbf{g}_j)\cdot\mbf{g}_k=\int_\Omega(\mbf{g}_k\times(\curl\mbf{g}_i))\cdot\mbf{g}_j=-\int_\Omega((\curl\mbf{g}_i)\times\mbf{g_k})\cdot\mbf{g_j}=-R_{ikj}\]
  And also :
  \[ R_{ijj}=\int_\Omega((\curl(\mbf{g}_i)\times\mbf{g}_j)\cdot\mbf{g}_j=\int_\Omega(\mbf{g}_j\times\mbf{g}_j)\cdot(\curl\mbf{g}_i)=0 \]
  And the same for $R_{aik}$. So instead of $M^3+2M^2+M$ terms, we need to compute only half $\frac{M^2(M-1)}{2}+M^2+\frac{M(M-1)}{2}+M=\frac{M^3}{2}+M^2+\frac{M}{2}$.
\end{rk}

%%% Local Variables:
%%% TeX-master: "../report.tex"
%%% eval: (flyspell-mode 1)
%%% ispell-local-dictionary: "english"
%%% End:
