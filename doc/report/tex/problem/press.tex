\section{Pressure : Post-traitment of the velocity}
\label{pressure}
To find the velocity $\mbf{v}$, we just need to add $\mbf{a}$ and $\mbf{u}$.\\
Since we need to post-trait the velocity to find the pressure term, we start from the equation of the problem \ref{start}.

We apply divergence on it and use the fact that the velocity is divergence free, and that the one of a curl is also always null. We have then :
\begin{equation*}
-\laplace q = \div((\curl \mbf{v})\times \mbf{v}) - \div \mbf{f}
\end{equation*}

To get a boundary condition, we use the normal component of the equation and the boundary conditions of $\mbf{v}$ :
\[ \grad q\cdot \mbf{n}\restr{\Gamma} =  \mbf{f}\cdot \mbf{n}\restr{\Gamma} - \frac{\partial\alpha_0}{\partial t} - ((\curl \mbf{v})\times \mbf{v})\cdot \mbf{n}\restr{\Gamma} - \frac{\alpha_2}{Re} \]
We want now the weak formulation of the problem :
\[ \int_\Omega -\laplace q\varphi = \int_\Omega (\div((\curl \mbf{v})\times \mbf{v}) -\div \mbf{f})\varphi \]
By integrating by parts the left term, we have :
\[ \int_\Omega \grad q\grad\varphi - \int_{\partial\Omega} (\grad q\cdot \mbf{n})\varphi = \int_\Omega (\div((\curl \mbf{v})\times \mbf{v}) -\div \mbf{f})\varphi \]

Again, we find the pressure up to a constant, we use the Lagrange multiplier to fix this constant. As in \ref{multLagrange}, we get :
\begin{pb}\label{fvq}
Find $p=q-\frac{\mbf{v}\cdot\mbf{v}}{2} \in L^2(\Omega)$ such that $\forall \varphi\in L^2(\Omega)$, we have :
\begin{align*}
\int_\Omega \grad q\grad\varphi + \int_\Omega \lambda\varphi + \int_\Omega q\nu &= \int_\Omega (\div((\curl \mbf{v})\times \mbf{v}) -\div \mbf{f})\varphi\\
&+ \int_{\partial\Omega} \left(f\cdot \mbf{n} - \frac{\partial\alpha_0}{\partial t} - ((\curl \mbf{v})\times \mbf{v})\cdot \mbf{n} - \frac{\alpha_2}{Re}\right)\varphi
\end{align*}\end{pb}

%%% Local Variables:
%%% TeX-master: "../report.tex"
%%% eval: (flyspell-mode 1)
%%% ispell-local-dictionary: "english"
%%% End:
