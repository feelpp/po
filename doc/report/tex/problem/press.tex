\section{Pression : Post-traitement de la vitesse}
\label{pression}
Pour retrouver la vitesse $\mathbf{v}$, il suffit maintenant d'additionner $\mathbf{a}$ et $\mathbf{u}$.\\
Le terme correspondant à la pression ayant été relayé en post-traitement de la vitesse, il faut le recalculer à partir de l'équation du problème \ref{depart}.

On applique la divergence sur cette équation et on utilise le fait que $\mathbf{v}$ soit à divergence nulle, et que la divergence d'un rotationnel soit toujours nulle. On a alors :
\begin{equation*}
-\laplace q = \div((\curl \mathbf{v})\times \mathbf{v}) - \div \mathbf{f}
\end{equation*}

Pour obtenir une condition au bord, on utilise la composante normale de l'équation (\ref{depart}) et les conditions aux bords de $\mathbf{v}$ :
\[ \grad q\cdot \mathbf{n}\restr{\Gamma} =  \mathbf{f}\cdot \mathbf{n}\restr{\Gamma} - \frac{\partial\alpha_0}{\partial t} - ((\curl \mathbf{v})\times \mathbf{v})\cdot \mathbf{n}\restr{\Gamma} - \frac{\alpha_2}{Re} \]
On cherche maintenant la forme variationnelle du problème :
\[ \int_\Omega -\laplace q\varphi = \int_\Omega (\div((\curl \mathbf{v})\times \mathbf{v}) -\div \mathbf{f})\varphi \]
En intégrant par partie le terme de gauche, on a :
\[ \int_\Omega \grad q\grad\varphi - \int_{\partial\Omega} (\grad q\cdot \mathbf{n})\varphi = \int_\Omega (\div((\curl \mathbf{v})\times \mathbf{v}) -\div \mathbf{f})\varphi \]

Toujours de même manière, on va trouver la pression à une constante près, on utilise donc encore une fois les multiplicateur de Lagrange afin de fixer cette constante. Comme dans \ref{multLagrange}, on obtient donc au final :
\begin{pb}\label{fvq}
Trouver $p=q-\frac{\mathbf{v}\cdot\mathbf{v}}{2} \in L^2(\Omega)$ tel que $\forall \varphi\in L^2(\Omega)$, on a :
\begin{align*}
\int_\Omega \grad q\grad\varphi + \int_\Omega \lambda\varphi + \int_\Omega q\nu &= \int_\Omega (\div((\curl \mathbf{v})\times \mathbf{v}) -\div \mathbf{f})\varphi\\
&+ \int_{\partial\Omega} \left(f\cdot \mathbf{n} - \frac{\partial\alpha_0}{\partial t} - ((\curl \mathbf{v})\times \mathbf{v})\cdot \mathbf{n} - \frac{\alpha_2}{Re}\right)\varphi
\end{align*}\end{pb}

%%% Local Variables:
%%% TeX-master: "../report.tex"
%%% eval: (flyspell-mode 1)
%%% ispell-local-dictionary: "french"
%%% End:
