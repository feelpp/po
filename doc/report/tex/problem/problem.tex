\part{Problem}
\label{partProb}
Let $\Omega\subset\R^3$ be a bounded simply-connected domain with a
Lipschitz continuous boundary $\Gamma$. Let $\Gamma_0,\dots,\Gamma_I$
be the connected components of $\Gamma$.\\

We will use the following notations :
\begin{align*}
gradient(v)&=(\partial_x v, \partial_y v, \partial_z v)=\grad v\\
gradient(\mbf{v})&=\begin{pmatrix}
\partial_x v_x & \partial_y v_x & \partial_z v_x\\
\partial_x v_y & \partial_y v_y & \partial_z v_y\\
\partial_x v_z & \partial_y v_z & \partial_z v_z
\end{pmatrix}=\grad\mbf{v}\\
divergence(\mbf{v})&=\frac{\partial v_x}{\partial x}+\frac{\partial v_y}{\partial y}+\frac{\partial v_z}{\partial z}=\div \mbf{v}\\
curl(\mbf{v})&=\begin{pmatrix}
\partial_y v_z - \partial_z v_y\\
\partial_z v_x - \partial_x v_z\\
\partial_x v_y - \partial_y v_x
\end{pmatrix}=\curl \mbf{v}\\
curl(curl(\mbf{v}))&=\curll \mbf{v}\\
H^1(\Omega) &= \{v \in L^2(\Omega)\;|\; \grad v\in L^2(\Omega)\}\\
H^1_0(\Omega) &= \{v \in H^1(\Omega)\; |\; v\restr{\Gamma} = 0\}\\
H(\mathrm{div};\Omega) &= \{\mbf{v} \in [L^2(\Omega)]^3\; |\; \div\mbf{v} \in L^2(\Omega) \}\\
H(\mathrm{curl};\Omega) &= \{\mbf{v} \in [L^2(\Omega)]^3\; |\; \curl\mbf{v} \in L^2(\Omega) \}\\
L^2_\sigma(\Omega) &= \{\mbf{v} \in [L^2(\Omega)]^3\; |\; \div \mbf{v} = 0\text{ et }\mbf{v}\cdot \mbf{n}\restr{\Gamma} = 0 \}\\
D^1(\Omega) &= \{\mbf{v} \in [H^1(\Omega)]^3\cap L^2_\sigma(\Omega)\; |\; (\curl \mbf{v}\cdot \mbf{n})\restr{\Gamma} = 0  \}
\end{align*}

We are looking for $(\mbf{v},p)$, corresponding to the velocity and the pressure, solutions of the incompressible Navier-Stokes equations in $Q_T=\Omega\times[0,T]$, where $\Omega$ is an open set in $\R^3$ and $\partial\Omega$ its boundary, with initial condition and general impermeable boundary conditions.
\begin{pb}\label{start}
Find $(\mbf{v},p)$ such that :
\begin{equation*}
\left\{\begin{aligned}
&\frac{\partial \mbf{v}}{\partial t} + (\curl  \mbf{v})\times \mbf{v} + \grad q + \frac{1}{Re}\curll  \mbf{v}-\mbf{f} = 0\\
&\div \mbf{v} = 0\\
&\mbf{v}\big\rvert_{t=0} = \mbf{v}_0\\
&\mbf{v}\cdot \mbf{n}\restr{\Gamma} = \alpha_0\\
&(\curl  \mbf{v})\cdot \mbf{n}\restr{\Gamma} = \alpha_1\\
&(\curll  \mbf{v})\cdot \mbf{n}\restr{\Gamma} = \alpha_2
\end{aligned}\right.
\end{equation*}
where $q = \frac{|\mbf{v}|^2}{2}+p$.\\
\end{pb}
This problem has non standards boundary conditions. Rather than Dirichlet conditions, we use $\nabla^k\times\mbf{v}\cdot\mbf{n}\restr{\Gamma}$ for $k=0,1,2$ with $\nabla^0\times=Id$.\\
There have been few studies about these conditions and the industry rarely uses it. We can cite the works of V. Girault \cite{girault90-1}, H. Bellout, J. Neustuppa and P. Penel \cite{Penel2004}. The principle is a Galerkin type decomposition that can be related to the works of E. Deriaz and V. Perrier \cite{Deriaz2009249}.\\

The interest for Plastic Omnium is to be able to split the variables in time and space, and hence compute longer simulations. Furtermore, this could mean that we wouldn't need wall functions boundary condition, and so the boundary layer become needless.\\ 

We can now explain the resolution's strategy in the chapter \ref{strategy}, then study each step of the resolution in more details in \ref{fv}.

\chapter{Resolution's Strategy}
\label{strategy}
According to \cite{Penel2004}, the existence and unicity of the solution are guaranteed if the solution lives in $D^1$, which means that $\nabla^k\times\mbf{v}\cdot\mbf{n}\restr{\Gamma}=0$ for $k=0,1$. We need to relief these boundary conditions with a function $\mbf{a}$. We want to write $\mbf{v}=\mbf{u}+\mbf{a}$, such that :\\
\begin{equation}\label{v}
\begin{array}{c|ccccc}
& \mbf{v} & = & \mbf{a} & + & \mbf{u}\\ \hline
\div\star & 0 & & 0 & & 0\\ \hline
\star\cdot \mbf{n}\restr{\Gamma} & \alpha_0 & & \alpha_0 & & 0\\ \hline
\curl\star\cdot \mbf{n}\restr{\Gamma} & \alpha_1 & & \alpha_1 & & 0\\ \hline
\curll\star\cdot \mbf{n}\restr{\Gamma} & \alpha_2 & & \alpha_2 & & 0
\end{array}
\end{equation}
We have $\mbf{u}\in D^1(\Omega)$ to define the solution in the interior of the domain and $\mbf{a}\in [L^2(\Omega)]^3$ that we use for the boundary conditions.
\begin{pb}\label{a}
Find $\mbf{a}\in [L^2(\Omega)]^3$ such that :
\begin{equation*}
\left\{\begin{aligned}
&\mbf{a}=\grad a_0 + \curl\mbf{a}_1 + \mbf{a}_2\\
&\div \mbf{a} =0\\
&\mbf{a}\cdot \mbf{n}\restr{\Gamma} = \alpha_0\\
&(\curl \mbf{a})\cdot \mbf{n}\restr{\Gamma} = \alpha_1\\
&(\curll\mbf{a})\cdot\mbf{n}\restr{\Gamma} = \alpha_2
\end{aligned}\right.
\end{equation*}
\end{pb}
By applying the divergence and the boundary contions to the first lign, it leads to :
\begin{center}
\begin{tabular}{c|ccccccc}
& $\mbf{a}$ & = & $\grad a_0$ & + & $\curl \mbf{a}_1$ & + & $\mbf{a}_2$ \\ \hline
$\div\star$ & 0 & & $\laplace a_0$ & & 0 & & 0\\ \hline
$\star\cdot \mbf{n}\restr{\Gamma}$ & $\alpha_0$ & & $\alpha_0$ & & 0 & & 0\\ \hline
  $\curl\star\cdot \mbf{n}\restr{\Gamma}$ & $\alpha_1$ & & 0 & & $\alpha_1$ & & 0\\ \hline
  $\curll\star\cdot\mbf{n}\restr{\Gamma}$ & $\alpha_2$ & & 0 & & 0 & & $\alpha_2$
\end{tabular}
\end{center}
With the two first ligns of the table, we get the problem :
\begin{pb}\label{pba0}
Find $a_0$ such that :
\begin{equation*}
\left\{\begin{aligned}
&-\laplace a_0 = 0\\
&\grad a_0\cdot \mbf{n}\restr{\Gamma}=\alpha_0
\end{aligned}\right.
\end{equation*}\end{pb}
This problem allows us to find $a_0$ up to a constant, we need to use Lagrange multiplier to add a constraint, for example $\int_Omega a_0=0$. We can choose the constant because we only need the gradient.\\
We'll discuss which way we solve the problem \ref{pba0} further.\\

Let $\mbf{a}_1$ solution of the mixed problem :
\begin{pb}\label{pba1}
Find $(\mbf{a}_1,\psi^1)$ tel que :
\begin{equation*}
\left\{\begin{aligned}
&\curll \mbf{a}_1 = \grad\psi^1\\
&\div \mbf{a}_1 = 0\\
&\mbf{a}_1\cdot \mbf{n}\restr{\Gamma} = 0\\
&\curl \mbf{a}_1\cdot \mbf{n}\restr{\Gamma} = 0\\
&\grad\psi^1\cdot \mbf{n}\restr{\Gamma} = \alpha_1
\end{aligned}\right.
\end{equation*}\end{pb}

Let $\mbf{a}_2$ solution of the following problem :
\begin{pb}\label{pba2}
  Find $\mbf{a}_2$ such that :
  \begin{equation*}
    \left\{\begin{aligned}
    &\curll\mbf{a}_2 = \bm{\epsilon}\\
    &\div \mbf{a}_2 = 0\\
    &\mbf{a}_2\cdot\mbf{n}\restr{\Gamma} = 0\\
    &\curl\mbf{a}_2\cdot\mbf{n}\restr{\Gamma} = 0\\
    &\curll\mbf{a}_2\cdot\mbf{n}\restr{\Gamma} = \alpha_2
    \end{aligned}\right.
  \end{equation*}
  Such that $\bm{\epsilon}\cdot\mbf{n} = \alpha_2$
\end{pb}

Once we know $\grad a_0$, $\curl\mbf{a}_1$ and $\mbf{a}_2$, we have $\mbf{a}$.\\

We can replace $\mbf{v}$ by $\mbf{u}+\mbf{a}$ in problem \ref{start} :
\[ \frac{\partial(\mbf{u}+\mbf{a})}{\partial t}+(\curl(\mbf{u}+\mbf{a}))\times(\mbf{u}+\mbf{a}) + \grad (\frac{|\mbf{u}+\mbf{a}|^2}{2}+p) + \frac{1}{Re}\curll(\mbf{u}+\mbf{a}) - \mbf{f} = 0 \]
Which leads to, by noting $\pi_a=\frac{|\mbf{u}+\mbf{a}|^2}{2}+p$ :
\[ \frac{\partial \mbf{u}}{\partial t}+\frac{\partial \mbf{a}}{\partial t} + (\curl \mbf{u}+\curl \mbf{a})\times(\mbf{u}+\mbf{a}) + \grad\pi_{\mbf{a}} + \frac{1}{Re}(\curll \mbf{u}+\curll \mbf{a}) - \mbf{f} = 0 \]
We have that $\curll \mbf{a} = \bm{\epsilon}$, then if $\mbf{f_a}=\mbf{f}-\frac{\partial \mbf{a}}{\partial t} - (\curl \mbf{a})\times \mbf{a} - \bm{\epsilon}$, we have the following problem :
\begin{pb}\label{pbu}
Find $\mbf{u}$ such that :
\begin{equation*}
\left\{\begin{aligned}
&\frac{\partial \mbf{u}}{\partial t} + (\curl \mbf{u})\times \mbf{u} + (\curl \mbf{u})\times \mbf{a} +(\curl \mbf{a})\times \mbf{u} + \grad \pi_{\mbf{a}} +\frac{1}{Re}\curll  \mbf{u} - \mbf{f_a} = 0\\
&\div \mbf{u} = 0\\
&\mbf{u}\big\rvert_{t=0} = \mbf{v}_0 - \mbf{a}(0,\cdot)\\
&\mbf{u}\cdot \mbf{n}\restr{\Gamma} = 0\\
&(\curl \mbf{u})\cdot \mbf{n}\restr{\Gamma} = 0\\
&(\curll  \mbf{u})\cdot \mbf{n}\restr{\Gamma} = 0
\end{aligned}\right.
\end{equation*}\end{pb}

Using a Galerkin decomposition, we have :
\begin{equation}\label{u}
\mbf{u}(t,\cdot) = \sum_{i=1}^{\infty} c_i(t)\mbf{g}_i(\cdot)
\end{equation}
where we have separeted the variables depending on time and space. The coefficients $c_i$ include all informations depending on the time, whereas the functions $\mbf{g}_i$ carry the spacial component.\\

Since $\mbf{u}\in D^1(\Omega)=D(\mathrm{rot}_{imperm})$, according to \cite{Penel2004}, a basis of this space is the eigen functions $\mbf{g}_i$ of the curl operator.
So, we want to solve the eigen problem :
\begin{pb}\label{pbcurl}
Find $(\lambda_i,\mbf{g}_i)\in\R\times D^1(\Omega)$ such that :
\begin{equation*}
\left\{\begin{aligned}
&\curl  \mbf{g}_i = \lambda_i \mbf{g}_i\\
&\div\mbf{g}_i = 0\\
&\mbf{g}_i\cdot \mbf{n}\restr{\Gamma} = 0\\
&\curl \mbf{g}_i\cdot \mbf{n}\restr{\Gamma} = 0
\end{aligned}\right.
\end{equation*}\end{pb}

By injecting \ref{u} in the problem \ref{pbu}, we get :
\begin{align*}
\frac{\partial}{\partial t}\left(\sum_{i=1}^\infty c_i\mbf{g}_i\right) &+ \left(\curl \left(\sum_{i=1}^\infty c_i\mbf{g}_i\right)\right)\times \left(\sum_{i=1}^\infty c_i\mbf{g}_i\right) + \left(\curl \left(\sum_{i=1}^\infty c_i\mbf{g}_i\right)\right)\times \mbf{a}&\\
&+ (\curl \mbf{a})\times \left(\sum_{i=1}^\infty c_i\mbf{g}_i\right) + \grad \pi_{\mbf{a}} +\frac{1}{Re}\curll  \left(\sum_{i=1}^\infty c_i\mbf{g}_i\right) - \mbf{f_a} =0\\
\end{align*}

Using the linearity of the derivative operator, as well as that of the curl operator, we have :
\begin{pb}\label{pbc}
Find $(c_i)$ such that :
\begin{align*}
\sum_{i=1}^\infty\frac{\partial c_i}{\partial t}\mbf{g}_i &+ \sum_{i=1}^\infty\sum_{j=1}^\infty c_i c_j((\curl\mbf{g}_i)\times \mbf{g}_j) + \sum_{i=1}^\infty c_i((\curl\mbf{g}_i)\times \mbf{a})\\
& + \sum_{i=1}^\infty c_i((\curl\mbf{a})\times \mbf{g}_i) + \grad \pi_{\mbf{a}} +\frac{1}{Re}\sum_{i=1}^\infty c_i\curll\mbf{g}_i - \mbf{f_a} = 0\\
\end{align*}
with \[ \sum_{i=1}^\infty c_i(0)\mbf{g}_i = \mbf{v_0}-\mbf{a}(0,\cdot) \]
\end{pb}

To summarize, we have to :
\begin{enumerate}
\item generate the basis $\{\mbf{g}_i\}$ from the eigen functions of the curl operator by solving \ref{pbcurl} as explain in the chapter \ref{eigen}.
\item find $\mbf{a}$ to decompose $\mbf{v}$ as $\mbf{u}+\mbf{a}$, for this, we solve the problems \ref{pba0},\ref{pba1} and \ref{pba2}. Which allows us to compute $\mbf{a}$ with \ref{a}. This part is details in \ref{relev}.
\item solve problem \ref{pbc} to find the coefficients $c_i$. This is explain in chapter \ref{spectre}.
\item recompose $\mbf{v}=\mbf{u}+\mbf{a}$, and find $p$ to get the solution of the problem \ref{start}. The chapter \ref{pressure} explains how to do it.
\end{enumerate}

The strategy is presented in the figure \ref{org1}.
\begin{figure}[H]
  \centering
  \begin{tikzpicture}
    \node[draw,scale=\taille,fill=green!50] (alpha0) at (-1,10) {$\alpha_0$};
    \node[draw,scale=\taille,fill=gray!50,label={[xshift=0.9cm]\ref{pba0}}] (pba0) at (-1,8){
      $\begin{aligned}
        -\laplace a_0&=0\\
        \grad a_0\cdot \mbf{n} &= \alpha_0
      \end{aligned}$
    };
    \node[draw,scale=\taille,fill=blue!50] (grada0) at (-1,5.5) {$\grad a_0$} ;
    \node[draw,scale=\taille,fill=green!50] (alpha1) at (1.5,10) {$\alpha_1$};
    \node[draw,scale=\taille,fill=gray!50,label={[xshift=1.1cm]\ref{pba1}}] (pba1) at (1.5,8){
      $\begin{aligned}
        \curll \mbf{a}_1 = \grad\psi^1\\
        \div \mbf{a}_1 = 0\\
        \mbf{a}_1\cdot \mbf{n}\restr{\Gamma} = 0\\
        \curl \mbf{a}_1\cdot \mbf{n}\restr{\Gamma} = 0\\
        \grad\psi^1\cdot \mbf{n}\restr{\Gamma} = \alpha_1
      \end{aligned}$
    } ;
    \node[draw,scale=\taille,fill=blue!50] (curla1) at (1.5,5.5) {$(\curl\mbf{a}_1,\grad\psi^1)$} ;
    \node[draw,scale=\taille,fill=green!50] (alpha2) at (4,10) {$\alpha_2$};
    \node[draw,scale=\taille,fill=gray!50,label={[xshift=1.1cm]\ref{pba2}}] (pba2) at (4,8){
      $\begin{aligned}
        \curll \mbf{a}_2 = \bm{\epsilon}\\
        \div \mbf{a}_2 = 0\\
        \mbf{a}_2\cdot \mbf{n}\restr{\Gamma} = 0\\
        \curl \mbf{a}_2\cdot \mbf{n}\restr{\Gamma} = 0\\
        \bm{\epsilon}\cdot \mbf{n}\restr{\Gamma} = \alpha_2
      \end{aligned}$
    } ;
    \node[draw,scale=\taille,fill=blue!50] (a2) at (4,5.5) {$(\curl\mbf{a}_2,\bm{\epsilon})$} ;
    \node[draw,scale=\taille,fill=gray!50,label={[xshift=1.5cm]\ref{a}}] (pba) at (1.5,4) {$\mbf{a}=\grad a_0+\curl\mbf{a}_1+\mbf{a}_2$};
    \node[draw,scale=\taillem,fill=blue!50] (a) at (1.5,3) {$\mbf{a}$} ;
    \node[draw,scale=\taille,fill=gray!50,label={[xshift=1.1cm]\ref{pbcurl}}] (pbcurl) at (10,8) {
      $\begin{aligned}
        \curl \mbf{g}_i = \lambda_i\mbf{g}_i\\
        \div\mbf{g}_i = 0\\
        \mbf{g}_i\cdot \mbf{n}\restr{\Gamma} = 0\\
        \curl\mbf{g}_i\cdot \mbf{n}\restr{\Gamma} = 0
      \end{aligned}$
    } ;
    \node[draw,scale=\taille,fill=yellow!50] (lambdagi) at (10, 5.5) {$(\lambda_i,\mbf{g}_i)$} ;
    \node[draw,scale=\taille,fill=green!50] (f) at (6.75,5.5) {$\mbf{f}$};
    \node[draw,scale=\taille,fill=green!50] (c0) at (7.75,5.5) {$c_i(0)$};
    \node[draw,scale=\taille,fill=gray!50,label={[xshift=4.5cm]\ref{pbc}}] (pbc) at (8.5,3) {
      $\begin{aligned}
        \sum_{i=1}^\infty\frac{\partial c_i}{\partial t}\mbf{g}_i &+ \sum_{i=1}^\infty\sum_{j=1}^\infty c_i c_j((\curl\mbf{g}_i)\times \mbf{g}_j) + \sum_{i=1}^\infty c_i((\curl\mbf{g}_i)\times \mbf{a})\\
        & + \sum_{i=1}^\infty c_i((\curl\mbf{a})\times \mbf{g}_i) + \grad \pi_{\mbf{a}} +\frac{1}{Re}\sum_{i=1}^\infty c_i\curll\mbf{g}_i - \mbf{f_a} = 0
      \end{aligned}$
    };
    \node[draw,scale=\taillem,fill=blue!50] (u) at (8.5,1) {$\mbf{u}$} ;
    \node[draw,scale=\taille,fill=gray!50] (pbv) at (5,-0.5) {$\mbf{v}=\mbf{a}+\mbf{u}$} ;
    \node[draw,scale=\tailleg,fill=red!50] (v) at (5,-2) {$(\mbf{v},p)$} ;

    \draw[->,>=latex] (alpha0) -- (pba0); \draw[->,>=latex] (pba0) -- (grada0); \draw[->,>=latex] (alpha1) -- (pba1); \draw[->,>=latex] (pba1) -- (curla1); \draw[->,>=latex] (alpha2) -- (pba2); \draw[->,>=latex] (pba2) -- (a2); \draw[->,>=latex] (grada0) -- (pba); \draw[->,>=latex] (curla1) -- (pba); \draw[->,>=latex] (a2) -- (pba); \draw[->,>=latex] (pba) -- (a); \draw[->,>=latex] (pbcurl) -- (lambdagi); \draw[->,>=latex] (a) -- (pbc); \draw[->,>=latex] (f) -- (pbc); \draw[->,>=latex] (c0) -- (pbc); \draw[->,>=latex] (lambdagi) -- (pbc); \draw[->,>=latex] (pbc) -- (u); \draw[->,>=latex] (u) -- (pbv); \draw[->,>=latex] (a) -- (pbv); \draw[->,>=latex] (pbv) -- (v);
  \end{tikzpicture}
  \caption{Flow chart of the resolution's strategy}\label{org1}
\end{figure}

\chapter{Weak Formulations}
\label{fv}
Now that how we proceed to solve the problem \ref{start} is clear, we can detail each step mentionned before.

\chapter{Modes propres}
\todo[inline]{intro ch + diff Curl/Lagrange}
\section{Cas général}
\todo[inline]{reformulation : mieux :utiliser Nedelec, d'où Girault, avec avantages}
Ici on veut utiliser les travaux de V. Girault \cite{girault90-1}, pour justifier l'utilisation des éléments de Nedelec. Pour cela, on a besoin de définir l'espace \[X = \{\bm{v}\in H(rot)\ |\ (\rot\bm{v}\cdot\bm{n})\restr=0 \}\]
On rappel les définitions suivantes :
\begin{align*}
L^2_\sigma(\Omega) &= \{\mathbf{v} \in L^2(\Omega)\ |\ \div \mathbf{v} = 0\text{ et }\mathbf{v}\cdot \mathbf{n}\restr = 0 \}\\
D^1(\Omega) &= \{\mathbf{v} \in H^1(\Omega)\cap L^2_\sigma(\Omega)\ |\ (\rot \mathbf{v}\cdot \mathbf{n})\restr = 0  \}
\end{align*}
De plus, on a (voir \cite{Girault79}) :
\[ H^1(\Omega)=H(rot)\cap H(div) \]
D'où:
\begin{align*}
D^1(\Omega) &= \{\bm{v}\in H^1(\Omega)\cap L^2_\sigma(\Omega)\ |\ (\rot \mathbf{v}\cdot \mathbf{n})\restr = 0  \}&\\
&=\{\bm{v}\in H(rot)\cap H(div)\cap L^2_\sigma(\Omega)\ |\ (\rot \mathbf{v}\cdot \mathbf{n})\restr = 0  \}&\\
&&\text{ or }L^2_\sigma\subset H(div)\\
&=\{\bm{v}\in H(rot)\cap L^2_\sigma(\Omega)\ |\ (\rot \mathbf{v}\cdot \mathbf{n})\restr = 0  \}&\\
&=\{\bm{v}\in H(rot)\ |\ (\rot \mathbf{v}\cdot \mathbf{n})\restr = 0  \}\cap L^2_\sigma(\Omega)&\\
&=X\cap L^2_\sigma(\Omega)&
\end{align*}

Le problème aux valeurs propres est :\\
Trouver $(\bm{g},\lambda)\in X\cap L^2_\sigma\times\R$ tel que :
\begin{align}
\rott\bm{g}&=\lambda^2\bm{g} \label{impPb1}\\
\div\bm{g}&=0 \label{impPb2}\\
(\bm{g}\cdot\bm{n})\restr&=0 \label{impPb3}\\
(\rot\bm{g}\cdot\bm{n})\restr&=0 \label{impPb4}\\
(\rott\bm{g}\cdot\bm{n})\restr&=0 \label{impPb5}
\end{align}
Mais les conditions (\ref{impPb2}-\ref{impPb3}) sont satisfaites par le fait que $\bm{g}\in L^2_\sigma$ et la condition (\ref{impPb4}) par l'appartenance à $X$.\\

En passant à la forme variationnelle, et en suivant les mêmes étapes que dans le chapitre (\ref{eigen}), on impose la condition (\ref{impPb5}) :
\[ \int_\Omega (\rot\bm{g})\cdot(\rot\bm{\varphi}) + \int_{\partial\Omega} \phi(\underbrace{(\rott \bm{g})\cdot\bm{n}}_{=0})= \lambda^2\int_\Omega \bm{g}\cdot\bm{\varphi} \]
On a donc le problème suivant :\\
Trouver $(\bm{g},\lambda)\in X\cap L^2_\sigma\times\R$ tel que $\forall \bm{\varphi}\in X\cap L^2_\sigma$ :
\[ \int_\Omega (\rot\bm{g})\cdot(\rot\bm{\varphi}) = \lambda^2\int_\Omega \bm{g}\cdot\bm{\varphi} \]

Ne pouvant pas utiliser directement des fonctions de bases à divergence nulle pour les éléments finis, on impose cette condition par un terme de pression fictif. On a donc :\\
Trouver $((\bm{g},p),\lambda)\in X\cap L^2_\sigma \times H^1 \times \R$ tel que $\forall (\bm{\varphi},q)\in X\cap L^2_\sigma \times H^1$ :
\begin{align*}
\int_\Omega (\rot\bm{g})\cdot(\rot\bm{\varphi}) + \int_\Omega\bm{\varphi}\grad p &= \lambda^2\int_\Omega \bm{g}\cdot\bm{\varphi}\\
\int_\Omega (\div\bm{g}) q &= 0
\end{align*}
En intégrant par partie la seconde équation, on obtient
\[ \int_\Omega (\div\bm{g}) q = \int_\Omega \bm{g}\grad q - \int_{\partial\Omega} (\underbrace{\bm{g}\cdot \bm{n}}_{=0})q = 0 \]
On peut donc imposer les contraintes (\ref{impPb2}-\ref{impPb3}), liées à $L^2_\sigma$, dans la formulation faible suivante :\\
Trouver $((\bm{g},p),\lambda)\in X \times H^1 \times \R$ tel que $\forall (\bm{\varphi},q)\in X \times H^1$ :
\[ \int_\Omega (\rot\bm{g})\cdot(\rot\bm{\varphi}) + \int_\Omega\bm{\varphi}\grad p + \int_\Omega \bm{g}\grad q = \lambda^2\int_\Omega \bm{g}\cdot\bm{\varphi} \]

Pour imposer la condition (\ref{impPb4}), on utilise une méthode de pénalisation, le problème devient donc :
Trouver $((\bm{g},p),\lambda)\in H(rot) \times H^1 \times \R$ tel que $\forall (\bm{\varphi},q)\in H(rot) \times H^1$ :
\begin{equation}\label{impFvEigen}
\int_\Omega (\rot\bm{g})\cdot(\rot\bm{\varphi}) + \int_\Omega\bm{\varphi}\grad p + \int_\Omega \bm{g}\grad q + \gamma\int_{\partial\Omega}(\rot\bm{g_i}\cdot\bm{n})(\rot\bm{\varphi}\cdot\bm{n}) = \lambda^2\int_\Omega \bm{g}\cdot\bm{\varphi}
\end{equation}
avec $\gamma$ une très grande valeur.\\

Comme on cherche la solution dans l'espace $H(rot)$, nous devrions utiliser des éléments conformes à cet espace, à savoir les éléments de Nedelec.\\
\todo[inline]{voir au dessus}
\subsection{Implémentation}

Cependant, les éléments de Nedelec n'étant pas prêt dans Feel++, on se place dans un espace plus large, $H^1$, et on utilise des éléments de Lagrange. Cela implique que $\rot\bm{g}$ n'appartient pas forcément à $L^2$. Cette méthode est donc moins précise et elle introduit de nouvelles erreurs de calcul.\\

Ces erreurs mènent à une divergence non nulle, on utilise donc un terme de pénalisation en plus afin de forcer la divergence. On résout donc le système suivant :
\[ \int_\Omega (\rot\bm{g})\cdot(\rot\bm{\varphi}) + \int_\Omega\bm{\varphi}\grad p + \int_\Omega \bm{g}\grad q + \gamma\int_{\partial\Omega}(\rot\bm{g_i}\cdot\bm{n})(\rot\bm{\varphi}\cdot\bm{n}) + \alpha\int_\Omega \div\bm{g}\div\bm{\varphi} = \lambda^2\int_\Omega \bm{g}\cdot\bm{\varphi} \]

Cependant, on a toujours la condition $\bm{g}\cdot\bm{n}$ qui n'est pas respecter. On ajoute donc un troisième terme de pénalisation :
\begin{align*}
\int_\Omega (\rot\bm{g})\cdot(\rot\bm{\varphi}) &+ \int_\Omega\bm{\varphi}\grad p + \int_\Omega \bm{g}\grad q & \\
&+ \gamma\int_{\partial\Omega}(\rot\bm{g_i}\cdot\bm{n})(\rot\bm{\varphi}\cdot\bm{n}) & (\ref{impPb4})\\
& + \alpha\int_\Omega \div\bm{g}\div\bm{\varphi} & (\ref{impPb2})\\
&+ \beta\int_{\partial\Omega}\bm{g_i}\cdot\bm{n})(\bm{\varphi}\cdot\bm{n})  = \lambda^2\int_\Omega \bm{g}\cdot\bm{\varphi} & (\ref{impPb3})
\end{align*}

Comme on cherche les solutions dans $[H^1(\Omega)]^3$, on utilise des éléments de Lagrange et non de Nedelec pour approcher les $\bm{g_i}$. Pour le terme de pression fictif, on utilise aussi des éléments de Lagrange, scalaire et d'ordre 1.\\ 
\lstinputlisting[linerange=space]{../../src/eigenprob.hpp}
Pour résoudre le problème $Ax=\lambda Bx$, on initialise $B$ comme une matrice de masse. On utilise pour cela une forme bilinéaire :\\
\lstinputlisting[linerange=rhsB]{../../src/eigenprob.cpp}
On ajoute maintenant les termes 
\[ \int_\Omega \rot \bm{g}\cdot\rot\bm{\varphi} + \int_\Omega\bm{\varphi}\grad p + \int_\Omega \bm{g}\grad q \]
à une autre forme bilinéaire qui formera la matrice $A$ :
\lstinputlisting[linerange={acurl,presgrad}]{../../src/eigenprob.cpp}
Afin de pouvoir contrôler les paramètres de pénalisation, on les ajoute en tant qu'options :
\lstinputlisting[linerange=options]{../../src/eigenprob.cpp}
On peut maintenant rajouter les trois termes de pénalisations :
\lstinputlisting[linerange={divdiv,bccurln,bcn}]{../../src/eigenprob.cpp}
Les deux matrices étant prêtes, on peut maintenant appeler une routine SLEPc pour résoudre le problème aux valeurs propres. Le solveurs et le pré-conditionneur sont définis en tant qu'options. Par défaut, on utilise un solveur de type Krylov-Schur (voir \ref{arnoldi}), avec une transformation de type \emph{shift and invert} qui facilite la recherche des valeurs propres de plus petites magnitudes.
\lstinputlisting[linerange=eigen]{../../src/eigenprob.cpp}
\todo[inline]{précision itératif/direct}

\subsection{Résultats}
Une fois les termes de pénalisation appliqués avec $\alpha=\beta=\gamma=1000$, les conditions imposées sont bien respectés, toutefois, la condition $\rot\bm{g}\cdot\bm{n}$ est plus difficile à imposer que les autres. Cela peut aussi être dû à des erreurs de calculs sur le rotationnel.
\begin{center}
\begin{tabular}{ >{$}c<{$} | >{$}c<{$} | >{$}c<{$} | >{$}c<{$} | >{$}c<{$} }
\no & \lambda^2 & \div\star & \star\cdot\bm{n} & \rot\star\cdot\bm{n} \\ \hline
0 & 40.9302 & 0.00210937 & 0.0232026 & 0.185115 \\ \hline
20 & 62.7893 & 0.00142931 & 0.00150083 & 0.0948699 \\ \hline
30 & 72.3563 & 0.00236504 & 0.0244683 & 0.240543 \\ \hline
45 & 90.406 & 0.00354548 & 0.0201967 & 0.32537
\end{tabular}
\end{center}

\begin{figure}[H]
	\makebox[\textwidth][c]{
		\subfloat[mode 0]{\includegraphics[scale=0.3]{curl-grad-1e3-mode0}}\ 
		\subfloat[mode 20]{\includegraphics[scale=0.3]{curl-grad-1e3-mode20}}
	}\\
	\makebox[\textwidth][c]{
		\subfloat[mode 30]{\includegraphics[scale=0.3]{curl-grad-1e3-mode30}}\ 
		\subfloat[mode 45]{\includegraphics[scale=0.3]{curl-grad-1e3-mode45}}
	}
	\caption{Modes propres}
	\label{resultats}
\end{figure}

Cependant, lorsqu'on calcul l'erreur entre le rotationnel du champ et le champ multiplier par sa valeur propre, on voit qu'il y a des erreurs, bien que la norme du rotationnel est bien $\lambda$ fois la norme du vecteur propre.
\begin{center}
\begin{tabular}{ >{$}c<{$} | >{$}c<{$} | >{$}c<{$} | >{$}c<{$} | >{$}c<{$} | >{$}c<{$} | >{$}c<{$} | >{$}c<{$} | >{$}c<{$} }
\no & \lambda^2 & \lambda & \div\star & \star\cdot\bm{n} & \rot\star\cdot\bm{n} & ||\rot\star-\star|| & ||\star|| & ||\rot\star|| \\ \hline
10 & 49.1504 & 7.0107 & 0.00217561 & 0.0250946 & 0.208425 & 9.62716 & 0.999049 & 6.94417
\end{tabular}
\end{center}

De plus, en regardant les deux champs, on voit qu'ils n'ont pas la même forme. Par exemple, la figure \ref{eigendiff} montre la différence entre le 10\ieme\ vecteur propre et son rotationnel :

\begin{figure}[H]
	\makebox[\textwidth][c]{
		\subfloat[mode 10]{\includegraphics[scale=0.3]{mode10}}\ 
		\subfloat[$\rot($mode 10$)$]{\includegraphics[scale=0.3]{curl10}}
	}
	\caption{Différences entre les modes et leur rotationnel}
	\label{eigendiff}
\end{figure}
\todo[inline]{problème calcul curl régularité pb ordre maillage, tant pis, utilisation de $\lambda\bm{g_i}$ à la place}
Cela peut être dû aux paramètres de pénalisation, en effet si on change ces paramètres, on trouve des valeurs propres complètement différentes :

\begin{center}
\begin{tabular}{ >{$}c<{$} | >{$}c<{$} | >{$}c<{$} | >{$}c<{$} | >{$}c<{$} | >{$}c<{$} | >{$}c<{$} }
\alpha & \beta & \gamma & \lambda^2 & \div\star & \star\cdot\bm{n} & \rot\star\cdot\bm{n} \\ \hline
0 & 0 & 0 & 5.24707e-11 & 15.0503 & 0.477078 & 1.23413e-07 \\ \hline
10^3 & 10^3 & 10^3 & 40.9302 & 0.00210937 & 0.0232026 & 0.185115 \\ \hline
10^6 & 10^6 & 10^6 & 57.1154 & 0.000159613 & 0.000673913 & 0.466915 \\ \hline
10^{10} & 10^{10} & 10^{10} & 61.731 & 1.3273e-06 & 6.06696e-06 & 0.847356 \\ \hline
10^{15} & 10^{15} & 10^{15} & 1.36827 & 2.30829e-09 & 5.37688e-08 & 5.57285 \\ \hline
10^{10} & 10^3 & 10^3 & 42.0751 & 6.2461e-09 & 0.0310199 & 0.299304 \\ \hline
10^3 & 10^{10} & 10^3 & 42.0096 & 0.00280563 & 0.0302683 & 0.280824 \\ \hline
10^3 & 10^3 & 10^{10} & 57.5605 & 0.00622557 & 1.23812e-06 & 0.621214
\end{tabular}
\end{center}

Il y a donc un problème dans la méthode de pénalisation. Une explication pourrait être que les différents termes de pénalisation sont largement supérieurs aux termes correspondant au problème aux valeurs propres, et donc on ne résoudrait plus ce problème, mais un problème avec seulement les termes de pénalisation.\\

\subsection{Temps de calculs}
La figure \ref{tpsMode} montre les temps de calculs en fonction du nombre de modes propres, entre 1 et 100, pour différentes tailles de maillage.\\
Ces tests on été effectués sur une machine virtuelle Linux Debian 6, invitée sur un ordinateur avec 4 processeurs Intel Xeon cadencés à 2,4 GHz et 64 Go de RAM, mais avec seulement 47 Go de RAM alloués à la machine virtuelle.\\
J'ai effectué les tests sur les maillages suivants avec des éléments de Lagrange d'ordre 3 vectoriel pour $\bm{g}$ et d'ordre 1 scalaire pour $p$. \\

\begin{center}
\begin{tabular}{c|c|c|c}
taille du maillage & nombre d'éléments & Degré de liberté & Degré de liberté/proc\\ \hline
0,15 & 5000 & 80000 & 20000\\ \hline
0.125 & 7500 & 125000 & 30000\\ \hline
0.1 & 12500 & 200000 & 50000
\end{tabular}
\end{center}

\begin{figure}[H]
\centering
\includegraphics{tpsMode}
\caption{Temps de calcul en fonction du nombre de modes propres}
\label{tpsMode}
\end{figure}

% \section{Composante $z$}
% \subsection{Implémentation}
% \subsection{Résultats}

%%% Local Variables:
%%% TeX-master: "../report.tex"
%%% eval: (flyspell-mode 1)
%%% ispell-local-dictionary: "french"
%%% End:


\chapter{Relèvement pour le calcul de $\mathbf{a}$}
On implémente ici le relèvement introduit dans le chapitre \ref{relev}, $\mathbf{a}=\grad\psi^0+\rot\mathbf{b}$.\\
Dans le cas du cylindre, on a $\alpha_1=0$, ce qui conduit à ce que $\rot \mathbf{b}=0$ et donc comme il ne reste plus que la condition $\mathbf{a}\cdot\mathbf{n}=\alpha_0$ à relever, on a \[ \mathbf{a}=\grad\psi^0 \]

\section{$\psi^0$ dans $H^1$}
\subsection{Discrétisation}\label{discGradh1}
Pour discrétiser le problème, on doit tout d'abord créer un maillage $\mathcal{T}_h$ de $\Omega$. Dans le volume on utilise des tétraèdres et sur les surfaces, pour un maillage conforme, on utilise des triangles. On a donc $\mathcal{T}_h=\{K_e\}_{e=1,\dots,N_e}$ où $N_e$ est le nombre d'éléments du maillage.\\
Si l'on veut résoudre le problème \ref{psi0} dans $H^1$, on va devoir utiliser des éléments de Lagrange (voir \ref{eltLagrange}) pour discrétiser $\psi^0$. On introduit donc l'espace
\[   P^k_{c,h} = \{ q_h \in C^0(\Omega) \; |\; q_h{}_{|_K} \in \mathbb{P}_k\; \forall\; K \in \mathcal{T}_h\} \subset H^1 \]
des fonctions continues, polynomial de degré $k$ sur chaque maille du maillage.\\
Cet espace est un sous-espace vectoriel de $H^1$, et donc lui-même un espace de Hilbert, mais il est de dimension finie.\\

Le problème \ref{fvpsi0} devient donc :
\begin{pb}\label{dcpsi0}
Trouver $(\psi^0_h, \lambda)\in P^k_{c,h}\times \R$ tel que $\forall (\varphi_h,\nu)\in P^k_{c,h}\times \R$ on a :
\[ \int_\Omega\grad \psi^0_h\grad\varphi_h + \int_\Omega \psi^0_h\nu + \int_\Omega \lambda\varphi_h = \int_{\partial\Omega} \alpha_0\varphi_h \]
\end{pb}

On note :
\begin{itemize}
\item $N_{hk}$ la dimension de $P^k_{c,h}$,
\item $\{\phi_i\}_{i=1,\dots,N_{hk}}$ une base de $P^k_{c,h}$,
\item pour tout $P^k_{c,h}\ni u_h=\sum_{i=1}^{N_{hk}} u_i\phi_i$, $U=(u_1,\dots,u_{N_{hk}})^T$,
\item $a$ une forme bilinéaire de $P^k_{c,h}\times P^k_{c,h}$ tel que : $a(u,v)=\int_\Omega \grad u\cdot \grad v$,
\item $b$ une forme bilinéaire de $P^k_{c,h}\times\R$ tel que : $b(u,\lambda) = \int_\Omega u\lambda$,
\item $f$ une forme linéaire de $P^k_{c,h}$ tel que : $f(v)=\int_{\partial\Omega} \alpha_0 v$
\end{itemize}

Alors la forme matricielle du problème \ref{dcpsi0} est :
\[ \begin{pmatrix} A & B^T\\ B & 0\end{pmatrix}\begin{pmatrix}U\\ \lambda\end{pmatrix} = \begin{pmatrix} F\\0 \end{pmatrix} \]
avec $A\in M(\R)^{N_{hk}\times N_{hk}}$, $B\in M(\R)^{N_{hk}\times 1}$ et $F=\R^{N_{hk}}$ où :
\[ A_{ij} = a(\phi_i,\phi_j)\quad B_{i1} b(\phi_i,1)\quad F=f(\phi_i) \]

\subsection{Implémentation}\label{impGradh1}
Pour résoudre le problème \ref{dcpsi0}, on doit d'abord choisir l'ordre $k$ de l'espace $P_{c,h}^k$. Ici on va prendre 2.\\
Pour utiliser un réel, on va prendre un élément de $P_{c,h}^0$, en effet, ce sont des fonctions continues et des polynômes de degré 0, des constantes sur chaque maille, ce sont donc des constantes sur le domaine.\\
On utilise le mot clé \texttt{FunctionSpace} pour créer un espace de fonctions, on lui donne le domaine sur lequel ces fonctions s'appliquent, et la base utilisée en paramètres. Ici, on utilise une base de lagrange d'ordre 2, scalaire, et une base de Lagrange d'ordre 0 scalaire pour créer $P^2_{c,h}\times P^0_{c,h}$.\\
\lstinputlisting[linerange={space}]{../../src/psi0.hpp}

On ajoute une fonction permettant de spécifier en option le profil d'entrée en fonction de $x$, $y$, du rayon du cylindre et de la vitesse souhaitée. Cela correspond à $2v\left(1-\frac{x^2+y^2}{R^2}\right)$.\\

\lstinputlisting[linerange={option}]{../../src/psi0.cpp}

Une fois les éléments de l'espace créé, on peut définir la forme bilinéaire $a$ de la façon suivante :\\

\lstinputlisting[linerange={bilinearA}]{../../src/psi0.cpp}

Ici, $u$ correspond à $\psi^0$ et $v$ à $\varphi$, \texttt{trial} est l'espace de solution alors que \texttt{test} est l'espace où vivent les fonctions tests. \texttt{inner} est le produit scalaire, \texttt{grad} calcul le gradient de la fonction test, et \texttt{gradt} calcul celui de la fonction solution. On ajoute ce \texttt{t} à la fin des opérateurs pour indiquer quel fonctions de bases utilisées.\\
La forme $b$ est défini par :\\
\lstinputlisting[linerange={bilinearB}]{../../src/psi0.cpp}

D'après (\ref{alpha0}), le second membre est :\\
\lstinputlisting[linerange={rhs}]{../../src/psi0.cpp}

Une fois le problème résolut, on veut projeter le gradient de $\psi^0$ sur $L^2$. Pour cela on résout le problème simple $u_h=\grad\psi_h^0$ avec $u_h\in [P^2_{c,h}]^3$ qui mène à la forme variationnelle suivante :
\[ \int_\Omega \mathbf{u}_h\cdot\mathbf{v}_h = \int_\Omega \grad\psi_h^0\cdot\mathbf{v}_h \]

\lstinputlisting[linerange={gradpsi0}]{../../src/psi0.cpp}

\subsection{Résultats}
Dans les figures \ref{az},\ref{aIn},\ref{aOut}, on peut observer $\mathbf{a}$ dans le cylindre. Ici, $\alpha_1=0$ et 
\[ \alpha_0(x,y)= \begin{cases} -2v\left(1-\frac{x^2+y^2}{r^2}\right) &\mbox{sur } \Gamma_1\\
2v\left(1-\frac{x^2+y^2}{r^2}\right)&\mbox{sur } \Gamma_2\\
0 &\mbox{sur } \Gamma_3 \end{cases} \]

\begin{figure}[H]
\centering
\includegraphics[scale=0.3]{az}
\caption{composante $z$ de $\mathbf{a}$}
\label{az}
\end{figure}
\begin{figure}[H]
\centering
\includegraphics[scale=0.5]{aIn}
\caption{entrée du cylindre}
\label{aIn}
\end{figure}
\begin{figure}[H]
\centering
\includegraphics[scale=0.5]{aOut}
\caption{sortie du cylindre}
\label{aOut}
\end{figure}

%\section{Gradient dans $\HH(\mathrm{div})$}

%%% Local Variables:
%%% TeX-master: "../report.tex"
%%% eval: (flyspell-mode 1)
%%% ispell-local-dictionary: "french"
%%% End:


\chapter{Problème spectral}
\section{Cas général}
\label{PSNewton}
Intéressons nous maintenant au problème (\ref{fvspec}).\\
Tout d'abord, dans le cylindre, comme $\alpha_1=0$, on a $\bm{a}=\grad\psi^0$ et donc $\rot \bm{a} = 0$. Ce qui amène le terme $\sum_i c_i((\rot \mathbf{a})\times \mathbf{g_i}, \mathbf{g_k})$ à être nul.\\
Cette équation comporte aussi le terme
\[ \sum_i\sum_j c_i\lambda_i c_j(\mathbf{g_i}\times \mathbf{g_j}, \mathbf{g_k}) \]
qui est non linéaire. Ce problème s'écrit donc sous la forme :
\[ F(c) = 0 \]
où $F:\R^M\rightarrow\R^M$, $c=(c_1,\ldots, c_M)$ et :
\begin{align}
F_k(c) = \frac{1}{Re} c_k\lambda_k^2 &+ \sum_i c_i\lambda_i(\mathbf{g_i}\times \mathbf{a}, \mathbf{g_k}) \nonumber \label{psf}\\
&+ \sum_{i,j} c_i\lambda_i c_j (\mathbf{g_i}\times \mathbf{g_j}, \mathbf{g_k}) - (\mathbf{h_a},\mathbf{g_k})
\end{align}

On va utiliser une méthode de Newton pour résoudre ce problème, on cherche donc :
\begin{equation}\label{Newton}
c^{(l+1)} = c^{(l)} - J(c^{(l)})^{-1}F(c^{(l)})\quad l=0,1,\ldots
\end{equation}
où $c^{(0)}$ est une donnée initiale et $J(c)$ est la matrice jacobienne de $F$ en $c$ :
\[ J(c)=
\begin{pmatrix}
\frac{\partial F_1(c)}{\partial c_1} & \ldots & \frac{\partial F_1(c)}{\partial c_M}\\
\vdots & \ddots & \vdots\\
\frac{\partial F_M(c)}{\partial c_1} & \ldots & \frac{\partial F_M(c)}{\partial c_M}
\end{pmatrix}\]
avec 
\begin{align}
J(c)_{ki} = \frac{\partial F_k(c)}{\partial c_i} &= \delta_{ki}\lambda_i^2 + \lambda_i(\mathbf{g_i}\times \mathbf{a},\mathbf{g_k}) \nonumber \label{psj}\\
&+ \sum_j\lambda_i c_j (\mathbf{g_i}\times\mathbf{g_j},\mathbf{g_k}) + \sum_j c_j\lambda_j (\mathbf{g_j}\times\mathbf{g_i},\mathbf{g_k})
\end{align}
avec $\delta_{ki}$ le symbole de Kronecker.\\

Résoudre (\ref{Newton}) est équivalent à résoudre 
\begin{align}
J(c^{(l)})\delta c^{(l)} = -F(c^{(l)})\label{INewton1}\\
c^{(l+1)}=c{(l)}+\delta c^{(l)}\label{INewton2}
\end{align}

On itère ce système jusqu'à ce que la différence entre deux itération soit inférieure à une certaine tolérance, c'est-à-dire $||\delta c^{(l)}||<tol$ ou que le nombre d'itération soit trop grand, ce qui indique la non convergence du système.

\subsection{Implémentation}

Comme on va manipuler des matrices pleines, au lieu d'utiliser PETSc, qui est plus adapté aux matrices creuses, on utilise la librairie Eigen \cite{eigenweb}.\\
Dans cette librairie, on déclare une matrice comme \texttt{Matrix<type, ligne, colonne>}. Afin de pouvoir utiliser des matrices de tailles différentes à chaque exécution, on utilise le mot clé \texttt{Dynamic} pour la taille. Il existe un type pré-déclaré pour le type \texttt{Matrix<Double, Dynamic, Dynamic>} : \texttt{MatrixXd}. De même pour le type \texttt{Matrix<Double, Dynamic, 1>}, qui est un vecteur de double de taille variable, on utilise le mot clé \texttt{VectorXd}.\\

Par exemple, si on a un vecteur de la librairie standard contenant les différentes valeurs des $\lambda_i$, on les stock dans un vecteur de la librairie Eigen avec les lignes de code suivantes : 
\lstinputlisting[linerange=lambda]{../../src/spectralproblem.cpp}
où $M$ est le nombre de valeurs propres que l'on a précédemment calculées. On remarque aussi que l'on peut accéder aux éléments du vecteur avec l'opérateur \texttt{()}.\\

Chaque somme de l'équation \ref{fvspec} peut être considérée comme un élément de la librairie Eigen. Ainsi $R_{iak}=\lambda_i\int (\bm{g_i}\times\bm{a})\cdot\bm{g_k}$ est un élément d'une matrice, celui de la $i$\ieme\ ligne et $k$\ieme\ colonne.\\
$R_{hk} = \int \bm{h_a}\cdot\bm{g_k}$ est un vecteur et $R_{ijk} = \lambda_i\int (\bm{g_i}\times\bm{a})\cdot\bm{g_k}$ est un élément de 3 dimensions, donc un vecteur de matrices. On a ainsi les déclarations suivantes :\\
\lstinputlisting[linerange=ri]{../../src/spectralproblem.hpp}

On initialise d'abord la mémoire nécessaire pour contenir le vecteur, puis on initialise chaque élément de la matrice $R_{iak}$ à l'aide de deux boucles imbriquées :\\
\lstinputlisting[linerange=riak]{../../src/spectralproblem.cpp}

De même, on initialise la mémoire utilisée par la matrice, puis chaque élément du vecteur $R_{hk}$ avec une boucle :\\
\lstinputlisting[linerange=rfk]{../../src/spectralproblem.cpp}

Pour $R_{ijk}$, on doit tout d'abord initialiser le vecteur contenant les matrices, puis pour chaque élément de ce vecteur, initialisé la mémoire de la matrice elle-même. Ensuite seulement, on peut initialiser chaque élément de $R_{ijk}$ :\\
\lstinputlisting[linerange=rijk]{../../src/spectralproblem.cpp}

Afin d'appliquer la méthode de Newton, il faut d'abord initialiser un vecteur pour stocker la solution $\delta c^{(l)}$ au problème \ref{INewton1} et choisir la tolérance pour laquelle on considère le système résolut.\\
\lstinputlisting[linerange=NSInit]{../../src/spectralproblem.cpp}

On peut maintenant appliquer la méthode de Newton décrite dans \ref{PSNewton}. On veut donc exprimer le vecteur $F(c)$ (\ref{psf}) dans la librairie Eigen. On utilise donc le mot clé \texttt{wiseProduct} pour faire un produit élément par élément de $c$ et de $\lambda$, et on multiplie la matrice $R_{iak}$ avec $c$, on retranche aussi le vecteur $R_{fk}$.\\
Puis pour la ligne $k$, le terme non linéaire $\sum_{i,j} c_i\lambda_i c_j (\mathbf{g_i}\times \mathbf{g_j}, \mathbf{g_k})$ est le produit $c^TR_{ijk}(k)c$ où $R_{ijk}(k)$ est la matrice se trouvant à la $k$\ieme\ position dans le vecteur $R_{ijk}$ et $c^T$ est le vecteur transposé de $c$.\\
\lstinputlisting[linerange=NSMatF]{../../src/spectralproblem.cpp}

Pour la matrice $J(c)$ (\ref{psj}), le terme $\delta_{ik}\lambda^2$, signifie que le vecteur $\lambda^2$ se trouve sur la diagonale de $J(c)$, on utilise donc le mot clé \texttt{asDiagonal} pour effectuer cette opération. On ajoute aussi la matrice $R_{iak}$ telle quelle à $J(c)$.\\
Le terme $\sum_j c_j\lambda_j (\mathbf{g_j}\times\mathbf{g_i},\mathbf{g_k})$ est égal au $i$\ieme\ élément du vecteur produit $c^TR_{ijk}$, tandis que $\sum_j\lambda_i c_j (\mathbf{g_i}\times\mathbf{g_j},\mathbf{g_k})$ est le $i$\ieme\ élément du vecteur $c^TR_{ijk}^T$.\\
\lstinputlisting[linerange=NSMatJ]{../../src/spectralproblem.cpp}

Il faut aussi initialiser le solveur avec lequel résoudre les systèmes \ref{INewton1}, par exemple, pour les résoudre avec une méthode de Householder, on utilisera les lignes suivantes :\\
\lstinputlisting[linerange={NSSys1,NSSys2}]{../../src/spectralproblem.cpp}

Il suffit maintenant d'additionner $c^{(l)}$ et $c^{(l+1)}$ pour obtenir la solution à l'itération suivante.\\
\lstinputlisting[linerange=NSAdd]{../../src/spectralproblem.cpp}

On itère ces étapes jusqu'à ce que la tolérance soit dépassée, ou que l'on pense que la méthode diverge.

\section{Stokes}
\subsection{Implémentation}

Si l'on supprime le terme non linéaire, l'équation s'écrit :
\[ \sum_i c_i\lambda_i(\bm{g_i}\times\bm{a},\bm{g_k}) +\frac{1}{Re}c_k\lambda^2 = (\bm{h_a},\bm{g_k}) \]

Ainsi, il suffit de résoudre l'équation $Ax=b$ où $A=(a_{kj})$ avec :
\[ a_{kj} = \delta_{kj}\frac{\lambda^2}{Re} + \lambda_j\int_\Omega (\bm{g_j}\times\bm{a})\cdot\bm{g_k} \]
et
\[ b_k = \int_\Omega \bm{h_a}\cdot\bm{g_k} \]

Il faut donc créer la matrice $A$ à l'aide des $\lambda$ et de $R_{iak}$ :\\
\lstinputlisting[linerange=StokesA]{../../src/spectralproblem.cpp}
et le vecteur $b$ avec $R_{hk}$ :\\
\lstinputlisting[linerange=StokesB]{../../src/spectralproblem.cpp}

Puis, on résout le système de la même manière que précédemment :\\
\lstinputlisting[linerange=StokesSolve]{../../src/spectralproblem.cpp}

\subsection{Résultats}

% \section{Composante $z$}
% \subsection{Implémentation}
% \subsection{Résultats}


%%% Local Variables:
%%% TeX-master: "../report.tex"
%%% eval: (flyspell-mode 1)
%%% ispell-local-dictionary: "french"
%%% End:


\section{Pressure : Post-traitment of the velocity}
\label{pressure}
To find the velocity $\mbf{v}$, we just need to add $\mbf{a}$ and $\mbf{u}$.\\
Since we need to post-trait the velocity to find the pressure term, we start from the equation of the problem \ref{start}.

We apply divergence on it and use the fact that the velocity is divergence free, and that the one of a curl is also always null. We have then :
\begin{equation*}
-\laplace q = \div((\curl \mbf{v})\times \mbf{v}) - \div \mbf{f}
\end{equation*}

To get a boundary condition, we use the normal component of the equation and the boundary conditions of $\mbf{v}$ :
\[ \grad q\cdot \mbf{n}\restr{\Gamma} =  \mbf{f}\cdot \mbf{n}\restr{\Gamma} - \frac{\partial\alpha_0}{\partial t} - ((\curl \mbf{v})\times \mbf{v})\cdot \mbf{n}\restr{\Gamma} - \frac{\alpha_2}{Re} \]
We want now the weak formulation of the problem :
\[ \int_\Omega -\laplace q\varphi = \int_\Omega (\div((\curl \mbf{v})\times \mbf{v}) -\div \mbf{f})\varphi \]
By integrating by parts the left term, we have :
\[ \int_\Omega \grad q\grad\varphi - \int_{\partial\Omega} (\grad q\cdot \mbf{n})\varphi = \int_\Omega (\div((\curl \mbf{v})\times \mbf{v}) -\div \mbf{f})\varphi \]

Again, we find the pressure up to a constant, we use the Lagrange multiplier to fix this constant. As in \ref{multLagrange}, we get :
\begin{pb}\label{fvq}
Find $p=q-\frac{\mbf{v}\cdot\mbf{v}}{2} \in L^2(\Omega)$ such that $\forall \varphi\in L^2(\Omega)$, we have :
\begin{align*}
\int_\Omega \grad q\grad\varphi + \int_\Omega \lambda\varphi + \int_\Omega q\nu &= \int_\Omega (\div((\curl \mbf{v})\times \mbf{v}) -\div \mbf{f})\varphi\\
&+ \int_{\partial\Omega} \left(f\cdot \mbf{n} - \frac{\partial\alpha_0}{\partial t} - ((\curl \mbf{v})\times \mbf{v})\cdot \mbf{n} - \frac{\alpha_2}{Re}\right)\varphi
\end{align*}\end{pb}

%%% Local Variables:
%%% TeX-master: "../report.tex"
%%% eval: (flyspell-mode 1)
%%% ispell-local-dictionary: "english"
%%% End:


\chapter{Recapitulatif}

The steps to resolve the problm are :
\begin{enumerate}
\item compute the eigenpairs of the curl operator with problem \ref{pbweak},
  \todo[inline]{Ok}
\item decompose those functions with \ref{fvgi0} and \ref{fvpsiml},
  \todo[inline]{May be not necessary}
\item find $a_0$, using one of the following :
  \begin{itemize}
  \item solve \ref{fva0} to have $a_0\in H^1$,
  \item solve \ref{fva0div} to have $a_0\in H(\mathrm{div})$,
  \end{itemize}
  \todo[inline]{Ok}
\item solve \ref{fva1} or \ref{pbbd1} to find $\curl\mbf{a}_1$,
  \todo[inline]{to check}
\item solve \ref{fva2} to find $\mbf{a}_2$;
  \todo[inline]{to check}
\item recompose $\mbf{a}$ thanks to $\grad a_0$, $\curl\mbf{a}_1$ and $\mbf{a}_2$,
\item \label{itemsp} solve the spectral proble \ref{fvc} in order to find the coefficients $c_i$ for $i=0\dots M$,
  \todo[inline]{Ok}
\item rebuild $\mbf{u}=\sum c_i \mbf{g}_i$ and $\mbf{v}=\mbf{a}+\mbf{u}$,
\item compute the pressure in post-traitment with \ref{fvq}.
\end{enumerate}

We now have find our solution $(\mbf{v},p)$.\\

The first step depends only on the geometry, so we can reuse the eigen functions even after changing some parameters. The step \ref{itemsp} is the most time consumming one. We need to compute a lot of coefficient before starting. Luckily, the most of them depend only on the geometry, so we can compute them only once, but the others depend on the parameters and even worse, can also depend on the time, si we could have to recompute them at each time step.\\

The figure \ref{org3} graphically represents the problems to solve.\\

\begin{figure}
  \centering
  \begin{tikzpicture}[scale=\taille]
    \node[scale=\taille,text width=10cm] (ccyan) at (5,4) {{\color{cyan} Chemins pour $\psi^0\in H^1$}} ;
    \node[scale=\taille,text width=10cm] (cmagenta) at (5,3.5) {{\color{magenta} Chemins pour $\psi^0\in H(\mathrm{div})$}} ;
    \node[draw,scale=\taille,fill=green!50] (di) at (12,5) {Données initiales} ;
    \node[draw,scale=\taille,fill=blue!50] (si) at (12,4) {Solutions intermédiaires} ;
    \node[draw,scale=\taille,fill=yellow!50] (sim) at (12,3) {Sol. inter. dép. de la géométrie} ;
    \node[draw,scale=\taille,fill=red!50] (sf) at (12,2) {Solutions finales} ;

    \node[draw,scale=\taille,fill=green!50] (alpha0) at (0.75,2) {$\alpha_0$} ;
    \node[draw,scale=\taille,fill=gray!50,label={[xshift=-0.7cm](\ref{fva0})}] (pba0) at (-0.5,-2){
      $\begin{aligned}
        -\laplace\psi^0&=0\\
        \grad\psi^0\cdot \mbf{n} &= \alpha_0
      \end{aligned}$
    } ;
    \node[draw,scale=\taille,fill=blue!50] (a0) at (-0.5,-3.5) {$a_0$} ;
    \node[draw,scale=\taille,fill=gray!50] (pbgrada0) at (-0.5,-4.75) {$w=\grad a_0$} ;
    \node[draw,scale=\taille,fill=gray!50,label={[xshift=0.8cm](\ref{fva0div})}] (pba0div) at (2,-2){
      $\begin{aligned}
        \mbf{w}&=\grad a_0\\
        \div\mbf{w}&=0\\
        \mbf{w}\cdot \mbf{n} &= \alpha_0
      \end{aligned}$
    } ;
    \node[draw,scale=\taille,fill=blue!50] (grada0) at (0.75,-6.5) {$\grad a_0$} ;

    \node[draw,scale=\taille,fill=green!50] (alpha1) at (5,2) {$\alpha_1$} ;
    \node[draw,scale=\taille,fill=gray!50,label={[xshift=0.8cm](\ref{fva1})}] (pba1) at (5,-2){
      $\begin{aligned}
        \curll \mbf{a}_1 &= \grad\psi^1\\
        \div \mbf{a}_1 &=0\\
        \mbf{a}_1\cdot \mbf{n} &= 0\\
        \curl \mbf{a}_1\cdot \mbf{n} &= 0\\
        \grad \psi^1\cdot \mbf{n} &= \alpha_1
      \end{aligned}$
    } ;
    \node[draw,scale=\taille,fill=blue!50] (a1) at (5,-6.5) {$\curl\mbf{a}_1$} ;

    \node[draw,scale=\taille,fill=green!50] (alpha2) at (8,2) {$\alpha_2$} ;
    \node[draw,scale=\taille,fill=gray!50,label={[xshift=0.8cm](\ref{fva2})}] (pba2) at (8,-2){
      $\begin{aligned}
        \curll \mbf{a}_2 &= \bm{\epsilon}\\
        \div \mbf{a}_2 &=0\\
        \mbf{a}_2\cdot \mbf{n} &= 0\\
        \curl \mbf{a}_2\cdot \mbf{n} &= 0\\
        \bm{\epsilon}\cdot \mbf{n} &= \alpha_2
      \end{aligned}$
    } ;
    \node[draw,scale=\taille,fill=blue!50] (a2) at (8,-6.5) {$\mbf{a}_2$} ;

    \node[draw,scale=\taille,fill=gray!50,label={[xshift=-1.3cm](\ref{a})}] (pba) at (3,-9) {$\mbf{a} = \grad a_0 + \curl \mbf{a}_1 + \mbf{a}_2$} ;
    \node[draw,scale=\taillem,fill=blue!50] (a) at (3,-12) {$\mbf{a}$} ;

    \node[draw,scale=\taille,fill=LimeGreen,label={[xshift=1.0cm](\ref{pbcond})}] (pbcond) at (11.5,-2){
      $\begin{aligned}
        \curl \mbf{u} = \lambda \mbf{u} & \quad \mbox{in }\Omega\\
        \curl \mbf{u}\cdot\mbf{n} = 0 & \quad \mbox{on }\Gamma
      \end{aligned}$
    } ;
    \node[draw,scale=\taille,fill=yellow!50] (lambdagi) at (11.5,-6.5) {$(\lambda_i^2,\mbf{g}_i)$} ;

    \node[draw,scale=\taille,fill=green!50] (f) at (8,-9) {$f$} ;
    \node[draw,scale=\taille,fill=green!50] (c0) at (9,-9) {$c_k^0$} ;
    \node[draw,scale=\taille,fill=ProcessBlue,label={[xshift=3.2cm](\ref{fvc})}] (pbsp) at (9,-12){
      $\begin{aligned}
        \frac{\partial c_k}{\partial t} &+ \sum_i\sum_j c_i c_j(\curl\mbf{g}_i\times \mbf{g_j}, \mbf{g_k}) \\
        &+ \sum_i c_i(\curl\mbf{g}_i\times \mbf{a},\mbf{g_k}) + \sum_i c_i((\curl \mbf{a})\times \mbf{g}_i, \mbf{g_k}) \\
        &+ \frac{1}{Re}c_k\lambda_k^2 = (\mbf{f_a},\mbf{g_k})
      \end{aligned}$
    } ;
    \node[draw,scale=\taille,fill=blue!50] (ck) at (9,-15) {$c_k$} ;
    \node[draw,scale=\taille,fill=gray!50,label={[xshift=0.7cm](\ref{u})}] (pbu) at (9,-16) {$\mbf{u}=\sum c_kg_k$} ;
    \node[draw,scale=\taillem,fill=blue!50] (u) at (9,-17) {$\mbf{u}$} ;
    \node[draw,scale=\taille,fill=gray!50,label={[xshift=0.6cm](\ref{v})}] (pbv) at (3,-18) {$\mbf{v}=\mbf{a}+\mbf{u}$} ;
    \node[draw,scale=\tailleg,fill=red!50] (v) at (3,-19) {$\mbf{v}$} ;
    \node[draw,scale=\taille,fill=gray!50,label={[xshift=3.1cm](\ref{fvq})}] (pbq) at (9,-19){
      $\begin{aligned}
        -\laplace q = \div((\curl \mbf{v})\times \mbf{v}) - \div \mbf{f}\\
        \grad q\cdot \mbf{n}\restr{\Gamma} =  \mbf{f}\cdot \mbf{n}\restr{\Gamma} - \frac{\partial\alpha_0}{\partial t} - ((\curl \mbf{v})\times \mbf{v})\cdot \mbf{n}\restr{\Gamma} - \frac{\alpha_2}{Re}
      \end{aligned}$
    } ;
    \node[draw,scale=\tailleg,fill=red!50] (q) at(15,-19) {$p$} ;

    \draw[->,>=latex,cyan] (-1,4) -- (ccyan) ; \draw[->,>=latex,magenta] (-1,3.5) -- (cmagenta) ;
    \draw (alpha0) -- (0.75,0); \draw[->,>=latex] (grada0) -- (pba) ; \draw[->,>=latex,cyan] (0.75,0) -| (pba0) ; \draw[->,>=latex,cyan] (pba0) -- (a0) ; \draw[->,>=latex,cyan] (a0) -- (pbgrada0) ; \draw[->,>=latex,cyan] (pbgrada0) -- (grada0) ; \draw[->,>=latex,magenta] (0.75,0) -| (pba0div) ; \draw[->,>=latex,magenta] (pba0div) -- (grada0) ; \draw[->,>=latex] (grada0) -- (pba) ;
    \draw[->,>=latex] (alpha1) -- (pba1) ; \draw[->,>=latex] (pba1) -- (a1) ; \draw[->,>=latex] (a1) -- (pba) ;
    \draw[->,>=latex] (alpha2) -- (pba2) ; \draw[->,>=latex] (pba2) -- (a2) ; \draw[->,>=latex] (a2) -- (pba) ;
    \draw[->,>=latex] (pba) -- (a);
    \draw[->,>=latex] (pbcond) -- (lambdagi); \draw[->,>=latex] (lambdagi) -- (pbsp) ;
    \draw[->,>=latex] (a) -- (pbsp); \draw[->,>=latex] (f) -- (pbsp); \draw[->,>=latex] (c0) -- (pbsp);
    \draw[->,>=latex] (pbsp) -- (ck); \draw[->,>=latex] (ck) -- (pbu); \draw[->,>=latex] (pbu) -- (u); \draw[->,>=latex] (u) -- (pbv); \draw[->,>=latex] (a) -- (pbv); \draw[->,>=latex] (pbv) -- (v); \draw[->,>=latex] (v) -- (pbq); \draw[->,>=latex] (pbq) -- (q);
  \end{tikzpicture}
  \caption{Flow chart of the problems to solve}
  \label{org3}
\end{figure}

%%% Local Variables:
%%% TeX-master: "../report.tex"
%%% eval: (flyspell-mode 1)
%%% ispell-local-dictionary: "english"
%%% End:
