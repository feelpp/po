\section{Boundary Conditions}
\label{relev}
We want to handle the boundary conditions into their own function $\mbf{a}$, where $\mbf{v}=\mbf{u}+\mbf{a}$ :
\[ \mbf{a}\cdot\mbf{n}=\alpha_0\quad \curl\mbf{a}\cdot\mbf{n}=\alpha_1 \quad \curll\mbf{a}\cdot\mbf{n}=\alpha_2 \]

We write $\mbf{a}=\grad a_0 + \curl\mbf{a}_1 + \mbf{a_2}$.\\
We first look into $\grad a_0$ which handle the condition $\mbf{a}\cdot\mbf{n}=\alpha_0$.\\

\subsection{$a_0$ dans $H^1$}
\label{seca0hdiv}
There is several possibilites to solve \ref{pba0}, we can first solve it in $H^1(\Omega)$ :
\begin{equation*}
  \left\{
  \begin{aligned}
    &-\laplace a_0 = 0\\
    &\grad a_0\cdot \mbf{n}\restr{\Gamma}=\alpha_0
  \end{aligned}
  \right.
\end{equation*}

We need to be careful that the compatibility condition $\int_{\partial\Omega} \alpha_0$ must be equal to 0 :
\[ 0=\int_\Omega \laplace a_0 = \int_\Omega \div(\grad a_0) = \int_{\partial\Omega} \grad a_0\cdot \mbf{n} = \int_{\partial\Omega} \alpha_0 \]

To get the weak formulation, we multiply by a test function $\varphi\in H^1(\Omega)$ and we integrate :
\[ \int_\Omega \laplace a_0 \varphi = 0 \]
We use the Green formula to get :
\[ -\int_\Omega \grad a_0\cdot\grad\varphi + \int_{\partial\Omega} \grad a_0\cdot \mbf{n}\varphi = 0 \]
But, $\grad a_0\cdot \mbf{n} = \alpha_0$ on $\partial\Omega$, so we have the following :
\begin{equation}\label{fva0LM}
  -\int_\Omega \grad a_0\cdot\grad\varphi + \int_{\partial\Omega} \alpha_0\varphi = 0
\end{equation}
This problem allows us to find $a_0$ only up to a constant, so we need to use Lagrange multiplier.
\label{multLagrange}
If we note $V=H^1(\Omega)$, $a(u,v)=\int \grad u \cdot \grad v$, $l(v)=\int \alpha_0v$ and $J(v)=\frac{1}{2}a(v,v)-l(v)$, then solving equation \ref{fva0LM} becomes finding $u$ such that :
\[ J(u) = \min_{v\in V} J(v) \]
If we add the constraint $b(v) = \int v = 0$, then, with $\lambda$ a Lagrange multiplier, the problem becomes finding $u$ such that :
\[ J(u) = \min_{v\in V} J(v) - \lambda b(v) \]
So, by adding the constraint's equation multiply by the Lagrange multiplier, and the term corresponding to $m=0$, we have :
\[ a(\psi_i,\varphi) + \lambda b(v) + \mu b(u) = l(\varphi) + m b(\mu) \]
So the problem is :
\begin{pb}\label{fva0}
  Find $( a_0,\lambda)\in H^1(\Omega)\times L^2(\Omega)$ such that $\forall (\varphi,\mu)$ :
  \begin{align*}
    \int_\Omega \grad a_0\cdot\grad\varphi + \int_\Omega \lambda\varphi + \int_\Omega a_0\mu &= \int_\Omega \alpha_0\varphi
  \end{align*}
\end{pb}
\begin{rk}
  This gives us $a_0$ but we still need to compute its gradient.
\end{rk}

\subsection{$a_0$ in $H(\mathrm{div})$}
The other way to compute $a_0$ is to look for $(\mbf{w}, a_0)\in H(\mathrm{div})\times L^2(\Omega)$ solution of the mixed problem :
\begin{equation*}
  \left\{
  \begin{aligned}
    \mbf{w} &= \grad a_0\\
    \div \mbf{w} &= 0\\
    \mbf{w}\cdot \mbf{n}\restr{\Gamma} &= \alpha_0
  \end{aligned}
  \right.
\end{equation*}
To get the weak formulation of it, we multiply by a test function $(\bm{\varphi},\nu)\in H(\mathrm{div})\times L^2(\Omega)$ and we integrate the first two equations :
\begin{align*}
  \int_\Omega \mbf{w}\cdot\bm{\varphi} &= \int_\Omega \grad a_0\cdot\bm{\varphi}\\
  \int_\Omega \div \mbf{w}\ \nu &= 0
\end{align*}
We integrate by parts the latter :
\[ \int_\Omega \div \mbf{w}\ \nu = \int_{\partial\Omega} \mbf{w}\cdot \mbf{n}\ q - \int_\Omega \mbf{w}\cdot\grad\nu = 0  \]
By inserting the boundary condition in the first equation, we get the weak formulation :
\[ -\int_\Omega \mbf{w}\cdot\bm{\varphi} + \int_\Omega \mbf{w}\cdot\grad\nu + \int_\Omega \grad a_0\cdot\bm{\varphi}  = \int_{\partial\Omega} \alpha_0\nu \]

To add the null mean constraint, we proceed the same way as in \ref{multLagrange}. The problem \ref{pba0} becomes :
\begin{pb}\label{fva0div}
  Find $(\mbf{w}, a_0,\lambda)\in H(\mathrm{div})\times L^2(\Omega)\times L^2(\Omega)$ such that $\forall (\bm{\varphi},\nu,\mu)\in H(\mathrm{div})\times L^2(\Omega)\times L^2(\Omega)$
  \begin{equation*}
    -\int_\Omega \mbf{w}\cdot\bm{\varphi} + \int_\Omega \mbf{w}\cdot\grad\nu + \int_\Omega \grad a_0\cdot\bm{\varphi} + \int_\Omega \lambda\varphi + \int_\Omega  a_0\mu = \int_{\partial\Omega} \alpha_0\nu
  \end{equation*}
\end{pb}

\begin{rk}
  A benefit to computing $\grad a_0$ in $H(\mathrm{div})$ is that we gain a regularity order for $\grad a_0$ with respect to solving \ref{fva0} and the compute the gradient.
\end{rk}

\subsection{$\mbf{a}_1$ in $H(\mathrm{rot})$}
We now want to handle the condition $\curl\mbf{a}\cdot\mbf{n}=\alpha_1$, thanks to $\curl\mbf{a}_1$.\\
For this, we solve the mixed problem \ref{pba1}, in $H(\mathrm{rot})$ : 
\begin{equation*}
  \left\{
  \begin{aligned}
    &\curll \mbf{a}_1 = \grad\psi^1\\
    &\div \mbf{a}_1 = 0\\
    &\mbf{a}_1\cdot \mbf{n}\restr{\Gamma} = 0\\
    &\curl \mbf{a}_1\cdot \mbf{n}\restr{\Gamma} = 0\\
    &\grad\psi^1\cdot \mbf{n}\restr{\Gamma} = \alpha_1
  \end{aligned}
  \right.
\end{equation*}

To get the weak formulation, we multiply by a test function in $H(\mathrm{rot})$ and we integrate :
\[ \int_\Omega (\curll \mbf{a}_1)\cdot\bm{\varphi} = \int_\Omega (\grad\psi^1)\cdot\bm{\varphi} \]
By integrating by parts each terms, we get the following problem :
\begin{pb} \label{fva1}
  Find $(\mbf{a}_1,\psi^1)\in H(\mathrm{rot})\times L^2$ such that $\forall \bm{\varphi}\in H(\mathrm{rot})$ :
  \begin{equation*}
    \int_\Omega (\curl \mbf{a}_1)\cdot(\curl\bm{\varphi}) - \int_{\partial\Omega} ((\curl\mbf{a}_1)\times\bm{\varphi})\cdot \mbf{n}) + \int_\Omega \psi^1(\div\bm{\varphi}) - \int_{\partial\Omega} \psi^1(\bm{\varphi}\cdot \mbf{n}) = 0
  \end{equation*}
\end{pb}
\begin{rk}
  We can see that the boundary condition doesn't appear in that formulation.
\end{rk}

\subsection{$\mbf{a}_1$ in $D^1$}
Another way to find $\mbf{a}_1$ is to use the basis of $D^1$ computed previously to write $\mbf{a}_1$. Indeed, we have :
\[ \div\mbf{a}_1=0\quad \mbf{a}_1\cdot\mbf{n}\restr{\Gamma}=0\quad \curl\mbf{a}_1\cdot\mbf{n}\restr{\Gamma}=0\]
hence $\mbf{a}_1\in D^1$.\\
We can resolve the problem
\begin{pb}
  Find $\psi^1$ such that :
  \begin{equation*}
    \left\{
    \begin{aligned}
      -\laplace\psi^1 &= 0\\
      \grad\psi^1\cdot\mbf{n} &= \alpha_1
    \end{aligned}
    \right.
  \end{equation*}
\end{pb}
the same way than the problem \ref{pba0}.\\
Then, find $\mbf{a}_1$ such that $\forall k=1,\dots,M$ :
\[ \int_\Omega \curll\mbf{a}_1\cdot\mbf{g}_k = \int_\Omega \grad\psi^1\cdot\mbf{g}_k \]
Using integrations by parts and vector identities, we have :
\[ \int_\Omega \curl\mbf{a}_1\cdot\curl\mbf{g_k} + \int_{\partial\Omega}(\curll\mbf{a}_1\cdot\mbf{n})\psi_k = \int_\Omega \grad\psi^1\cdot\mbf{g_k} \]
where $\mbf{g}_k=\mbf{g}^0_k+\grad\psi_k$.\\
Using the fact that $\curll\mbf{a}_1=\grad\psi^1$ and that $\grad\psi^1\cdot\mbf{n}=\alpha_1$, and by noting $\mbf{a}_1=\sum d_i\mbf{g}_i$ we get the following problem :
\begin{pb}\label{pbbd1}
  Find $(d_i)_{i=1,\dots,M}$ such that $\forall k=1,\dots,M$ we have :
  \[ d_k\lambda_k^2 = \int_\Omega \grad\psi^1\cdot\mbf{g_k} - \int_{\partial\Omega} \alpha_1\psi_k \]
\end{pb}

\begin{rk}
  This formulation force us to decompose each eigen functions.
\end{rk}

\subsection{$\mbf{a}_2$}
Solving the problem \ref{pba2} is quite the same as the problem \ref{pba1}, but none of the solution above are suitable. I propose the following :\\
Solving :
\begin{equation*}
  \left\{
  \begin{aligned}
    -\laplace\psi^2 &=0\\
    \grad\psi^2\cdot\mbf{n} &= \alpha_2
  \end{aligned}
  \right.
\end{equation*}
and then find $\mbf{a}_2$ such that :
\begin{equation*}
  \left\{
  \begin{aligned}
    \curll\mbf{a}_2 &= \grad\psi^2\\
    (\curl\mbf{a})\times\mbf{n} &= 0
  \end{aligned}
  \right.
\end{equation*}

This leads to :
\begin{gather*}
  \int_\Omega \curll\mbf{a}_2\cdot\bm{\varphi} = \int_\Omega\grad\psi^2\cdot\bm{\varphi}\\
  \int_\Omega \curl\mbf{a}_2\cdot\curl\bm{\varphi} + \int_\Gamma (\mbf{n}\times(\curl\mbf{a}_2))\cdot\bm{\varphi} = \int_\Omega \grad\psi^2\cdot\bm{\varphi}\\
  \int_\Omega \curl\mbf{a}_2\cdot\curl\bm{\varphi} = \int_\Omega \grad\psi^2\cdot\bm{\varphi}
\end{gather*}

So the problem \ref{pba2} would becomes :
\begin{pb}\label{fva2}
  Find $\psi^2$ such that :
  \[ \int_\Omega \grad\psi^2\cdot\grad\varphi = \int_\Gamma \alpha_2\varphi \]
  Find $\mbf{a}_2$ such that :
  \[ \int_\Omega \curl\mbf{a}_2\cdot\curl\bm{\phi} = \int_\Omega \grad\psi^2\cdot\bm{\phi} \]
\end{pb}
  
%%% Local Variables:
%%% TeX-master: "../../report.tex"
%%% eval: (flyspell-mode 1)
%%% ispell-local-dictionary: "english"
%%% End:
