\section{Problème aux valeurs propres}
\label{eigen}
On s'intéresse ici plus particulièrement au problème \ref{curlcurl},
\begin{equation*}
(\lambda_i^2,\mathbf{g}_i)\in\R\times D^1(\Omega)\quad \left\{\begin{aligned}
&\curll  \mathbf{g}_i = \lambda_i^2 \mathbf{g}_i\\
&\div \mathbf{g}_i = 0\\
&\mathbf{g}_i\cdot \mathbf{n}\restr{\Gamma} = 0\\
&\curl \mathbf{g}_i\cdot \mathbf{n}\restr{\Gamma} = 0\\
&(\curll  \mathbf{g}_i\cdot \mathbf{n}\restr{\Gamma} = 0)
\end{aligned}\right.
\end{equation*}
qui va nous donner les valeurs et fonctions propres de l'opérateur rotationnel avec les conditions aux limites d'imperméabilité. Cela nous permettra ensuite d'exprimer $\mathbf{u}$ dans la base de $D^1$ formée par ces fonctions.\\

Tout d'abord, notons que d'après P. Penel \cite{Penel2004} lemme 3.3, on a :
\[ D^1 = \{\mathbf{v}=\mathbf{v_0}+\grad\phi\; |\; \mathbf{v_0}\in H^1_0(\Omega),\ -\laplace\phi=\div\bm{v_0},\ \grad\phi\cdot \mathbf{n}\restr{\Gamma} = 0 \} \]
Ainsi, sur $\partial\Omega$, $\mathbf{v}=\grad\phi$.\\

On va maintenant chercher la formulation variationnelle, soit $\mathbf{g}\in D^1(\Omega)$ solution du problème \ref{curlcurl}, alors pour tout $\bm{\varphi}\in D^1(\Omega)$ nous avons :
\[ \int_\Omega (\curll \mathbf{g})\cdot\bm{\varphi}\ dX = \int_\Omega\lambda^2 \mathbf{g}\cdot\bm{\varphi}\ dX \]
puis en intégrant par partie :
\[ \int_\Omega (\curl \mathbf{g})\cdot(\curl\bm{\varphi})\ dX + \int_{\partial\Omega} ((\curl \mathbf{g})\times \bm{\varphi})\cdot \mathbf{n}\ d\Gamma = \lambda^2\int_\Omega \mathbf{g}\cdot\bm{\varphi}\ dX \]
On a $\bm{\varphi}=\bm{\varphi}_0+\grad\phi$ et sur $\partial\Omega,\, \bm{\varphi}\restr{\Gamma}=\grad\phi$, d'où :
\[ \int_\Omega (\curl \mathbf{g})\cdot(\curl\bm{\varphi})\ dX + \int_{\partial\Omega} ((\curl \mathbf{g})\times \grad\phi)\cdot \mathbf{n}\ d\Gamma = \lambda^2\int_\Omega \mathbf{g}\cdot\bm{\varphi}\ dX \]
En utilisant le théorème de flux-divergence aussi appelé théorème de Green-Ostrogradski :
\[ \int_\Omega (\curl \mathbf{g})\cdot(\curl\bm{\varphi})\ dX + \int_\Omega \div((\curl \mathbf{g})\times \grad\phi)\ dX = \lambda^2\int_\Omega \mathbf{g}\cdot\bm{\varphi}\ dX \]
En utilisant la formule $\div(\mathbf{F}\times \mathbf{G}) = \mathbf{G}\cdot \curl \mathbf{F} - \mathbf{F}\cdot \curl \mathbf{G}$, on a :
\[ \int_\Omega (\curl \mathbf{g})\cdot(\curl\bm{\varphi})\ dX + \int_\Omega \grad\phi\cdot(\curll \mathbf{g})\ dX -\int_\Omega (\curl \mathbf{g})\cdot (\curl\grad\phi)\ dX  = \lambda^2\int_\Omega \mathbf{g}\cdot\bm{\varphi}\ dX \]
Comme le rotationnel d'un gradient est nul, on a :
\[ \int_\Omega (\curl \mathbf{g})\cdot(\curl\bm{\varphi})\ dX + \int_\Omega \grad\phi\cdot(\curll \mathbf{g})\ dX  = \lambda^2\int_\Omega \mathbf{g}\cdot\bm{\varphi}\ dX \]
En intégrant le deuxième terme par partie, on obtient :
\[ \int_\Omega (\curl \mathbf{g})\cdot(\curl\bm{\varphi})\ dX + \int_{\partial\Omega} \phi((\curll \mathbf{g})\cdot \mathbf{n})\ d\Gamma - \int_\Omega \phi(\div(\curll \mathbf{g}))\ dX  = \lambda^2\int_\Omega \mathbf{g}\cdot\bm{\varphi}\ dX \]
Comme $\curll  \mathbf{g}_i\cdot \mathbf{n}\restr{\Gamma} = 0$, le deuxième terme s'annule et comme la divergence d'un rotationnel est nulle, le troisième terme s'annule aussi. Le problème \ref{curlcurl} devient donc :
\begin{pb}\label{fveigen}
Trouver $(\lambda_i^2,\mathbf{g}_i)\in \R\times D^1$ tel que $\forall \bm{\varphi}\in D^1(\Omega)$ :
\begin{equation*}
\int_\Omega (\curl \mathbf{g})\cdot(\curl\bm{\varphi})\ dX = \lambda^2\int_\Omega \mathbf{g}\cdot\bm{\varphi}\ dX
\end{equation*}
\end{pb}
On obtient donc $(\lambda_i^2,\mathbf{g}_i)_{i=1,\dots,M}$, où $(\mathbf{g}_i)$ tend vers une base de $D^1(\Omega)$ lorsque $M\rightarrow \infty$.
%\iffalse
\section{Décomposition des fonctions de $D^1$}
\label{decomp}

Comme énoncé plus tôt, tout élément de $D^1$ peut s'écrire $\bm{\varphi} = \bm{\varphi}_0 + \grad\phi$, y compris bien sûr les $\mathbf{g}_i$. Comme on va en avoir besoin pour les problèmes suivants, on va les décomposer en $\mathbf{g}_i=\mathbf{g_i^0}+\grad\psi_i$ avec $\mathbf{g_i^0}\restr{\Gamma} = 0$ et $\grad\psi_i\cdot \mathbf{n}\restr{\Gamma} = 0$.\\
On applique donc le rotationnel du rotationnel sur cette relation.\\
\[ \curll \mathbf{g_i^0} +\curll\grad\psi_i = \curll \mathbf{g}_i \]
Le dernier terme est nul car c'est le rotationnel d'un gradient. On utilise la formule $\curll \mathbf{v}=\grad(\div \mathbf{v})-\laplace \mathbf{v}$ sur le premier terme :
\[ \grad(\div \mathbf{g_i^0})-\laplace \mathbf{g_i^0} = \lambda_i^2 \mathbf{g}_i \]
On obtient donc le tableau suivant :
\begin{center}
\begin{tabular}{c|ccccc}
& $\mathbf{g}_i$ & = & $\mathbf{g_i^0}$ & + & $\grad\psi_i$ \\ \hline
$\curll\star$ & $\lambda_i^2\mathbf{g}_i$ & & $\grad(\div \mathbf{g_i^0})-\laplace \mathbf{g_i^0}$ & & 0\\ \hline
$\div\star$ & 0 & & $\div \mathbf{g_i^0}$ & & $\laplace\psi_i$\\ \hline
$\star\cdot \mathbf{n}\restr{\Gamma}$ & 0 & & 0 & & 0
\end{tabular}
\end{center}

\subsection{$g_i^0$}
En utilisant la première ligne, on parvient au problème :
\begin{equation*}
\left\{\begin{aligned}
\grad(\div \mathbf{g_i^0})-\laplace \mathbf{g_i^0} &= \lambda_i^2\mathbf{g}_i\\
\mathbf{g_i^0}\restr{\Gamma} &= 0
\end{aligned}\right.
\end{equation*}
On cherche $\mathbf{g_i^0}$ dans $[H^1_0(\Omega)]^3$. On multiplie donc cette équation par une fonction test de $[H^1_0(\Omega)]^3$ et on intègre :
\[ \int_\Omega \grad(\div \mathbf{g_i^0})\cdot\bm{\varphi} - \int_\Omega \laplace \mathbf{g_i^0}\cdot\bm{\varphi} = \int_\Omega \lambda_i^2\mathbf{g}_i\cdot\bm{\varphi} \]
On utilise ensuite la formule d'intégration par partie $\int_\Omega \grad{\mathbf{u}}\bm{\varphi} = -\int_\Omega \mathbf{u}\div\bm{\varphi} + \int_{\partial\Omega} \mathbf{u}\bm{\varphi}\cdot \mathbf{n}$ sur le premier terme :
\[ -\int_\Omega (\div \mathbf{g_i^0})(\div\bm{\varphi}) + \int_{\partial\Omega} (\div \mathbf{g_i^0})(\bm{\varphi}\cdot \mathbf{n}) - \int_\Omega \laplace \mathbf{g_i^0}\cdot\bm{\varphi} = \int_\Omega \lambda_i^2\mathbf{g}_i\cdot\bm{\varphi} \]
Comme $\bm{\varphi}\in [H^1_0(\Omega)]^3$, la seconde intégrale est nul. On intègre par partie le terme en laplacien :
\[ -\int_\Omega (\div \mathbf{g_i^0})(\div\bm{\varphi}) + \int_\Omega \overline{\grad \mathbf{g_i^0}}:\overline{\grad\bm{\varphi}} - \int_{\partial\Omega} (\overline{\grad \mathbf{g_i^0}}\cdot \mathbf{n})\cdot\bm{\varphi} = \int_\Omega \lambda_i^2\mathbf{g}_i\cdot\bm{\varphi} \]
Encore une fois, comme $\bm{\varphi}\in [H^1_0(\Omega)]^3$, le terme sur les bords s'annule. On obtient donc le problème suivant :
\begin{pb}\label{fvgi0}
Trouver $\mathbf{g_i^0}\in H^1_0$ tel que $\forall \bm{\varphi}\in [H^1_0(\Omega)]^3$ :
\begin{equation*}
-\int_\Omega (\div \mathbf{g_i^0})(\div\bm{\varphi}) + \int_\Omega \overline{\grad \mathbf{g_i^0}}:\overline{\grad\bm{\varphi}} = \int_\Omega \lambda_i^2\mathbf{g}_i\cdot\bm{\varphi}
\end{equation*}\end{pb}

\subsection{Gradient $\psi_i$}
\label{multLagrange}
D'autre part, les deux dernières lignes du tableau nous donnent le problème de Poisson suivant :
\begin{equation*}
\left\{\begin{aligned}
-\laplace\psi_i &= \div \mathbf{g_i^0}\\
\grad\psi_i\cdot \mathbf{n}\restr{\Gamma} &= 0
\end{aligned}\right.
\end{equation*}
Cette fois-ci, on cherche $\psi_i$ dans $H^1(\Omega)$. On a donc la forme variationnelle suivante :
\begin{equation}\label{fvpsi}
\int_\Omega \grad\psi_i\cdot\grad\varphi = \int_\Omega (\div \mathbf{g_i^0})\varphi
\end{equation}

Ce problème permet de trouver $\psi_i$ seulement à une constante près, on va donc devoir imposer une constante de notre choix, par exemple pour que $\int_\Omega \psi_i = 0$. Ceci va donc créer une translation dans le résultat, qu'il va falloir corriger en post-traitement.\\
Afin d'appliquer cette contrainte supplémentaire, on va utiliser la méthode des multiplicateurs de Lagrange.\\
Si l'on note $V=H^1(\Omega)$, $a(u,v)=\int \grad u \cdot \grad v$, $l(v)=\int (\div \mathbf{g_i^0})v$ et $J(v)=\frac{1}{2}a(v,v)-l(v)$, alors résoudre l'équation \ref{fvpsi} revient à trouver $u$ tel que :
\[ J(u) = \min_{v\in V} J(v) \]
Si l'on ajoute la contrainte $b(v) = \int v = 0$, alors, avec $\lambda$ un multiplicateur de Lagrange, le problème devient trouver $u$ tel que :
\[ J(u) = \min_{v\in V} J(v) - \lambda b(v) \]
Soit, en ajoutant l'équation de la contrainte multipliée par le multiplicateur de Lagrange correspondant à $\varphi$, et le terme correspondant à la moyenne $m$ de $\psi_i$ :
\[ a(\psi_i,\varphi) + \lambda b(v) + \mu b(u) = l(\varphi) + m b(\mu) \]
Ce qui donne le problème suivant :
\begin{pb}\label{fvpsiml}
Trouver $(\psi_i,\lambda)\in H^1(\Omega)\times L^2(\Omega)$ tel que $\forall (\varphi,\mu)$ :
\begin{equation*}
\int_\Omega \grad\psi_i\cdot\grad\varphi + \int_\Omega \lambda\varphi + \int_\Omega \psi_i\mu = \int_\Omega (\div \mathbf{g_i^0})\varphi + \int_\Omega m\ \mu
\end{equation*}\end{pb}
%\fi

%%% Local Variables:
%%% TeX-master: "../report.tex"
%%% eval: (flyspell-mode 1)
%%% ispell-local-dictionary: "french"
%%% End:
