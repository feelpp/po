\chapter{Problème spectral pour le calcul de $\mathbf{u}$}
On peut maintenant résoudre le problème \ref{fvc} :
\[ \frac{\partial c_k}{\partial t} + \frac{1}{Re}c_k\lambda_k^2 + \sum_{i=1}^M\sum_{j=1}^Mc_ic_jR_{ijk} + \sum_{i=1}^Mc_iR_{iak} + \sum_{i=1}^Mc_iR_{raij} = R_{fk}+R_{pk} \]
où
\begin{align*}
R_{ijk} &= \lambda_i\int_\Omega(\mathbf{g_i}\times \mathbf{g_j})\cdot\mathbf{g_k} & R_{iak} &= \lambda_i\int_\Omega(\mathbf{g_i}\times \mathbf{a})\cdot\mathbf{g_k}\\
R_{raij} &= \int_\Omega((\rot\mathbf{a})\times \mathbf{g_i})\cdot\mathbf{g_k} & R_{fk} &= \int_\Omega\mathbf{f_a}\cdot\mathbf{g_k}\\
R_{pk} &= \int_{\partial\Omega} \alpha_2\phi_k
\end{align*}
Cela nous permet de connaître les coefficients $c_i$ dans la décomposition $\mathbf{u}=\sum_{i=1}^M c_i\mathbf{g_i}$.\\

%% Dans l'état actuel, comme nous n'avons pas encore déterminé comment relever $\alpha_2$, on veut faire porter sa contribution sur $\mathbf{f}$ et donc dans le terme $R_{fk}$. La conséquence est que $\mathbf{a}$ ne relève pas la condition $\rott\mathbf{v}\cdot\mathbf{n}=\alpha_2$, ainsi $\mathbf{a}=\grad\psi^0+\rot\mathbf{b}$.\\

De plus, dans le cylindre, comme $\alpha_1=0$, on a $\mathbf{a}=\grad\psi^0$ et donc comme le rotationnel d'un gradient est nul, $\rot \mathbf{a} = 0$. Ce qui amène le terme $R_{raij}$ à être nul. On a donc plus que :
\[ \frac{\partial c_k}{\partial t} + \frac{1}{Re}c_k\lambda_k^2 + \sum_{i=1}^M\sum_{j=1}^Mc_ic_jR_{ijk} + \sum_{i=1}^Mc_iR_{iak} = R_{fk}+R_{pk} \]



\todo[inline]{pression comme second membre, $\alpha_2$ dépend de $\delta p$}

\section{Navier-Stokes}
\label{PSNewton}
Cette équation comporte le terme
\[ \sum_{i=1}^M\sum_{j=1}^M c_i c_jR_{ijk} \]
qui est non linéaire. Ce problème s'écrit donc sous la forme :
\[ F(c) = 0 \]
où $F:\R^M\rightarrow\R^M$, $c=(c_1,\ldots, c_M)$ et :
\begin{equation}\label{psf}
 F_k(c) = \frac{1}{Re} c_k\lambda_k^2 + \sum_i c_i R_{iak} + \sum_{i,j} c_i c_j R_{ijk} - R_{fk} - R_{pk}
\end{equation}

On va utiliser une méthode de Newton pour résoudre ce problème, on cherche donc :
\begin{equation}\label{Newton}
c^{(l+1)} = c^{(l)} - J(c^{(l)})^{-1}F(c^{(l)})\quad l=0,1,\ldots
\end{equation}
où $c^{(0)}$ est une donnée initiale et $J(c)$ est la matrice jacobienne de $F$ en $c$ :
\[ J(c)=
\begin{pmatrix}
\frac{\partial F_1(c)}{\partial c_1} & \ldots & \frac{\partial F_1(c)}{\partial c_M}\\
\vdots & \ddots & \vdots\\
\frac{\partial F_M(c)}{\partial c_1} & \ldots & \frac{\partial F_M(c)}{\partial c_M}
\end{pmatrix}\]
avec 
\begin{equation}\label{psj}
J(c)_{ki} = \frac{\partial F_k(c)}{\partial c_i} = \delta_{ki}\lambda_i^2 + R_{iak} + \sum_j c_j \underbrace{\lambda_i(\mathbf{g_i}\times\mathbf{g_j},\mathbf{g_k})}_{R_{ijk}} + \sum_j c_j\underbrace{\lambda_j (\mathbf{g_j}\times\mathbf{g_i},\mathbf{g_k})}_{R_{jik}}
\end{equation}
avec $\delta_{ki}$ le symbole de Kronecker et où on a inversé le rôle de $i$ et de $j$ dans $R_{jik}$.\\

Résoudre (\ref{Newton}) est équivalent à résoudre 
\begin{align}
J(c^{(l)})\delta c^{(l)} = -F(c^{(l)})\label{INewton1}\\
c^{(l+1)}=c{(l)}+\delta c^{(l)}\nonumber
\end{align}

On itère ce système jusqu'à ce que la différence entre deux itération soit inférieure à une certaine tolérance, c'est-à-dire $||\delta c^{(l)}||<tol$ ou que le nombre d'itération soit trop grand, ce qui indique la non convergence du système.

\subsection{Implémentation}
Comme on va manipuler des matrices pleines, au lieu d'utiliser PETSc, qui est plus adapté aux matrices creuses, on utilise la librairie Eigen \cite{eigenweb}.\\
Dans cette librairie, on déclare une matrice comme \texttt{Matrix<type, ligne, colonne>}. Afin de pouvoir utiliser des matrices de tailles différentes à chaque exécution, on utilise le mot clé \texttt{Dynamic} pour la taille. Il existe un type pré-déclaré pour le type \texttt{Matrix<Double, Dynamic, Dynamic>} : \texttt{MatrixXd}. De même pour le type \texttt{Matrix<Double, Dynamic, 1>}, qui est un vecteur de double de taille variable, on utilise le mot clé \texttt{VectorXd}.\\

Par exemple, si on a un vecteur de la librairie standard contenant les différentes valeurs des $\lambda_i$, on les stock dans un vecteur de la librairie Eigen avec les lignes de code suivantes : 
\lstinputlisting[linerange=lambda]{../../src/spectralproblem.cpp}
où $M$ est le nombre de valeurs propres que l'on a précédemment calculées. On remarque aussi que l'on peut accéder aux éléments du vecteur avec l'opérateur \texttt{()}.\\

Chaque somme de l'équation \ref{fvc} peut être considérée comme un élément de la librairie Eigen. Ainsi $R_{iak}=\lambda_i\int (\mathbf{g_i}\times\mathbf{a})\cdot\mathbf{g_k}$ est un élément d'une matrice, celui de la $i$\ieme\ ligne et $k$\ieme\ colonne.\\
$R_{fk} = \int \mathbf{f_a}\cdot\mathbf{g_k}$ et $R_{pk} = \int_{\partial\Omega}\alpha_2\psi_k$ sont des vecteurs et $R_{ijk} = \lambda_i\int (\mathbf{g_i}\times\mathbf{a})\cdot\mathbf{g_k}$ est un élément en 3 dimensions, donc un vecteur de matrices. On a ainsi les déclarations suivantes :\\
\lstinputlisting[linerange=ri]{../../src/spectralproblem.hpp}

On initialise d'abord la mémoire nécessaire pour contenir le vecteur, puis on initialise chaque élément de la matrice $R_{iak}$ à l'aide de deux boucles imbriquées :\\
\lstinputlisting[linerange=riak]{../../src/spectralproblem.cpp}

De même, on initialise la mémoire utilisée par la matrice, puis chaque élément du vecteur $R_{fk}$ avec une boucle :\\
\lstinputlisting[linerange=rfk]{../../src/spectralproblem.cpp}

On procède pareil pour $R_{pk}$ en ayant pris le soin de créer une expression avec $\alpha_2$, rentré en paramètre :\\
\lstinputlisting[linerange=rpk]{../../src/spectralproblem.cpp}

Pour $R_{ijk}$, on doit tout d'abord initialiser le vecteur contenant les matrices, puis pour chaque élément de ce vecteur, initialisé la mémoire de la matrice elle-même. Ensuite seulement, on peut initialiser chaque élément de $R_{ijk}$ :\\
\lstinputlisting[linerange=rijk]{../../src/spectralproblem.cpp}

Afin d'appliquer la méthode de Newton, il faut d'abord initialiser un vecteur pour stocker la solution $\delta c^{(l)}$ au problème \ref{INewton1} et choisir la tolérance pour laquelle on considère le système résolut.\\
\lstinputlisting[linerange=NSInit]{../../src/spectralproblem.cpp}

On peut maintenant appliquer la méthode de Newton décrite dans \ref{PSNewton}. On veut donc exprimer le vecteur $F(c)$ (\ref{psf}) dans la librairie Eigen. On utilise donc le mot clé \texttt{wiseProduct} pour faire un produit élément par élément de $c$ et de $\lambda$, et on multiplie la matrice $R_{iak}$ avec $c$, on retranche aussi le vecteur $R_{fk}$.\\
Puis pour la ligne $k$, le terme non linéaire $\sum_{i,j} c_i\lambda_i c_j (\mathbf{g_i}\times \mathbf{g_j}, \mathbf{g_k})$ est le produit $c^TR_{ijk}(k)c$ où $R_{ijk}(k)$ est la matrice se trouvant à la $k$\ieme\ position dans le vecteur $R_{ijk}$ et $c^T$ est le vecteur transposé de $c$.\\
\lstinputlisting[linerange=NSMatF]{../../src/spectralproblem.cpp}

Pour la matrice $J(c)$ (\ref{psj}), le terme $\delta_{ik}\lambda^2$, signifie que le vecteur $\lambda^2$ se trouve sur la diagonale de $J(c)$, on utilise donc le mot clé \texttt{asDiagonal} pour effectuer cette opération. On ajoute aussi la matrice $R_{iak}$ telle quelle à $J(c)$.\\
Le terme $\sum_j c_j\lambda_j (\mathbf{g_j}\times\mathbf{g_i},\mathbf{g_k})$ est égal au $i$\ieme\ élément du vecteur produit $c^TR_{ijk}$, tandis que $\sum_j\lambda_i c_j (\mathbf{g_i}\times\mathbf{g_j},\mathbf{g_k})$ est le $i$\ieme\ élément du vecteur $c^TR_{ijk}^T$, car $R_{jik}=R_{ijk}^T$.\\
\lstinputlisting[linerange=NSMatJ]{../../src/spectralproblem.cpp}

Il faut aussi initialiser le solveur avec lequel résoudre les systèmes \ref{INewton1}, par exemple, pour les résoudre avec une méthode de Householder, on utilisera les lignes suivantes :\\
\lstinputlisting[linerange={NSSys1,NSSys2}]{../../src/spectralproblem.cpp}

Il suffit maintenant d'additionner $c^{(l)}$ et $c^{(l+1)}$ pour obtenir la solution à l'itération suivante.\\
\lstinputlisting[linerange=NSAdd]{../../src/spectralproblem.cpp}

On itère ces étapes jusqu'à ce que la tolérance soit dépassée, ou que l'on pense que la méthode diverge.

\section{Stokes}
\subsection{Implémentation}
Si l'on supprime le terme non linéaire, l'équation s'écrit :
\[ \sum_i c_iR_{iak} +\frac{1}{Re}c_k\lambda^2 = R_{fk} + R_{pk} \]

Ainsi, il suffit de résoudre l'équation $Ax=b$ où $A=(a_{ki})$ avec :
\[ a_{ki} = \delta_{ki}\frac{\lambda^2}{Re} + R_{iak} \]
et
\[ b_k = R_{fk} + R_{pk} \]

Il faut donc créer la matrice $A$ à l'aide des $\lambda$ et de $R_{iak}$ :\\
\lstinputlisting[linerange=StokesA]{../../src/spectralproblem.cpp}
et le vecteur $b$ avec $R_{fk}$ :\\
\lstinputlisting[linerange=StokesB]{../../src/spectralproblem.cpp}

Puis, on résout le système de la même manière que précédemment :\\
\lstinputlisting[linerange=StokesSolve]{../../src/spectralproblem.cpp}

\subsection{Résultats}

% \section{Composante $z$}
% \subsection{Implémentation}
% \subsection{Résultats}


%%% Local Variables:
%%% TeX-master: "../report.tex"
%%% eval: (flyspell-mode 1)
%%% ispell-local-dictionary: "french"
%%% End:
